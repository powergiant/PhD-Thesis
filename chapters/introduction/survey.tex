\section{Previous research}


\quad\, (1) \textit{Early physics research.} In the physics literature, the inception of the kinetic description can be traced to the work of Peierls \cite{peierls1929kinetischen} in the study of anharmonic crystals, leading to the so-called phonon Boltzmann equation. Since then, the kinetic theory has been developed for various models and has become a systematic paradigm starting in the 1960s, with immense applications in various fields of physics and science \cite{benney1969random}, \cite{benney1966nonlinear}, \cite{davidson2012methods}, \cite{hasselmann1962non}, \cite{hasselmann1963non}, \cite{janssen2008progress}, \cite{nazarenko2011wave}, \cite{world1998guide}, \cite{spohn2006phonon}, \cite{spohn2008boltzmann}, \cite{vedenov1967theory}, \cite{zaslavskii1967limits}. The name wave turbulence theory comes from the spectral energy dynamics and cascades that the wave kinetic equation predicts for nonlinear wave systems, which yields similar conclusions to Kolmogorov spectra in hydrodynamic turbulence; this connection was is a major contribution of Zakharov \cite{zakharov2012kolmogorov}, \cite{zakharov1965weak}.

(2) \textit{Rigorous results about wave turbulence theory.} The WKE was first verified in the linear setting \cite{erdHos2000linear}, \cite{erdHos2008quantum}, \cite{soffer2020energy}. In the nonlinear setting, the WKE was rigorously verified for the Gibbs measure initial data by Lukkarinen and Spohn \cite{lukkarinen2011weakly}. Then the basic concepts of general wave turbulence were rigorously formulated by Buckmaster, Germain, Hani, and Shatah \cite{buckmaster2021onset} and a non-trivial result that verified WKE for a short time scale was also proved by them. 

The WKE was proved for an almost sharp time scale independently by Deng and Hani \cite{deng2021derivation} and Collot and Germain \cite{collot2019derivation}, \cite{collot2020derivation} using the ideas from the study of randomly initialized PDE. The full WKE for the sharp time was proved independently by the deep works of Deng and Hani \cite{deng2021full} and Staffilani and Tran \cite{staffilani2021wave} for a four-wave problem and a three-wave problem respectively. One key contribution of \cite{deng2021full} and \cite{staffilani2021wave} was the classification of Feynman diagrams in the contexts of normal form expansion and Liouville equation respectively. Later in \cite{deng2022rigorous}, \cite{deng2023derivation}, WKE for NLS was derived for all scaling laws by Deng and Hani.


WKE for the space-inhomogeneous case was derived by Ampatzoglou, Collot, and Germain \cite{ampatzoglou2021derivation} for an almost sharp time scale for a three wave problem. The result that reaching wave kinetic time was obtained by \cite{hannani2022wave}. The higher-order correlation functions were studied by Deng and Hani \cite{deng2021propagation}. A linearized wave kinetic equation near the Rayleigh-Jeans spectrum was derived by Faou \cite{faou2020linearized}. The discrete wave turbulence was studied by Dymov and Kuksin \cite{dymov2021formal}, \cite{dymov2020zakharov}, \cite{dymov2023formal}, \cite{dymov2021large}.

(3) \textit{Rigorous results about the dynamics of WKE.} There are also many papers about the dynamics of WKE itself. For references, see \cite{collot2022stability}, \cite{escobedo2015finite}, \cite{escobedo2015theory}, \cite{germain2020optimal}, \cite{gamba2020wave}, \cite{soffer2018dynamics}, \cite{rumpf2021wave}, \cite{soffer2020energy}, \cite{staffilani2023wave} and the reference therein.

%(4) numerical simulation

%(4) about boltzmann equation and randomized initial data


