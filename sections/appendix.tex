\appendix

\section{Remarks about other four wave systems}\label{app.rem}


In this section we introduce the new variable $\psi$ and rewrite \eqref{eq.NKLG} into the standard form of a dispersive equation. 



Notice that in \eqref{eq.NKLG} $\partial_{tt} u - \Delta u +u = -\lambda^2 u^3$, the operator
$\partial_{tt}-\Delta +1$ can be factorized into $(-i\partial_t+\Lambda(\nabla))\cdot(i\partial_t+\Lambda(\nabla))$, where $\Lambda(\nabla)$ is the Fourier multiplier with symbol $\Lambda(\xi)=\sqrt{1+|\xi|^2}$. Therefore the equation can be simplified by introducing a new variable
\begin{equation}
    \psi=i\partial_tu+\Lambda(\nabla)u.
\end{equation}
Then we have
\begin{equation}\label{eq.firstorderderivation}
    (-i\partial_t+\Lambda(\nabla))\psi=(-i\partial_t+\Lambda(\nabla))\cdot(i\partial_tu+\Lambda(\nabla))u=-\lambda^2 u^3.
\end{equation} 

We can solve $u$ in terms of $\psi$, $\bar{\psi}$
\begin{equation}\label{eq.defpsi}
    u=(2\Lambda(\nabla))^{-1}(\psi+\bar{\psi}).
\end{equation} 
We can derive the equation of $\psi$ by substituting \eqref{eq.defpsi} into the right hand side of \eqref{eq.firstorderderivation}

\begin{equation}\label{eq.firstorder}
    i\partial_t\psi-\Lambda(\nabla)\psi=\frac{\lambda^2}{8} \Big(\Lambda(\nabla)^{-1}\psi+\Lambda(\nabla)^{-1}\bar{\psi}\Big)^3.
\end{equation} 


In this section we explain the renormalization argument.
Let $\psi_k$ be the Fourier coefficient of $\psi$. Then in term of $\psi_k$ equation (\ref{eq.firstorder}) becomes

\begin{equation}\label{eq.mainfourier'}
\begin{split}
i \dot{\psi}_{k} =& \Lambda_k u_k+\frac{\lambda^2}{8L^{2d}} \sum_{(k_1,k_2,k_{3}) \in (\mathbb{Z}^d_L)^3} m^{3}_{k_1k_2k_3}\psi_{k_1}\psi_{k_2}  \psi_{k_3}
+m^{2}_{k_1k_2k_3}\psi_{k_1}\overline{\psi}_{k_2}  \psi_{k_3}
\\
&+m^{1}_{k_1k_2k_3}\psi_{k_1}\overline{\psi}_{k_2}  \overline{\psi}_{k_3}
+m^{0}_{k_1k_2k_3}\overline{\psi}_{k_1}\overline{\psi}_{k_2}  \overline{\psi}_{k_3},
\end{split}
\end{equation}
where 
\begin{equation}
\begin{cases}
m^{3}_{k_1k_2k_3}=(\Lambda_{k_1}\Lambda_{k_2}\Lambda_{k_3})^{-1} \delta_{k_1 + k_2 +k_3 = k}
\\
m^{2}_{k_1k_2k_3}=(\Lambda_{k_1}\Lambda_{k_2}\Lambda_{k_3})^{-1} \delta_{k_1 - k_2 +k_3 = k}
\\
m^{1}_{k_1k_2k_3}=(\Lambda_{k_1}\Lambda_{k_2}\Lambda_{k_3})^{-1} \delta_{k_1 - k_2 -k_3 = k}
\\
m^{0}_{k_1k_2k_3}=(\Lambda_{k_1}\Lambda_{k_2}\Lambda_{k_3})^{-1} \delta_{k_1 + k_2 +k_3 + k =0}
\end{cases}
\end{equation}

According to \eqref{eq.wellprepared}, the initial data of \eqref{eq.mainfourier'} can be written as 

\begin{equation}
    \psi_k(0) =  \sqrt{n_{\textrm{in}}(k)/2} \,(\alpha_k+i\beta_k)
\end{equation}
To keep the notation simpler we define
\begin{equation}
\begin{split}
    &\eta_{k}(\omega)=\frac{1}{\sqrt{2}}(\alpha_k+i\beta_k),
    \\
    &\xi_k = \psi_k(0) = \sqrt{n_{\textrm{in}}(k)} \, \eta_{k}(\omega).
\end{split}
\end{equation}
Therefore $\eta_{k}(\omega)$ are i.i.d complex normal distributions and $\xi_k$ are Fourier coefficients of the initial data of $\psi$.

As explained in Appendix \ref{app.rem}, the random data Cauchy problem of defocusing Klein-Gordon equation \eqref{eq.NKLG'} is almost equivalent to above equation after change of variable $\psi=i\partial_tu+\Lambda(\nabla)u$.
\begin{equation}\label{eq.NKLG'}
\begin{cases}
\partial_{tt} u = \Delta u -u - \lambda^2 u^3,  \quad x\in \mathbb{T}^d_{L_1\cdots L_d},   \\[.6em]
u(0,x) = u_{\textrm{in}}(x).
\\[.6em]
\partial_tu(0,x) = u^{1}_{\textrm{in}}(x)
\end{cases}    
\end{equation}


\section{Number theoretic results}
The mains results of this appendix is to prove Theorem \ref{th.numbertheory1} and Theorem \ref{th.numbertheory} 

\begin{thm}\label{th.numbertheory1}
Let $\Lambda(k)=\sqrt{1+|k|^2}$, then for all $\beta\in [1,2]^d$, the following number theory estimate is true

\begin{equation}\label{eq.numbertheory1}
    \sup_{\substack{c,n\in\mathbb{Z}_L^d\\T\le L}}
    T\#\left\{
        \begin{matrix}
            k_1,k_2,k_3\in\mathbb{Z}_L^d \\
            |k_1|,|k_2|,|k_3|\lesssim 1
        \end{matrix}
        :
        \begin{matrix}
            k_1-k_2+k_3=c \\
            \Lambda(k_1)-\Lambda(k_2)+\Lambda(k_3)=\Lambda(c)+n+O(T^{-1})
        \end{matrix}
    \right\}\le L^{2d}.
\end{equation}

\begin{equation}\label{eq.numbertheory1'}
    \sup_{\substack{c_1,c_2,n\in\mathbb{Z}_L^d\\T\le L}}
    T\#\left\{
        \begin{matrix}
            k_1,k_2\in\mathbb{Z}_L^d \\
            |k_1|, |k_2|\lesssim 1
        \end{matrix}
        :
        \begin{matrix}
            k_1+ k_2=c_1+ c_{2} \\
            \Lambda(k_1)+\Lambda(k_2)=\Lambda(c_1)+ \Lambda(c_2)+n+O(T^{-1})
        \end{matrix}
        \right\}\le L^{2d}.
\end{equation}

\begin{equation}\label{eq.numbertheory1''}
    \sup_{\substack{c_1,c_2,n\in\mathbb{Z}_L^d\\T\le L^2}}
    T\#\left\{
        \begin{matrix}
            k_1,k_2\in\mathbb{Z}_L^d \\
            |k_1|, |k_2|\lesssim 1
        \end{matrix}
        :
        \begin{matrix}
            k_1- k_2=c_1- c_{2},\ k_1\ne k_2 \\
            \Lambda(k_1)-\Lambda(k_2)=\Lambda(c_1)- \Lambda(c_2)+n+O(T^{-1})
        \end{matrix}
        \right\}\le L^{2d+1}.
\end{equation}

More generally, the proof of this theorem works for many other choices of $\Lambda(k)$.
\end{thm}


\begin{thm}\label{th.numbertheory}
Let $\Lambda(k)=\sqrt{1+|k|^2}$, then for sufficiently generic $\beta\in [1,2]^d$, the following number theory estimate is true for all $\theta\ll 1$

\begin{equation}\label{eq.numbertheory}
    \sup_{\substack{c,n\in\mathbb{Z}_L^d\\T\le L^{2}}}
    T\#\left\{
        \begin{matrix}
            k_1,k_2,k_3\in\mathbb{Z}_L^d \\
            |k_1|,|k_2|,|k_3|\le L^\theta
        \end{matrix}
        :
        \begin{matrix}
            k_1-k_2+k_3=c \\
            \Lambda(k_1)-\Lambda(k_2)+\Lambda(k_3)=\Lambda(c)+n+O(T^{-1})
        \end{matrix}
    \right\}\le L^{2d+O(\theta)}
\end{equation}

\begin{equation}\label{eq.numbertheory'}
    \sup_{\substack{c_1,c_2,n\in\mathbb{Z}_L^d\\T\le L^{2}}}
    T\#\left\{
        \begin{matrix}
            k_1,k_2\in\mathbb{Z}_L^d \\
            |k_1|, |k_2|\le L^\theta
        \end{matrix}
        :
        \begin{matrix}
            k_1\pm k_2=c_1\pm c_{2},\ k_1\ne k_2 \\
            \Lambda(k_1)\pm\Lambda(k_2)=\Lambda(c_1)\pm \Lambda(c_2)+n+O(T^{-1})
        \end{matrix}
    \right\}\le L^{2d+O(\theta)}
\end{equation}

More generally, the proof of this theorem works for many other choices of $\Lambda(k)$.
\end{thm}

% k_1,k_2,k_3\in\mathbb{Z}_L^d \\
% |k_1|,|k_2|\le L^\theta
% \end{matrix}
% :
% \begin{matrix}
% k_1-k_2+k_3=k \\
% \Lambda(k_1)-\Lambda(k_2)+\Lambda(k_3)=\Lambda(k)+n+O(T^{-1})
% \end{matrix}
% \right\}\le L^{2d+O(\theta)}
% \end{equation}
% More generally, the proof of this theorem works for many other choices of $\Lambda(k)$.
% \end{thm}

% \begin{rem}
% Although the proof of Theorem \ref{th.numbertheory} works for much more general $\Lambda(k)$, it's quite hard to formulate a general theorem that covers all physically interesting cases. Therefore, we only illustrate the ideas by proving some special cases.
% \end{rem}

\begin{proof}[Proof of Theorem \ref{th.numbertheory1}]\eqref{eq.numbertheory1} and \eqref{eq.numbertheory1'} are corollaries of the following lemma. 

\begin{lem}\label{lem.rationallemma} For any $\beta$
\begin{equation}
    \sup_{\substack{m,n\\m\lesssim 1}} \#\left\{\begin{array}{cc}
         x,y\in\mathbb{Z}^d_L  \\
         |x|,|y|\lesssim 1
    \end{array}:\begin{array}{cc}
         x+y=m  \\
         \Lambda_{\beta}(x)+\Lambda_{\beta}(y)=n+O(L^{-1})
    \end{array}\right\}\lesssim_{\beta} L^{d-2} .
\end{equation}
\end{lem}
\begin{proof}
It is easy to show that
\begin{equation}
\begin{split}
    &\#\left\{\begin{array}{cc}
         x,y\in\mathbb{Z}^d_L  \\
         |x|,|y|\lesssim 1
    \end{array}:\begin{array}{cc}
         x+y=m  \\
         \Lambda_{\beta}(x)+\Lambda_{\beta}(y)=n+O(L^{-1})
    \end{array}\right\}
    \\
    =&\#\left\{
         x\in \mathbb{Z}^d_L,\ |x|\lesssim 1:\Lambda_{\beta}(x)+\Lambda_{\beta}(m-x)=n+O(L^{-1})\right\}
\end{split}
\end{equation}

Define $\mathcal{D}_{m,n}=\left\{x\in \mathbb{Z}^d_L,\ |x|\lesssim 1: \Lambda_{\beta}(x)+\Lambda_{\beta}(m-x)=n+O(L^{-1})\right\}$, then we just need to show that 
\begin{equation}\label{eq.goalofrationallemma}
    \sup_{\substack{m,n\\m\lesssim 1}} \#\mathcal{D}_{m,n}\lesssim_{\beta} L^{d-1} .
\end{equation}

We prove \eqref{eq.goalofrationallemma} using volume bound. The proof is divided into two steps.

\textbf{Step 1.} In this step, we show that 
\begin{equation}\label{eq.goalofrationallemmastep1}
    \#\mathcal{D}_{m,n}\le L^{d} \text{vol}(\mathcal{D}^{\mathbb{R}}_{m,n}),
\end{equation}
where $\mathcal{D}^{\mathbb{R}}_{m,n}=\left\{x\in \mathbb{R}^d,\ |x|\lesssim 1:\Lambda_{\beta}(x)+\Lambda_{\beta}(m-x)=n+O(L^{-1})\right\}$.

\eqref{eq.goalofrationallemmastep1} can be proved from the following claim.

\textit{Claim.} If $x\in \mathcal{D}_{m,n}$, then $D_{1/(2L)}(x)\subseteq \mathcal{D}^{\mathbb{R}}_{m,n}$. Here  $D_{r}(x)=\{x'\in \mathbb{R}^d: \sup_{i=1,\cdots, d} |x'_i-x_i|\le r\}$ ($x_i$ are the components of $x$). 

We prove the claim now. $x\in \mathcal{D}_{m,n}$ is equivalent to $\Lambda_{\beta}(x)+\Lambda_{\beta}(m-x)=n+O(L^{-1})$. For any $x'\in D_{1/(2L)}(x)$, $|x'-x|\lesssim 1/L$, because $\Lambda$ is a Lifschitz function, we have $|\Lambda_{\beta}(x)-\Lambda_{\beta}(x')|\lesssim L^{-1}$. Therefore, 
\begin{equation}
\begin{split}
    &|\Lambda_{\beta}(x')+\Lambda_{\beta}(m-x')-n|
    \\
    \le &|\Lambda_{\beta}(x)+\Lambda_{\beta}(m-x)-n|+|\Lambda_{\beta}(x)-\Lambda_{\beta}(x')|+|\Lambda_{\beta}(m-x)-\Lambda_{\beta}(m-x')|
    \\
    \lesssim & L^{-1}.
\end{split}
\end{equation}
Therefore, we have $\Lambda_{\beta}(x')+\Lambda_{\beta}(m-x')-n=O(L^{-1})$ and thus $x'\in \mathcal{D}^{\mathbb{R}}_{m,n}$. This is true for any $x'\in D_{1/(2L)}(x)$, so $D_{1/(2L)}(x)\subseteq \mathcal{D}^{\mathbb{R}}_{m,n}$.

Since for different $x_1,x_2\in \mathcal{D}_{m,n}$, $D_{1/(2L)}(x_1)\cap D_{1/(2L)}(x_2)=\emptyset$, we have 
\begin{equation}
    \sum_{x\in \mathcal{D}_{m,n}} \text{vol}( D_{1/(2L)}(x))=\text{vol}\left( \bigcup_{x\in \mathcal{D}_{m,n}} D_{1/(2L)}(x)\right)\le \text{vol}(\mathcal{D}^{\mathbb{R}}_{m,n}).
\end{equation}
The left hand side equals to $L^{-d}\#\mathcal{D}_{m,n}$, so we get
\begin{equation}
    L^{-d}\#\mathcal{D}_{m,n}\le \text{vol}(\mathcal{D}^{\mathbb{R}}_{m,n}),
\end{equation}
which implies \eqref{eq.goalofrationallemmastep1}.

\textbf{Step 2.} In this step, we show that 
\begin{equation}\label{eq.goalofrationallemmastep2}
    \text{vol}(\mathcal{D}^{\mathbb{R}}_{m,n})\le L^{-1}.
\end{equation}
Combining \eqref{eq.goalofrationallemmastep1} and \eqref{eq.goalofrationallemmastep2}, we get
\eqref{eq.goalofrationallemma}, which proves Lemma \ref{lem.rationallemma}.


By calculating the Hessian we can show that $F_{\beta,m,n}=\Lambda_{\beta}(x)+\Lambda_{\beta}(m-x)-n$ is a convex function. Let $x_{\beta, m}$ be its unique critical point. By Morse lemma, we know that there exists $r_0$ such that $F_{\beta,m,n}=|y|^2-a$ after a change of variable.

In $\mathcal{D}^{\mathbb{R}}_{m,n}\backslash B_{r_0}(x_{\beta, m})$, $|\nabla F_{\beta,m,n}|\gtrsim 1$. Then by the coarea formula, we get
\begin{equation}
\begin{split}
    \text{vol}(\mathcal{D}^{\mathbb{R}}_{m,n}\backslash B_{r_0}(x_{\beta, m})) =&  \text{vol}(\{|F_{\beta,m,n}|\lesssim L^{-1}\}\backslash B_{r_0}(x_{\beta, m}))
    \\
    =& \int^{CL^{-1}}_{-CL^{-1}} \int_{\{F_{\beta,m,n}=s\}\backslash B_{r_0}(x_{\beta, m})}\frac{1}{|\nabla F_{\beta,m,n}|}d\sigma ds
    \\
    \lesssim& \int^{CL^{-1}}_{-CL^{-1}} \text{Area}(\{F_{\beta,m,n}=s\}\backslash B_{r_0}(x_{\beta, m})) ds
    \\
    \lesssim& L^{-1}.
\end{split}
\end{equation}

In $\mathcal{D}^{\mathbb{R}}_{m,n}\cap  B_{r_0}(x_{\beta, m})$, $F_{\beta,m,n}=|y|^2-a$ after a change of variable $y=y(x)$ and $y(B_{r_0}(x_{\beta, m}))\subseteq B_{Cr_0}(0)$ 
\begin{equation}
\begin{split}
    \text{vol}(\mathcal{D}^{\mathbb{R}}_{m,n}\cap B_{r_0}(x_{\beta, m})) \lesssim \text{vol}(\{||y|^2-a|\lesssim L^{-1}\}\cap B_{Cr_0}(0))\lesssim L^{-1}.
\end{split}
\end{equation}
Here the proof of the second inequality is elementary and is thus skipped. 

Therefore, we complete the proof of Lemma \ref{lem.rationallemma}
\end{proof}

We now return to the proof of Theorem \ref{th.numbertheory1}. Define 
\begin{equation}
    \mathcal{D}^{(1)}_{m,n}([-T^{-1},T^{-1} ])=\left\{\begin{matrix}
k_1,k_2,k_3\in\mathbb{Z}_L^d \\
|k_1|,|k_2|,|k_3|\lesssim 1
\end{matrix}
:
\begin{matrix}
k_1-k_2+k_3=c \\
\Lambda(k_1)-\Lambda(k_2)+\Lambda(k_3)=\Lambda(c)+n+O(T^{-1})
\end{matrix}
\right\}
\end{equation}

\begin{equation}
    \mathcal{D}^{(2)}_{m,n}([-T^{-1},T^{-1} ])=\left
    \{\begin{matrix}
k_1,k_2\in\mathbb{Z}_L^d \\
|k_1|, |k_2|\lesssim 1
\end{matrix}
:
\begin{matrix}
k_1+ k_2=c_1+ c_{2} \\
\Lambda(k_1)+\Lambda(k_2)=\Lambda(c_1)+ \Lambda(c_2)+n+O(T^{-1})
\end{matrix}
\right\}
\end{equation}


Let us first prove \eqref{eq.numbertheory1'}. For any $T\le L$, let $N=[L/T]+1$, then $\cup_{j=-N}^N [jL^{-1}, (j+1)L^{-1}]$ is a cover of $[-T^{-1}, T^{-1}]$. Therefore, $\cup_{j=-N}^N\mathcal{D}^{(2)}_{m,n}([jL^{-1}, (j+1)L^{-1}])$ is a cover of $\mathcal{D}^{(2)}_{m,n}([-T^{-1}$ $,T^{-1} ])$

By Lemma \ref{lem.rationallemma} we get 
\begin{equation}\label{eq.thrationalexpand}
    \#\mathcal{D}^{(2)}_{m,n}([-T^{-1},T^{-1} ])\lesssim \sum_{j=-N}^N\#\mathcal{D}^{(2)}_{m,n}([jL^{-1}, (j+1)L^{-1}])\lesssim NL^{d-1}\lesssim L^dT^{-1}.
\end{equation}

This proves \eqref{eq.numbertheory1'}.

Now we prove \eqref{eq.numbertheory1}. 

\begin{equation}\label{eq.thrationalexpandlong}
\begin{split}
\#\mathcal{D}^{(1)}_{m,n}([-T^{-1},T^{-1} ])=&\#\left\{\begin{matrix}
k_1,k_2,k_3\in\mathbb{Z}_L^d \\
|k_1|,|k_2|,|k_3|\lesssim 1
\end{matrix}
:
\begin{matrix}
k_1-k_2+k_3=c \\
\Lambda(k_1)-\Lambda(k_2)+\Lambda(k_3)=\Lambda(c)+n+O(T^{-1})
\end{matrix}
\right\}
\\
=&\sum_{|c_2|\lesssim 1} \#\left\{\begin{matrix}
k_1,k_3\in\mathbb{Z}_L^d \\
|k_1|,|k_3|\lesssim 1
\end{matrix}
:
\begin{matrix}
k_1+k_3=c+c_2 \\
\Lambda(k_1)+\Lambda(k_3)=\Lambda(c)+\Lambda(c_2)+n+O(T^{-1})
\end{matrix}
\right\}
\\
= &  \sum_{|c_2|\lesssim 1} \mathcal{D}^{(2)}_{m,n}([-T^{-1},T^{-1} ])\lesssim \sum_{|c_2|\lesssim 1} L^{d} T^{-1}
\\
=&L^{2d} T^{-1}.
\end{split}
\end{equation}

This proves \eqref{eq.numbertheory1}.

\eqref{eq.numbertheory1''} is a simple corollary of the following lemma. 

\begin{lem}\label{lem.rationallemma2} For any $\beta$
\begin{equation}
    \sup_{\substack{m,n\\0\ne m\lesssim 1}} \#\left\{\begin{array}{cc}
         x,y\in\mathbb{Z}^d_L  \\
         |x|,|y|\lesssim 1
    \end{array}:\begin{array}{cc}
         x-y=m\ne 0  \\
         \Lambda_{\beta}(x)-\Lambda_{\beta}(y)=n+O(L^{-2})
    \end{array}\right\}\lesssim_{\beta} L^{d-1} .
\end{equation}
\end{lem}
\begin{proof} The proof of this lemma is similar to that of Lemma \ref{lem.rationallemma}. 

Since $m\in\mathbb{Z}^d_L$ and $m\ne 0$, we have $|m|\ge L^{-1}$. Therefore, we have

\begin{equation}
\begin{split}
    &\left\{x\in \mathbb{Z}^d_L,\ |x|\lesssim 1: \Lambda_{\beta}(x)-\Lambda_{\beta}(x-m)-n=O(L^{-2})\right\}
    \\
    \subseteq& \left\{x\in \mathbb{Z}^d_L,\ |x|\lesssim 1: |m|^{-1}(\Lambda_{\beta}(x)-\Lambda_{\beta}(x-m)-n)=O(L^{-1})\right\}
\end{split}
\end{equation}

Define $\mathcal{D}_{m,n}=\left\{x\in \mathbb{Z}^d_L,\ |x|\lesssim 1: |m|^{-1}(\Lambda_{\beta}(x)-\Lambda_{\beta}(x-m)-n)=O(L^{-1})\right\}$, then we just need to show that 
\begin{equation}\label{eq.goalofrationallemma2}
    \sup_{\substack{m,n\\m\lesssim 1}} \#\mathcal{D}_{m,n}\lesssim_{\beta} L^{d-1} .
\end{equation}

We prove \eqref{eq.goalofrationallemma2} using volume bound. The proof is divided into two steps.

\textbf{Step 1.} In this step, we show that 
\begin{equation}\label{eq.goalofrationallemma2step1}
    \#\mathcal{D}_{m,n}\le L^{d} \text{vol}(\mathcal{D}^{\mathbb{R}}_{m,n}),
\end{equation}
where $\left\{x\in \mathbb{R}^d,\ |x|\lesssim 1: |m|^{-1}(\Lambda_{\beta}(x)-\Lambda_{\beta}(x-m)-n)=O(L^{-1})\right\}$.

Since the Lifschitz norm of the function $F_{\beta,m,n}(x)=|m|^{-1}(\Lambda_{\beta}(x)-\Lambda_{\beta}(x-m)-n)$ is bounded by a universal constant, the same argument of step 1 in the proof of Lemma \ref{lem.rationallemma} works and we do not repeat it.

\textbf{Step 2.} In this step, we show that 
\begin{equation}\label{eq.goalofrationallemma2step2}
    \text{vol}(\mathcal{D}^{\mathbb{R}}_{m,n})\le L^{-1}.
\end{equation}
Combining \eqref{eq.goalofrationallemma2step1} and \eqref{eq.goalofrationallemma2step2}, we get
\eqref{eq.goalofrationallemma2}, which proves Lemma \ref{lem.rationallemma2}.


By calculating the Hessian we can show that $\Lambda_{\beta}(x)$ is a convex function, so
\begin{equation}
    |m|^{-1}\left|\nabla\Lambda_{\beta}(x)-\nabla\Lambda_{\beta}(x-m)\right|\gtrsim 1.
\end{equation}

Above equation implies that $|\nabla F_{\beta,m,n}|\gtrsim 1$ By the coarea formula, we get
\begin{equation}
\begin{split}
    \text{vol}(\mathcal{D}^{\mathbb{R}}_{m,n}) =&  \text{vol}(\{|F_{\beta,m,n}|\lesssim L^{-1}\})
    \\
    =& \int^{CL^{-1}}_{-CL^{-1}} \int_{\{F_{\beta,m,n}=s\}}\frac{1}{|\nabla F_{\beta,m,n}|}d\sigma ds
    \\
    \lesssim& \int^{CL^{-1}}_{-CL^{-1}} \text{Area}(\{F_{\beta,m,n}=s\}) ds
    \\
    \lesssim& L^{-1}.
\end{split}
\end{equation}


Therefore, we complete the proof of Lemma \ref{lem.rationallemma2}

\end{proof}

Finally we prove \eqref{eq.numbertheory1''}. Define
\begin{equation}
    \mathcal{D}^{(3)}_{m,n}([-T^{-1},T^{-1} ])=\left\{\begin{matrix}
k_1,k_2\in\mathbb{Z}_L^d \\
|k_1|, |k_2|\lesssim 1
\end{matrix}
:
\begin{matrix}
k_1- k_2=c_1- c_{2},\ k_1\ne k_2 \\
\Lambda(k_1)-\Lambda(k_2)=\Lambda(c_1)- \Lambda(c_2)+n+O(T^{-1})
\end{matrix}
\right\}
\end{equation}

Apply the same argument of getting \eqref{eq.thrationalexpand},
\begin{equation}
    \#\mathcal{D}^{(3)}_{m,n}([-T^{-1},T^{-1} ])\lesssim \sum_{j=-N}^N\#\mathcal{D}^{(3)}_{m,n}([jL^{-1}, (j+1)L^{-1}])\lesssim NL^{d-1}\lesssim L^dT^{-1}.
\end{equation}
This proves \eqref{eq.numbertheory1''} and we complete the proof of Theorem \ref{th.numbertheory1}.
\end{proof}



\begin{proof}[Proof of Theorem \ref{th.numbertheory}]
The $-$ case of \eqref{eq.numbertheory'} is equivalent to \eqref{eq.numbertheory1''}.

\eqref{eq.numbertheory} and the $+$ case of \eqref{eq.numbertheory'} are corollaries of the following lemma. 

\begin{lem}\label{lem.irrationallemma} For any $\theta$, for almost all $\beta$
\begin{equation}
    \sup_{\substack{m,n\\m\lesssim 1}} \#\left\{\begin{array}{cc}
         x,y\in\mathbb{Z}^d_L  \\
         |x|,|y|\lesssim L^{\theta}
    \end{array}:\begin{array}{cc}
         x+y=m  \\
         \Lambda_{\beta}(x)+\Lambda_{\beta}(y)=n+O(L^{-2+\theta})
    \end{array}\right\}\lesssim_{\theta,\beta} L^{d-2+\theta} .
\end{equation}
\end{lem}
\begin{proof} Let 
\begin{equation}
    \#_{m,n,L,\beta}=\#\left\{\begin{array}{cc}
         x,y\in\mathbb{Z}^d_L  \\
         |x|,|y|\lesssim L^{\theta}
    \end{array}:\begin{array}{cc}
         x+y=m  \\
         \Lambda_{\beta}(x)+\Lambda_{\beta}(y)=n+O(L^{-2+\theta})
    \end{array}\right\}
\end{equation}


Then the lemma is equivalent to 
\begin{equation}
    \mathbb{P}_{\beta}\left(\sup_{\substack{m,n\\m\lesssim 1}} \#_{m,n,L,\beta} \lesssim_{\theta,\beta} L^{d-2+\theta}\right)=1.
\end{equation}
which is also equivalent to
\begin{equation}\label{eq.irrationallemmaexpand}
    \mathbb{P}_{\beta}\left(\sup_{L}\sup_{\substack{m,n\\m\lesssim 1}} L^{-(d-2+\theta)}\#_{m,n,L,\beta} <\infty \right)=1.
\end{equation}

The proof of \eqref{eq.irrationallemmaexpand} is divided into several steps.

\textbf{Step 1.} In this step, we show that in the proof of \eqref{eq.irrationallemmaexpand}, we may replace $\sup_{L}$ and $\sup_{m,n}$ by supremum over discrete set $\mathcal{L}$ and $\mathcal{M}$, $\mathcal{N}$, i.e. $\sup_{L\in \mathcal{L}}$ and $\sup_{m\in\mathcal{M},n\in\mathcal{N}}$.

Since $x,y\in \mathbb{Z}_L^d$, we may assume that $m\in \mathbb{Z}_{L}^d$ and we take $\mathcal{M}=\mathbb{Z}_{L}^d$.

Since if $|n-n'|\lesssim L^{-2}$, then $\Lambda_{\beta}(x)+\Lambda_{\beta}(y)=n+O(L^{-2+\theta})$ $\Rightarrow$ $\Lambda_{\beta}(x)+\Lambda_{\beta}(y)=n'+O(L^{-2+\theta})$, we know that \eqref{eq.irrationallemmaexpand} is true for all $n\in\mathbb{Z}_{L^2}^d$ implies that it is true for all $n'\in\mathbb{R}^d$. We can thus take $\mathcal{N}=\mathbb{Z}_{L^2}^d$.

Let $K=[\text{log}_2 L]$ and $l=L/2^K$, then $L=l\cdot 2^K$ and $l\in [1,2]$. Since if $|l-l'|\lesssim L^{-2}$, then $\Lambda_{\beta}(x)+\Lambda_{\beta}(y)=n+O(L^{-2+\theta})$ $\Rightarrow$ $\Lambda_{\beta}(x)+\Lambda_{\beta}(y)=n'+O(L^{-2+\theta})$, we know that \eqref{eq.irrationallemmaexpand} is true for all $l\in [1,2]$ implies that it is true for all $l\in [1,2]\cap \mathbb{Z}_{L^2}$. We can thus take $\mathcal{L}=\{L=l\cdot 2^K:\ l\in [1,2]\cap \mathbb{Z}_{L^2},\ K\in \mathbb{N}_+\}$.

Therefore, to prove \eqref{eq.irrationallemmaexpand}, it suffices to show that 
\begin{equation}\label{eq.irrationallemmaexpand'}
    \mathbb{P}_{\beta}\left(\sup_{L\in \mathcal{L}}\sup_{\substack{m\in \mathcal{M},m\lesssim 1\\n\in \mathcal{N}}}  L^{-(d-2+\theta)}\#_{m,n,L,\beta} <\infty\right)=1.
\end{equation}

In what follows, we show that 
\begin{equation}\label{eq.irrationallemmaexpand''}
    \mathbb{P}_{\beta}\left(\sup_{L\in \mathcal{L}}L^{-(d-2+\theta)}\#_{m,n,L,\beta} <\infty \right)=1.
\end{equation}
\eqref{eq.irrationallemmaexpand''} implies that $\sup_{L\in \mathcal{L}}L^{-(d-2+\theta)}\#_{m,n,L,\beta} =\infty$ is a null set. Since a countable union of null sets is still a null set, \eqref{eq.irrationallemmaexpand'} is a corollary of \eqref{eq.irrationallemmaexpand''}.

\textbf{Step 2.} In this step, we derive \eqref{eq.irrationallemmaexpand''} from the following expectation bound.
\begin{equation}\label{eq.irrationallemmaexpectation}
    \mathbb{E}(\#_{m,n,L,\beta})\lesssim L^{d-2+\theta/2}.
\end{equation}

\eqref{eq.irrationallemmaexpectation} implies that
\begin{equation}\label{eq.irrationallemmaexpectation'}
    \mathbb{E}(\#_{m,n,l\cdot 2^K,\beta})\lesssim 2^{(d-2)K+\theta K/2}.
\end{equation}

Therefore, we have
\begin{equation}
    \mathbb{E}\left(\sum_{K} 2^{-(d-2+\theta)K}\#_{m,n,l\cdot 2^K,\beta}\right)\lesssim \sum_K 2^{-\theta K/2}\le C_{\theta}<\infty.
\end{equation}

This implies that almost surely
\begin{equation}
    \sum_{K} 2^{-(d-2+\theta)K}\#_{m,n,l\cdot 2^K,\beta}<\infty.
\end{equation}

Let $C_{\beta,\theta}=\sum_{K} 2^{-(d-2+\theta)K}\#_{m,n,l\cdot 2^K,\beta}$ which is finite almost surely. We get 
\begin{equation}
    \#_{m,n,l\cdot 2^K,\beta}\le C_{\beta,\theta} 2^{(d-2+\theta)K},
\end{equation}
which is equivalent to \eqref{eq.irrationallemmaexpand''}.

\textbf{Step 3.} In this step, we prove the expectation bound \eqref{eq.irrationallemmaexpectation}.

Let's calculate the expectation.
\begin{equation}\label{eq.irrationallemmastep2}
\begin{split}
    \mathbb{E}(\#_{m,n,L,\beta})=&\mathbb{E}\left(\sum_{\substack{x,y\in\mathbb{Z}^d_L,x+y=m \\ |x|,|y|\lesssim L^{\theta}}} \chi_{\Lambda_{\beta}(x)+\Lambda_{\beta}(y)=n+O(L^{-2})}\right)
    \\
    =&\sum_{\substack{x,y\in\mathbb{Z}^d_L,x+y=m \\ |x|,|y|\lesssim L^{\theta}}} \mathbb{P}_{\beta}(|\Lambda_{\beta}(x)+\Lambda_{\beta}(y)-n|\lesssim L^{-2})
    \\
    =&\sum_{\substack{x,y\in\mathbb{Z}^d_L,x+y=m \\ |x|,|y|\lesssim L^{\theta}}} \mathbb{P}_{\beta}(|F_{x,y}(\beta)|\lesssim L^{-2})
\end{split}
\end{equation}
Here we define $F_{x,y}(\beta)=\Lambda_{\beta}(x)+\Lambda_{\beta}(y)-n$.

We estimate $\mathbb{P}_{\beta}(|F_{x,y}(\beta)|\lesssim L^{-2})$ using coarea formula. 
\begin{equation}\label{eq.irrationallemmastep2'}
\begin{split}
    \mathbb{P}_{\beta}(|F_{x,y}(\beta)|\lesssim L^{-2})=&\text{Area}(|F_{x,y}(\beta)|\lesssim L^{-2})
    \\
    =& \int^{CL^{-2}}_{-CL^{-2}} \int_{\{F_{x,y}=s\}}\frac{1}{|\nabla F_{x,y}|}d\sigma ds
    \\
    \lesssim& L^{-2} \sup |\nabla F_{x,y}|^{-1} \text{Area}(F_{x,y}=s)
    \\
    \lesssim& L^{-2} \sup |\nabla F_{x,y}|^{-1}
\end{split}
\end{equation}

We thus need to get lower bound of $|\nabla F_{x,y}|$.

A simple calculation gives
\begin{equation}
    \nabla_{\beta} F_{x,y}(\beta)=\frac{1}{2\Lambda_{\beta}(x)}\begin{bmatrix} x_1^2\\\vdots\\x_d^2
    \end{bmatrix}+\frac{1}{2\Lambda_{\beta}(y)}\begin{bmatrix} y_1^2\\\vdots\\y_d^2
    \end{bmatrix},
\end{equation}
which implies that 
\begin{equation}
\begin{split}
    |\nabla_{\beta} F_{x,y}(\beta)|^2=&\frac{1}{4\Lambda^2_{\beta}(x)} (x_1^4+\cdots x_d^4) +\frac{1}{4\Lambda^2_{\beta}(y)} (y_1^4+\cdots y_d^4)+\frac{1}{2\Lambda_{\beta}(x)\Lambda_{\beta}(y)} (x_1^2y_1^2+\cdots x_d^2y_d^2)
    \\
    \gtrsim& L^{-} (|x|^2+|y|^2)^2.
\end{split}
\end{equation}

Therefore, we have 
\begin{equation}
    |\nabla_{\beta} F_{x,y}(\beta)|\gtrsim L^{-} (|x|^2+|y|^2).
\end{equation}

By \eqref{eq.irrationallemmastep2'}, we get for $(x,y)\ne (0,0)$
\begin{equation}
    \mathbb{P}_{\beta}(|F_{x,y}(\beta)|\lesssim L^{-2})\lesssim L^{-2+} (|x|^2+|y|^2)^{-1}.
\end{equation}
and for $x=y=0$
\begin{equation}
    \mathbb{P}_{\beta}(|F_{0,0}(\beta)|\le 1.
\end{equation}

By \eqref{eq.irrationallemmastep2},
\begin{equation}
\begin{split}
    \mathbb{E}(\#_{m,n,L,\beta})\lesssim & 1+L^{-2+}\sum_{\substack{x,y\in\mathbb{Z}^d_L,x+y=m \\ |x|,|y|\lesssim L^{\theta}}}  (|x|^2+|y|^2)^{-1}    
    \\
    \lesssim& L^{d-2+\theta/2}.
\end{split}
\end{equation}

Therefore, we complete the proof of the expectation bound \eqref{eq.irrationallemmaexpectation} and thus the proof of Lemma \ref{lem.irrationallemma}
\end{proof}

% \begin{lem}For any $\theta$, for almost all $\beta$
% \begin{equation}
%     \sup_{\substack{m,n\\m\gg 1}} \#\{x\in\mathbb{Z}^d_L, |x|\lesssim L^{\theta}:\Lambda_{\beta}(x)+\Lambda_{\beta}(m-x)=\Lambda_{\beta}(m)+n+O(L^{-2+\theta})\}\lesssim_{\theta,\beta} L^{d-2+\theta}. 
% \end{equation}

% \end{lem}

% \begin{proof}

% \end{proof}

The derivation of \eqref{eq.numbertheory} and the $+$ case of \eqref{eq.numbertheory'} from Lemma \ref{lem.irrationallemma} can be done by applying a similar argument of \eqref{eq.thrationalexpand} and \eqref{eq.thrationalexpandlong}.
\end{proof}
