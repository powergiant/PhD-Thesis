\section{Lattice points counting and convergence results}
In this section, we prove Proposition \ref{prop.treetermsupperbound} and \ref{prop.operatorupperbound} which gives upper bounds for tree terms $\mathcal{J}_{T,k}$ and the linearization operator $\mathcal{P}_T$. As explained before, these results are crucial in the proof of the main theorem. The proof is divided into several steps.

In section \ref{sec.refexp}, we calculate the coefficients of $\mathcal{J}_{T,k}$ as polynomials of Gaussian random variables.

In section \ref{sec.uppcoef}, we obtain upper bounds for the coefficients of these Gaussian polynomials.

Large deviation theory suggests that an upper bound of a Gaussian polynomial can be derived from an upper bound of its expectation and variance.

In section \ref{sec.coupwick}, we introduce the concept of couples which is a graphical method of calculating the expectation of Gaussian polynomials.

In section \ref{sec.numbertheory}, we use couple to establish a lattice point counting result.

In section \ref{sec.treetermsupperbound} and section \ref{sec.randommatrices}, we apply the lattice points counting result to derive upper bounds for  $\mathcal{J}_{T,k}$ and $\mathcal{P}_T$ respectively. This finishes the proof of Proposition \ref{prop.treetermsupperbound} and \ref{prop.operatorupperbound} and therefore the proof of the main theorem.


\subsection{Refined expression of coefficients}\label{sec.refexp} From \eqref{eq.treeterm.threewave}, it is easy to show that $\mathcal{J}_{T,k}$ are polynomials of $\xi$. In this section, we calculate the coefficients of $\mathcal{J}_{T,k}$ using the definition \eqref{eq.treeterm.threewave} of them.

Notice that all non-leaf nodes other than the root in a tree have degree $3$. For convenience, we add a new edge, called \underline{leg} $\mathfrak{l}$, to the root node, which makes the root also of degree $3$. This process is illustrated by Figure \ref{fig.leg}.
    \begin{figure}[H]
    \centering
    \scalebox{0.5}{
    \begin{tikzpicture}[level distance=80pt, sibling distance=100pt]
        \node[fillcirc] {} 
            child {node[fillcirc]  {}
                child {node[fillstar] {}}
                child {node[fillstar] {}}
                }
            child {node[fillstar] {}};
        \node[draw, single arrow,
              minimum height=33mm, minimum width=8mm,
              single arrow head extend=2mm,
              anchor=west, rotate=0] at (6,-3) {};  
        \node[] at (15,1.5) {} 
            child {node[fillcirc] {} 
                child {node[fillcirc]  {}
                    child {node[fillstar] {}}
                    child {node[fillstar] {}}
                }
                child {node[fillstar] {}}
            };
    \end{tikzpicture}
    }
        \caption{Adding a leg to a tree}
        \label{fig.leg}
    \end{figure}


We also need the following concepts about trees.
\begin{defn}\label{def.treemore}
\begin{enumerate}
    \item \textbf{Nodes and children of an edge:} As in Figure \ref{fig.childrenofedge}, let $\mathfrak{n}_{u}$ and $\mathfrak{n}_{l}$ be two endpoints of an edge $\mathfrak{e}$ and assume that $\mathfrak{n}_{l}$ is a children of $\mathfrak{n}_{u}$. We define $\mathfrak{n}_{u}$ (resp. $\mathfrak{n}_{l}$) to the \underline{upper node} (resp. \underline{lower node}) of $\mathfrak{e}$. Let $\mathfrak{n}_1$, $\mathfrak{n}_2$ be the two children of $\mathfrak{n}_{l}$ and let $\mathfrak{e}_1$(resp. $\mathfrak{e}_2$) be the edge between two nodes $\mathfrak{n}_{l}$ and $\mathfrak{n}_1$(resp. $\mathfrak{n}_2$). $\mathfrak{e}_1$, $\mathfrak{e}_2$ are defined to be the two \underline{children edges} of $\mathfrak{e}$. 
    \begin{figure}[H]
    \centering
    \scalebox{0.5}{
    \begin{tikzpicture}[level distance=80pt, sibling distance=100pt]
        \node[scale=2.0] at (13.8,-2.7) {$\mathfrak{e}$};
        \node[scale=2.0] at (14.3,-1.5) {$\mathfrak{n}_{u}$};
        \node[scale=2.0] at (12.6,-4) {$\mathfrak{n}_{l}$};
        \node[scale=2.0] at (12,-5.5) {$\mathfrak{e}_1$};
        \node[scale=2.0] at (14.5,-5.5) {$\mathfrak{e}_2$};
        \node[scale=2.0] at (11.5,-7.5) {$\mathfrak{n}_1$};
        \node[scale=2.0] at (15.2,-7.5) {$\mathfrak{n}_2$};
        \node[] at (15,1.5) {} 
            child {node[fillcirc] {} 
                child {node[fillcirc]  {}
                    child {node[fillstar] {}}
                    child {node[fillstar] {}}
                }
                child {node[fillstar] {}}
            };
    \end{tikzpicture}
    }
        \caption{Children $\mathfrak{e}_1$, $\mathfrak{e}_2$ of an edge $\mathfrak{e}$}
        \label{fig.childrenofedge}
    \end{figure}
    \item \textbf{Direction of an edge:} As in Figure \ref{fig.orientation}, each edge $\mathfrak{e}$ is  assigned with a \underline{direction}. This concept is mostly used to decide the value of variables $\iota_{\mathfrak{e}}\in\{\pm\}$ that will be defined later. Although it can be shown that the final result does not depend on the choices of the direction of each edge, for definiteness, we assign a downward direction to each edge. The orientation in Figure \ref{fig.orientation} is one example of this choice.
    \begin{figure}[H]
    \centering
    \scalebox{0.5}{
    \begin{tikzpicture}[level distance=80pt, sibling distance=100pt]
        \node[scale=2.0] at (11.5,-7.5) {$1$};
        \node[scale=2.0] at (15.0,-7.5) {$2$};
        \node[scale=2.0] at (16.8,-4.7) {$3$};
        \node[] at (15,1.5) (1) {} 
            child {node[fillcirc] (2) {} 
                child {node[fillcirc] (3) {}
                    child {node[fillstar] (4) {}}
                    child {node[fillstar] (5) {}}
                }
                child {node[fillstar] (6) {}}
            };
        \draw[-{Stealth[length=5mm, width=3mm]}] (1) -- (2);
        \draw[-{Stealth[length=5mm, width=3mm]}] (2) -- (3);
        \draw[-{Stealth[length=5mm, width=3mm]}] (2) -- (6);
        \draw[-{Stealth[length=5mm, width=3mm]}] (3) -- (4);
        \draw[-{Stealth[length=5mm, width=3mm]}] (3) -- (5);
    \end{tikzpicture}
    }
        \caption{Children $\mathfrak{e}_1$, $\mathfrak{e}_2$ of an edge $\mathfrak{e}$}
        \label{fig.orientation}
    \end{figure}
    \item \textbf{Labelling of leaves:} As in Figure \ref{fig.orientation}, each leaf is labelled by $1$, $2$, $\cdots$, $l(T)+1$ from left to right. An edge pointing to a leaf $\mathfrak{n}$ is also labeled by $j$ if $\mathfrak{n}$ is labeled by $j$.
\end{enumerate}

\end{defn}

Now we calculate the coefficients of $\mathcal{J}_{T,k}$.

\begin{lem}\label{lem.treeterms} Given a tree $T$ of depth $l=l(T)$, denote by $T_{\text{in}}$ the tree formed by all non-leaf nodes $\mathfrak{n}$, then associate each node $\mathfrak{n}\in T_{\text{in}}$ and edge $\mathfrak{l}\in T$ with variables $t_{\mathfrak{n}}$ and $k_{\mathfrak{l}}$ respectively. Given a labelling of all leaves by $1$, $2$, $\cdots$, $l+1$, we identify $k_{\mathfrak{e}}$ with $k_j$ if $\mathfrak{e}$ is connected to a leaf labelled by $j$. Given a node $\mathfrak{n}$, let $\mathfrak{e}_1$, $\mathfrak{e}_2$ and $\mathfrak{e}$ be the three edges from or pointing to $\mathfrak{n}$ ($\mathfrak{e}$ is the parent of $\mathfrak{e}_1$ and $\mathfrak{e}_2$), $\mathfrak{n}_1$, $\mathfrak{n}_2$ be children of $\mathfrak{n}$ and $\hat{\mathfrak{n}}$ be the parent of $\mathfrak{n}$. 

Let $\mathcal{J}_T$ be terms defined in Definition \ref{def.treeterms}, then their Fourier coefficients $\mathcal{J}_{T,k}$ are degree $l$ polynomials of $\xi$ given by the following formula

\begin{equation}\label{eq.coefterm.threewave}
\mathcal{J}_{T,k}=\left(\frac{i\lambda}{L^{d}}\right)^l\sum_{k_1,\, k_2,\, \cdots,\, k_{l+1}} H^T_{k_1\cdots k_{l+1}}  \xi_{k_1}\xi_{k_2}\cdots\xi_{k_{l+1}}   
\end{equation}
where $H^T_{k_1\cdots k_{l+1}}$ is given by
\begin{equation}\label{eq.coef.threewave}
H^T_{k_1\cdots k_{l+1}}=\int_{\cup_{\mathfrak{n}\in T_{\text{in}}} A_{\mathfrak{n}}} e^{\sum_{\mathfrak{n}\in T_{\text{in}}} it_{\mathfrak{n}}\Omega_{\mathfrak{n}}-\nu(t_{\widehat{\mathfrak{n}}}-t_{\mathfrak{n}})|k_{\mathfrak{e}}|^2} \prod_{\mathfrak{n}\in T_{\text{in}}} dt_{\mathfrak{n}} %\prod_{\mathfrak{n}\in T_{\text{in}} %\textit{with children $\mathfrak{n}_1$, $\mathfrak{n}_2$, $\mathfrak{n}_3$ }}
\ \delta_{\cap_{\mathfrak{n}\in T_{\text{in}}} \{S_{\mathfrak{n}}=0\}}\ \prod_{\mathfrak{e}\in T_{\text{in}}}\iota_{\mathfrak{e}}k_{\mathfrak{e},x} ,
\end{equation}
and $\iota$, $A_{\mathfrak{n}}$, $S_{\mathfrak{n}}$, $\Omega_{\mathfrak{n}}$ are defined by 
\begin{equation}\label{eq.iotadef.threewave}
    \iota_{\mathfrak{e}}=\begin{cases}
        +1 \qquad \textit{if $\mathfrak{e}$ pointing inwards to $\mathfrak{n}$}
        \\
        -1 \qquad  \textit{if $\mathfrak{e}$ pointing outwards from $\mathfrak{n}$}
    \end{cases}
\end{equation}
\begin{equation}
    A_{\mathfrak{n}}=
        \begin{cases}
            \{t_{\mathfrak{n}_1},\, t_{\mathfrak{n}_2},\, t_{\mathfrak{n}_3}\le t_{\mathfrak{n}}\} \qquad \textit{if $\mathfrak{n}\ne$ the root $\mathfrak{r}$}
            \\
            \{t_{\mathfrak{r}}\le t\} \qquad\qquad\qquad\  \textit{if $\mathfrak{n}= \mathfrak{r}$ }
        \end{cases}
\end{equation}
\begin{equation}\label{eq.defnS_n.threewave}
    S_{\mathfrak{n}}=\iota_{\mathfrak{e}_1}k_{\mathfrak{e}_1}+\iota_{\mathfrak{e}_2}k_{\mathfrak{e}_2}+\iota_{\mathfrak{e}}k_{\mathfrak{e}}
\end{equation}
\begin{equation}
    \Omega_{\mathfrak{n}}=\iota_{\mathfrak{e}_1}\Lambda_{k_{\mathfrak{e}_1}}+\iota_{\mathfrak{e}_2}\Lambda_{k_{\mathfrak{e}_2}}+\iota_{\mathfrak{e}}\Lambda_{k_{\mathfrak{e}}}
\end{equation}
For root node $\mathfrak{r}$, we impose the constrain that $k_{\mathfrak{r}}=k$ and $t_{\widehat{\mathfrak{r}}}=t$ (notice that $\mathfrak{r}$ does not have a parent so $\widehat{\mathfrak{r}}$ is not well defined). 
\end{lem}

\begin{proof}
We can check that $\mathcal{J}_T$ is defined by \eqref{eq.coefterm.threewave} and \eqref{eq.coef.threewave} satisfies the recursive formula \eqref{eq.treeterm.threewave} by direct substitution, so they are the unique solution of that recursive formula, and this proves Lemma \ref{lem.treeterms}.
\end{proof}



\subsection{An upper bound of coefficients in expansion series}\label{sec.uppcoef} In this section, we derive an upper bound for coefficients $H^T_{k_1\cdots k_{l+1}}$.

Notice that in \eqref{eq.coefterm.threewave}, $H^T_{k_1\cdots k_{l+1}}$ are integral of some oscillatory functions. An upper bound can be derived by the standard integration by parts arguments.

Associate each $\mathfrak{n}\in T_{\text{in}}$ with two variables $a_{\mathfrak{n}}$, $b_{\mathfrak{n}}$. Then we define
\begin{equation}\label{eq.defF_T.threewave}
F_{T}(t,\{a_{\mathfrak{n}}\}_{\mathfrak{n}\in T_{\text{in}}},\{b_{\mathfrak{n}}\}_{\mathfrak{n}\in T_{\text{in}}})=\int_{\cup_{\mathfrak{n}\in T_{\text{in}}} A_{\mathfrak{n}}} e^{\sum_{\mathfrak{n}\in T_{\text{in}}} it_{\mathfrak{n}} a_{\mathfrak{n}} - \nu(t_{\widehat{\mathfrak{n}}}-t_{\mathfrak{n}})b_{\mathfrak{n}}} \prod_{\mathfrak{n}\in T_{\text{in}}} dt_{\mathfrak{n}} 
\end{equation}

\begin{lem}\label{lem.boundcoef'}
We have the following upper bound for $F_{T}(t,\{a_{\mathfrak{n}}\}_{\mathfrak{n}},\{b_{\mathfrak{n}}\}_{\mathfrak{n}})$,
\begin{equation}\label{eq.boundcoef'.threewave}
    \sup_{\{b_{\mathfrak{n}}\}_{\mathfrak{n}}\lesssim 1} |F_{T}(t,\{a_{\mathfrak{n}}\}_{\mathfrak{n}},\{b_{\mathfrak{n}}\}_{\mathfrak{n}})|\lesssim \sum_{\{d_{\mathfrak{n}}\}_{\mathfrak{n}\in T_{\text{in}}}\in\{0,1\}^{l(T)}}\prod_{\mathfrak{n}\in T_{\text{in}}}\frac{1}{|q_{\mathfrak{n}}|+T^{-1}_{\text{max}}}.
\end{equation}
Fix a sequence $\{d_{\mathfrak{n}}\}_{\mathfrak{n}\in T_{\text{in}}}$ whose elements $d_{\mathfrak{n}}$ takes boolean values $\{0,1\}$. We define the two sequences $\{q_{\mathfrak{n}}\}_{\mathfrak{n}\in T_{\text{in}}}$, $\{r_{\mathfrak{n}}\}_{\mathfrak{n}\in T_{\text{in}}}$ by following recursive formula
\begin{equation}\label{eq.q_n'.threewave}
    q_{\mathfrak{n}}=
    \begin{cases}
    a_{\mathfrak{r}}, \qquad\qquad \textit{ if $\mathfrak{n}=$ the root $\mathfrak{r}$.}
    \\
    a_{\mathfrak{n}}+d_{\mathfrak{n}}q_{\mathfrak{n}'},\ \ \textit{ if $\mathfrak{n}\neq\mathfrak{r}$ and $\mathfrak{n}'$ is the parent of $\mathfrak{n}$.}
    \end{cases}
\end{equation}
\begin{equation}\label{eq.r_n'.threewave}
    r_{\mathfrak{n}}=
    \begin{cases}
    b_{\mathfrak{r}}, \qquad\qquad \textit{ if $\mathfrak{n}=$ the root $\mathfrak{r}$.}
    \\
    b_{\mathfrak{n}}+d_{\mathfrak{n}}q_{\mathfrak{n}'},\ \ \textit{ if $\mathfrak{n}\neq\mathfrak{r}$ and $\mathfrak{n}'$ is the parent of $\mathfrak{n}$.}
    \end{cases}
\end{equation}

% \begin{equation}
% Q=\{\{q_{\mathfrak{n}}\}_{\mathfrak{n}\in T_{\text{in}}}:q_{\mathfrak{n}}= q_{\mathfrak{n}}(),\ \forall\mathfrak{n}\in T_{\text{in}}\}    
% \end{equation}
% and the set $Q_{\mathfrak{n}}$ is defined inductively by 
% \begin{equation}\label{eq.treeterm.threewave}
%     Q_{\mathfrak{n}}=
%     \begin{cases}
%     \{a_{\mathfrak{r}}\}, \qquad\qquad\qquad\qquad \textit{ if $\mathfrak{n}=$ the root $\mathfrak{r}$.}
%     \\
%     \mathcal{T}_1(\mathcal{J}_{T_{\mathfrak{n}_1}}, \mathcal{J}_{T_{\mathfrak{n}_2}}, \mathcal{J}_{T_{\mathfrak{n}_3}}), \textit{ if $\mathfrak{n}\neq\mathfrak{r}$.}
%     \end{cases}
% \end{equation}
\end{lem}
\begin{proof} The lemma is proved by induction.

For a tree $T$ contains only one node $\mathfrak{r}$, $F_{T}=1$ and \eqref{eq.boundcoef'.threewave} is obviously true.

Assume that \eqref{eq.boundcoef'.threewave} is true for trees with $\le n-1$ nodes. We prove the $n$ nodes case. 

For general $T$, let $T_1$, $T_2$ be the two subtrees and $\mathfrak{n}_1$, $\mathfrak{n}_2$ be the two children of the root $\mathfrak{r}$, then by the definition of $F_T$\eqref{eq.defF_T.threewave}, we get
\begin{equation}\label{eq.lemboundcoef'1.threewave}
\begin{split}
    F_{T}(t)=&\int_{\cup_{\mathfrak{n}\in T_{\text{in}}} A_{\mathfrak{n}}} e^{\sum_{\mathfrak{n}\in T_{\text{in}}}it_{\mathfrak{n}} a_{\mathfrak{n}} - \nu(t_{\widehat{\mathfrak{n}}}-t_{\mathfrak{n}})b_{\mathfrak{n}}} \prod_{\mathfrak{n}\in T_{\text{in}}} dt_{\mathfrak{n}}    
    \\
    =&\int_{\cup_{\mathfrak{n}\in T_{\text{in}}} A_{\mathfrak{n}}}e^{it_{\mathfrak{r}} a_{\mathfrak{r}} - \nu(t-t_{\mathfrak{r}})b_{\mathfrak{r}}} e^{\sum_{\mathfrak{n}\in T_{\text{in},1}\cup T_{\text{in},2}} it_{\mathfrak{n}} a_{\mathfrak{n}} - \nu(t_{\widehat{\mathfrak{n}}}-t_{\mathfrak{n}})b_{\mathfrak{n}}}  \left(dt_{\mathfrak{r}}\prod_{j=1}^2\prod_{\mathfrak{n}\in T_{\text{in},j}}dt_{\mathfrak{n}}  \right)
    \\
    =&\int_{\cup_{\mathfrak{n}\in T_{\text{in}}} A_{\mathfrak{n}}}e^{it_{\mathfrak{r}}(a_{\mathfrak{r}}+T^{-1}_{\text{max}}\, \text{sgn}(a_{\mathfrak{r}}))- \nu(t-t_{\mathfrak{r}})b_{\mathfrak{r}}} e^{-iT^{-1}_{\text{max}}t_{\mathfrak{r}} \text{sgn}(a_{\mathfrak{r}})} e^{\sum_{\mathfrak{n}\in T_{\text{in},1}\cup T_{\text{in},2}} it_{\mathfrak{n}} a_{\mathfrak{n}} - \nu(t_{\widehat{\mathfrak{n}}}-t_{\mathfrak{n}})b_{\mathfrak{n}}}  \left(dt_{\mathfrak{r}}\prod_{j=1}^2\prod_{\mathfrak{n}\in T_{\text{in},j}}dt_{\mathfrak{n}}  \right)
\end{split}
\end{equation}



We do integration by parts in the above integrals using the Stokes formula. Notice that for $t_{\mathfrak{r}}$, there are three inequality constrains, $t_{\mathfrak{r}}\le t$ and $t_{\mathfrak{r}}\ge t_{\mathfrak{n}_1},t_{\mathfrak{n}_2}$. 


% \begin{equation}\label{eq.lemboundcoefexpand.threewave}
% \begin{split}
%     F_{T}(t)=&\frac{i}{a_{\mathfrak{r}}+\alpha \text{sgn}(a_{\mathfrak{r}}) }\int_{\cup_{\mathfrak{n}\in T_{\text{in}}} A_{\mathfrak{n}}} \frac{d}{dt_{\mathfrak{r}}}e^{-i(t_{\mathfrak{r}}a_{\mathfrak{r}}+\alpha\, \text{sgn}(a_{\mathfrak{r}}))} 
%     \\
%     &\qquad\qquad\qquad\qquad\qquad e^{-i\alpha(c(t_{\mathfrak{n}}) b_{\mathfrak{n}}- \text{sgn}(a_{\mathfrak{r}}))} e^{-i\sum_{\mathfrak{n}\in T_{\text{in}}} (t_{\mathfrak{n}}a_{\mathfrak{n}}+\alpha c(t_{\mathfrak{n}}) b_{\mathfrak{n}})}  \left(dt_{\mathfrak{r}}\prod_{j=1}^3\prod_{\mathfrak{n}\in T_{\text{in},j}}dt_{\mathfrak{n}}  \right)
%     \\
%     =&\frac{i}{a_{\mathfrak{r}}+\alpha \text{sgn}(a_{\mathfrak{r}}) }\left(\int_{\cup_{\mathfrak{n}\in T_{\text{in}}} A_{\mathfrak{n}},\ t_{\mathfrak{r}}=t}-\int_{\cup_{\mathfrak{n}\in T_{\text{in}}} A_{\mathfrak{n}},\ t_{\mathfrak{r}}=t_{\mathfrak{n}_1}}-\int_{\cup_{\mathfrak{n}\in T_{\text{in}}} A_{\mathfrak{n}},\ t_{\mathfrak{r}}=t_{\mathfrak{n}_2}}-\int_{\cup_{\mathfrak{n}\in T_{\text{in}}} A_{\mathfrak{n}},\ t_{\mathfrak{r}}=t_{\mathfrak{n}_3}}\right) 
%     \\
%     &\qquad\qquad\qquad e^{-i(t_{\mathfrak{r}}a_{\mathfrak{r}}+\alpha\, \text{sgn}(a_{\mathfrak{r}}))}e^{-i\alpha(c(t_{\mathfrak{n}}) b_{\mathfrak{n}}- \text{sgn}(a_{\mathfrak{r}}))} e^{-i\sum_{\mathfrak{n}\in T_{\text{in}}} (t_{\mathfrak{n}}a_{\mathfrak{n}}+\alpha c(t_{\mathfrak{n}}) b_{\mathfrak{n}})}  \left(dt_{\mathfrak{r}}\prod_{j=1}^3\prod_{\mathfrak{n}\in T_{\text{in},j}}dt_{\mathfrak{n}}  \right)
%     \\
%     -&\frac{i}{a_{\mathfrak{r}}+\alpha \text{sgn}(a_{\mathfrak{r}}) }\int_{\cup_{\mathfrak{n}\in T_{\text{in}}} A_{\mathfrak{n}}} e^{-i(t_{\mathfrak{r}}a_{\mathfrak{r}}+\alpha\, \text{sgn}(a_{\mathfrak{r}}))} 
%     \\
%     &\qquad\qquad\qquad\qquad\qquad \frac{d}{dt_{\mathfrak{r}}}e^{-i\alpha(c(t_{\mathfrak{n}}) b_{\mathfrak{n}}- \text{sgn}(a_{\mathfrak{r}}))} e^{-i\sum_{\mathfrak{n}\in T_{\text{in}}} (t_{\mathfrak{n}}a_{\mathfrak{n}}+\alpha c(t_{\mathfrak{n}}) b_{\mathfrak{n}})}  \left(dt_{\mathfrak{r}}\prod_{j=1}^3\prod_{\mathfrak{n}\in T_{\text{in},j}}dt_{\mathfrak{n}}  \right)
%     \\
%     =& \frac{i}{a_{\mathfrak{r}}+\alpha \text{sgn}(a_{\mathfrak{r}}) }(F_{I}-F_{T^{(1)}}-F_{T^{(2)}}-F_{T^{(3)}}-F_{II})
% \end{split}
% \end{equation}
\begin{equation}\label{eq.lemboundcoefexpand.threewave}
\begin{split}
    F_{T}(t)=\frac{1}{ia_{\mathfrak{r}}+iT^{-1}_{\text{max}} \text{sgn}(a_{\mathfrak{r}})+\nu b_{\mathfrak{r}} }&\int_{\cup_{\mathfrak{n}\in T_{\text{in}}} A_{\mathfrak{n}}} \frac{d}{dt_{\mathfrak{r}}}e^{it_{\mathfrak{r}}(a_{\mathfrak{r}}+T^{-1}_{\text{max}}\, \text{sgn}(a_{\mathfrak{r}}))- \nu(t-t_{\mathfrak{r}})b_{\mathfrak{r}}}  
    \\
    &e^{-iT^{-1}_{\text{max}}t_{\mathfrak{r}} \text{sgn}(a_{\mathfrak{r}})} e^{\sum_{\mathfrak{n}\in T_{\text{in},1}\cup T_{\text{in},2}} it_{\mathfrak{n}} a_{\mathfrak{n}} - \nu(t_{\widehat{\mathfrak{n}}}-t_{\mathfrak{n}})b_{\mathfrak{n}}}  \left(dt_{\mathfrak{r}}\prod_{j=1}^2\prod_{\mathfrak{n}\in T_{\text{in},j}}dt_{\mathfrak{n}}  \right)
\end{split}
\end{equation}
\begin{flalign*}
\hspace{1.3cm}
=&\frac{1}{ia_{\mathfrak{r}}+iT^{-1}_{\text{max}} \text{sgn}(a_{\mathfrak{r}})+\nu b_{\mathfrak{r}} }\left(\int_{\cup_{\mathfrak{n}\in T_{\text{in}}} A_{\mathfrak{n}},\ t_{\mathfrak{r}}=t}-\int_{\cup_{\mathfrak{n}\in T_{\text{in}}} A_{\mathfrak{n}},\ t_{\mathfrak{r}}=t_{\mathfrak{n}_1}}-\int_{\cup_{\mathfrak{n}\in T_{\text{in}}} A_{\mathfrak{n}},\ t_{\mathfrak{r}}=t_{\mathfrak{n}_2}}\right) &&
\\
& e^{it_{\mathfrak{r}}(a_{\mathfrak{r}}+T^{-1}_{\text{max}}\, \text{sgn}(a_{\mathfrak{r}}))- \nu(t-t_{\mathfrak{r}})b_{\mathfrak{r}}} e^{-iT^{-1}_{\text{max}}t_{\mathfrak{r}} \text{sgn}(a_{\mathfrak{r}})} e^{\sum_{\mathfrak{n}\in T_{\text{in},1}\cup T_{\text{in},2}} it_{\mathfrak{n}} a_{\mathfrak{n}} - \nu(t_{\widehat{\mathfrak{n}}}-t_{\mathfrak{n}})b_{\mathfrak{n}}} \left(dt_{\mathfrak{r}}\prod_{j=1}^2\prod_{\mathfrak{n}\in T_{\text{in},j}}dt_{\mathfrak{n}}  \right) &&
\end{flalign*}
\begin{flalign*}
\hspace{1.3cm}
-&\frac{1}{ia_{\mathfrak{r}}+iT^{-1}_{\text{max}} \text{sgn}(a_{\mathfrak{r}})+\nu b_{\mathfrak{r}} }\int_{\cup_{\mathfrak{n}\in T_{\text{in}}} A_{\mathfrak{n}}}e^{it_{\mathfrak{r}}(a_{\mathfrak{r}}+T^{-1}_{\text{max}}\, \text{sgn}(a_{\mathfrak{r}}))- \nu(t-t_{\mathfrak{r}})b_{\mathfrak{r}}} &&
    \\
    &\qquad\qquad\qquad\qquad\qquad \frac{d}{dt_{\mathfrak{r}}}(e^{-iT^{-1}_{\text{max}}t_{\mathfrak{r}} \text{sgn}(a_{\mathfrak{r}})}) e^{\sum_{\mathfrak{n}\in T_{\text{in},1}\cup T_{\text{in},2}} it_{\mathfrak{n}} a_{\mathfrak{n}} - \nu(t_{\widehat{\mathfrak{n}}}-t_{\mathfrak{n}})b_{\mathfrak{n}}}  \left(dt_{\mathfrak{r}}\prod_{j=1}^2\prod_{\mathfrak{n}\in T_{\text{in},j}}dt_{\mathfrak{n}}  \right) &&
\end{flalign*}
\begin{flalign*}
\hspace{1.3cm}
= \frac{1}{ia_{\mathfrak{r}}+iT^{-1}_{\text{max}} \text{sgn}(a_{\mathfrak{r}})+\nu b_{\mathfrak{r}} }(F_{I}-F_{T^{(1)}}-F_{T^{(2)}}-F_{II}) &&
\end{flalign*}

Here $T^{(j)}$, $j=1,2$ are trees that are obtained by deleting the root $\mathfrak{r}$, adding edges connecting $\mathfrak{n}_j$ with another node and defining $\mathfrak{n}_j$ to be the new root. For $T^{(j)}$, we can define the term $F_{T^{(j)}}$ by \eqref{eq.defF_T.threewave}. It can be shown that $F_{T^{(j)}}$ defined in this way is the same as the $\int_{\cup_{\mathfrak{n}\in T_{\text{in}}} A_{\mathfrak{n}},\ t_{\mathfrak{r}}=t_{\mathfrak{n}_j}}$ term in the second equality of \eqref{eq.lemboundcoefexpand.threewave}, so the last equality of \eqref{eq.lemboundcoefexpand.threewave} is true. $F_{I}$ is the $\int_{\cup_{\mathfrak{n}\in T_{\text{in}}} A_{\mathfrak{n}},\ t_{\mathfrak{r}}=t}$ term and $F_{II}$ is the last term containing $\frac{d}{dt_{\mathfrak{r}}}$.

We can apply the induction assumption to $F_{T^{(j)}}$ and show that $\frac{1}{ia_{\mathfrak{r}}+iT^{-1}_{\text{max}} \text{sgn}(a_{\mathfrak{r}})+\nu b_{\mathfrak{r}} } F_{T^{(j)}}$ can be bounded by the right hand side of \eqref{eq.boundcoef'.threewave}.

A direct calculation gives that 
\begin{equation}
    F_{I}(t)=e^{it a_{\mathfrak{r}} } F_{T_1}(t)F_{T_2}(t).
\end{equation}
Then the induction assumption implies that $\frac{1}{ia_{\mathfrak{r}}+iT^{-1}_{\text{max}} \text{sgn}(a_{\mathfrak{r}})+\nu b_{\mathfrak{r}} } F_{I}$ can be bounded by the right hand side of \eqref{eq.boundcoef'.threewave}.

Another direct calculation gives that 
\begin{equation}
    F_{II}(t)=\int^t_0  e^{it_{\mathfrak{r}}(a_{\mathfrak{r}}+T^{-1}_{\text{max}}\, \text{sgn}(a_{\mathfrak{r}}))- \nu(t-t_{\mathfrak{r}})b_{\mathfrak{r}}} \frac{d}{dt_{\mathfrak{r}}}(e^{-iT^{-1}_{\text{max}}t_{\mathfrak{r}} \text{sgn}(a_{\mathfrak{r}})})  F_{T_1}(t_{\mathfrak{r}})F_{T_2}(t_{\mathfrak{r}}) dt_{\mathfrak{r}}.
\end{equation}
Apply the induction assumption
\begin{equation}
\begin{split}
    &\left| \frac{1}{ia_{\mathfrak{r}}+iT^{-1}_{\text{max}} \text{sgn}(a_{\mathfrak{r}})+\nu b_{\mathfrak{r}} } F_{II}(t)\right|
    \\
    \le& \frac{1}{|q_{\mathfrak{r}}|+T^{-1}_{\text{max}}}\prod_{j=1}^2\left(\sum_{\{d_{\mathfrak{n}}\}_{\mathfrak{n}\in T_{\text{in},j}}\in\{0,1\}^{l(T_j)}}\prod_{\mathfrak{n}\in T_{\text{in},j}}\frac{1}{|q_{\mathfrak{n}}|+T^{-1}_{\text{max}}}\right)
    \\
    \le& \sum_{\{d_{\mathfrak{n}}\}_{\mathfrak{n}\in T_{\text{in}}}\in\{0,1\}^{l(T)}}\prod_{\mathfrak{n}\in T_{\text{in}}}\frac{1}{|q_{\mathfrak{n}}|+T^{-1}_{\text{max}}}.
\end{split}
\end{equation}

Combining the bounds of $F_{I}$, $F_{T^{(1)}}$, $F_{T^{(2)}}$, $F_{II}$, we conclude that $F_T$ can be bounded by the right hand side of \eqref{eq.boundcoef'.threewave} and thus complete the proof of Lemma \ref{lem.boundcoef'}.
\end{proof}
 
% For general $T$, let $T_1$, $T_2$, $T_3$ be the three subtrees of the root, then we have the following recursive formula for $F_{T}$
% \begin{equation}\label{eq.lemboundcoefrecur.threewave}
%     F_{T}(t)=\int^t_0 e^{-i(t_{\mathfrak{r}}a_{\mathfrak{r}}+\alpha c(t_{\mathfrak{r}})b_{\mathfrak{r}})} F_{T_1}(t_{\mathfrak{r}})F_{T_2}(t_{\mathfrak{r}})F_{T_3}(t_{\mathfrak{r}}) dt_{\mathfrak{r}}.
% \end{equation}





A straightforward application of the above lemma gives the following upper bound of the coefficients $H^T_{k_1\cdots k_{l+1}}$. 

\begin{lem}\label{lem.boundcoef}
We have the following upper bound for $H^T_{k_1\cdots k_{l+1}}$,
\begin{equation}\label{eq.boundcoef.threewave}
    |H^T_{k_1\cdots k_{l+1}}|\lesssim \sum_{\{d_{\mathfrak{n}}\}_{\mathfrak{n}\in T_{\text{in}}}\in\{0,1\}^{l(T)}}\prod_{\mathfrak{n}\in T_{\text{in}}}\frac{1}{|q_{\mathfrak{n}}|+T^{-1}_{\text{max}}}\ \prod_{\mathfrak{e}\in T_{\text{in}}}|k_{\mathfrak{e},x}|\ \delta_{\cap_{\mathfrak{n}\in T_{\text{in}}} \{S_{\mathfrak{n}}=0\}}.
\end{equation}
Fix a sequence $\{d_{\mathfrak{n}}\}_{\mathfrak{n}\in T_{\text{in}}}$ whose elements $d_{\mathfrak{n}}$ takes boolean values $\{0,1\}$. We define the two sequences $\{q_{\mathfrak{n}}\}_{\mathfrak{n}\in T_{\text{in}}}$, $\{r_{\mathfrak{n}}\}_{\mathfrak{n}\in T_{\text{in}}}$ by following recursive formula
\begin{equation}\label{eq.q_n.threewave}
    q_{\mathfrak{n}}=
    \begin{cases}
    \Omega_{\mathfrak{r}}, \qquad\qquad \textit{ if $\mathfrak{n}=$ the root $\mathfrak{r}$.}
    \\
    \Omega_{\mathfrak{n}}+d_{\mathfrak{n}}q_{\mathfrak{n}'},\ \ \textit{ if $\mathfrak{n}\neq\mathfrak{r}$ and $\mathfrak{n}'$ is the parent of $\mathfrak{n}$.}
    \end{cases}
\end{equation}
\begin{equation}\label{eq.r_n.threewave}
    r_{\mathfrak{n}}=
    \begin{cases}
    |k_{\mathfrak{r}}|^2, \qquad\qquad \textit{ if $\mathfrak{n}=$ the root $\mathfrak{r}$.}
    \\
    |k_{\mathfrak{n}}|^2+d_{\mathfrak{n}}q_{\mathfrak{n}'},\ \ \textit{ if $\mathfrak{n}\neq\mathfrak{r}$ and $\mathfrak{n}'$ is the parent of $\mathfrak{n}$.}
    \end{cases}
\end{equation}


\end{lem}
\begin{proof}
This is a direct corollary of Lemma \ref{eq.boundcoef'.threewave} if we take $a_{\mathfrak{n}}=\Omega_{\mathfrak{n}}$, $b_{\mathfrak{n}}=|k_{\mathfrak{e}}|^2$. 
\end{proof}


Lemma \ref{lem.boundcoef} suggests that the coefficients are small when $|q_{\mathfrak{n}}|\gg T^{-1}_{\text{max}}$. Therefore, in order to bound $\mathcal{J}_{T,k}$, we should count the lattice points on $|q_{\mathfrak{n}}|\lesssim T^{-1}_{\text{max}}$
\begin{equation}\label{eq.diophantineeq''.threewave}
    \{k_{\mathfrak{e}}\in \mathbb{Z}^d_L,\ |k_{\mathfrak{e}}|\lesssim 1,\ \forall \mathfrak{e}\in T: |q_{\mathfrak{n}}|\lesssim T^{-1}_{\text{max}},\ S_{\mathfrak{n}}=0,\ \forall \mathfrak{n}\in T.\ k_{\mathfrak{l}}=k\}
\end{equation}

By solving \eqref{eq.q_n.threewave}, we know that $\Omega_{\mathfrak{n}}$ is a linear combination of $q_{\mathfrak{n}}$, so there exist constants $c_{\mathfrak{n},\mathfrak{n}'}$ such that $\Omega_{\mathfrak{n}}=\sum_{\mathfrak{n}'}c_{\mathfrak{n},\mathfrak{n}'}q_{\mathfrak{n}'}$. Therefore, $|q_{\mathfrak{n}}|\lesssim T^{-1}_{\text{max}}$ implies that $|\Omega_{\mathfrak{n}}|\le\sum_{\mathfrak{n}'}|c_{\mathfrak{n},\mathfrak{n}'}q_{\mathfrak{n}'}|\lesssim T^{-1}_{\text{max}}$.

%For the sake of concreteness let's take $d_{\mathfrak{n}}=0$ for all nodes ${\mathfrak{n}}$, then $q_{\mathfrak{n}}=\Omega_{\mathfrak{n}}$ for all ${\mathfrak{n}}$ and
$|\Omega_{\mathfrak{n}}|\lesssim T^{-1}_{\text{max}}$ implies that \eqref{eq.diophantineeq''.threewave} is a subset of
\begin{equation}\label{eq.diophantineeq'.threewave}
    \{k_{\mathfrak{e}}\in \mathbb{Z}^d_L,\ |k_{\mathfrak{e}}|\lesssim 1,\ \forall \mathfrak{e}\in T: |\Omega_{\mathfrak{n}}|\lesssim T^{-1}_{\text{max}},\ S_{\mathfrak{n}}=0,\ \forall \mathfrak{n}\in T. \ k_{\mathfrak{l}}=k\}.
\end{equation}
To bound the number of elements of \eqref{eq.diophantineeq''.threewave}, we just need to do the same thing for \eqref{eq.diophantineeq'.threewave}.

\eqref{eq.diophantineeq'.threewave} can be read from the tree diagrams $T$. As in Figure \ref{fig.equations}, each edge corresponds to a variable $k_{\mathfrak{e}}$. The leg $\mathfrak{l}$ corresponds to equation $k_{\mathfrak{l}}=k$. Each node $\mathfrak{n}$ is connected with three edges $\mathfrak{e}_1$, $\mathfrak{e}_2$, $\mathfrak{e}$ whose corresponding variables $k_{\mathfrak{e}_1}$, $k_{\mathfrak{e}_2}$, $k_{\mathfrak{e}}$ satisfy the momentum conservation equation
\begin{equation}
\iota_{\mathfrak{e}_1}k_{\mathfrak{e}_1}+\iota_{\mathfrak{e}_2}k_{\mathfrak{e}_2}+\iota_{\mathfrak{e}}k_{\mathfrak{e}}=0
\end{equation}
and the energy conservation equation (if the node is decorated by $\bullet$)
\begin{equation}
    \iota_{\mathfrak{e}_1}\Lambda_{k_{\mathfrak{e}_1}}+\iota_{\mathfrak{e}_2}\Lambda_{k_{\mathfrak{e}_2}}+\iota_{\mathfrak{e}}\Lambda_{k_{\mathfrak{e}}} = O(T^{-1}_{\text{max}}).
\end{equation}



\begin{figure}[H]
    \centering
    \scalebox{0.5}{
    \begin{tikzpicture}[level distance=80pt, sibling distance=150pt]
        \node[scale=2.0] at (14.3,0) {$k_{\mathfrak{e}}$};
        \node[scale=2.0] at (14.5,-1.3) {$\mathfrak{n}$};
        \node[scale=2.0] at (13,-2.4) {$k_{\mathfrak{e}_1}$};
        \node[scale=2.0] at (17,-2.4) {$k_{\mathfrak{e}_2}$};
        \node[] at (15,1.5) (1) {} 
            child {node[fillcirc] (2) {} 
                child {node[fillstar] (3) {}
                }
                child {node[fillstar] (6) {}}
            };
        \draw[-{Stealth[length=5mm, width=3mm]}] (1) -- (2);
        \draw[-{Stealth[length=5mm, width=3mm]}] (2) -- (3);
        \draw[-{Stealth[length=5mm, width=3mm]}] (2) -- (6);
        \node[scale=2.0] at (15,-5.5) {$\iota_{\mathfrak{e}_1}=\iota_{\mathfrak{e}_2}=1, \iota_{\mathfrak{e}}=-1$};
        \node[scale=2.0] at (15,-6.5) {$k_{\mathfrak{e}_1}+k_{\mathfrak{e}_2}-k_{\mathfrak{e}}=0$};
        \node[scale=2.0] at (15,-7.5) {$\Lambda_{k_{\mathfrak{e}_1}}+\Lambda_{k_{\mathfrak{e}_2}}-\Lambda_{k_{\mathfrak{e}}}=O(T^{-1}_{\text{max}})$};
    \end{tikzpicture}
    }
        \caption{Equations of a node $\mathfrak{n}$}
        \label{fig.equations}
    \end{figure}
    

The goal of the next two sections is to count the number of solutions of a modified version of the above equation \eqref{eq.diophantineeq'.threewave}. 

\subsection{Couples and the Wick theorem}\label{sec.coupwick} In this section, we calculate $\mathbb{E}|\mathcal{J}_{T,k}|^2$ using Wick theorem. We also introduce another type of diagrams, the couple diagrams, to represent the result.

By the upper bound in the last section, the coefficients $H^T_{k_1\cdots k_{l+1}}$ concentrate near the surface $q_{\mathfrak{n}}=0$, $\forall \mathfrak{n}$. But to get an upper bound of $\mathcal{J}_{T,k}$, we need an upper bound of their variance $\mathbb{E}|\mathcal{J}_{T,k}|^2$. The coefficients of $\mathbb{E}|\mathcal{J}_{T,k}|^2$ also concentrate near a surface whose expression is similar to \eqref{eq.diophantineeq''.threewave}. 

Let's derive the expression of the coefficients of $\mathbb{E}|\mathcal{J}_{T,k}|^2$ and its concentration surface.

%$\mathbb{E}(\mathcal{J}_{T,k}\overline{\mathcal{J}_{T',k'}})$

%In this section, we derive the formula \eqref{eq.couples.threewave} for calculating expectations $\mathbb{E}(\mathcal{J}_{T,k}\overline{\mathcal{J}_{T',k'}})$ in terms of couples. 

By Lemma \ref{lem.treeterms}, we know that $\mathcal{J}_{T,k}$ is a polynomial of $\xi$ which are proportional to i.i.d Gaussians. Therefore, 
\begin{equation}\label{eq.termexp1.threewave}
\begin{split}
    \mathbb{E}|\mathcal{J}_{T,k}|^2=&\mathbb{E}(\mathcal{J}_{T,k}\overline{\mathcal{J}_{T,k}})=\left(\frac{\lambda}{L^{d}}\right)^{2l(T)}
    \sum_{k_1,\, k_2,\, \cdots,\, k_{l(T)+1}}\sum_{k'_1,\, k'_2,\, \cdots,\, k'_{l(T)+1}}
    \\[0.5em]
    & H^T_{k_1\cdots k_{l(T)+1}} \overline{H^{T}_{k'_1\cdots k'_{l(T)+1}}}  \mathbb{E}\Big(\xi_{k_1}\xi_{k_2}\cdots\xi_{k_{l(T)+1}}\xi_{k'_1}\xi_{k'_2}\cdots\xi_{k'_{l(T)+1}}\Big)
\end{split}
\end{equation}
% \begin{equation}\label{eq.termexp1.threewave}
% \begin{split}
%     \mathbb{E}(\mathcal{J}_{T,k}\overline{\mathcal{J}_{T',k'}})&=\left(\frac{-i\lambda^2}{L^{2d}}\right)^{l(T)+l(T')}
%     \sum_{k_1,\, k_2,\, \cdots,\, k_{l(T)+1}}\sum_{k'_1,\, k'_2,\, \cdots,\, k'_{2l(T')+1}}
%     \\[0.5em]
%     & H^T_{k_1\cdots k_{l(T)+1}} H^{T'}_{k'_1\cdots k'_{2l(T')+1}}   \mathbb{E}\Big([\xi_{k_1}\xi_{k_2}\cdots\xi_{k_{l(T)+1}}]_{R(T)}
%     [\xi_{k'_1}\xi_{k'_2}\cdots\xi_{k'_{2l(T')+1}}]_{R(T')}\Big)
% \end{split}
% \end{equation}

We just need to calculate 
\begin{equation}\label{eq.expectation''.threewave}
    \mathbb{E}\Big(\xi_{k_1}\xi_{k_2}\cdots\xi_{k_{l(T)+1}}
    \xi_{k'_1}\xi_{k'_2}\cdots\xi_{k'_{l(T)+1}}\Big).
\end{equation}
Notice that $\xi_k=\sqrt{n_{\textrm{in}}(k)} \, \eta_{k}(\omega)$ and $\eta_{k}$ are i.i.d Gaussians. We can apply the Wick theorem to calculate the above expectations. 

To introduce the Wick theorem, we need the following definition.

\begin{defn}
\begin{enumerate}
    \item \textbf{Pairing:} Suppose that we have a set $A=\{a_1,\cdots,a_{2m}\}$. A \underline{pairing} is a partition of $A=\{a_{i_1},a_{i_2}\}\cup\cdots\cup \{a_{i_{2m-1}},a_{i_{2m}}\}$ into $m$ disjoint subsets which have exactly two elements. Given a pairing $p$, elements $a_{i_{k}}$, $a_{i_{k'}}$ in the same subset of $p$ are called \underline{paired} with each other, which is denoted by $a_{i_{k}}\sim_{p} a_{i_{k'}}$.
    \item \textbf{$\mathcal{P}(A)$:} Denote by $\mathcal{P}(A)$ the set of all pairings of $A$.
\end{enumerate}

% Pairing: $\wick{\c i_1 \c i_{2} \c i_3 \c i_{4} \cdots \c i_{2m-1} \c i_{2m}}$ (representation of a pairing is unique if $i_1<i_3<\cdots<i_{2m-1}$).

% components of a pairing: $p_{k}=i_k$


\end{defn}

\begin{lem}[Wick theorem]\label{th.wick}
Let $\{\eta_k\}_{k\in\mathbb{Z}^d_L}$ be i.i.d complex Gaussian random variables with reflection symmetry (i.e. $\eta_{k}=\bar{\eta}_{-k}$). Let $\mathcal{P}$ be the set of all pairings of $\{k_1,k_2,\cdots,k_{2m}\}$, then
\begin{equation}
    %\mathbb{E}(\xi_{k_1}\cdots \xi_{k_{2m}})=\sum_{p\in \mathcal{P}} \prod_{i=1}^{m} \delta_{k_{p_{2i-1}}=k_{p_{2i}}}
    \mathbb{E}(\eta_{k_1}\cdots \eta_{k_{2m}})=\sum_{p\in \mathcal{P}}  \delta_{p}(k_1,\cdots,k_{2m}), 
\end{equation}
where 
\begin{equation}\label{eq.deltapairing.threewave}
\delta_{p}=\begin{cases}
1\qquad \textit{if $k_{i}=-k_{j}$ for all $k_{i}\sim_{p}k_{j}$,}
\\
0\qquad \textit{otherwise.}
\end{cases}
\end{equation}
\end{lem}
\begin{proof}
By Isserlis' theorem, for $X_1$, $X_2$, $\cdots$, $X_n$ zero-mean i.i.d Gaussian, we have 
\begin{equation}
    \mathbb{E} [X_1 X_2 \cdots X_n] = \sum_{p\in\mathcal{P}} \prod_{i\sim_{p}j} \mathbb{E} [X_i X_j]
\end{equation}
Here $\mathcal{P}$ is the set of pairings of $\{1,2,\cdots,n\}$.

Since $\mathbb{E} [\eta_{k_i}\eta_{k_j}]=\delta_{k_i=-k_j}$, take $X_1=\eta_{k_1}$, $\cdots$, $X_{2m}=\eta_{k_{2m}}$, then we can check that $\prod_{i\sim_{p}j} \mathbb{E} [X_i X_j]=\delta_{p}(k_1,\cdots,k_{2m})$. This finishes the proof of the Wick theorem.
\end{proof}

Applying Wick theorem to \eqref{eq.termexp1.threewave}, we get
\begin{equation}\label{eq.termexp'.threewave}
\begin{split}
    &\mathbb{E}|\mathcal{J}_{T,k}|^2=\left(\frac{\lambda}{L^{d}}\right)^{2l(T)}
    \sum_{p\in \mathcal{P}(\{k_1,\cdots, k_{l(T)+1}, k'_1,\cdots, k'_{l(T)+1}\})}
    \\[0.5em]
    & \underbrace{\sum_{k_1,\, k_2,\, \cdots,\, k_{l(T)+1}}\sum_{k'_1,\, k'_2,\, \cdots,\, k'_{l(T)+1}} H^T_{k_1\cdots k_{l(T)+1}} \overline{H^{T}_{k'_1\cdots k'_{l(T)+1}}} \delta_{p}(k_1,\cdots, k_{l(T)+1}, k'_1,\cdots, k'_{l(T)+1})\sqrt{n_{\textrm{in}}(k_1)}\cdots}_{Term(T, p)_k}.
\end{split}
\end{equation}
% \begin{equation}\label{eq.termexp1.threewave}
% \begin{split}
%     &\mathbb{E}(\mathcal{J}_{T,k}\overline{\mathcal{J}_{T',k'}})=\left(\frac{-i\lambda^2}{L^{2d}}\right)^{l(T)+l(T')}
%     \sum_{p\in \mathcal{P}(\{k_1,\cdots, k_{l(T)+1}, k'_1,\cdots, k'_{2l(T')+1}\})}
%     \\[0.5em]
%     & \underbrace{\sum_{k_1,\, k_2,\, \cdots,\, k_{l(T)+1}}\sum_{k'_1,\, k'_2,\, \cdots,\, k'_{2l(T')+1}} H^T_{k_1\cdots k_{l(T)+1}} H^{T'}_{k'_1\cdots k'_{2l(T')+1}} \delta_{p}(k_1,\cdots, k_{l(T)+1}, k'_1,\cdots, k'_{2l(T')+1})\sqrt{n_{\textrm{in}}(k_1)}\cdots}_{Term(T, T', p)}.
% \end{split}
% \end{equation}

We see that the correlation of two tree terms is a sum of smaller expressions $Term(T, p)$. By \eqref{eq.diophantineeq'.threewave}, the coefficients $H^T_{k_1\cdots k_{l(T)+1}} H^{T}_{k'_1\cdots k'_{l(T)+1}}$ of $Term(T, p)$ concentrate near the subset 

\begin{equation}\label{eq.diophantineequnpaired.threewave}
\{k_{\mathfrak{e}}, k_{\mathfrak{e}'}\in \mathbb{Z}^d_L,\ |k_{\mathfrak{e}}|, |k_{\mathfrak{e}'}|\lesssim 1,\ \forall \mathfrak{e},\mathfrak{e}'\in T:|\Omega_{\mathfrak{n}}|,|\Omega_{\mathfrak{n}'}|\lesssim T^{-1}_{\text{max}},\ S_{\mathfrak{n}}=S_{\mathfrak{n}'}=0\ \forall \mathfrak{n},\mathfrak{n}'\in T.\ k_{\mathfrak{l}}=-k_{\mathfrak{l}}'=k\}.
\end{equation}


The pairing $p$ in Wick theorem introduces new equations $k_{i}=-k'_{j}$ (defined in \eqref{eq.deltapairing.threewave}) and the coefficients $H^T_{k_1\cdots k_{l(T)+1}} H^{T}_{k'_1\cdots k'_{l(T)+1}} \delta_{p}$ concentrate near the subset 
\begin{equation}\label{eq.diophantineeqpaired.threewave}
\begin{split}
    \{k_{\mathfrak{e}}, k_{\mathfrak{e}'}\in \mathbb{Z}^d_L,\ &|k_{\mathfrak{e}}|, |k_{\mathfrak{e}'}|\lesssim 1,\ \forall \mathfrak{e},\mathfrak{e}'\in T: |\Omega_{\mathfrak{n}}|,|\Omega_{\mathfrak{n}'}|\lesssim T^{-1}_{\text{max}},\ S_{\mathfrak{n}}=S_{\mathfrak{n}'}=0\ \forall \mathfrak{n},\mathfrak{n}'\in T.
    \\
    k_{\mathfrak{l}}=-k_{\mathfrak{l}}'=k.\  
    &\textit{$k_{i}=-k'_{j}$ (and $k_{i}=-k_{j}$, $k'_{i}=-k'_{j}$) for all $k_{i}\sim_{p}k'_{j}$ (and $k_{i}\sim_{p}k_{j}$, $k'_{i}\sim_{p}k'_{j}$)}\}.
\end{split}
\end{equation}

As in the case of \eqref{eq.diophantineeq'.threewave}, there is a graphical representation of \eqref{eq.diophantineeqpaired.threewave}. To explain this, we need the concept of couples.

%It turns out we can find a very compact formula for them. To do this, we need to introduce the concept of couples.


\begin{defn}[Construction of couples]\label{def.conple} Given two trees $T$ and $T'$, we flip the orientation of all edges in $T'$ (as in the two left trees in Figure \ref{fig.treepairing}). We also label their leaves by $1, 2, \cdots, l(T)+1$ and $1, 2, \cdots, l(T')+1$ so that the corresponding variables of these leaves are $k_1, k_2, \cdots, k_{l(T)+1}$ and $k_1, k_2, \cdots, k_{l(T')+1}$. Assume that we have a pairing $p$ of the set $\{k_1, k_2, \cdots, k_{l(T)+1}, k_1, k_2, \cdots, k_{l(T')+1}\}$, then this pairing induces a pairing between leaves (if $k_i\sim_p k_j$ then define $\textit{the $i$-th leaf}\sim_p \textit{the $j$-th leaf}$). Given this pairing of leaves, we define the following procedure which glues two trees $T$ and $T'$ into a couple  $\mathcal{C}(T,T,p)$. Some examples of pairing can be found in Figure \ref{fig.treepairing}. 

\begin{figure}[H]
    \centering
    \scalebox{0.3}{
    \begin{tikzpicture}[level distance=80pt, sibling distance=100pt]
        \node[] at (15,1.5) (1) {} 
            child {node[fillcirc] (2) {} 
                child {node[fillcirc] (3) {}
                    child {node[fillstar] (4) {}}
                    child {node[fillstar] (5) {}}
                }
                child {node[fillstar] (6) {}}
            };
        \draw[-{Stealth[length=5mm, width=3mm]}] (1) -- (2);
        \draw[-{Stealth[length=5mm, width=3mm]}] (2) -- (3);
        \draw[-{Stealth[length=5mm, width=3mm]}] (2) -- (6);
        \draw[-{Stealth[length=5mm, width=3mm]}] (3) -- (4);
        \draw[-{Stealth[length=5mm, width=3mm]}] (3) -- (5);
        \node[] at (22,1.5) (11) {} 
            child {node[fillcirc] (12) {} 
                child {node[fillcirc] (13) {}
                    child {node[fillstar] (14) {}}
                    child {node[fillstar] (15) {}}
                }
                child {node[fillstar] (16) {}}
            };
        \draw[{Stealth[length=5mm, width=3mm]}-] (11) -- (12);
        \draw[{Stealth[length=5mm, width=3mm]}-] (12) -- (13);
        \draw[{Stealth[length=5mm, width=3mm]}-] (12) -- (16);
        \draw[{Stealth[length=5mm, width=3mm]}-] (13) -- (14);
        \draw[{Stealth[length=5mm, width=3mm]}-] (13) -- (15);
        \draw[bend right =20, dashed] (4) edge (14);
        \draw[bend right =20, dashed] (5) edge (15);
        \draw[bend right =20, dashed] (6) edge (16);
        
        
        \node[] at (33,1.5) (new1) {} 
            child {node[fillcirc] (new2) {} 
                child {node[fillcirc] (new3) {}
                    child {node[fillstar] (new4) {}}
                    child {node[fillstar] (new5) {}}
                }
                child {node[fillstar] (new6) {}}
            };
        \draw[-{Stealth[length=5mm, width=3mm]}] (new1) -- (new2);
        \draw[-{Stealth[length=5mm, width=3mm]}] (new2) -- (new3);
        \draw[-{Stealth[length=5mm, width=3mm]}] (new2) -- (new6);
        \draw[-{Stealth[length=5mm, width=3mm]}] (new3) -- (new4);
        \draw[-{Stealth[length=5mm, width=3mm]}] (new3) -- (new5);
        \node[] at (40,1.5) (new11) {} 
            child {node[fillcirc] (new12) {} 
                child {node[fillcirc] (new13) {}
                    child {node[fillstar] (new14) {}}
                    child {node[fillstar] (new15) {}}
                }
                child {node[fillstar] (new16) {}}
            };
        \draw[{Stealth[length=5mm, width=3mm]}-] (new11) -- (new12);
        \draw[{Stealth[length=5mm, width=3mm]}-] (new12) -- (new13);
        \draw[{Stealth[length=5mm, width=3mm]}-] (new12) -- (new16);
        \draw[{Stealth[length=5mm, width=3mm]}-] (new13) -- (new14);
        \draw[{Stealth[length=5mm, width=3mm]}-] (new13) -- (new15);
        \draw[bend right =90, dashed] (new4) edge (new6);
        \draw[bend right =20, dashed] (new5) edge (new15);
        \draw[bend right =90, dashed] (new14) edge (new16);
    \end{tikzpicture}
    }
        \caption{Example of pairings between trees.}
        \label{fig.treepairing}
    \end{figure}

\begin{enumerate}
    \item \textbf{Merging edges connected to leaves:} Given two edges with opposite orientations connected to two paired leaves, these two edges can be \underline{merged} into one edge as in Figure \ref{fig.pairingleaves}.
    \begin{figure}[H]
    \centering
    \scalebox{0.5}{
    \begin{tikzpicture}[level distance=80pt, sibling distance=100pt]
        \node[draw, circle, minimum size=1cm, scale=2] at (0,0) (1) {$T_1$} [grow =300] 
            child {node[fillstar] (2) {}};
        \node[draw, circle, minimum size=1cm, scale=2] at (5,0) (3) {$T_2$} [grow =240] 
            child {node[fillstar] (4) {}};
        \draw[-{Stealth[length=5mm, width=3mm]}] (1) -- (2);
        \draw[{Stealth[length=5mm, width=3mm]}-] (3) -- (4);
        \draw[bend right =40, dashed] (2) edge (4);
        
        \node[draw, single arrow,
              minimum height=33mm, minimum width=8mm,
              single arrow head extend=2mm,
              anchor=west, rotate=0] at (7,-1.5) {}; 
              
        \node[draw, circle, minimum size=1cm, scale=2] at (13,-1.5) (5) {$T_1$}; 
        \node[draw, circle, minimum size=1cm, scale=2] at (18,-1.5) (6) {$T_2$};
        \draw[-{Stealth[length=5mm, width=3mm]}] (5) -- (6);
            
    \end{tikzpicture}
    }
        \caption{Pairing and merging of two edges}
        \label{fig.pairingleaves}
    \end{figure}
    We know that two edges connected to leaves correspond to two indices $k_i$, $k_j$. Merging two such edges is a graphical interpretation that $k_i=-k_j$. 
    \item \textbf{Pairing of trees and couples:} Given a pairing $p$ of the set of leaves in $T$, $T'$ we merge all edges paired by $p$ as in Figure \ref{fig.couple} and the resulting combinatorial structure is called a \underline{couple}. 
     \begin{figure}[H]
    \centering
    \scalebox{0.3}{
    \begin{tikzpicture}[level distance=80pt, sibling distance=100pt]
        \node[] at (0,0) (1) {} 
            child {node[fillcirc] (2) {} 
                child {node[fillstar] (3) {}}
                child {node[fillstar] (4) {}}
            };
        \draw[-{Stealth[length=5mm, width=3mm]}] (1) -- (2);
        \draw[-{Stealth[length=5mm, width=3mm]}] (2) -- (3);
        \draw[-{Stealth[length=5mm, width=3mm]}] (2) -- (4);
        
        \node[] at (0,-14.5) (11) {} [grow =90] 
            child {node[fillcirc] (12) {} 
                child {node[fillstar] (13) {}}
                child {node[fillstar] (14) {}}
            };    
        \draw[{Stealth[length=5mm, width=3mm]}-] (11) -- (12);
        \draw[{Stealth[length=5mm, width=3mm]}-] (12) -- (13);
        \draw[{Stealth[length=5mm, width=3mm]}-] (12) -- (14);    
            
        \draw[dashed] (3) edge (14);
        \draw[dashed] (4) edge (13);
        
        \node[draw, single arrow,
              minimum height=66mm, minimum width=16mm,
              single arrow head extend=4mm,
              anchor=west, rotate=0] at (7,-7.5) {};
        
        \node[] at (20,0) (21) {}; 
        \node[fillcirc] at (20,-4) (22) {};
        \node[fillcirc] at (20,-11) (23) {};
        \node[] (24) at (20,-15) {};
        \draw[-{Stealth[length=5mm, width=3mm]}] (21) edge (22);
        \draw[-{Stealth[length=5mm, width=3mm]}, bend right =40] (22) edge (23);
        \draw[-{Stealth[length=5mm, width=3mm]}, bend left =40] (22) edge (23);
        \draw[-{Stealth[length=5mm, width=3mm]}] (23) edge (24);
    \end{tikzpicture}
    }
        \caption{The construction of a couple}
        \label{fig.couple}
    \end{figure}
    
    We know that each edge connected to a leaf corresponds to a variable $k_i$. A pairing $p$ of $\{k_1,k_2,\cdots,k_{2m}\}$ in \eqref{eq.diophantineeqpaired.threewave} induces a pairing of edges connected to leaves. Merging paired edges corresponds to $k_{i}=-k'_{j}$ for all $k_{i}\sim_{p}k'_{j}$ in \eqref{eq.diophantineeqpaired.threewave}. 
\end{enumerate}
\end{defn}

The following proposition introduces the graphical representation of \eqref{eq.diophantineeqpaired.threewave}.

\begin{prop}\label{prop.couple}
\eqref{eq.diophantineeqpaired.threewave} can be read from a couple diagram $\mathcal{C}(T,T,p)$. Each edge corresponds to a variable $k_{\mathfrak{e}}$. The leg $\mathfrak{l}$ corresponds to equation $k_{\mathfrak{l}}=k$. Each node corresponds to a momentum conservation equation
\begin{equation}
    \iota_{\mathfrak{e}_1}k_{\mathfrak{e}_1}+\iota_{\mathfrak{e}_2}k_{\mathfrak{e}_2}+\iota_{\mathfrak{e}}k_{\mathfrak{e}}=0,
\end{equation} 
and an energy conservation equation \begin{equation}
    \iota_{\mathfrak{e}_1}\Lambda_{k_{\mathfrak{e}_1}}+\iota_{\mathfrak{e}_2}\Lambda_{k_{\mathfrak{e}_2}}+\iota_{\mathfrak{e}}\Lambda_{k_{\mathfrak{e}}} = O(T^{-1}_{\text{max}}).
\end{equation}  
\end{prop}
\begin{rem}
In a couple diagram, we only have nodes decorated by $\bullet$. Nodes decorated by $\star$ have been removed in (1), (2) of Definition \ref{def.conple}.
\end{rem}
\begin{rem}
Through the process of (1), (2) in Definition \ref{def.conple}, a couple diagram can automatically encode the equation $k_{i}=-k'_{j}$ for all $k_{i}\sim_{p}k'_{j}$. Therefore, they do not appear in Proposition \ref{prop.couple}.
\end{rem}

\begin{proof}
This directly follows from the definition of couples. 
\end{proof}


The calculations of this section are summarized in the following proposition.  

\begin{prop}\label{prop.termcouple} (1) Define $Term(T,p)$ in the same way as in \eqref{eq.termexp'.threewave},
\begin{equation}\label{eq.termTp.threewave}
\begin{split}
    Term(T, p)_k=&\sum_{k_1,\, k_2,\, \cdots,\, k_{l(T)+1}}\sum_{k'_1,\, k'_2,\, \cdots,\, k'_{l(T)+1}}
    \\
    &H^T_{k_1\cdots k_{l(T)+1}} \overline{H^{T}_{k'_1\cdots k'_{l(T)+1}}} \delta_{p}(k_1,\cdots, k_{l(T)+1}, k'_1,\cdots, k'_{l(T)+1})\sqrt{n_{\textrm{in}}(k_1)}\cdots\sqrt{n_{\textrm{in}}(k'_1)}\cdots
\end{split}
\end{equation}
then $\mathbb{E}|\mathcal{J}_{T,k}|^2$ is a sum of $Term(T,p)_k$ for all $p\in \mathcal{P}$, (in \eqref{eq.termexp'.threewave} the sum is over set of all possible pairing $\mathcal{P}$)
\begin{equation}\label{eq.termexp.threewave}
\begin{split}
    \mathbb{E}|\mathcal{J}_{T,k}|^2=\left(\frac{\lambda}{L^{d}}\right)^{2l(T)}
    \sum_{p\in \mathcal{P}(\{k_1,\cdots, k_{l(T)+1}, k'_1,\cdots, k'_{l(T)+1}\})} Term(T, p).
\end{split}
\end{equation}

(2) $Term(T,p)$ concentrates near the subset \eqref{eq.diophantineeqpaired.threewave} which has a simple graphical representation given by Proposition \ref{prop.couple}. 
\end{prop}

\begin{proof} The proof of (1), (2) is easy and thus skipped. 
\end{proof}
% \begin{lem}
% Formula for $\mathbb{E}(\mathcal{J}_{T,k}\overline{\mathcal{J}_{T',k'}})$ in terms of couple $C$

% \begin{equation}\label{eq.couples.threewave}
%     1
% \end{equation}
% \end{lem}

\subsection{Counting lattice points}\label{sec.numbertheory} In this section, we use the connection between couples and concentration subsets \eqref{eq.diophantineeqpaired.threewave} to count the number of solutions of a generalized version of \eqref{eq.diophantineeqpaired.threewave},



\begin{equation}\label{eq.diophantineeqpairedsigma.threewave}
\begin{split}
    &\{k_{\mathfrak{e}}, k_{\mathfrak{e}'}\in \mathbb{Z}^d_L,\ |k_{\mathfrak{e}}|, |k_{\mathfrak{e}}'|\lesssim 1,\ \forall \mathfrak{e},\mathfrak{e}'\in T:\  |k_{\mathfrak{e}x}|\sim \kappa_{\mathfrak{e}}, |k'_{\mathfrak{e}x}|\sim \kappa_{\mathfrak{e}'},\ \forall \mathfrak{e},\mathfrak{e}'\in T, 
    \\
    &|\Omega_{\mathfrak{n}}-\sigma_{\mathfrak{n}}|,|\Omega'_{\mathfrak{n}}-\sigma'_{\mathfrak{n}}|\lesssim T^{-1}_{\text{max}},\ S_{\mathfrak{n}}=S_{\mathfrak{n}'}=0,\ \forall \mathfrak{n}.\ k_{\mathfrak{l}}=-k_{\mathfrak{l}}'=k.
    \\
    &\textit{$k_{i}=-k'_{j}$ (and $k_{i}=-k_{j}$, $k'_{i}=-k'_{j}$) for all $k_{i}\sim_{p}k'_{j}$ (and $k_{i}\sim_{p}k_{j}$, $k'_{i}\sim_{p}k'_{j}$)}\}.
\end{split}
\end{equation}

In \eqref{eq.diophantineeqpairedsigma.threewave}, $\kappa_{\mathfrak{e}}\in \{0\}\cup  \mathcal{D}(\alpha,1)$, where $\mathcal{D}(\alpha,1)\coloneqq\{2^{-K_{\mathfrak{e}}}:K_{\mathfrak{e}}\in  \mathbb{Z}\cap [0,ln\ \alpha^{-1}]\}$. The relation $|k_{\mathfrak{e}x}|\sim \kappa_{\mathfrak{e}}$ is defined by 
\begin{equation}\label{eq.kappa.threewave}
    |k_{\mathfrak{e}x}|\sim \kappa_{\mathfrak{e}}\text{ if and only if }\left\{\begin{aligned}
        & \frac{1}{2}\kappa_{\mathfrak{e}}\le  |k_{\mathfrak{e}x}|\le 2\kappa_{\mathfrak{e}} \qquad && \text{ if  $\kappa_{\mathfrak{e}}\ne 0$}
        \\[1em]
        &  |k_{\mathfrak{e}x}|\lesssim \alpha^2,\ k_{\mathfrak{e}x}\ne 0   \qquad && \text{ if  $\kappa_{\mathfrak{e}}= 0$}
    \end{aligned}
    \right.
\end{equation}

\eqref{eq.diophantineeqpairedsigma.threewave} is obtained by replacing $\Omega_{\mathfrak{n}}$, $\Omega'_{\mathfrak{n}}$ by $\Omega_{\mathfrak{n}}-\sigma_{\mathfrak{n}}$, $\Omega'_{\mathfrak{n}}-\sigma'_{\mathfrak{n}}$ in \eqref{eq.diophantineeqpaired.threewave} and adding conditions $|k_{\mathfrak{e}}|\sim \kappa_{\mathfrak{e}}$, where  $\sigma_{\mathfrak{n}}$, $\sigma'_{\mathfrak{n}}$ and $\kappa_{\mathfrak{e}}$ are some given constants. The counterpart of Proposition \ref{prop.couple} in this case is

\begin{prop}\label{prop.couple'}
\eqref{eq.diophantineeqpairedsigma.threewave} can be read from a couple diagram $\mathcal{C}=\mathcal{C}(T,T,p)$. Each edge corresponds to a variable $k_{\mathfrak{e}}$. The leg $\mathfrak{l}$ corresponds to equation $k_{\mathfrak{l}}=k$. Each node corresponds to a momentum conservation equation
\begin{equation}\label{eq.momentumconservationunit.threewave}
    \iota_{\mathfrak{e}_1}k_{\mathfrak{e}_1}+\iota_{\mathfrak{e}_2}k_{\mathfrak{e}_2}+\iota_{\mathfrak{e}}k_{\mathfrak{e}}=0,
\end{equation} 
and a energy conservation equation 
\begin{equation}\label{eq.energyconservationunit.threewave}
    \iota_{\mathfrak{e}_1}\Lambda_{k_{\mathfrak{e}_1}}+\iota_{\mathfrak{e}_2}\Lambda_{k_{\mathfrak{e}_2}}+\iota_{\mathfrak{e}}\Lambda_{k_{\mathfrak{e}}} = \sigma_{\mathfrak{n}} + O(T^{-1}_{\text{max}}).
\end{equation}  
Denote the momentum and energy conservation equations by $MC_{\mathfrak{n}}$ and $EC_{\mathfrak{n}}$ respectively, then \eqref{eq.diophantineeqpairedsigma.threewave} can be rewritten as 
\begin{equation}\label{eq.diophantineeqpairedsigma'.threewave}
    \text{\eqref{eq.diophantineeqpairedsigma.threewave}}=\{k_{\mathfrak{e}}\in \mathbb{Z}^d_L,\ |k_{\mathfrak{e}}|\lesssim 1\ \forall \mathfrak{e}\in \mathcal{C}:\  |k_{\mathfrak{e}x}| \sim \kappa_{\mathfrak{e}},\ \forall \mathfrak{e}\in \mathcal{C}_{\text{norm}}.\ MC_{\mathfrak{n}},\  EC_{\mathfrak{n}},\ \forall \mathfrak{n}\in \mathcal{C}.\ k_{\mathfrak{l}} = - k_{\mathfrak{l}}'= k.\}
\end{equation}
\end{prop}
\begin{proof}
This directly follows from Proposition \ref{prop.couple}. 
\end{proof}

To explain the counting argument in this paper, we need the following definitions related to couples.

\begin{defn}\label{def.morecouple}
\begin{enumerate}
    \item \textbf{Connected couples:} A couple $\mathcal{C}$ is a \underline{connected couple} if it is connected as a graph.  
    \item \textbf{Equations of a couple $Eq(\mathcal{C})$:} Given a couple $\mathcal{C}$ and constants $k$, $\sigma_{\mathfrak{n}}$, let $Eq(\mathcal{C},\{\sigma_{\mathfrak{n}}\}_{\mathfrak{n}}, k)$ (or simply $Eq(\mathcal{C})$) be the system of equation \eqref{eq.diophantineeqpairedsigma'.threewave} constructed in Proposition \ref{prop.couple'}. For any system of equations $Eq$, let $\#(Eq)$ be its number of solutions.
    \item \textbf{Normal edges and leaf edges:} Remember that any couple $\mathcal{C}$ is constructed from a pairing of two trees $T$, $T'$ and therefore all edges in $\mathcal{C}$ comes from $T$, $T'$. We define edges coming from $T_{\text{in}}$, $T'_{\text{in}}$ to be \underline{normal edges} and edges from those connected to leaves in $T$, $T'$ to be \underline{leaf edges}. The set of all normal edges is denoted by $\mathcal{C}_{\text{norm}}$. A leg in $\mathcal{C}$ which is a normal edge is called a \underline{normal leg}.
\end{enumerate}
\end{defn}

The main goal of this section is to prove an upper bound of $\#Eq(\mathcal{C})$. The main idea of proving this is to decompose a large couple $\mathcal{C}$ into smaller pieces and then prove this for the smaller piece using the induction hypothesis. To explain the idea, let us first focus on an example. Let $\mathcal{C}$ be the left couple in the following picture. (The corresponding variables of each edge are labeled near these edges.)
\begin{figure}[H]
    \centering
    \scalebox{0.4}{
    \begin{tikzpicture}[level distance=80pt, sibling distance=100pt]
        \node[] at (0,0) (1) {}; 
        \node[fillcirc] at (3,0) (2) {}; 
        \node[fillcirc] at (6,-2) (3) {}; 
        \node[fillcirc] at (9,-2) (4) {}; 
        \node[fillcirc] at (12,0) (5) {}; 
        \node[] at (15,0) (6) {}; 
        \draw[-{Stealth[length=5mm, width=3mm]}] (1) edge (2);
        \draw[-{Stealth[length=5mm, width=3mm]}] (2) edge (3);
        \draw[-{Stealth[length=5mm, width=3mm]}, bend left =40] (3) edge (4);
        \draw[-{Stealth[length=5mm, width=3mm]}, bend right =40] (3) edge (4);
        \draw[-{Stealth[length=5mm, width=3mm]}] (4) edge (5);
        \draw[-{Stealth[length=5mm, width=3mm]}] (5) edge (6);
        \draw[-{Stealth[length=5mm, width=3mm]}, bend left =40] (2) edge (5);
         
         \node[scale=2.0] at (3,-0.7) {$\mathfrak{n}_{1}$};
         \node[scale=2.0] at (6,-2.7) {$\mathfrak{n}_{2}$};
         \node[scale=2.0] at (9,-2.7) {$\mathfrak{n}_{3}$};
         \node[scale=2.0] at (12,-0.7) {$\mathfrak{n}_{4}$};
         \node[scale=2.0] at (1.5,-0.5) {$k$};
         \node[scale=2.0] at (4.3,-1.4) {$a$};
         \node[scale=2.0] at (7.5,-3.1) {$b$};
         \node[scale=2.0] at (7.5,-1) {$c$};
         \node[scale=2.0] at (10.7,-1.4) {$d$};
         \node[scale=2.0] at (7.5,2.2) {$e$};
         \node[scale=2.0] at (13.5,-0.5) {$-k$};
         
         
        \node[draw, single arrow,
              minimum height=33mm, minimum width=8mm,
              single arrow head extend=2mm,
              anchor=west, rotate=0] at (16,0) {};  
         
         
        \node[] at (20,0) (11) {}; 
        \node[fillcirc] at (23,0) (12) {}; 
        \node[fillstar] at (26,-2) (13) {};
        \node[fillstar] at (26,2) (14) {};
        \draw[-{Stealth[length=5mm, width=3mm]}] (11) edge (12);
        \draw[-{Stealth[length=5mm, width=3mm]}] (12) edge (13);
        \draw[-{Stealth[length=5mm, width=3mm]}] (12) edge (14);
        
        \node[scale=2.0] at (23,-0.7) {$\mathfrak{n}_{1}$};
        \node[scale=2.0] at (21.5,-0.5) {$k$};
        \node[scale=2.0] at (24.3,-1.4) {$a$};
        \node[scale=2.0] at (24.3,1.4) {$e$};
        \node[scale=2.0] at (23,-1.8) {$A$};
        
        
        \node[] at (28,-2) (32) {}; 
        \node[fillcirc] at (31,-2) (33) {}; 
        \node[fillcirc] at (34,-2) (34) {}; 
        \node[fillcirc] at (37,0) (35) {}; 
        \node[] at (40,0) (36) {}; 
        \node[] at (34,2) (37) {};
        \draw[-{Stealth[length=5mm, width=3mm]}] (32) edge (33);
        \draw[-{Stealth[length=5mm, width=3mm]}, bend left =40] (33) edge (34);
        \draw[-{Stealth[length=5mm, width=3mm]}, bend right =40] (33) edge (34);
        \draw[-{Stealth[length=5mm, width=3mm]}] (34) edge (35);
        \draw[-{Stealth[length=5mm, width=3mm]}] (35) edge (36);
        \draw[-{Stealth[length=5mm, width=3mm]}] (37) edge (35);
        
        \node[scale=2.0] at (31,-2.7) {$\mathfrak{n}_{2}$};
        \node[scale=2.0] at (34,-2.7) {$\mathfrak{n}_{3}$};
        \node[scale=2.0] at (37,-0.7) {$\mathfrak{n}_{4}$};
        \node[scale=2.0] at (29.5,-2.5) {$a$};
        \node[scale=2.0] at (32.5,-3.1) {$b$};
        \node[scale=2.0] at (32.5,-1) {$c$};
        \node[scale=2.0] at (35.7,-1.4) {$d$};
        \node[scale=2.0] at (35.7,1.4) {$e$};
        \node[scale=2.0] at (38.5,-0.5) {$-k$};
        \node[scale=2.0] at (37,-2.2) {$B_{a,e}$};
    \end{tikzpicture}
    }
        \caption{An example of decomposing a couple}
        \label{fig.exampleofcuttingidea}
    \end{figure}

By \eqref{eq.diophantineeqpairedsigma'.threewave}, we know that the couple $\mathcal{C}$ corresponds to the following equations.
\begin{equation}\label{eq.cuttingexmaple.threewave}
    \begin{split}
        \{a, b, c, d, e:\ &(|a|\text{ to }|e|)\lesssim 1,\ (|a_x|\text{ to }|e_x|)\sim (\kappa_{a}\text{ to }\kappa_{e})
        \\
        &a+e=k,\ \Lambda(a) + \Lambda(e) - \Lambda(k) =\sigma_{1} + O(T^{-1}_{\text{max}})
        \\
        &a+c=b,\ \Lambda(a) + \Lambda(c) - \Lambda(b) =\sigma_{2} + O(T^{-1}_{\text{max}})
        \\
        &b+c=d,\ \Lambda(b) + \Lambda(c) - \Lambda(d) =\sigma_{3} + O(T^{-1}_{\text{max}})
        \\
        &d+e+k=0,\ \Lambda(d) + \Lambda(e) + \Lambda(k) =\sigma_{4} + O(T^{-1}_{\text{max}})\}
    \end{split}
\end{equation}

We know that \eqref{eq.cuttingexmaple.threewave} can be rewritten into the form $\bigcup_{a,e\in A} B_{a,e}$, where
\begin{equation}\label{eq.couplequationA.threewave}
    A=\{a, e:\ |a|,|e|\lesssim 1,\ |a_x|\sim \kappa_{a},|e_x|\sim \kappa_{e}, a+e=k,\ \Lambda(a) + \Lambda(e) - \Lambda(k) =\sigma_{1} + O(T^{-1}_{\text{max}})\}
\end{equation}
\begin{equation}\label{eq.couplequationB.threewave}
    \begin{split}
        B_{a,e}=\{b, c, d:\ &|b|,|c|,|d|\lesssim 1,\ |b_x|\sim \kappa_{b},|c_x|\sim \kappa_{c},|d_x|\sim \kappa_{d}
        \\
        &a+c=b,\ \Lambda(a) + \Lambda(c) - \Lambda(b) =\sigma_{2} + O(T^{-1}_{\text{max}})
        \\
        &b+c=d,\ \Lambda(b) + \Lambda(c) - \Lambda(d) =\sigma_{3} + O(T^{-1}_{\text{max}})
        \\
        &d+e+k=0,\ \Lambda(d) + \Lambda(e) + \Lambda(k) =\sigma_{4} + O(T^{-1}_{\text{max}})\}
    \end{split}    
\end{equation}

Since an upper bound of $\# Eq(\mathcal{C})$ can be derived from upper bounds of $\# A$, $\# B_{a,e}$, we just need to consider $A$, $B_{a,e}$ which are systems of equations of smaller size. We can reduce the size of systems of equations in this way and prove upper bounds by induction.

One problem of applying an induction argument is that $A$, $B_{a,e}$ cannot be represented by couple defined by Definition \ref{def.conple} that can contain at most two legs (an edge just connected to one node). In Definition \ref{def.conple}, a leg is used to represent a variable that is fixed, as in the condition $k_{\mathfrak{l}} = - k_{\mathfrak{l}}'= k$ in \eqref{eq.diophantineeqpairedsigma'.threewave}. The definition of $\# B_{a,e}$ contains three fixed variables $a$, $e$, $k$ which cannot be represented by just two legs. Therefore, we have to define a new type of couple that allows multiple legs.

Except for the lack of legs, we also have the problem of representing free variables. We know that the couple representation of $A$ should contain one node and three edges if we insist on the rule that a node corresponds to an equation and the variables in the equation correspond to edges connected to this node. All these edges are legs, but two of three edges correspond to variable $a$, $b$ which are not fixed. Therefore, we have to define a type of legs that can correspond to unfixed variables.

To solve the above problems, we introduce the following definition.

\begin{defn}\label{def.couplemultileg}
\begin{enumerate}
    \item \textbf{Couples with multiple legs:} A graph in which all nodes have degree $1$ or $3$ is called a \underline{couples with multiple legs}. The graph $A$ and $B_{a,e}$ in Figure \ref{fig.exampleofcuttingidea} are examples of this definition.
    \item \textbf{Legs:} In a couple with multiple legs, an edge connected to a degree one node is called a \underline{leg}. Remember that we have encountered this concept in the second paragraph of section \ref{sec.refexp} and in what follows we call the leg defined there the \underline{root leg} of a tree.   
    \item \textbf{Free legs and fixed legs:} In a couple with multiple legs, we use two types of node decoration for degree $1$ nodes as in Figure \ref{fig.decorationdegreeone}. One is $\star$ and the other one is \underline{invisible}. 
    \begin{figure}[H]
    \centering
    \scalebox{0.5}{
    \begin{tikzpicture}[level distance=80pt, sibling distance=100pt]
        \node[draw, circle, minimum size=1cm, scale=2] at (0,0) (1) {$\mathcal{C}_1$} 
            child {node[fillstar] (2) {}};
        \node[draw, circle, minimum size=1cm, scale=2] at (5,0) (3) {$\mathcal{C}_2$}  
            child {node[] (4) {}};
        \draw[{Stealth[length=5mm, width=3mm]}-] (1) -- (2);
        \draw[{Stealth[length=5mm, width=3mm]}-] (3) -- (4);
        %\draw[bend right =40, dashed] (2) edge (4);
            
    \end{tikzpicture}
    }
        \caption{Node decoration of degree one node}
        \label{fig.decorationdegreeone}
    \end{figure}
    An edge connected to a $\star$ or invisible nodes is called a \underline{free leg} or \underline{fixed leg} respectively. 
    
    
    \item \textbf{Equations of a couple $Eq(\mathcal{C},\{c_{\mathfrak{l}}\}_{\mathfrak{l}})$:} We define the corresponding equations for couples with multiple legs.
    \begin{equation}\label{eq.Eq(C,c).threewave}
    \begin{split}
        &Eq(\mathcal{C},\{c_{\mathfrak{l}}\}_{\mathfrak{l}})
        \\
        =&\{k_{\mathfrak{e}}\in \mathbb{Z}^d_L,\ |k_{\mathfrak{e}}|\lesssim 1\ \forall \mathfrak{e}\in \mathcal{C}:\  |k_{\mathfrak{e}x}| \sim \kappa_{\mathfrak{e}},\ \forall \mathfrak{e}\in \mathcal{C}_{\text{norm}}.\ MC_{\mathfrak{n}},\  EC_{\mathfrak{n}},\ \forall \mathfrak{n}\in \mathcal{C}.\ k_{\mathfrak{l}}=c_{\mathfrak{l}},\ \forall \mathfrak{l}.\}   
    \end{split}
    \end{equation}
    In this representation, the corresponding variable of a fixed leg $\mathfrak{l}$ is fixed to be the constant $c_{\mathfrak{l}}$ and the corresponding variable of a free leg $\mathfrak{l}$ is not fixed.
\end{enumerate}
\end{defn}

With the above definition, it's easy to show that the couple $A$ and $B_{a,e}$ in Figure \ref{fig.exampleofcuttingidea} correspond to the system of equations \eqref{eq.couplequationA.threewave} and \eqref{eq.couplequationB.threewave} respectively.


Using the above argument, we can prove the following proposition which gives an upper bound of a number of solutions of \eqref{eq.diophantineeqpairedsigma.threewave} (or \eqref{eq.diophantineeqpairedsigma'.threewave}).

\begin{prop}\label{prop.counting}
Let $\mathcal{C}=\mathcal{C}(T,T',p)$ be a connected couple with exactly one free and one fixed leg, $n$ be the total number of nodes in $\mathcal{C}$ and $Q=L^{d}T^{-1}_{\text{max}}$. We fix $k\in \mathbb{R}$ for the legs $\mathfrak{l}$, $\mathfrak{l}'$ and $\sigma_{\mathfrak{n}}\in\mathbb{R}$ for each $\mathfrak{n}\in \mathcal{C}$. Assume that $\alpha$ satisfies \eqref{eq.conditionalpha.threewave}. Then the number of solutions $M$ of \eqref{eq.diophantineeqpairedsigma.threewave} (or \eqref{eq.diophantineeqpairedsigma'.threewave}) is bounded by 
\begin{equation}\label{eq.countingbd0.threewave}
M\leq L^{O(n\theta)} Q^{\frac{n}{2}}\ \prod_{\mathfrak{e}\in \mathcal{C}_{\text{norm}}} \kappa^{-1}_{\mathfrak{e}}.
\end{equation}
\end{prop}
\begin{proof} 
The proof is lengthy and therefore divided into several steps. The main idea of the proof is to use the operation of edge cutting to decompose the couple $\mathcal{C}$ into smaller ones $\mathcal{C}_1$, $\mathcal{C}_2$, then apply Lemma \ref{lem.Eq(C)cutting} which relates $\#Eq(\mathcal{C})$ and $\#Eq(\mathcal{C}_i)$. The desired upper bounds of $\#Eq(\mathcal{C})$ can be obtained from that of $\#Eq(\mathcal{C}_i)$ inductively.


\textbf{Step 1.} In this step, we explain the cutting edge argument and prove the Lemma \ref{lem.Eq(C)cutting} which relates $\#Eq(\mathcal{C}_1)$, $\#Eq(\mathcal{C}_2)$ and $\#Eq(\mathcal{C})$.

Here is the formal definition of cutting 

\begin{defn}
\begin{enumerate}
    \item \textbf{Cutting an edge:} Given an edge $\mathfrak{e}$, we can cut it into two edges (a fixed and a free leg) as in Figure \ref{fig.cutedge}.
    
   \begin{figure}[H]
    \centering
    \scalebox{0.5}{
    \begin{tikzpicture}[level distance=80pt, sibling distance=100pt]
        \node[draw, circle, minimum size=1cm, scale=2] at (0,0) (1) {$\mathcal{C}_1$}; 
        \node[draw, circle, minimum size=1cm, scale=2] at (5,0) (2) {$\mathcal{C}_2$};
        \draw[-{Stealth[length=5mm, width=3mm]}] (1) -- (2);
        \node[scale =3] at (2.5,0) {$\times$};
        
        
        \node[draw, single arrow,
              minimum height=33mm, minimum width=8mm,
              single arrow head extend=2mm,
              anchor=west, rotate=0] at (7.5,0) {}; 
              
              
        \node[draw, circle, minimum size=1cm, scale=2] at (13,0) (11) {$\mathcal{C}_1$}; 
        \node[fillstar] at (16,0) (12) {};
        \draw[-{Stealth[length=5mm, width=3mm]}] (11) -- (12);
         
        \node[] at (17,0) (13) {}; 
        \node[draw, circle, minimum size=1cm, scale=2] at (20,0) (14) {$\mathcal{C}_2$};
        \draw[-{Stealth[length=5mm, width=3mm]}] (13) -- (14);    
    \end{tikzpicture}
    }
        \caption{An example of cutting an edge}
        \label{fig.cutedge}
    \end{figure}

    \item \textbf{Cut:} A \underline{cut} $c$ of a couple $\mathcal{C}$ is a set of edges such that $\mathcal{C}$ is disconnected after cutting all edges in $c$. A \underline{refined cut} is a cut together with a map $\text{rc}:c\rightarrow \{\text{left}, \text{right}\}$. For each $\mathfrak{e}\in c$, if $\text{rc}(\mathfrak{e})=\text{left}$ (resp. right), then as in Figure \ref{fig.cutedge} the left node (resp. right node) produced by cutting $\mathfrak{e}$ is a $\star$ node (resp. invisible node). The map $\text{rc}$ describes which one should be the free or fixed leg in the two legs produced by cutting an edge.
    \item \textbf{$c(\mathfrak{e})$, $c(\mathfrak{n})$ and $c(\mathfrak{l})$:} Given an edge $\mathfrak{e}$ that is not a leg, define  $c(\mathfrak{l})$ to be the cut that contains only one edge $\mathfrak{e}$. Given a node $\mathfrak{n}\in \mathcal{C}$, let $\{\mathfrak{e}_{i}\}$ be edges that are connected to $\mathfrak{n}$, then define $c(\mathfrak{n})$ to be the cut that consists of edges $\{\mathfrak{e}_{i}\}$. Given a leg $\mathfrak{l}$, let $\mathfrak{n}$ be the unique node connected to it, then define  $c(\mathfrak{e})$ to be the cut $c(\mathfrak{n})$. An example of cutting $c(\mathfrak{e})$ is given by Figure \ref{fig.cutedge}. The following picture gives an example of cutting $c(\mathfrak{n})$ or $c(\mathfrak{l})$ (in this picture $\mathfrak{n}=\mathfrak{n}_1$ and $\mathfrak{l}$ is the leg labelled by $k$.)
    
    \begin{figure}[H]
    \centering
    \scalebox{0.4}{
    \begin{tikzpicture}[level distance=80pt, sibling distance=100pt]
        \node[] at (0,0) (1) {}; 
        \node[fillcirc] at (3,0) (2) {}; 
        \node[fillcirc] at (6,-2) (3) {}; 
        \node[fillcirc] at (9,-2) (4) {}; 
        \node[fillcirc] at (12,0) (5) {}; 
        \node[] at (15,0) (6) {}; 
        \draw[-{Stealth[length=5mm, width=3mm]}] (1) edge (2);
        \draw[-{Stealth[length=5mm, width=3mm]}] (2) edge (3);
        \draw[-{Stealth[length=5mm, width=3mm]}, bend left =40] (3) edge (4);
        \draw[-{Stealth[length=5mm, width=3mm]}, bend right =40] (3) edge (4);
        \draw[-{Stealth[length=5mm, width=3mm]}] (4) edge (5);
        \draw[-{Stealth[length=5mm, width=3mm]}] (5) edge (6);
        \draw[-{Stealth[length=5mm, width=3mm]}, bend left =40] (2) edge (5);
         
         \node[scale=2.0] at (3,-0.7) {$\mathfrak{n}_{1}$};
         \node[scale=2.0] at (6,-2.7) {$\mathfrak{n}_{2}$};
         \node[scale=2.0] at (9,-2.7) {$\mathfrak{n}_{3}$};
         \node[scale=2.0] at (12,-0.7) {$\mathfrak{n}_{4}$};
         \node[scale=2.0] at (1.5,-0.5) {$k$};
         \node[scale=2.0] at (4.3,-1.4) {$a$};
         \node[scale=2.0] at (7.5,-3.1) {$b$};
         \node[scale=2.0] at (7.5,-1) {$c$};
         \node[scale=2.0] at (10.7,-1.4) {$d$};
         \node[scale=2.0] at (7.5,2.2) {$e$};
         \node[scale=2.0] at (13.5,-0.5) {$-k$};
         \node[scale=3.0, rotate =45] at (4.3,-0.85) {$\times$};
         \node[scale=3.0, rotate = 0] at (7.5,1.75) {$\times$};
         
         
        \node[draw, single arrow,
              minimum height=33mm, minimum width=8mm,
              single arrow head extend=2mm,
              anchor=west, rotate=0] at (16,0) {};  
         
         
        \node[] at (20,0) (11) {}; 
        \node[fillcirc] at (23,0) (12) {}; 
        \node[fillstar] at (26,-2) (13) {};
        \node[fillstar] at (26,2) (14) {}; 
        \draw[-{Stealth[length=5mm, width=3mm]}] (11) edge (12);
        \draw[-{Stealth[length=5mm, width=3mm]}] (12) edge (13);
        \draw[-{Stealth[length=5mm, width=3mm]}] (12) edge (14);
        
        \node[scale=2.0] at (23,-0.7) {$\mathfrak{n}_{1}$};
        \node[scale=2.0] at (21.5,-0.5) {$k$};
        \node[scale=2.0] at (24.3,-1.4) {$a$};
        \node[scale=2.0] at (24.3,1.4) {$e$};
        \node[scale=2.0] at (23,-1.8) {$A$};
        
        
        \node[] at (28,-2) (32) {}; 
        \node[fillcirc] at (31,-2) (33) {}; 
        \node[fillcirc] at (34,-2) (34) {}; 
        \node[fillcirc] at (37,0) (35) {}; 
        \node[] at (40,0) (36) {}; 
        \node[] at (34,2) (37) {};
        \draw[-{Stealth[length=5mm, width=3mm]}] (32) edge (33);
        \draw[-{Stealth[length=5mm, width=3mm]}, bend left =40] (33) edge (34);
        \draw[-{Stealth[length=5mm, width=3mm]}, bend right =40] (33) edge (34);
        \draw[-{Stealth[length=5mm, width=3mm]}] (34) edge (35);
        \draw[-{Stealth[length=5mm, width=3mm]}] (35) edge (36);
        \draw[-{Stealth[length=5mm, width=3mm]}] (37) edge (35);
        
        \node[scale=2.0] at (31,-2.7) {$\mathfrak{n}_{2}$};
        \node[scale=2.0] at (34,-2.7) {$\mathfrak{n}_{3}$};
        \node[scale=2.0] at (37,-0.7) {$\mathfrak{n}_{4}$};
        \node[scale=2.0] at (29.5,-2.5) {$a$};
        \node[scale=2.0] at (32.5,-3.1) {$b$};
        \node[scale=2.0] at (32.5,-1) {$c$};
        \node[scale=2.0] at (35.7,-1.4) {$d$};
        \node[scale=2.0] at (35.7,1.4) {$e$};
        \node[scale=2.0] at (38.5,-0.5) {$-k$};
        \node[scale=2.0] at (37,-2.2) {$B_{a,e}$};
    \end{tikzpicture}
    }
        \caption{An example of cuts, $c(\mathfrak{n})$ and $c(\mathfrak{l})$}
        \label{fig.c(n)c(e)}
    \end{figure}
    \item \textbf{Normal edges in couples with multiple legs:} In this paper, all couples with multiple legs are produced by cutting a couple defined in Definition \ref{def.conple}. If a normal edge $\mathfrak{e}$ is cut into $\mathfrak{e}_1$ and $\mathfrak{e}_2$, then $\mathfrak{e}_1$ and $\mathfrak{e}_2$ are defined to be normal in the resulting couples with multiple legs.
\end{enumerate}
\end{defn}
\begin{rem}
Explicitly writing down the full definition of $\text{rc}$ is often complicated, so in what follows, when defining $\text{rc}$, we will only describe which one should be the free or fixed leg in the two legs produced by cutting an edge.
%Explicitly writing down the full definition of $\text{rc}$ is often complicated, so I what follows, when defining $\text{rc}$, we will just describe which node is $\star$ nodes or invisible nodes in the two nodes after cutting one edge.
\end{rem}

The couples in Proposition \ref{prop.counting} contains just $2$ fixed legs, but after cutting, these couples may contain more fixed or free legs. 

By Definition \ref{def.couplemultileg}, for a couple $\mathcal{C}$ with multiple legs, given constants $c_{\mathfrak{l}}$ for each fixed leg $\mathfrak{l}$, the corresponding equation of $\mathcal{C}$ is denoted by $Eq(\mathcal{C},\{c_{\mathfrak{l}}\}_{\mathfrak{l}})$.
In $Eq(\mathcal{C},\{c_{\mathfrak{l}}\}_{\mathfrak{l}})$ each edge $\mathfrak{e}$ is associated with a variable $k_{\mathfrak{e}}$ and each node $\mathfrak{n}$ is still associated with equations $MC_{\mathfrak{n}}$, $EC_{\mathfrak{n}}$. The corresponding variables of free (resp. fixed) legs are free (resp. fixed to be a constant $c_{\mathfrak{l}}$).


Let us explain how does $Eq(\mathcal{C})$ and $\#Eq(\mathcal{C})$ changes after cutting. The result is summarized in the following lemma.
\begin{lem}\label{lem.Eq(C)cutting}
Let $c$ be a cut of $\mathcal{C}$ that consists of edges $\{\mathfrak{e}_{i}\}$ and $\mathcal{C}_1$, $\mathcal{C}_2$ be two components after cutting. Let $\mathfrak{e}_{i}^{(1)}\in \mathcal{C}_1$, $\mathfrak{e}_{i}^{(2)}\in \mathcal{C}_2$ be two edges obtained by cutting $\mathfrak{e}_{i}$. The $\text{rc}$ map is defined by assigning $\{\mathfrak{e}_{i}^{(1)}\}$ to be free legs and $\{\mathfrak{e}_{i}^{(2)}\}$ to be fixed legs. Then we have 
\begin{equation}\label{eq.Eq(C)cutting.threewave}
    Eq(\mathcal{C},\{c_{\mathfrak{l}}\}_{\mathfrak{l}})=\left\{(k_{\mathfrak{e}_1},k_{\mathfrak{e}_{2}}):\ k_{\mathfrak{e}_1}\in Eq(\mathcal{C}_1,\{c_{\mathfrak{l}_1}\}),\  k_{\mathfrak{e}_{2}}\in Eq\left(\mathcal{C}_{2}, \{c_{\mathfrak{l}_2}\}, \left\{k_{\mathfrak{e}_{i}^{(1)}}\right\}_{i}\right)\right\}.
\end{equation}
and
\begin{equation}\label{eq.Eq(C)cuttingcounting.threewave}
    \sup_{\{c_{\mathfrak{l}}\}_{\mathfrak{l}}}\#Eq(\mathcal{C},\{c_{\mathfrak{l}}\}_{\mathfrak{l}})\le
    \sup_{\{c_{\mathfrak{l}_1}\}_{\mathfrak{l}_1\in \text{leg}(\mathcal{C}_1)} } \# Eq(\mathcal{C}_1,\{c_{\mathfrak{l}_1}\}) \sup_{\{c_{\mathfrak{l}_2}\}_{\mathfrak{l}_2\in \text{leg}(\mathcal{C}_2)} }\# Eq(\mathcal{C}_{2}, \{c_{\mathfrak{l}_2}\}).
\end{equation}
Here $\text{leg}(\mathcal{C})$ is the set of fixed legs in $\mathcal{C}$ (not the set of all legs!).
\end{lem}
\begin{proof}
%Let $c$ be a cut of $\mathcal{C}$ that consists of edges $\{\mathfrak{e}_{i}\}$. Let $\mathcal{C}_1$ and $\mathcal{C}_2$ be two components after cutting and $\mathfrak{e}_{i}^{(1)}\in \mathcal{C}_1$, $\mathfrak{e}_{i}^{(2)}\in \mathcal{C}_2$ be two edges obtained by cutting $\mathfrak{e}_{i}$. We assume that $\{\mathfrak{e}_{i}^{(1)}\}$ are free legs and $\{\mathfrak{e}_{i}^{(2)}\}$ are legs. 
By definition \eqref{eq.Eq(C,c).threewave} we have
\begin{equation}
\begin{split}
    Eq(\mathcal{C},\{c_{\mathfrak{l}}\}_{\mathfrak{l}})=&\{k_{\mathfrak{e}}\in \mathbb{Z}^d_L,\ |k_{\mathfrak{e}}|\lesssim 1:\  |k_{\mathfrak{e}x}| \sim \kappa_{\mathfrak{e}},\ \forall \mathfrak{e}\in \mathcal{C}_{\text{norm}}.\ MC_{\mathfrak{n}},\  EC_{\mathfrak{n}},\ \forall \mathfrak{n}.\  k_{\mathfrak{l}}=c_{\mathfrak{l}},\ \forall \mathfrak{l}\in \text{leg}(\mathcal{C}).\} 
    \\
    =&\{(k_{\mathfrak{e}_1},k_{\mathfrak{e}_2}):\ |k_{\mathfrak{e}_1}| \lesssim 1,\ MC_{\mathfrak{n}_1},\  EC_{\mathfrak{n}_1}.\ \forall \mathfrak{e}_1, \mathfrak{n}_1\in\mathcal{C}_1.\ |k_{\mathfrak{e}x}| \sim \kappa_{\mathfrak{e}},\ \forall \mathfrak{e}\in \mathcal{C}_{\text{norm}}.
    \\
    &k_{\mathfrak{l}_1}=c_{\mathfrak{l}_1},\ \forall \mathfrak{l}_1\in \text{leg}(\mathcal{C})\cap \text{leg}(\mathcal{C}_1)
    \\
    &|k_{\mathfrak{e}_2}| \lesssim 1,\ MC_{\mathfrak{n}_2},\  EC_{\mathfrak{n}_2}.\ \forall \mathfrak{e}_2, \mathfrak{n}_2\in\mathcal{C}_2.\ |k_{\mathfrak{e}x}| \sim \kappa_{\mathfrak{e}},\ \forall \mathfrak{e}\in \mathcal{C}_{\text{norm}}.
    \\
    &k_{\mathfrak{l}_2}=c_{\mathfrak{l}_2},\ \forall \mathfrak{l}_2\in \text{leg}(\mathcal{C})\cap \text{leg}(\mathcal{C}_2),\ k_{\mathfrak{e}_{i}^{(2)}}=k_{\mathfrak{e}_{i}^{(1)}},\ \forall\mathfrak{e}_{i}\in c\}
    \\
    =&\left\{(k_{\mathfrak{e}_1},k_{\mathfrak{e}_{2}}):\ k_{\mathfrak{e}_1}\in Eq(\mathcal{C}_1,\{c_{\mathfrak{l}_1}\}),\  k_{\mathfrak{e}_{2}}\in Eq\left(\mathcal{C}_{2}, \{c_{\mathfrak{l}_2}\}, \left\{k_{\mathfrak{e}_{i}^{(1)}}\right\}_{i}\right)\right\}
\end{split}
\end{equation}
Here in $Eq\left(\mathcal{C}_{2}, \{c_{\mathfrak{l}_2}\}, \left\{k_{\mathfrak{e}_{i}^{(1)}}\right\}_{i}\right)$, $k_{\mathfrak{e}_{i}^{(1)}}$ are view as a constant value and $k_{\mathfrak{e}_{i}^{(2)}}$ are fixed to be this constant value.

Therefore, we have the following identity of $Eq(\mathcal{C},\{c_{\mathfrak{l}}\}_{\mathfrak{l}})$
\begin{equation}
    Eq(\mathcal{C},\{c_{\mathfrak{l}}\}_{\mathfrak{l}})=\left\{(k_{\mathfrak{e}_1},k_{\mathfrak{e}_{2}}):\ k_{\mathfrak{e}_1}\in Eq(\mathcal{C}_1,\{c_{\mathfrak{l}_1}\}),\  k_{\mathfrak{e}_{2}}\in Eq\left(\mathcal{C}_{2}, \{c_{\mathfrak{l}_2}\}, \left\{k_{\mathfrak{e}_{i}^{(1)}}\right\}_{i}\right)\right\}.
\end{equation}
which proves \eqref{eq.Eq(C)cutting.threewave}.

We can also find the relation between $\#Eq(\mathcal{C}_1)$, $\#Eq(\mathcal{C}_2)$ and $\#Eq(\mathcal{C})$. Applying \eqref{eq.Eq(C)cutting.threewave},
\begin{equation}
\begin{split}
    \#Eq(\mathcal{C},\{c_{\mathfrak{l}}\}_{\mathfrak{l}})=&\sum_{(k_{\mathfrak{e}_1},k_{\mathfrak{e}_{2}})\in \#Eq(\mathcal{C},\{c_{\mathfrak{l}}\}_{\mathfrak{l}})} 1
    \\
    =&\sum_{\left\{(k_{\mathfrak{e}_1},k_{\mathfrak{e}_{2}}):\ k_{\mathfrak{e}_1}\in Eq(\mathcal{C}_1,\{c_{\mathfrak{l}_1}\}),\  k_{\mathfrak{e}_{2}}\in Eq\left(\mathcal{C}_{2}, \{c_{\mathfrak{l}_2}\}, \left\{k_{\mathfrak{e}_{i}^{(1)}}\right\}_{i}\right)\right\}} 1
    \\
    =&\sum_{k_{\mathfrak{e}_1}\in Eq(\mathcal{C}_1,\{c_{\mathfrak{l}_1}\})} \sum_{k_{\mathfrak{e}_{2}}\in Eq\left(\mathcal{C}_{2}, \{c_{\mathfrak{l}_2}\}, \left\{k_{\mathfrak{e}_{i}^{(1)}}\right\}_{i}\right)} 1
    \\
    =&\sum_{k_{\mathfrak{e}_1}\in Eq(\mathcal{C}_1,\{c_{\mathfrak{l}_1}\})} \# Eq\left(\mathcal{C}_{2}, \{c_{\mathfrak{l}_2}\}, \left\{k_{\mathfrak{e}_{i}^{(1)}}\right\}_{i}\right)
\end{split}
\end{equation}
Take $\sup$ in the above equation
\begin{equation}
\begin{split}
    &\sup_{\{c_{\mathfrak{l}}\}_{\mathfrak{l}}}\#Eq(\mathcal{C},\{c_{\mathfrak{l}}\}_{\mathfrak{l}})
    =\sup_{\{c_{\mathfrak{l}}\}_{\mathfrak{l}}}\sum_{k_{\mathfrak{e}_1}\in Eq(\mathcal{C}_1,\{c_{\mathfrak{l}_1}\})} \# Eq\left(\mathcal{C}_{2}, \{c_{\mathfrak{l}_2}\}, \left\{k_{\mathfrak{e}_{i}^{(1)}}\right\}_{i}\right)
    \\
    \le &\sup_{\{c_{\mathfrak{l}_1}\}_{\mathfrak{l}_1\in \text{leg}(\mathcal{C})\cap \text{leg}(\mathcal{C}_1)} }\sum_{k_{\mathfrak{e}_1}\in Eq(\mathcal{C}_1,\{c_{\mathfrak{l}_1}\})} \sup_{\{c_{\mathfrak{l}_2}\}_{\mathfrak{l}_2\in \text{leg}(\mathcal{C})\cap \text{leg}(\mathcal{C}_2)} }\# Eq\left(\mathcal{C}_{2}, \{c_{\mathfrak{l}_2}\}, \left\{k_{\mathfrak{e}_{i}^{(1)}}\right\}_{i}\right)
    \\
    \le &\sup_{\{c_{\mathfrak{l}_1}\}_{\mathfrak{l}_1\in \text{leg}(\mathcal{C}_1)} }\sum_{k_{\mathfrak{e}_1}\in Eq(\mathcal{C}_1,\{c_{\mathfrak{l}_1}\})} \sup_{\{c_{\mathfrak{l}_2}\}_{\mathfrak{l}_2\in \text{leg}(\mathcal{C}_2)} }\# Eq(\mathcal{C}_{2}, \{c_{\mathfrak{l}_2}\})
    \\
    = &\sup_{\{c_{\mathfrak{l}_1}\}_{\mathfrak{l}_1\in \text{leg}(\mathcal{C}_1)} } \# Eq(\mathcal{C}_1,\{c_{\mathfrak{l}_1}\}) \sup_{\{c_{\mathfrak{l}_2}\}_{\mathfrak{l}_2\in \text{leg}(\mathcal{C}_2)} }\# Eq(\mathcal{C}_{2}, \{c_{\mathfrak{l}_2}\})
\end{split}
\end{equation}

This proves \eqref{eq.Eq(C)cuttingcounting.threewave}.
\end{proof}









\textbf{Step 2.} In this step, we specify the cutting procedure. 

Notice that Proposition \ref{prop.countingind}, the multiple leg analog of Proposition \ref{prop.counting}, is only true for couples satisfying the "property P". When designing the cutting procedure, we must make sure that all couples generated during the execution of this procedure satisfy the "property P". The following proposition guarantees the existence of such a procedure.

\begin{prop}\label{prop.cuttingalgorithm}
There exists a recursive algorithm that repeatedly decomposes $\mathcal{C}$ into smaller pieces and satisfies the following requirements. In the rest of this paper, we will call this algorithm "the cutting algorithm".

(1) The input of the $0$-th step of this algorithm is $\mathcal{C}(0)=\mathcal{C}$. The inputs of other steps are the outputs of previous steps of the algorithm itself.

(2) In step $k$, $\mathcal{C}(k)$ is decomposed into 2 or 3 connected components by cutting edges and all components with more than one node are outputted. For $\mathcal{C}(1)$, $\# Eq(\mathcal{C}(1))=\# Eq(\mathcal{C})$ and $\mathcal{C}_{\text{norm}}(1)=\mathcal{C}_{\text{norm}}$.

(3) One of the connected components in (2) contains exactly one node $\mathfrak{n}$ and one fixed normal leg $\mathfrak{l}$. We call this component $\mathcal{C}(k)_{
\mathfrak{l}}$. There are only two possibilities of $\mathcal{C}(k)_{\mathfrak{l}}$ as in Figure \ref{fig.2possibilities}. We label them by $\mathcal{C}_{I}$, $\mathcal{C}_{II}$.

\begin{figure}[H]
    \centering
    \scalebox{0.3}{
    \begin{tikzpicture}[level distance=80pt, sibling distance=100pt]
        \node[] at (0,0) (1) {} 
            child {node[fillcirc] (2) {} 
                child {node[fillstar] (3) {}}
                child {node[fillstar] (4) {}}
            };
        \draw[-{Stealth[length=5mm, width=3mm]}] (1) -- (2);
        \draw[-{Stealth[length=5mm, width=3mm]}] (2) -- (3);
        \draw[-{Stealth[length=5mm, width=3mm]}] (2) -- (4);
        
        \node[] at (12,0) (1) {} 
            child {node[fillcirc] (2) {} 
                child {node[fillstar] (3) {}}
                child {node[xshift = 5pt, yshift = -10pt] (4) {}}
            };
        \draw[-{Stealth[length=5mm, width=3mm]}] (1) -- (2);
        \draw[-{Stealth[length=5mm, width=3mm]}] (2) -- (3);
        \draw[-{Stealth[length=5mm, width=3mm]}] (2) -- (4);
    \end{tikzpicture}
    }
        \caption{Two possibilities of $\mathcal{C}_\mathfrak{l}$.}
        \label{fig.2possibilities}
    \end{figure}
    
(4) The cutting algorithm satisfies the requirement that all connected components in (2) generated in each step satisfy property P, where property P is defined below

\underline{Property P of a couple $\widetilde{\mathcal{C}}$}: $\widetilde{\mathcal{C}}$ is connected and contains exactly one free leg and at least one fixed normal leg.

% We need the following definition.

% \underline{Property P of a couple $\widetilde{\mathcal{C}}$}: $\widetilde{\mathcal{C}}$ is connected and contains exactly one free leg and at least one fixed normal leg.

% With the above definition, the cutting algorithm should satisfy the requirement that the output components of each steps of the algorithm satisfies property P.
\end{prop}

\begin{rem}
Although the definition of cut is rather general, only three special types of cuts, $c(\mathfrak{e})$, $c(\mathfrak{n})$ and $c(\mathfrak{l})$, are used in the cutting algorithm.
\end{rem}

\begin{rem}
The couple $\mathcal{C}$ does not satisfy the property P because it does not have any free leg.
\end{rem}

\begin{proof}[Proof of Proposition \ref{prop.cuttingalgorithm}.] Consider the following algorithm.

\medskip


\begin{mdframed}

\centerline{\textbf{The cutting algorithm}}

\medskip 

\ \ \ \textbf{Step 0.} The input $\mathcal{C}(0)$ of this step is $\mathcal{C}$. In this step, we replace one of the two fixed legs of $\mathcal{C}$ with a free leg to obtain a new couple $\widehat{\mathcal{C}}$. By Lemma \ref{lem.freeleg} (3), $\#Eq(\mathcal{C})=\#Eq(\widehat{\mathcal{C}})$. The output $\mathcal{C}(1)$ of this step is $\widehat{\mathcal{C}}$. An example of step 0 can be found in the following picture.
\begin{figure}[H]
    \centering
    \scalebox{0.4}{
    \begin{tikzpicture}[level distance=80pt, sibling distance=100pt]
        \node[] at (0,0) (1) {}; 
        \node[fillcirc] at (3,0) (2) {}; 
        \node[fillcirc] at (7,0) (3) {}; 
        \node[] at (10,0) (4) {}; 
        \draw[-{Stealth[length=5mm, width=3mm]}] (1) edge (2);
        \draw[-{Stealth[length=5mm, width=3mm]}, bend left =40] (2) edge (3);
        \draw[-{Stealth[length=5mm, width=3mm]}, bend right =40] (2) edge (3);
        \draw[-{Stealth[length=5mm, width=3mm]}] (3) edge (4);
        \node[scale=2.0] at (5.1,-2) {$\mathcal{C}(0)=\mathcal{C}$};
 
        \node[draw, single arrow,
              minimum height=33mm, minimum width=8mm,
              single arrow head extend=2mm,
              anchor=west, rotate=0] at (11,0) {};  
              
      
        \node[] at (15,0) (11) {}; 
        \node[fillcirc] at (18,0) (12) {}; 
        \node[fillcirc] at (22,0) (13) {}; 
        \node[fillstar] at (25,0) (14) {}; 
        \draw[-{Stealth[length=5mm, width=3mm]}] (11) edge (12);
        \draw[-{Stealth[length=5mm, width=3mm]}, bend left =40] (12) edge (13);
        \draw[-{Stealth[length=5mm, width=3mm]}, bend right =40] (12) edge (13);
        \draw[-{Stealth[length=5mm, width=3mm]}] (13) edge (14);      
        \node[scale=2.0] at (20.1,-2) {$\mathcal{C}(1)=\widehat{\mathcal{C}}$};
    \end{tikzpicture}
    }
        \caption{An example of step $0$}
        \label{fig.step0}
    \end{figure}

\textbf{Step $k$.} Assume that the step $k-1$ have been finished. The input $\mathcal{C}(k)$ of step $k$ is the output of step $k-1$. (If there are two output couples from step $k-1$, apply to step $k$ to these two couples separately.)

By property P, there exists a fixed normal leg in $\mathcal{C}(k)$. Choose one such leg $\mathfrak{l}$ and define $\mathcal{C}(k)_{\mathfrak{l}}$ to be the component which contains $\mathfrak{l}$ after cutting $c(\mathfrak{l})$ and define $\mathcal{C}(k)'=\mathcal{C}(k)\backslash \mathcal{C}(k)_{\mathfrak{l}}$. Check how many components $\mathcal{C}(k)'$ have. Jump to case 1 if the number of components is equal to 1, otherwise, jump to case 2. Examples of case 1 and case 2 can be found in the following picture.
\begin{figure}[H]
    \centering
    \scalebox{0.4}{
    \begin{tikzpicture}[level distance=80pt, sibling distance=100pt]
        \node[] at (0,0) (1) {};
        \node[fillcirc] at (0, -3) (2) {};
        \node[draw, circle, minimum size=1cm, scale=2] at (0,-7) (3) {$\mathcal{C}(k)'$}; 
        \draw[-{Stealth[length=5mm, width=3mm]}] (1) edge (2);
        \draw[-{Stealth[length=5mm, width=3mm]}, bend left =40] (2) edge (3);
        \draw[-{Stealth[length=5mm, width=3mm]}, bend right =40] (2) edge (3);
        
        \node[] at (10,0) (1) {};
        \node[fillcirc] at (10, -3) (2) {};
        \node[draw, circle, minimum size=1cm, scale=1.5] at (7,-7) (3) {$(\mathcal{C}(k)')_1$};
        \node[draw, circle, minimum size=1cm, scale=1.5] at (13,-7) (4) {$(\mathcal{C}(k)')_2$};
        \draw[-{Stealth[length=5mm, width=3mm]}] (1) edge (2);
        \draw[-{Stealth[length=5mm, width=3mm]}] (2) edge (3);
        \draw[-{Stealth[length=5mm, width=3mm]}] (2) edge (4);
    \end{tikzpicture}
    }
        \caption{Examples of case 1 and case 2 (Left is case 1 and right is case two. $(\mathcal{C}(k)')_1$ and $(\mathcal{C}(k)')_2$ are two components of $\mathcal{C}(k)'$)}
        \label{fig.step1case}
    \end{figure}

\textbf{Case 1 of step $k$.} In this case $\mathcal{C}(k)'$ has one components. We rename  $\mathcal{C}(k)'$ to $\mathcal{C}(k)_1$. By property P, there exists a unique free leg $\mathfrak{l}_{fr}$ in $\mathcal{C}(k)$. Check if $\mathfrak{l}_{fr}$ and $\mathfrak{l}$ are connected to the same node. If yes, jump to case 1.1, otherwise jump to case 1.2.

\textbf{Case 1.1.} Cut edges in $c(\mathfrak{l})$ into $\{\mathfrak{e}_{i}^{(1)}\}_{i=1,2}$ and $\{\mathfrak{e}_{i}^{(2)}\}_{i=1,2}$, then $\mathcal{C}(k)$ is decomposed into $\mathcal{C}(k)_{\mathfrak{l}}$, $\mathcal{C}(k)_1=\mathcal{C}\backslash \mathcal{C}_{\mathfrak{l}}$. As in Lemma \ref{lem.Eq(C)cutting}, define $\{\mathfrak{e}_{i}^{(1)}\}\subseteq \mathcal{C}(k)_{\mathfrak{l}}$ to be free legs and $\{\mathfrak{e}_{i}^{(2)}\}\subseteq \mathcal{C}(k)_1$ to be fixed legs. 

If $\mathcal{C}(k)_1$ satisfies the property P, define $\mathcal{C}(k+1)=\mathcal{C}(k)_1$ to be the output of step $k$ and apply step $k+1$ to $\mathcal{C}(k+1)$. 

Otherwise, by Lemma \ref{lem.normleg} (2) there exists exactly one free normal leg and at least one fixed leg. By Lemma \ref{lem.freeleg}, we can define a new couple $\widehat{\mathcal{C}}$ such that the free normal leg becomes fixed and the fixed leg becomes free.  Finally, define $\mathcal{C}(k+1)=\widehat{\mathcal{C}}$ to be the output. 

Examples of cutting in case 1.1 can be found in the following picture.
\begin{figure}[H]
    \centering
    \scalebox{0.4}{
    \begin{tikzpicture}[level distance=80pt, sibling distance=100pt]
        \node[] at (0,0) (1) {};
        \node[fillcirc] at (0, -3) (2) {};
        \node[draw, circle, minimum size=1cm, scale=2] at (0,-7) (3) {$\mathcal{C}(k)_1$}; 
        \draw[-{Stealth[length=5mm, width=3mm]}] (1) edge (2);
        \draw[-{Stealth[length=5mm, width=3mm]}, bend left =40] (2) edge (3);
        \draw[-{Stealth[length=5mm, width=3mm]}, bend right =40] (2) edge (3);
        \node[scale =3] at (-1.1,-4.8) {$\times$};
        \node[scale =3] at (1.1,-4.8) {$\times$};
        
        \node[draw, single arrow,
              minimum height=33mm, minimum width=8mm,
              single arrow head extend=2mm,
              anchor=west, rotate=0] at (4,-4.8) {};  
        
        \node[] at (12,-1.5) (11) {} 
            child {node[fillcirc] (12) {} 
                child {node[fillstar] (13) {}}
                child {node[fillstar] (14) {}}
            };
        \draw[-{Stealth[length=5mm, width=3mm]}] (11) -- (12);
        \draw[-{Stealth[length=5mm, width=3mm]}] (12) -- (13);
        \draw[-{Stealth[length=5mm, width=3mm]}] (12) -- (14);
        \node[scale =2] at (12,-8) {$\mathcal{C}(k)_{\mathfrak{l}}$};
        
        \node[] at (17,-3) (21) {};
        \node[] at (21,-3) (22) {};
        \node[draw, circle, minimum size=1cm, scale=2] at (19,-6) (23) {$\mathcal{C}(k)_1$}; 
        \draw[-{Stealth[length=5mm, width=3mm]}] (21) edge (23);
        \draw[-{Stealth[length=5mm, width=3mm]}] (22) edge (23);
    \end{tikzpicture}
    }
        \caption{Examples of cutting in case 1.1}
        \label{fig.step1case1.1}
    \end{figure}

\textbf{Case 1.2.} In this case, $\mathfrak{l}_{fr}$ and $\mathfrak{l}$ are connected to the same node $\mathfrak{n}$. Let $\mathfrak{e}$ be the edge connecting $\mathfrak{n}$ and another interior node. Cut $\mathfrak{e}$ into $\{\mathfrak{e}^{(1)},\mathfrak{e}^{(2)}\}$ and then $\mathcal{C}(k)$ is decomposed into $\mathcal{C}(k)_{\mathfrak{l}}$, $\mathcal{C}(k)_1=\mathcal{C}(k)\backslash \mathcal{C}(k)_{\mathfrak{l}}$. Define $\mathfrak{e}^{(1)}\in \mathcal{C}(k)_{\mathfrak{l}}$ to be fixed legs and $\mathfrak{e}^{(2)}\in \mathcal{C}(k)_1$ to be free legs. 

If $\mathcal{C}(k)_1$ satisfies the property P, define $\mathcal{C}(k+1)=\mathcal{C}(k)_1$.
%to be the output of step $k$ and apply step $k+1$ to $\mathcal{C}(k+1)$


Otherwise by Lemma \ref{lem.normleg} (2), %there exist exactly one free normal leg and at least one fixed leg.
%By Lemma \ref{lem.freeleg}, we can define a new couple $\widehat{\mathcal{C}}$ such that the free normal leg becomes fixed and the fixed leg become free and we define $\mathcal{C}(k+1)=\widehat{\mathcal{C}}$ in this case.
%According to Lemma \ref{lem.normleg} (2), $\mathcal{C}(k)_1$ contains at least one normal legs and two legs in $\mathcal{C}(k)_1$. 
$\mathfrak{e}^{(2)}$ is the only normal legs. Using Lemma \ref{lem.freeleg} (2), we may construct a new couple $\mathcal{C}(k+1)$ by assigning $\mathfrak{e}^{(2)}$ to be fixed and another leg to be free. 
%In any other case $\mathcal{C}(k)_1$ contains a fixed normal leg and we define $\mathcal{C}(k+1)=\mathcal{C}(k)_1$. 
Finally define $\mathcal{C}(k+1)$ to be the output of step $k$ and apply step $k+1$ to $\mathcal{C}(k+1)$. 

Examples of cutting in case 1.2 can be found in the following picture.
\begin{figure}[H]
    \centering
    \scalebox{0.4}{
    \begin{tikzpicture}[level distance=80pt, sibling distance=100pt]
        \node[] at (-2,0) (0) {};
        \node[fillstar] at (1.8,-0.2) (1) {};
        \node[fillcirc] at (0, -3) (2) {};
        \node[draw, circle, minimum size=1cm, scale=2] at (0,-7) (3) {$\mathcal{C}(k)_1$}; 
        \draw[-{Stealth[length=5mm, width=3mm]}] (0) edge (2);
        \draw[-{Stealth[length=5mm, width=3mm]}] (1) edge (2);
        \draw[-{Stealth[length=5mm, width=3mm]}] (2) edge (3);
        \node[scale =3] at (0,-4.4) {$\times$};
        
        \node[draw, single arrow,
              minimum height=33mm, minimum width=8mm,
              single arrow head extend=2mm,
              anchor=west, rotate=0] at (4,-4.8) {};  
        
        \node at (12,-7) (11) {} [grow =90]
            child {node[fillcirc] (12) {} 
                child {node[fillstar, xshift = -0.2cm, yshift = -0.2cm] (13) {}}
                child {node[] (14) {}}
            };
        \draw[{Stealth[length=5mm, width=3mm]}-] (11) -- (12);
        \draw[{Stealth[length=5mm, width=3mm]}-] (12) -- (13);
        \draw[{Stealth[length=5mm, width=3mm]}-] (12) -- (14);
        \node[scale =2] at (12,-8) {$\mathcal{C}(k)_{\mathfrak{l}}$};
        
        \node[fillstar] at (19,-2) (22) {};
        \node[draw, circle, minimum size=1cm, scale=2] at (19,-6) (23) {$\mathcal{C}(k)_1$}; 
        \draw[-{Stealth[length=5mm, width=3mm]}] (22) edge (23);
    \end{tikzpicture}
    }
        \caption{Examples of cutting in case 1.2}
        \label{fig.step1case1.2}
    \end{figure}

\textbf{Case 2 of step $k$.} Let the two connected components of $\mathcal{C}(k)\backslash \mathcal{C}(k)_{\mathfrak{l}}$ be $\mathcal{C}(k)_2$ and $\mathcal{C}(k)_3$. Let $\mathfrak{e}_{2}$, $\mathfrak{e}_{3}$ be the two edges that connect $\mathfrak{l}$ and  $\mathcal{C}(k)_2$, $\mathcal{C}(k)_3$ respectively. Cut
$\mathfrak{e}_{2}$, $\mathfrak{e}_{3}$ into $\{\mathfrak{e}_{2}^{(1)},\mathfrak{e}_{3}^{(1)}\}\subseteq \mathcal{C}(k)_{\mathfrak{l}}$ and $\mathfrak{e}_{2}^{(2)}\in \mathcal{C}(k)_2$, $\mathfrak{e}_{3}^{(2)}\in \mathcal{C}(k)_3$. 
By Lemma \ref{lem.normleg}, $\mathcal{C}(k)_2$, $\mathcal{C}(k)_3$ contain at least one normal leg and two legs. By symmetry, we can just consider $\mathcal{C}(k)_2$. 

If $\mathcal{C}(k)_2$ contains free legs, define $\mathfrak{e}_{2}^{(2)}\in \mathcal{C}(k)_2$ to be fixed, otherwise define $\mathfrak{e}_{2}^{(2)}$ to be free. 

In the case that $\mathcal{C}(k)_2$ contains free legs, define the output $\mathcal{C}(k+1)$ to be $\mathcal{C}(k)_2$ or $\mathcal{C}(k)_3$ and apply step $k+1$ to them separately. 

In the case that $\mathcal{C}(k)_2$ contains no free legs, $\mathfrak{e}_{2}^{(2)}$ is the only normal leg and it is defined to be free. Use Lemma \ref{lem.freeleg} (2) to construct a new couple $\widehat{\mathcal{C}}_2$ by assigning $\mathfrak{e}_2^{(2)}$ to be fixed and another leg to be free. Then define $\mathcal{C}(k+1)$ to be $\widehat{\mathcal{C}}_2$ or $\widehat{\mathcal{C}}_3$ and apply step $k+1$ to them separately. 

Examples of cutting in case 2 can be found in the following picture.
    \begin{figure}[H]
    \centering
    \scalebox{0.4}{
    \begin{tikzpicture}[level distance=80pt, sibling distance=100pt]
    
    \node[] at (0,0) (1) {};
        \node[fillcirc] at (0, -3) (2) {};
        \node[draw, circle, minimum size=1cm, scale=2] at (-3,-7) (3) {$\mathcal{C}(k)_2$};
        \node[draw, circle, minimum size=1cm, scale=2] at (3,-7) (4) {$\mathcal{C}(k)_3$};
        \draw[-{Stealth[length=5mm, width=3mm]}] (1) edge (2);
        \draw[-{Stealth[length=5mm, width=3mm]}] (2) edge (3);
        \draw[-{Stealth[length=5mm, width=3mm]}] (2) edge (4);
        \node[scale =3, rotate = 44] at (-1.18,-4.5) {$\times$};
        \node[scale =3, rotate = 44] at (1.18,-4.5) {$\times$};
        
        \node[draw, single arrow,
              minimum height=33mm, minimum width=8mm,
              single arrow head extend=2mm,
              anchor=west, rotate=0] at (5,-4.5) {};  
        
        \node[] at (12,-1.5) (11) {} 
            child {node[fillcirc] (12) {} 
                child {node[fillstar] (13) {}}
                child {node[fillstar] (14) {}}
            };
        \draw[-{Stealth[length=5mm, width=3mm]}] (11) -- (12);
        \draw[-{Stealth[length=5mm, width=3mm]}] (12) -- (13);
        \draw[-{Stealth[length=5mm, width=3mm]}] (12) -- (14);
        \node[scale =2] at (12,-8) {$\mathcal{C}(k)_{\mathfrak{l}}$};
        
        \node[] at (18,-1.5) (11) {}; 
        \node[draw, circle, minimum size=1cm, scale=2] at (18,-6.5) (12) {$\mathcal{C}(k)_2$}; 
        \node[] at (23,-1.5) (13) {}; 
        \node[draw, circle, minimum size=1cm, scale=2] at (23,-6.5) (14) {$\mathcal{C}(k)_3$};
        \draw[-{Stealth[length=5mm, width=3mm]}] (11) -- (12);
        \draw[-{Stealth[length=5mm, width=3mm]}] (13) -- (14);
    \end{tikzpicture}
    }
        \caption{Examples of cutting in case 2}
        \label{fig.step1case2}
    \end{figure}
\end{mdframed}


% (1) Replace one of the two fixed legs of $\mathcal{C}$ by a free leg to obtain a new couple $\widehat{\mathcal{C}}$. By Lemma \ref{lem.freeleg} (ii), $\#Eq(\mathcal{C})=\#Eq(\widehat{\mathcal{C}})$.

% (2) By Lemma \ref{lem.normleg} (ii), there exists a fixed normal leg in $\mathcal{C}$. Choose one such leg $\mathfrak{l}$ and define $\mathcal{C}_{\mathfrak{l}}$ to be the component which contains $\mathfrak{l}$ after cutting $c(\mathfrak{l})$ and define $\mathcal{C}'=\mathcal{C}\backslash \mathcal{C}_{\mathfrak{l}}$. Check how many components does $\mathcal{C}'$ have. Jump to (3) if number of components equals to 1, otherwise jump to (4).

% (3) In this case $\mathcal{C}'$ has one components, By Lemma \ref{lem.normleg} (ii), there exists a unique free leg $\mathfrak{l}_{fr}$ in $\mathcal{C}$. Check if $\mathfrak{l}_{fr}$ and $\mathfrak{l}$ are connected to a same node. If yes, jump to (3.1), otherwise jump to (3.2).

% (3.1) Cut edges in $c(\mathfrak{l})$, then $\mathcal{C}$ is decomposed into $\mathcal{C}_{\mathfrak{l}}$, $\mathcal{C}'=\mathcal{C}\backslash \mathcal{C}_{\mathfrak{l}}$. As in Lemma \ref{lem.Eq(C)cutting}, define $\{\mathfrak{e}_{i}^{(1)}\}\subseteq \mathcal{C}_{\mathfrak{l}}$ to be free legs and $\{\mathfrak{e}_{i}^{(2)}\}\subseteq \mathcal{C}'$ to be fixed legs. Then jump to (2) and apply (2)-(4) to $\mathcal{C}'$.

% (3.2) In this case $\mathfrak{l}_{fr}$ and $\mathfrak{l}$ are connected to a same node $\mathfrak{n}$. Let $\mathfrak{e}$ be the last edge that is connected $\mathfrak{n}$. Cut $\mathfrak{e}$ and then $\mathcal{C}$ is decomposed into $\mathcal{C}_{\mathfrak{l}}$, $\mathcal{C}'=\mathcal{C}\backslash \mathcal{C}_{\mathfrak{l}}$. Define $\mathfrak{e}^{(1)}\in \mathcal{C}_{\mathfrak{l}}$ to be fixed legs and $\mathfrak{e}^{(2)}\in \mathcal{C}'$ to be free legs. According to Lemma \ref{lem.normleg} (i), there should be at least one normal legs and two legs in $\mathcal{C}'$. If $\mathfrak{e}^{(2)}$ is the only normal legs, then use Lemma \ref{lem.freeleg} (i), we may construct a new couple $\widehat{\mathcal{C}}'$ by assigning $\mathfrak{e}^{(2)}$ to be fixed and another leg to be free. In any other case $\mathcal{C}'$ contains a fixed normal leg and we define $\widehat{\mathcal{C}}'=\mathcal{C}'$. Finally jump to (2) and apply (2)-(4) to $\widehat{\mathcal{C}}'$.

% (4) Let the two connected components of $\mathcal{C}\backslash \mathcal{C}_{\mathfrak{l}}$ be $\mathcal{C}_2$ and $\mathcal{C}_3$. Let $\mathfrak{e}_{2}$, $\mathfrak{e}_{3}$ be the two edges that connect $\mathfrak{l}$ and  $\mathcal{C}_2$, $\mathcal{C}_3$ respectively. Cut
% $\mathfrak{e}_{2}$, $\mathfrak{e}_{3}$ into $\{\mathfrak{e}_{2}^{(1)},\mathfrak{e}_{3}^{(1)}\}\subseteq \mathcal{C}_{\mathfrak{l}}$ and $\mathfrak{e}_{2}^{(2)}\in \mathcal{C}_2$, $\mathfrak{e}_{3}^{(2)}\in \mathcal{C}_3$. By Lemma \ref{lem.normleg} (i), there should be at least one normal legs and two legs in $\mathcal{C}_2$, $\mathcal{C}_3$. By symmetry, we can just consider $\mathcal{C}_2$. If $\mathcal{C}_2$ contains free legs, then define  $\mathfrak{e}_{2}^{(2)}\in \mathcal{C}_2$ to be fixed, otherwise define $\mathfrak{e}_{2}^{(2)}$ to be free. In this case, jump to (2) and apply (2)-(4) to $\mathcal{C}_2$, $\mathcal{C}_3$ separately. If in $\mathcal{C}_2$ $\mathfrak{e}_{2}^{(2)}$ is the only normal legs and it's free, then use Lemma \ref{lem.freeleg} (i) to construct a new couple $\widehat{\mathcal{C}}_2$ by assigning $\mathfrak{e}^{(2)}$ to be fixed and another leg to be free. Then jump to (2) and apply (2)-(4) to $\widehat{\mathcal{C}}_2$, $\widehat{\mathcal{C}}_3$ separately.

(1), (3) are true by definition. (2) is true because
$\#Eq(\mathcal{C})=\#Eq(\widehat{\mathcal{C}})$ as explained in step $0$ of the algorithm. Since in step $0$, we just replace a fixed leg with a free leg, $\mathcal{C}_{\text{norm}}$ should not change, so we get $\mathcal{C}_{\text{norm}}(1)=\mathcal{C}_{\text{norm}}$. 

The only non-trivial part is (4), which is a corollary of Lemma \ref{lem.normleg} below.

Therefore, we complete the proof.
\end{proof}


\begin{lem}\label{lem.freeleg} %Free leg moving lemma
Given a connected couple $\mathcal{C}$ with multiple legs, then we have the following conclusions.

(1) Let $\{k_{\mathfrak{l}_i}\}_{i=1,\cdots,n_{\text{leg}}}$ be the variables corresponding to legs in couple $\mathcal{C}$. Let $\iota_{\mathfrak{e}}$ be the same as \eqref{eq.iotadef.threewave}, then $Eq(\mathcal{C})$ implies the following momentum conservation equation
\begin{equation}\label{eq.momentumconservation.threewave}
    \sum_{i=1}^{n_{\text{leg}}} \iota_{\mathfrak{l}_i}k_{\mathfrak{l}_i}=0,
\end{equation}

(2) Assume that there is exactly one free leg $\mathfrak{l}_{i_0}$ in $\mathcal{C}$ and all other variables $\{k_{\mathfrak{l}_{i}}\}_{i\ne i_0}$ corresponding to fix legs are fixed to be constants $\{c_{\mathfrak{l}_{i}}\}_{i\ne i_0}$. For any $i_{1}=1,\cdots,n_{\text{leg}}$, we can construct a new couple $\widehat{\mathcal{C}}$ by replacing the $i_0$ leg by a fixed leg and $i_1$ leg by a free leg. If $i\ne i_0, i_1$, fix $k_{\mathfrak{l}_{i}}$ to be the constant $c_{\mathfrak{l}_{i}}$, if $i=i_0$, fix $k_{\mathfrak{l}_{i_0}}$ to be the constant $-\iota_{\mathfrak{l}_{i_0}}\sum_{i\ne i_0} \iota_{\mathfrak{l}_i}k_{\mathfrak{l}_i}$. Under the above assumptions, we have
\begin{equation}
    Eq(\mathcal{C}, \{c_{\mathfrak{l}_{i}}\}_{i\ne i_0})=Eq\left(\widehat{\mathcal{C}}, \{c_{\mathfrak{l}_{i}}\}_{i\ne i_0, i_1}\cup \{-\iota_{\mathfrak{l}_{i_0}}\sum_{i\ne i_0} \iota_{\mathfrak{l}_i}k_{\mathfrak{l}_i}\}\right).
\end{equation}

(3) Assume that there is no free leg in $\mathcal{C}$ and all $\{k_{\mathfrak{l}_{i}}\}_{i\ne i_0}$ are fixed to be constants $\{c_{\mathfrak{l}_{i}}\}_{i\ne i_0}$. For any $i_{1}=1,\cdots,n_{\text{leg}}$, we can construct a new couple $\widehat{\mathcal{C}}$ by replacing the $i_0$ leg by a free leg. Then we have
\begin{equation}
    Eq(\mathcal{C}, \{c_{\mathfrak{l}_{i}}\}_{i})=Eq(\widehat{\mathcal{C}}, \{c_{\mathfrak{l}_{i}}\}_{i\ne i_0}).
\end{equation}

(4) If the couple $\mathcal{C}$ contains any leg, then it contains at least two legs.

\end{lem}
\begin{proof}
We first prove (1). Given a node $\mathfrak{n}$ and an edge $\mathfrak{e}$ connected to it, we define $\iota_{\mathfrak{e}}(\mathfrak{n})$ by the following rule
\begin{equation}
    \iota_{\mathfrak{e}}(\mathfrak{n})=\begin{cases}
        +1 \qquad \textit{if $\mathfrak{e}$ pointing towards $\mathfrak{n}$}
        \\
        -1 \qquad  \textit{if $\mathfrak{e}$ pointing outwards from $\mathfrak{n}$}
    \end{cases}
\end{equation}
For a leg $\mathfrak{l}$, since it is connected to just one node, we may omit the $(\mathfrak{n})$ and just write $\iota_{\mathfrak{l}}$ as in the statement of the lemma.

For each node $\mathfrak{n}$, let $\mathfrak{e}_1(\mathfrak{n})$, $\mathfrak{e}_2(\mathfrak{n})$, $\mathfrak{e}(\mathfrak{n})$ be the three edges connected to it. For each edge $\mathfrak{e}$, let $\mathfrak{n}_1(\mathfrak{e})$, $\mathfrak{n}_2(\mathfrak{e})$ be the two nodes connected to it. Then we know that $\iota_{\mathfrak{e}}(\mathfrak{n}_1(\mathfrak{e}))+\iota_{\mathfrak{e}}(\mathfrak{n}_2(\mathfrak{e}))$, since $\mathfrak{n}_1(\mathfrak{e})$ and $\mathfrak{n}_2(\mathfrak{e})$ have the opposite direction. 

Since $k_{\mathfrak{e}}$ satisfy $Eq(\mathcal{C})$, by \eqref{eq.momentumconservationunit.threewave}, we get $\iota_{\mathfrak{e}_1(\mathfrak{n})}(\mathfrak{n})k_{\mathfrak{e}_1(\mathfrak{n})}+\iota_{\mathfrak{e}_2(\mathfrak{n})}(\mathfrak{n})k_{\mathfrak{e}_2(\mathfrak{n})}+\iota_{\mathfrak{e}(\mathfrak{n})}(\mathfrak{n})k_{\mathfrak{e}(\mathfrak{n})}=0$. Summing over $\mathfrak{n}$ gives 
\begin{equation}
\begin{split}
    0=&\sum_{\mathfrak{n}\in \mathcal{C}}\iota_{\mathfrak{e}_1(\mathfrak{n})}(\mathfrak{n})k_{\mathfrak{e}_1(\mathfrak{n})}+\iota_{\mathfrak{e}_2(\mathfrak{n})}(\mathfrak{n})k_{\mathfrak{e}_2(\mathfrak{n})}+\iota_{\mathfrak{e}(\mathfrak{n})}(\mathfrak{n})k_{\mathfrak{e}(\mathfrak{n})}
    \\
    =& \sum_{\mathfrak{e}\text{ is not a leg}} 
    (\iota_{\mathfrak{e}}(\mathfrak{n}_1(\mathfrak{e}))+\iota_{\mathfrak{e}}(\mathfrak{n}_2(\mathfrak{e}))) k_{\mathfrak{e}}+ \sum_{\mathfrak{l}\text{ is a leg}} 
    \iota_{\mathfrak{l}} k_{\mathfrak{l}}
    \\
    =& \sum_{i=1}^{n_{\text{leg}}} \iota_{\mathfrak{l}_i}k_{\mathfrak{l}_i}
\end{split}
\end{equation}
This proves \eqref{eq.momentumconservation.threewave} and thus proves (1).

Now we prove (2). Since in $Eq\left(\widehat{\mathcal{C}}, \{c_{\mathfrak{l}_{i}}\}_{i\ne i_0, i_1}\cup \{-\iota_{\mathfrak{l}_{i_0}}\sum_{i\ne i_0} \iota_{\mathfrak{l}_i}k_{\mathfrak{l}_i}\}\right)$, $\{k_{\mathfrak{l}_{i}}\}_{i\ne i_0, i_1}$ are fixed to be constants $\{c_{\mathfrak{l}_{i}}\}_{i\ne i_0, i_1}$ and $k_{\mathfrak{l}_{i_0}}$ is fixed to be the constant $-\iota_{\mathfrak{l}_{i_0}}\sum_{i\ne i_0} \iota_{\mathfrak{l}_i}k_{\mathfrak{l}_i}$, by \eqref{eq.momentumconservation.threewave}, we know that 
\begin{equation}
    \iota_{\mathfrak{l}_{i_0}}\left(-\iota_{\mathfrak{l}_{i_0}}\sum_{i\ne i_0} \iota_{\mathfrak{l}_i}k_{\mathfrak{l}_i}\right)+ \iota_{\mathfrak{l}_{i_1}}k_{\mathfrak{l}_{i_1}}+\sum_{i\ne i_0, i_1} \iota_{\mathfrak{l}_i}c_{\mathfrak{l}_i}=0.
\end{equation}

This implies that $k_{\mathfrak{l}_{i_1}}=c_{\mathfrak{l}_{i_1}}$ in $Eq(\widehat{\mathcal{C}})$. Therefore, equations in $Eq(\widehat{\mathcal{C}})$ automatically imply $k_{\mathfrak{l}_{i_1}}=c_{\mathfrak{l}_{i_1}}$. Notice that whether or not containing $k_{\mathfrak{l}_{i_1}}=c_{\mathfrak{l}_{i_1}}$ is the only difference between $Eq(\mathcal{C})$ and $Eq(\widehat{\mathcal{C}})$. We conclude that $Eq(\mathcal{C})=Eq(\widehat{\mathcal{C}})$. We thus complete the proof of (2).

The proof of (3) is similar to (2). Whether or not containing $k_{\mathfrak{l}_{i_0}}=c_{\mathfrak{l}_{i_0}}$ is the only difference between $Eq(\mathcal{C})$ and $Eq(\widehat{\mathcal{C}})$. But if $\{k_{\mathfrak{l}_{i}}\}_{i\ne i_0}$ are fixed to be constants $\{c_{\mathfrak{l}_{i}}\}_{i\ne i_0}$, by momentum conservation we know that 
\begin{equation}
     k_{\mathfrak{l}_{i_0}}=-\iota_{\mathfrak{l}_{i_0}}\sum_{i\ne i_0} \iota_{\mathfrak{l}_i}c_{\mathfrak{l}_i}.
\end{equation}
Therefore, $k_{\mathfrak{l}_{i_0}}$ is fixed to be the constant $-\iota_{\mathfrak{l}_{i_0}}\sum_{i\ne i_0} \iota_{\mathfrak{l}_i}c_{\mathfrak{l}_i}$ in $Eq(\widehat{\mathcal{C}})$ and we conclude that $Eq(\mathcal{C})=Eq(\widehat{\mathcal{C}})$. We thus complete the proof of (3).

If (4) is wrong, then $\mathcal{C}$ just has one leg $\mathfrak{l}$. By \eqref{eq.momentumconservation.threewave}, $k_{\mathfrak{l}}=0$. This contradicts with $k_{\mathfrak{l},x}\ne 0$ in \eqref{eq.kappa.threewave}.
\end{proof}


% \begin{lem}\label{lem.freeleg} %Free leg moving lemma
% Given a connected couple $\mathcal{C}$ with multiple legs, 

% (1) Moving

% (2) adding

% (3) momentum conservation

% \end{lem}

\begin{lem}\label{lem.normleg} %Normal leg lemma \textbf{need connectivity}
(1) The output $\mathcal{C}(k+1)$ of step $k$ of the cutting algorithm satisfies the property P.

(2) All the intermediate results $\mathcal{C}(k)_1$, $\mathcal{C}(k)_2$, $\mathcal{C}(k)_3$ satisfy the weak property P: either they satisfy the property P or they contain exactly one free normal leg and at least one fixed leg. Notice that the weak property implies that these couples contain at least one normal leg and two legs.
\end{lem}
\begin{proof}
We prove the following stronger result by induction.

\medskip

\textit{Claim.} Assume that the couple $\mathcal{C}=\mathcal{C}(T,T,p)$ is the input the cutting algorithm. Then for any $k$, there exist a finite number of trees disjoint subtrees $T^{(k)}_1$, $T^{(k)}_2$, $\cdots$, $T^{(k)}_{m^{(k)}}$ of the two copies of $T$ which satisfy the following property.

(1) Let $\text{leaf}(k)$ be the set of all leaves of these subtrees. Assume that $p$ induce a pairing $p|_{\text{leaf}_1(k)}$ of a subset $\text{leaf}_1(k)\subseteq\text{leaf}(k)$. Apply a similar construction to Definition \ref{def.conple} we can construct a couple from $T^{(k)}_1$, $T^{(k)}_2$, $\cdots$, $T^{(k)}_{m^{(k)}}$ from $p$ and this couple equals exactly to $\mathcal{C}(k)$.

(2) The root legs of $T^{(k)}_1$, $T^{(k)}_2$, $\cdots$, $T^{(k)}_{m^{(k)}}$ are exactly the normal legs of $\mathcal{C}(k)$. The edges in $\text{leaf}_2(k)=\text{leaf}(k)\backslash \text{leaf}_1(k)$ are exactly the leaf legs (legs that are leaf edges) of $\mathcal{C}(k)$.

(3) $\mathcal{C}(k)$ satisfies the property P.

(4) All the intermediate results  $\mathcal{C}(k)_1$, $\mathcal{C}(k)_2$, $\mathcal{C}(k)_3$ satisfy the weak property P.

\medskip

We show that $\mathcal{C}(0)$ satisfies (1) -- (4) of the above claim.

Let $\mathcal{C}$ be the input couple of step $0$ obtained by pairing two copies of $T$. In step $0$, $\mathcal{C}(0)$ is obtained by replacing a fixed leg by a free leg. Therefore, if we define $T^{(0)}_{1}=T$ and $T^{(0)}_{2}$ to be the tree obtained by replacing the fixed root leg in $T$ by a free leg, then $\mathcal{C}(0)$ is the couple obtained by pairing $T^{(0)}_{1}$ and $T^{(0)}_{2}$. Here the pairing is $p$ and $\text{leaf}_1(0)=\text{leaf}(0)$ and $\text{leaf}_2(0)=\emptyset$. Therefore, (1) is true for $\mathcal{C}(0)$.

The two fixed legs in $\mathcal{C}$ are all normal edges because they come from the two root legs which belong to $T_{\text{in}}$. Therefore, the two legs in $\mathcal{C}(1)$ are also normal. Since the two legs in $\mathcal{C}(1)$ come from the root legs in $T^{(0)}_{1}$ and $T^{(0)}_{2}$, (2) is also true.

Since in step $0$, we replace a fixed leg with a free leg, $\mathcal{C}(0)$ contains exactly one free and one fixed leg which are all normal. Therefore, the output $\mathcal{C}(0)$ of step $0$ satisfies property P and (3) is proved.

In step $0$, (4) does not need any proof.

\medskip

Assume that $\mathcal{C}(k)$ satisfies (1) -- (4) of the above claim, then we prove the same for $\mathcal{C}(k+1)$.

Remember that in step $k$ of the algorithm, the input is $\mathcal{C}(k)$. The induction assumption implies that $\mathcal{C}(k)$ is obtained by pairing $T^{(k)}_1$, $T^{(k)}_2$, $\cdots$, $T^{(k)}_{m^{(k)}}$. There are several different cases in step $k$ and we treat them separately.

\underline{In case 1.1 of the cutting algorithm}, we cut $c(\mathfrak{l})$ into $\{\mathfrak{e}_{i}^{(1)}\}_{i=1,2}$ and $\{\mathfrak{e}_{i}^{(2)}\}_{i=1,2}$. By (2) $\mathfrak{l}$ is the root leg of some subtree $T^{(k)}_{j_0}$. Assume that in $T^{(k)}_{j_0}$ the two subtrees of the root are $T^{(k)}_{j_0,1}$, $T^{(k)}_{j_0,2}$. After cutting $c(\mathfrak{l})$, $T^{(k)}_{j_0}$ becomes two trees $T^{(k)}_{j_0,1}$, $T^{(k)}_{j_0,2}$ and $T^{(k)}_{j}$ $(j\ne j_0)$ do not change. We treat three different cases separately.

\underline{Case 1.1 (i).} (Both $T^{(k)}_{j_0,1}$ and $T^{(k)}_{j_0,2}$ are one node trees.) In this case all edges in $\{\mathfrak{e}_{1}^{(1)}, \mathfrak{e}_{2}^{(1)}, \mathfrak{e}_{1}^{(2)}, \mathfrak{e}_{2}^{(2)}\}$ are leaf edges of some trees in $\{T_{j}^{(k)}\}$. Define $T^{(k+1)}_{j}=T^{(k)}_{j}$ for $j< j_0$ and $T^{(k+1)}_{j}=T^{(k)}_{j+1}$ for $j> j_0$, then $\mathcal{C}(k)_1$ can be constructed from these trees with $\text{leaf}_1(k+1)=\text{leaf}_1(k)\backslash\{\mathfrak{e}_{1}^{(1)}, \mathfrak{e}_{2}^{(1)}, \mathfrak{e}_{1}^{(2)}, \mathfrak{e}_{2}^{(2)}\}$. Remember that in case 1.1 of the algorithm, the final result $\mathcal{C}(k+1)$ is obtained by replacing (or not) some free normal leg with a fixed normal leg in $\mathcal{C}(k)_1$. Therefore, if we change some free root leg of $\{T^{(k)}_{j}\}$ to be fixed, then $\mathcal{C}(k+1)$ can also be obtained from pairing these trees. Therefore, (1) is true for $\mathcal{C}(k+1)$ with $\text{leaf}_1(k+1)=\text{leaf}_1(k)\backslash\{\mathfrak{e}_{1}^{(1)}, \mathfrak{e}_{2}^{(1)}, \mathfrak{e}_{1}^{(2)}, \mathfrak{e}_{2}^{(2)}\}$.

Since both the sets of root legs and normal legs are deleted by one element $\mathfrak{l}$ and in step $k-1$ they are the same, they continue to be the same in step $k$. In case 1.1 (i), $\text{leaf}_2(k+1)=\text{leaf}_2(k)\cup \{\mathfrak{e}_{i}^{(2)}\}_{i=1,2}$ and the set of leaf legs in $\mathcal{C}(k)$ also changes in this way, so they continue to be the same in step $k$. Therefore, (2) is true.

We prove (4) by contradiction. The weak property P is equivalent to that $\mathcal{C}(k)_i$ contains exactly one free leg and at least one normal leg. In case 1.1 (i), $i=1$, assume that the weak property P is not true, then either $\mathcal{C}(k)_1$ does not contain normal leg or the number of free legs is not 1. If $\mathcal{C}(k)_1\ne \emptyset$, then the number of $\{T^{(k)}_{j}\}$ is not zero. Because the root of $\{T^{(k)}_{j}\}$ are normal legs, there exists at least one normal leg in $\mathcal{C}(k)_1$. By our hypothesis that weak property P is wrong, the number of free legs is not 1. Since $\mathcal{C}(k)_1$ is obtained from $\mathcal{C}(k)$ by cutting a fixed leg, the number of free legs in $\mathcal{C}(k)_1$ equals to  $\mathcal{C}(k)$ which is one. Therefore, we find a contradiction.

Notice that in case 1.1 of the algorithm, we change the free normal leg in $\mathcal{C}(k)_1$ to be fixed if $\mathcal{C}(k)_1$ does not satisfy the property P. The result $\mathcal{C}(k+1)$ must satisfy the property P, so we prove (3). 

\underline{Case 1.1 (ii).} (One of $T^{(k)}_{j_0,1}$ or $T^{(k)}_{j_0,2}$ is a one node tree.) Without loss of generality assume that $T^{(k)}_{j_0,2}$ is a one node tree. In this case, $\mathfrak{e}_{2}^{(1)}$, $\mathfrak{e}_{2}^{(2)}$ are leaf edges of some trees in $\{T_{j}^{(k)}\}$. Define $T^{(k+1)}_{j}=T^{(k)}_{j}$ for $j\ne j_0$ and $T^{(k+1)}_{j_0}=T^{(k)}_{j_0,1}$, then $\mathcal{C}(k)_1$ can be constructed from these trees with  $\text{leaf}_1(k+1)=\text{leaf}_1(k)\backslash\{\mathfrak{e}_{2}^{(1)}, \mathfrak{e}_{2}^{(2)}\}$. By the same reason as in case 1.1 (i), (1) is true for $\mathcal{C}(k+1)$ with $\text{leaf}_1(k+1)=\text{leaf}_1(k)\backslash\{\mathfrak{e}_{2}^{(1)}, \mathfrak{e}_{2}^{(2)}\}$.

By the same reason as in case 1.1 (i), the set of root legs and the set of normal legs continue to be the same in step $k$. In case 1.1 (ii), $\text{leaf}_2(k+1)=\text{leaf}_2(k)\cup \{ \mathfrak{e}_{2}^{(2)}\}$ and the set of leaf legs also changes in this way, so they continue to be the same in step $k$. Therefore, (2) is true.

The proof of (3), and (4) in case 1.1 (ii) is the same as that in case 1.1 (i).


\underline{Case 1.1 (iii).} (Neither $T^{(k)}_{j_0,1}$ or $T^{(k)}_{j_0,2}$ is an one node tree.) In this case, no edge in $\{\mathfrak{e}_{1}^{(1)}, \mathfrak{e}_{2}^{(1)}, \mathfrak{e}_{1}^{(2)}, \mathfrak{e}_{2}^{(2)}\}$ is leaf edge of $T_{j}^{(k)}$. Define $T^{(k+1)}_{j}=T^{(k)}_{j}$ for $j\le j_0$, $T^{(k+1)}_{j}=T^{(k)}_{j-1}$ for $j\ge j_0+2$, $T^{(k+1)}_{j_0}=T^{(k)}_{j_0,1}$ and $T^{(k+1)}_{j_0+1}=T^{(k)}_{j_0,2}$, then $\mathcal{C}(k)_1$ can be constructed from these trees with  $\text{leaf}_1(k+1)=\text{leaf}_1(k)$. By the same reason as in case 1.1 (i), (1) is true for $\mathcal{C}(k+1)$ with $\text{leaf}_1(k+1)=\text{leaf}_1(k)$.

By the same reason as in case 1.1 (i), the set of root legs and the set of normal legs continue to be the same in step $k$. In case 1.1 (iii), $\text{leaf}_2(k+1)=\text{leaf}_2(k)$ and the set of leaf legs also changes in this way, so they continue to be the same in step $k$. Therefore, (2) is true.

The proof of (3), and (4) in case 1.1 (iii) is the same as that in case 1.1 (i).

\underline{In case 1.2 of the cutting algorithm}, we cut $\mathfrak{e}$ into $\{\mathfrak{e}^{(1)},\mathfrak{e}^{(2)}\}$. By (2), $\mathfrak{l}$ and $\mathfrak{l}_{fr}$ are the root leg and leaf of some subtrees $T^{(k)}_{j_0}$. Assume that in $T^{(k)}_{j_0}$ one subtree of the root is $T^{(k)}_{j_0,1}$ and the other one is a one node tree with edge $\mathfrak{l}_{fr}$. After cutting $c(\mathfrak{l})$, $T^{(k)}_{j_0}$ becomes $T^{(k)}_{j_0,1}$ and $T^{(k)}_{j}$ $(j\ne j_0)$ do not change. 
We treat two different cases separately.

\underline{Case 1.2 (i).} ($T^{(k)}_{j_0,1}$ is an one node trees.) In this case $\mathfrak{e}^{(1)}$, $\mathfrak{e}^{(2)}$ are leaf edges of some trees in $\{T_{j}^{(k)}\}$. Define $T^{(k+1)}_{j}=T^{(k)}_{j}$ for $j< j_0$ and $T^{(k+1)}_{j}=T^{(k)}_{j+1}$ for $j> j_0$, then $\mathcal{C}(k)_1$ can be constructed from these trees with  $\text{leaf}_1(k+1)=\text{leaf}_1(k)\backslash\{\mathfrak{e}^{(1)}, \mathfrak{e}^{(2)}\}$. By the same reason as in case 1.1 (i), (1) is true for $\mathcal{C}(k+1)$ with $\text{leaf}_1(k+1)=\text{leaf}_1(k)\backslash\{\mathfrak{e}^{(1)}, \mathfrak{e}^{(2)}\}$.

By the same reason as in case 1.1 (i), the set of root legs and the set of normal legs continue to be the same in step $k$. In case 1.2 (i), $\text{leaf}_2(k+1)=\text{leaf}_2(k)\cup \{ \mathfrak{e}^{(2)}\}$ and the set of leaf legs also changes in this way, so they continue to be the same in step $k$. Therefore, (2) is true.

The proof of (3), and (4) in case 1.2 (i) is the same as that in case 1.1 (i).


\underline{Case 1.2 (ii).} ($T^{(k)}_{j_0,1}$ is not a one node trees.) In this case $\mathfrak{e}^{(1)}$, $\mathfrak{e}^{(2)}$ are not leaf edges of any trees in $\{T_{j}^{(k)}\}$. Define $T^{(k+1)}_{j}=T^{(k)}_{j}$ for $j\ne j_0$ and $T^{(k+1)}_{j_0}=T^{(k)}_{j_0,1}$, then $\mathcal{C}(k)_1$ can be constructed from these trees with  $\text{leaf}_1(k+1)=\text{leaf}_1(k)$. By the same reason as in case 1.1 (i), (1) is true for $\mathcal{C}(k+1)$ with $\text{leaf}_1(k+1)=\text{leaf}_1(k)$.

By the same reason as in case 1.1 (i), the set of root legs and the set of normal legs continue to be the same in step $k$. In case 1.2 (ii), $\text{leaf}_2(k+1)=\text{leaf}_2(k)$ and the set of leaf legs also changes in this way, so they continue to be the same in step $k$. Therefore, (2) is true.

The proof of (3), and (4) in case 1.2 (ii) is the same as that in case 1.1 (i).

\underline{In case 2 of the cutting algorithm}, we cut $c(\mathfrak{l})=\{\mathfrak{e}_2,\mathfrak{e}_3\}$ into $\{\mathfrak{e}_{2}^{(1)}, \mathfrak{e}_{3}^{(1)}\}$ and $\{\mathfrak{e}_{2}^{(2)}, \mathfrak{e}_{3}^{(2)}\}$. By (2), $\mathfrak{l}$ is the root leg of some subtree $T^{(k)}_{j_0}$. Assume that in $T^{(k)}_{j_0}$ the two subtrees of the root are $T^{(k)}_{j_0,1}$, $T^{(k)}_{j_0,2}$. After cutting $c(\mathfrak{l})$, $T^{(k)}_{j_0}$ becomes two trees $T^{(k)}_{j_0,1}$, $T^{(k)}_{j_0,2}$ and $T^{(k)}_{j}$ $(j\ne j_0)$ do not change. Since $\mathcal{C}(k)_{2}$ and $\mathcal{C}(k)_{3}$ are disjoint, for each $j\ne j_0$, $T^{(k)}_{j}$ should be a subset of one of these couples. Without loss of generality we assume that $j_0=1$ and $T^{(k)}_2, \cdots, T^{(k)}_{m^{(k)'}}\subseteq \mathcal{C}(k)_{2}$ and $T^{(k)}_{m^{(k)'}+1}, \cdots, T^{(k)}_{m^{(k)}}\subseteq \mathcal{C}(k)_{3}$. Since the output, $\mathcal{C}(k+1)$ either comes from $\mathcal{C}(k)_{2}$ or $\mathcal{C}(k)_{3}$, without loss of generality we assume that the output comes from $\mathcal{C}(k)_{2}$. We treat two different cases separately.

\underline{Case 2 (i).} ($T^{(k)}_{j_0,1}$ is a one node tree.) In this case, all edges in $\{\mathfrak{e}_{2}^{(1)}, \mathfrak{e}_{2}^{(2)}\}$ are leaf edges of some trees in $\{T_{j}^{(k)}\}$. Define $T^{(k+1)}_{j}=T^{(k)}_{j+1}$, then $\mathcal{C}(k)_2$ can be constructed from $T^{(k)}_2, \cdots, T^{(k)}_{m^{(k)'}}$ with $\text{leaf}_1(k+1)=\{\text{all leaves in }T^{(k)}_2, \cdots,$ $T^{(k)}_{m^{(k)'}}\}$. Remember that in case 2 of the algorithm, the final result $\mathcal{C}(k+1)$ is obtained by replacing (or not) some free normal leg wtih a fixed normal leg in $\mathcal{C}(k)_2$, so by the same argument as in case 1.1 (i), (1) is true for $\mathcal{C}(k+1)$.

By the same reason as in case 1.1 (i), the set of root legs and the set of normal legs continue to be the same in step $k$. In case 2 (i), in the output $\mathcal{C}(k+1)$ coming from $\mathcal{C}(k)_2$, $\text{leaf}_2(k+1)=(\text{leaf}_1(k+1)\cap \mathcal{C}(k)_2)\cup \{ \mathfrak{e}_{2}^{(2)}\}$ and the set of leaf legs also changes in this way, so they continue to be the same in step $k$. Therefore, (2) is true.

The proof of (3), and (4) in case 2 (i) is the same as that in case 1.1 (i).

\underline{Case 2 (ii).} ($T^{(k)}_{j_0,1}$ is not a one node tree.) In this case $\{\mathfrak{e}_{2}^{(1)}$, $\mathfrak{e}_{2}^{(2)}\}$ are not leaf edges of any trees in $\{T_{j}^{(k)}\}$. Define $T^{(k+1)}_{j}=T^{(k)}_{j}$ for $j>1$ and $T^{(k+1)}_{1}=T^{(k)}_{j_0,1}$, then $\mathcal{C}(k)_2$ can be constructed from $T^{(k+1)}_1, \cdots, T^{(k+1)}_{m^{(k)'}}$ with $\text{leaf}_1(k+1)=\{\text{all leaves in }T^{(k+1)}_1, \cdots, T^{(k+1)}_{m^{(k)'}}\}$. For the same reason as in case 1.1 (i), (1) is true for $\mathcal{C}(k+1)$.

For the same reason as in case 1.1 (i), the set of root legs and the set of normal legs continue to be the same in step $k$. In case 2 (ii), in the output $\mathcal{C}(k+1)$ coming from $\mathcal{C}(k)_2$, $\text{leaf}_2(k+1)=(\text{leaf}_1(k+1)\cap \mathcal{C}(k)_2)$ and the set of leaf legs also changes in this way, so they continue to be the same in step $k$. Therefore, (2) is true.

The proof of (3), and (4) in case 2 (i) is the same as that in case 1.1 (i).
\end{proof}



\textbf{Step 3.} In this step, we state Proposition \ref{prop.countingind} which is a stronger version of Proposition \ref{prop.counting} and derive Proposition \ref{prop.counting} from it. In the end, we prove parts (1), (2) and the 1 node case of part (3) of Proposition \ref{prop.countingind}.


\begin{prop}\label{prop.countingind}
Let $\mathcal{C}(k)$ be a couple which is the output of step $k$ of the cutting algorithm Proposition \ref{prop.cuttingalgorithm}. For any couple $\mathcal{C}$, let $n(\mathcal{C})$ be the total number of nodes in $\mathcal{C}$ and $n_e(\mathcal{C})$ (resp. $n_{fx}(\mathcal{C})$, $n_{\textit{fr}}(\mathcal{C})$) be the total number of non-leg edges (resp. fixed legs, free legs). We fix $\sigma_{\mathfrak{n}}\in\mathbb{R}$ for each $\mathfrak{n}\in \mathcal{C}(k)$ and $c_{\mathfrak{l}}\in \mathbb{R}$ for each fixed leg $\mathfrak{l}$. Assume that $\alpha$ satisfies \eqref{eq.conditionalpha.threewave}. Then we have

(1) We have following relation of $n(\mathcal{C})$, $n_e(\mathcal{C})$, $n_{fx}(\mathcal{C})$ and $n_{\textit{fr}}(\mathcal{C})$
\begin{equation}
    2n_e(\mathcal{C})+n_{fx}(\mathcal{C})+n_{\textit{fr}}(\mathcal{C})=3n(\mathcal{C})
\end{equation}

(2) For any couple $\mathcal{C}$, define $\chi(\mathcal{C})=n_e(\mathcal{C})+n_{\textit{fr}}(\mathcal{C})-n(\mathcal{C})$. Let $c$, $\mathcal{C}_1$, $\mathcal{C}_2$ be the same as in Lemma \ref{lem.Eq(C)cutting} and also assume that $\{\mathfrak{e}_{i}^{(1)}\}$ are free legs and $\{\mathfrak{e}_{i}^{(2)}\}$ are fixed legs. Then 
\begin{equation}
    \chi(\mathcal{C})=\chi(\mathcal{C}_1)+\chi(\mathcal{C}_2).
\end{equation}

(3) If we assume that $\mathcal{C}(k)$ satisfies the property P, then (recall that $Q=L^{d}T^{-1}_{\text{max}}$)
\begin{equation}\label{eq.countingbd3.threewave}
\sup_{\{c_{\mathfrak{l}}\}_{\mathfrak{l}}}\#Eq(\mathcal{C}(k),\{c_{\mathfrak{l}}\}_{\mathfrak{l}})\leq L^{O\left(\chi(\mathcal{C}(k))\theta\right)} Q^{\chi(\mathcal{C}(k))}\prod_{\mathfrak{e}\in \mathcal{C}_{\text{norm}}(k)} \kappa^{-1}_{\mathfrak{e}} .
\end{equation}

\end{prop}

\underline{Derivation of Proposition \ref{prop.counting} from Proposition \ref{prop.countingind}:} 
By Proposition \ref{prop.cuttingalgorithm} (2), $\mathcal{C}_{\text{norm}}(1)=\mathcal{C}_{\text{norm}}$ and $\# Eq(\mathcal{C})=\# Eq(\mathcal{C}(1))$. Then by Proposition \ref{prop.countingind} (3), 
\begin{equation}\label{eq.countinglemstep3.threewave}
    \# Eq(\mathcal{C})=\# Eq(\mathcal{C}(1))\leq L^{O(\chi(\mathcal{C}(1))\theta)} Q^{\chi(\mathcal{C}(1))}\prod_{\mathfrak{e}\in \mathcal{C}_{\text{norm}}(1)} \kappa^{-1}_{\mathfrak{e}}
    %=L^{O(\theta)} Q^{\chi(\mathcal{C})}\prod_{\mathfrak{e}\in \mathcal{C}_{\text{norm}}} \kappa^{-1}_{\mathfrak{e}}
\end{equation}
Since in $\mathcal{C}(1)$, $n_{fx}(\mathcal{C}(1))=n_{\textit{fr}}(\mathcal{C}(1))=1$, by Proposition \ref{prop.countingind} (1), $2n_e(\mathcal{C}(1))+2=3n(\mathcal{C}(1))$. Using this fact and the definition of $\chi$, we get $\chi(\mathcal{C}(1))=n_e(\mathcal{C}(1))+1-n(\mathcal{C}(1))=n(\mathcal{C}(1))/2$. Substituting this expression of $\chi(\mathcal{C}(1))$ into \eqref{eq.countinglemstep3.threewave} proves the conclusion of Proposition \ref{prop.counting}. 

The proof of Proposition \ref{prop.countingind} is the main goal of the rest of the proof.

%In the cutting algorithm, we keep cutting nodes $\mathfrak{n}$ that is connected to normal fixed legs $\mathfrak{l}$ using cut $c(\mathfrak{l})$. Suppose that $\mathcal{C}$ has $n$ nodes, let $\mathcal{C}_{\mathfrak{l}}$ and $\mathcal{C}'$ be the two components after cutting (as in (2) of the cutting algorithm), then  $\mathcal{C}'$ contains $n-1$ nodes and $\mathcal{C}_{\mathfrak{l}}$ contains $1$ nodes. We prove Proposition \ref{prop.countingind} for $\mathcal{C}_{\mathfrak{l}}$, i.e. prove the 1 node case of it.

\underline{Proof Proposition \ref{prop.countingind} (1):} Consider the set $\mathcal{S}=\{(\mathfrak{n}, \mathfrak{e})\in \mathcal{C}: \mathfrak{n} \textit{ is an end point of }\mathfrak{e}\}$, then 
\begin{equation}
\#\mathcal{S}=\sum_{\substack{(\mathfrak{n}, \mathfrak{e})\in \mathcal{C}\\ \mathfrak{n} \textit{ is an end point of }\mathfrak{e}}} 1
\end{equation}

First sum over $\mathfrak{e}$ and then over $\mathfrak{n}$, we get 
\begin{equation}
\#\mathcal{S}=\sum_{\mathfrak{n}}\sum_{\substack{\mathfrak{e}\in \mathcal{C}\\ \mathfrak{n} \textit{ is an end point of }\mathfrak{e}}} 1=\sum_{\mathfrak{n}}\ 3=3n(\mathcal{C}).
\end{equation}
In the second equality, $\sum_{\mathfrak{e}\in \mathcal{C}: \mathfrak{n} \textit{ is an end point of }\mathfrak{e}} 1 =3$ because for each node $\mathfrak{n}$ there are $3$ edges connected to it.

Switch the order of summation, we get 
\begin{equation}
\begin{split}
\#\mathcal{S}=&\sum_{\mathfrak{e} \textit{ is a non-leg edges}}\sum_{\substack{\mathfrak{n}\in \mathcal{C}\\ \mathfrak{n} \textit{ is an endpoint of }\mathfrak{e}}} 1+\sum_{\mathfrak{e} \textit{ is a leg}}\sum_{\substack{\mathfrak{n}\in \mathcal{C}\\ \mathfrak{n} \textit{ is an endpoint of }\mathfrak{e}}} 1
\\
=&\sum_{\mathfrak{e} \textit{ is a non-leg edges}} 2+\sum_{\mathfrak{e} \textit{ is a leg}} 1
\\
=& 2n_e(\mathcal{C})+n_{fx}(\mathcal{C})+n_{\textit{fr}}(\mathcal{C})
\end{split}
\end{equation}
In the second equality, $\sum_{\substack{\mathfrak{n}\in \mathcal{C}\\ \mathfrak{n} \textit{ is an endpoint of }\mathfrak{e}}} 1$ equals to $1$ or $2$ because for each non-leg edge (resp. leg) there are $2$ (resp. 1) nodes connected to it.

Because the value of $\#\mathcal{S}$ does not depend on the order of summation, we conclude that $2n_e(\mathcal{C})+n_{fx}(\mathcal{C})+n_{\textit{fr}}(\mathcal{C})=3n(\mathcal{C})$, which proves Proposition \ref{prop.countingind} (1).


\underline{Proof Proposition \ref{prop.countingind} (2):} Since cutting does change the number of nodes, we have $n(\mathcal{C})=n(\mathcal{C}_1)+n(\mathcal{C}_2)$. Let $c$ be the cut that consists of edges $\{\mathfrak{e}_{i}\}$ and $n(c)$ be the number of edges in $c$. 
%Given a couple $\mathcal{C}$, let $n(\mathcal{C})$, $n_e(\mathcal{C})$, $n_{\textit{fx}}(\mathcal{C})$ and $n_{\textit{fr}}(\mathcal{C})$ be the total number of the number of nodes, non-leg edges, fixed legs and free legs respectively. 
When cutting $\mathcal{C}$ into $\mathcal{C}_1$ and $\mathcal{C}_2$, $n(c)$ non-leg edges are cut into pairs of free and fixed legs, so $n_e(\mathcal{C})=n(\mathcal{C}_1)+n(\mathcal{C}_2)+n(c)$. Because we have $n(c)$ additional free legs after cutting, so $n_{\textit{fr}}(\mathcal{C})=n_{\textit{fr}}(\mathcal{C}_1)+n_{\textit{fr}}(\mathcal{C}_2)-n(c)$. Therefore, we get
\begin{equation}
\begin{split}
    \chi(\mathcal{C})=&n_e(\mathcal{C})+n_{\textit{fr}}(\mathcal{C})-n(\mathcal{C})
    \\
    =&n(\mathcal{C}_1)+n(\mathcal{C}_2)+n(c)+n_{\textit{fr}}(\mathcal{C}_1)+n_{\textit{fr}}(\mathcal{C}_2)-n(c)-(n(\mathcal{C}_1)+n(\mathcal{C}_2))
    \\
    =&(n(\mathcal{C}_1)+n_{\textit{fr}}(\mathcal{C}_1)-n(\mathcal{C}_1))+(n(\mathcal{C}_2)+n_{\textit{fr}}(\mathcal{C}_2)-n(\mathcal{C}_2))
    \\
    =&\chi(\mathcal{C}_1)+\chi(\mathcal{C}_2).
\end{split}
\end{equation}

We complete the proof of (2) of Proposition \ref{prop.countingind}.

\underline{Proof of 1 node case of Proposition \ref{prop.countingind} (3):} We prove the following Lemma \ref{lem.countingbdunit} which is the 1 node case of Proposition \ref{prop.countingind} (3). Recall that $\mathcal{C}_{I}$ and $\mathcal{C}_{II}$ are the two possibilities of the 1 node couple $\mathcal{C}_{\mathfrak{l}}$ in Proposition \ref{prop.cuttingalgorithm} (3).

\begin{lem}\label{lem.countingbdunit}
$\mathcal{C}_{I}$, $\mathcal{C}_{II}$ satisfy the bound \eqref{eq.countingbd3.threewave} in Proposition \ref{prop.countingind}. In other words, fix $c_1$ (resp. $c_1$, $c_2$) for the fixed legs of $\mathcal{C}_{I}$ (resp. $\mathcal{C}_{II}$), then we have 
\begin{equation}\label{eq.countingbdunit.threewave}
    \# Eq(\mathcal{C}_{I})\leq L^\theta Q\kappa^{-1}_{\mathfrak{e}},\qquad \# Eq(\mathcal{C}_{II})\leq Q^0=1.
\end{equation}
\end{lem}
\begin{proof} Given $c_1$, $c_2$, the equation of $\mathcal{C}_{II}$ is 
\begin{equation}
    \begin{cases}
    k_{\mathfrak{e}_1}+k_{\mathfrak{e}_2}-k_{\mathfrak{e}}=0,\ k_{\mathfrak{e}_1}=c_1,\ k_{\mathfrak{e}}=c_2,\ |k_{\mathfrak{e}x}|\sim \kappa_{\mathfrak{e}}
    \\
    \Lambda_{k_{\mathfrak{e}_1}}+\Lambda_{k_{\mathfrak{e}_2}}-\Lambda_{k_{\mathfrak{e}}}=\sigma_{\mathfrak{n}}+O(T^{-1}_{\text{max}})
    \end{cases}
\end{equation}
It's obvious that there is at most one solution to this system of equations.

Given $c_1$, the equation of $\mathcal{C}_{I}$ is 
\begin{equation}
    \begin{cases}
    k_{\mathfrak{e}_1}+k_{\mathfrak{e}_2}-k_{\mathfrak{e}}=0,\ k_{\mathfrak{e}}=c_1,\ |k_{\mathfrak{e}x}|\sim \kappa_{\mathfrak{e}}
    \\
    \Lambda_{k_{\mathfrak{e}_1}}+\Lambda_{k_{\mathfrak{e}_2}}-\Lambda_{k_{\mathfrak{e}}}=\sigma_{\mathfrak{n}}+O(T^{-1}_{\text{max}})
    \end{cases}
\end{equation}
By Theorem \ref{th.numbertheory1.threewave}, the number of solutions of the above system of equations can be bounded by $L^\theta L^dT^{-1}_{\text{max}}|k_{\mathfrak{e}x}|^{-1}\lesssim L^\theta Q\kappa^{-1}_{\mathfrak{e}}$

Therefore, we complete the proof of this lemma.
\end{proof}



\textbf{Step 4.} In this step, we apply the edge cutting algorithm Proposition \ref{prop.cuttingalgorithm} to prove Proposition \ref{prop.countingind} (3) by induction.

If $\mathcal{C}$ has only one node ($n=1$), then $\mathcal{C}$ equals to $\mathcal{C}_{I}$ or $\mathcal{C}_{II}$ and Proposition \ref{prop.countingind} (3) in this case follows from Lemma \ref{lem.countingbdunit}.

Suppose that Proposition \ref{prop.countingind} (3) holds true for couples with the number of nodes $\le n-1$. We prove it for couples with the number of nodes $n$. 

Given the couple $\mathcal{C}(k)$ which is the output of the $k-1$-th step, apply the cutting algorithm to $\mathcal{C}(k)$. Then according to Proposition \ref{prop.cuttingalgorithm}, $\mathcal{C}(k)$ (2) is decomposed into 2 or 3 components.

In the first case, denote by $\mathcal{C}(k)_{
\mathfrak{l}}$ and $\mathcal{C}(k)_1$ the two components after cutting. In the second case, denote by $\mathcal{C}(k)_{
\mathfrak{l}}$, $\mathcal{C}(k)_2$ and $\mathcal{C}(k)_3$ the three components after cutting. By Proposition \ref{prop.cuttingalgorithm} (4), $\mathcal{C}(k)_1$, $\mathcal{C}(k)_2$, $\mathcal{C}(k)_3$ are couples of nodes $\le n-1$ that satisfy the property P and thus satisfy the assumption of Proposition \ref{prop.countingind} (3). Therefore, the induction assumption is applicable and Proposition \ref{prop.countingind} (3) is true for these three couples.

\textbf{Case 1.} Assume that there are two components after cutting. One example of this case is the left couple in Figure \ref{fig.step1case}.
    
Then applying Lemma \ref{lem.Eq(C)cutting} gives

\begin{equation}
\begin{split}
    \sup_{\{c_{\mathfrak{l}}\}_{\mathfrak{l}}}\#Eq(\mathcal{C}(k),\{c_{\mathfrak{l}}\}_{\mathfrak{l}})\le&
    \sup_{\{c_{\mathfrak{l}_1}\}_{\mathfrak{l}_1\in \text{leg}(\mathcal{C}(k)_{
\mathfrak{l}})} } \# Eq(\mathcal{C}(k)_{
\mathfrak{l}},\{c_{\mathfrak{l}_1}\}) \sup_{\{c_{\mathfrak{l}_2}\}_{\mathfrak{l}_2\in \text{leg}(\mathcal{C}(k)_1)} }\# Eq(\mathcal{C}(k)_1, \{c_{\mathfrak{l}_2}\})
    \\
    \lesssim&  L^{\theta} Q^{\chi(\mathcal{C}_1)} L^{O(\chi(\mathcal{C}_2)\theta)} Q^{\chi(\mathcal{C}_2)} = L^{O((\chi(\mathcal{C}_1)+\chi(\mathcal{C}_2))\theta)} Q^{\chi(\mathcal{C}_1)+\chi(\mathcal{C}_2)} \\
    =& L^{O(\chi(\mathcal{C})\theta)} Q^{\chi(\mathcal{C})}.
\end{split}
\end{equation}

Here the second inequality follows from the induction assumption and Lemma \ref{lem.countingbdunit}. The last equality follows from Proposition \ref{prop.countingind} (2).


\textbf{Case 2.} Assume that there are three components after cutting. One example of this case is the right couple in Figure \ref{fig.step1case}.

Then applying Lemma \ref{lem.Eq(C)cutting} gives

\begin{equation}\label{eq.case2expand.threewave}
\begin{split}
    \sup_{\{c_{\mathfrak{l}}\}_{\mathfrak{l}}}\#Eq(\mathcal{C}(k),\{c_{\mathfrak{l}}\}_{\mathfrak{l}})\le&
    \sup_{\{c_{\mathfrak{l}_1}\}_{\mathfrak{l}_1\in \text{leg}(\mathcal{C}(k)\backslash\mathcal{C}(k)_2)} } \# Eq(\mathcal{C}(k)\backslash\mathcal{C}(k)_2,\{c_{\mathfrak{l}_1}\}) \sup_{\{c_{\mathfrak{l}_2}\}_{\mathfrak{l}_2\in \text{leg}(\mathcal{C}(k)_2)} }\# Eq(\mathcal{C}(k)_2, \{c_{\mathfrak{l}_2}\})
    \\
    \lesssim&  L^{O(\chi(\mathcal{C}(k)_2)\theta)} Q^{\chi(\mathcal{C}(k)_2)}\sup_{\{c_{\mathfrak{l}_1}\}_{\mathfrak{l}_1\in \text{leg}(\mathcal{C}(k)\backslash\mathcal{C}(k)_2)} } \# Eq(\mathcal{C}(k)\backslash\mathcal{C}(k)_2,\{c_{\mathfrak{l}_1}\}).
\end{split}
\end{equation}
Here in the second inequality, we apply the induction assumption to $\mathcal{C}(k)_2$.

Applying Lemma \ref{lem.Eq(C)cutting} and \eqref{eq.case2expand.threewave} gives
\begin{equation}\label{eq.case2expand'.threewave}
\begin{split}
    &\sup_{\{c_{\mathfrak{l}}\}_{\mathfrak{l}}}\#Eq(\mathcal{C}(k),\{c_{\mathfrak{l}}\}_{\mathfrak{l}})
    \lesssim  L^{O(\mathcal{C}(k)_2\theta)} Q^{\chi(\mathcal{C}(k)_2)}\sup_{\{c_{\mathfrak{l}_1}\}_{\mathfrak{l}_1\in \text{leg}(\mathcal{C}(k)\backslash\mathcal{C}(k)_2)} } \# Eq(\mathcal{C}(k)\backslash\mathcal{C}(k)_2,\{c_{\mathfrak{l}_1}\})
    \\
    \lesssim& L^{O(\chi(\mathcal{C}(k)_2)\theta)} Q^{\chi(\mathcal{C}(k)_2)}\sup_{\{c_{\mathfrak{l}_1}\}_{\mathfrak{l}_1\in \text{leg}(\mathcal{C}(k)_{\mathfrak{l}})} } \# Eq(\mathcal{C}(k)_{\mathfrak{l}},\{c_{\mathfrak{l}_1}\}) \sup_{\{c_{\mathfrak{l}_3}\}_{\mathfrak{l}_3\in \text{leg}(\mathcal{C}(k)_{3})} }\# Eq(\mathcal{C}(k)_{3}, \{c_{\mathfrak{l}_3}\})
    \\
    \lesssim& L^{O((\chi(\mathcal{C}(k)_2)+\chi(\mathcal{C}(k)_3))\theta)} Q^{\chi(\mathcal{C}(k)_2)+\chi(\mathcal{C}(k)_3)}\sup_{\{c_{\mathfrak{l}_1}\}_{\mathfrak{l}_1\in \text{leg}(\mathcal{C}(k)_{\mathfrak{l}})} } \# Eq(\mathcal{C}(k)_{\mathfrak{l}},\{c_{\mathfrak{l}_1}\}) 
    \\
    \lesssim& (L^{O(\theta)} Q)^{\chi(\mathcal{C}(k)_2)+\chi(\mathcal{C}(k)_3)+\chi(\mathcal{C}(k)_{\mathfrak{l}})}=L^{O(\chi(\mathcal{C})\theta)} Q^{\chi(\mathcal{C})}.
\end{split}
\end{equation}
Here in the third inequality, we apply the induction assumption to $\mathcal{C}(k)_3$. The last equality follows from Proposition \ref{prop.countingind} (2).

Therefore, we complete the proof of Proposition \ref{prop.countingind}
and thus the proof of Proposition \ref{prop.counting}.
\end{proof}






























\subsection{An upper bound of tree terms}\label{sec.treetermsupperbound} In this section, we first prove the following Proposition which gives an upper bound of the variance of $\mathcal{J}_{T,k}$ and then prove Proposition \ref{prop.treetermsupperbound} as its corollary.

\begin{prop}\label{prop.treetermsvariance}
Assume that $\alpha$ satisfies \eqref{eq.conditionalpha.threewave} and $\rho=\alpha\, T^{\frac{1}{2}}_{\text{max}}$. For any $\theta>0$, we have
\begin{equation}
    \sup_k\, \mathbb{E}|(\mathcal{J}_T)_k|^2\lesssim L^{O(l(T)\theta)} \rho^{2l(T)}.
\end{equation}
and $\mathbb{E}|(\mathcal{J}_T)_k|^2=0$ if $|k|\gtrsim 1$.
\end{prop}
\begin{proof}
By \eqref{eq.termTp.threewave} and \eqref{eq.termexp.threewave}, we know that $\mathcal{J}_{T,k}$ is a linear combination of $Term(T,p)_k$.
\begin{equation}
\begin{split}
    \mathbb{E}|\mathcal{J}_{T,k}|^2=\left(\frac{\lambda}{L^{d}}\right)^{2l(T)}
    \sum_{p\in \mathcal{P}(\{k_1,\cdots, k_{l(T)+1}, k'_1,\cdots, k'_{l(T)+1}\})} Term(T, p)_k.
\end{split}
\end{equation}

Since $\alpha=\frac{\lambda}{L^{\frac{d}{2}}}$, $\frac{\lambda}{L^{d}}=\alpha L^{-\frac{d}{2}}$. Since the number of elements in $\mathcal{P}$ can be bounded by a constant, by Lemma \ref{lem.Tpvariance} proved below, we get
\begin{equation}\label{eq.proptreetermsvariance1.threewave}
\begin{split}
    \mathbb{E}|\mathcal{J}_{T,k}|^2\lesssim (\alpha L^{-\frac{d}{2}})^{2l(T)}
    L^{O(n\theta)} Q^{\frac{n}{2}}  T^{n}_{\text{max}}  .
\end{split}
\end{equation}

By definition, $n$ is the total number of $\bullet$ nodes in the couple constructed from tree $T$ and pairing $p$. Therefore, $n$ equals to $2l(T)$.  Replacing $n$ by $2l(T)$ and $Q$ by $L^dT^{-1}_{\text{max}}$ in \eqref{eq.proptreetermsvariance1.threewave}, we get
\begin{equation}
\begin{split}
    \mathbb{E}|\mathcal{J}_{T,k}|^2\lesssim& (\alpha L^{-\frac{d}{2}})^{2l(T)}
    L^{O(l(T)\theta)} (L^dT^{-1}_{\text{max}})^{l(T)}  T_{\text{max}}^{2l(T)}  
    \\
    =& L^{O(l(T)\theta)} (\alpha^2 T_{\text{max}})^{l(T)}  
    \\
    =&  L^{O(l(T)\theta)} \rho^{2l(T)}.  
\end{split}
\end{equation}

By Lemma \ref{lem.Tpvariance} below $Term(T,p)_k=0$ if $|k|\gtrsim 1$, we know that the same is true for $\mathbb{E}|(\mathcal{J}_T)_k|^2=0$. 

Therefore, we complete the proof of this proposition.
\end{proof}


\begin{lem}\label{lem.Tpvariance} Let $n$ and $Q=L^dT^{-1}_{\text{max}}$ be the same as in Proposition \ref{prop.counting}. Assume that $\alpha$ satisfies \eqref{eq.conditionalpha.threewave} and $n_{\mathrm{in}} \in C^\infty_0(\mathbb{R}^d)$ is compactly supported. Let $\mathcal{C}$ be the couple constructed from tree $T$ and pairing $p$. Then for any $\theta>0$, we have
\begin{equation}
    \sup_k\, |Term(T,p)_k|\le L^{O(n\theta)} Q^{\frac{n}{2}} T_{\text{max}}^{n}.
\end{equation}
and $Term(T,p)_k=0$ if $|k|\gtrsim 1$.
\end{lem}
\begin{proof} By \eqref{eq.termTp.threewave}, we get

\begin{equation}
\begin{split}
    Term(T, p)_k=&\sum_{k_1,\, k_2,\, \cdots,\, k_{l(T)+1}}\sum_{k'_1,\, k'_2,\, \cdots,\, k'_{l(T)+1}} H^T_{k_1\cdots k_{l(T)+1}} H^{T}_{k'_1\cdots k'_{l(T)+1}}
    \\
    & \delta_{p}(k_1,\cdots, k_{l(T)+1}, k'_1,\cdots, k'_{l(T)+1})\sqrt{n_{\textrm{in}}(k_1)}\cdots\sqrt{n_{\textrm{in}}(k'_1)}\cdots
\end{split}
\end{equation}

Since $n_{\mathrm{in}}$ are compactly supported and there are bounded many of them in $Term(T, p)_k$, by $k_1 + k_2 + \cdots + k_{l(T)+1}=k$, we know that $Term(T, p)_k=0$ if $|k|\gtrsim 1$.

By \eqref{eq.boundcoef.threewave}, we get  %$\frac{1}{\sqrt{(|c(\Omega)_{\mathfrak{n}}|+T^{-2}_{\text{max}})^2+|r_{\mathfrak{n}}|^2}}\lesssim \frac{1}{|c(\Omega)_{\mathfrak{n}}|+T^{-2}_{\text{max}}}$
\begin{equation}\label{eq.termlemmaeq1.threewave}
    |H^T_{k_1\cdots k_{l+1}}|\lesssim \sum_{\{d_{\mathfrak{n}}\}_{\mathfrak{n}\in T_{\text{in}}}\in\{0,1\}^{l(T)}}\prod_{\mathfrak{n}\in T_{\text{in}}}\frac{1}{|q_{\mathfrak{n}}|+T^{-1}_{\text{max}}}\ \prod_{\mathfrak{e}\in T_{\text{in}}}|k_{\mathfrak{e},x}|\ \delta_{\cap_{\mathfrak{n}\in T_{\text{in}}} \{S_{\mathfrak{n}}=0\}}.
\end{equation}



% If $k_1,\cdots,k_{l(T)+1}$ are bounded, then the above argument proves the boundedness of $|k_{\mathfrak{e},x}|$, so by \eqref{eq.termlemmaeq1.threewave} we have
% \begin{equation}\label{eq.termlemmaeq2.threewave}
%     H^T_{k_1\cdots k_{l+1}}\lesssim \sum_{\{d_{\mathfrak{n}}\}_{\mathfrak{n}\in T_{\text{in}}}\in\{0,1\}^{l(T)}}\prod_{\mathfrak{n}\in T_{\text{in}}}\frac{1}{\sqrt{(|q_{\mathfrak{n}}|+T^{-1}_{\text{max}})^2+|r_{\mathfrak{n}}|^2}}\ \delta_{\cap_{\mathfrak{n}\in T_{\text{in}}} \{S_{\mathfrak{n}}=0\}}.
% \end{equation}


By \eqref{eq.q_n.threewave}, $q_{\mathfrak{n}}$ is a linear combination of $\Omega_{\mathfrak{n}}$, so there exist constants $c_{\mathfrak{n},\widetilde{\mathfrak{n}}}$ such that $q_{\mathfrak{n}}=\sum_{\widetilde{\mathfrak{n}}}c_{\mathfrak{n},\widetilde{\mathfrak{n}}}\Omega_{\widetilde{\mathfrak{n}}}$. Let $c$ be the matrix $[c_{\mathfrak{n},\widetilde{\mathfrak{n}}}]$ and $\mathscr{M}(T)$ be the set of all possible such matrices, then the number of elements in $\mathscr{M}(T)$ can be bounded by a constant. Let $c(\Omega)$ be the vector $\{\sum_{\widetilde{\mathfrak{n}}}c_{\mathfrak{n},\widetilde{\mathfrak{n}}}\Omega_{\widetilde{\mathfrak{n}}}\}_{\mathfrak{n}}$ and $c(\Omega)_{\mathfrak{n}}=\sum_{\widetilde{\mathfrak{n}}}c_{\mathfrak{n},\widetilde{\mathfrak{n}}}\Omega_{\widetilde{\mathfrak{n}}}$ be the components of $c(\Omega)$.

With this notation, we know that $q_{\mathfrak{n}}=c(\Omega)_{\mathfrak{n}}$ and the right hand side of \eqref{eq.termlemmaeq1.threewave} becomes $\sum_{c\in \mathscr{M}(T) }\prod_{\mathfrak{n}\in T_{\text{in}}} \frac{1}{|c(\Omega)_{\mathfrak{n}}|+T^{-1}_{\text{max}}}$. Therefore, using the fact that $n_{\textrm{in}}$ are compactly supported, we have
\begin{equation}\label{eq.termlemmaeq3.threewave}
\begin{split}
    &|Term(T, p)_k|\lesssim \sum_{\substack{k_1,\, \cdots,\, k_{l(T)+1},\, k'_1,\, \cdots,\, k'_{l(T)+1}\\ |k_{j}|, |k'_j|\lesssim 1, \forall j}} \sum_{c\in \mathscr{M}(T) }\prod_{\mathfrak{n}\in T_{\text{in}}}\frac{1}{|c(\Omega)_{\mathfrak{n}}|+T^{-1}_{\text{max}}} \prod_{\mathfrak{e}\in T_{\text{in}}} |k_{\mathfrak{e},x}|\ \delta_{\cap_{\mathfrak{n}\in T_{\text{in}}} \{S_{\mathfrak{n}}=0\}} 
    \\
    &\sum_{c'\in \mathscr{M}(T)}\prod_{\mathfrak{n}'\in T_{\text{in}}}\frac{1}{|c'(\Omega)_{\mathfrak{n}'}|+T^{-1}_{\text{max}}}\prod_{\mathfrak{e}'\in T_{\text{in}}}|k_{\mathfrak{e}',x}|\ \delta_{\cap_{\mathfrak{n}'\in T_{\text{in}}} \{S_{\mathfrak{n}'}=0\}} \delta_{p}(k_1,\cdots, k_{l(T)+1}, k'_1,\cdots, k'_{l(T)+1})
\end{split}
\end{equation}

Switch the order of summations and products in \eqref{eq.termlemmaeq3.threewave}, then we get
\begin{equation}\label{eq.termlemmaeq2.threewave}
\begin{split}
    &|Term(T, p)_k|\lesssim \sum_{\substack{k_1,\, \cdots,\, k_{l(T)+1},\, k'_1,\, \cdots,\, k'_{l(T)+1}\\ |k_{j}|, |k'_j|\lesssim 1, \forall j}} \sum_{c, c'\in \mathscr{M}(T) }\prod_{\mathfrak{n}, \mathfrak{n}'\in T_{\text{in}}}\frac{1}{|c(\Omega)_{\mathfrak{n}}|+T^{-1}_{\text{max}}}
    \\
    &\frac{1}{|c'(\Omega)_{\mathfrak{n}'}|+T^{-1}_{\text{max}}}\prod_{\mathfrak{e},\mathfrak{e}'\in T_{\text{in}}}(|k_{\mathfrak{e},x}||k_{\mathfrak{e}',x}|)\ \delta_{\cap_{\mathfrak{n},\mathfrak{n}'\in T_{\text{in}}} \{S_{\mathfrak{n}}=0, S_{\mathfrak{n}'}=0\}} \delta_{p}(k_1,\cdots, k_{l(T)+1}, k'_1,\cdots, k'_{l(T)+1}).
\end{split}
\end{equation}

Given a tree $T$ and pairing $p$, we can construct a couple $\mathcal{C}$. We show that 
\begin{equation}\label{eq.termlemmaeq4.threewave}
\sum_{c, c'\in \mathscr{M}(T) }=\sum_{c\in \mathscr{M}(\mathcal{C}) },\qquad \prod_{\mathfrak{n}, \mathfrak{n}'\in T_{\text{in}}}=\prod_{\mathfrak{n}\in \mathcal{C}}, \qquad \prod_{\mathfrak{e},\mathfrak{e}'\in T_{\text{in}}}=\prod_{\mathfrak{e}\in \mathcal{C}_{\text{norm}}},\qquad \cap_{\mathfrak{n},\mathfrak{n}'\in T_{\text{in}}}=\cap_{\mathfrak{n}\in \mathcal{C}}.    
\end{equation}


Remember that $\mathcal{C}$ is constructed by gluing two copies of $T$ by $p$. In \eqref{eq.termlemmaeq2.threewave}, $\mathfrak{n}$, $\mathfrak{e}$, $\mathfrak{n}'$, $\mathfrak{e}'$ denote nodes and edges in the first or second copy respectively. Since all nodes in $\mathcal{C}$ come from the two copies of $T$, we get $\prod_{\mathfrak{n}, \mathfrak{n}'\in T_{\text{in}}}=\prod_{\mathfrak{n}\in \mathcal{C}}$ and $\cap_{\mathfrak{n},\mathfrak{n}'\in T_{\text{in}}}=\cap_{\mathfrak{n}\in \mathcal{C}}$. Remember that $T_{\text{in}}$ is the tree formed by all non-leaf nodes, so edges of $\mathcal{C}_{\text{norm}}$ all come from the two copies of $T_{\text{in}}$. Therefore $\prod_{\mathfrak{e},\mathfrak{e}'\in T_{\text{in}}}=\prod_{\mathfrak{e}\in \mathcal{C}_{\text{norm}}}$. Given two matrices $c$, and $c'$, we can construct a new matrix $c\oplus c'$ as the following. Consider two vectors $\Omega=\{\Omega_{\mathfrak{n}}\}_{\mathfrak{n}'\in T}$, $\Omega'=\{\Omega_{\mathfrak{n}'}\}_{\mathfrak{n}'\in T}$, then define $\Omega\oplus\Omega'\coloneqq \{\Omega_{\mathfrak{n}},\Omega_{\mathfrak{n}'}\}_{\mathfrak{n},\mathfrak{n}'\in T}$. We know that $c(\Omega)=\sum_{\widetilde{\mathfrak{n}}}c_{\mathfrak{n},\widetilde{\mathfrak{n}}}\Omega_{\widetilde{\mathfrak{n}}}$ and $c(\Omega')=\sum_{\widetilde{\mathfrak{n}}'}c_{\mathfrak{n}',\widetilde{\mathfrak{n}}'}\Omega_{\widetilde{\mathfrak{n}}'}$. Define $c\oplus c'$ to be the linear map whose domain is all vector of the form $\Omega\oplus\Omega'$ and whose action is $c\oplus c'(\Omega\oplus\Omega')=\{\sum_{\widetilde{\mathfrak{n}}}c_{\mathfrak{n},\widetilde{\mathfrak{n}}}\Omega_{\widetilde{\mathfrak{n}}},\sum_{\widetilde{\mathfrak{n}}'}c_{\mathfrak{n}',\widetilde{\mathfrak{n}}'}\Omega_{\widetilde{\mathfrak{n}}'}\}_{\mathfrak{n},\mathfrak{n}'\in T}$. Define $\mathscr{M}(\mathcal{C})=\{c\oplus c':c, c'\in \mathscr{M}(T)\}$, then we get $\sum_{c, c'\in \mathscr{M}(T) }=\sum_{c\in \mathscr{M}(\mathcal{C}) }$.



By \eqref{eq.termlemmaeq4.threewave} and the fact that leaves corresponding to $k_j, k_j'$ are merged in $\mathcal{C}$, \eqref{eq.termlemmaeq2.threewave} is equivalent to 
\begin{equation}\label{eq.termlemmaeq5.threewave}
|Term(T, p)_k|\lesssim \sum_{\substack{k_1,\, \cdots,\, k_{l(T)+1},\, k'_1,\, \cdots,\, k'_{l(T)+1}\\ |k_{j}|, |k'_j|\lesssim 1, \forall j}} \sum_{c\in \mathscr{M}(\mathcal{C}) }\prod_{\mathfrak{n}\in \mathcal{C}}\frac{1}{|c(\Omega)_{\mathfrak{n}}|+T^{-1}_{\text{max}}} \prod_{\mathfrak{e}\in \mathcal{C}_{\text{norm}}}|k_{\mathfrak{e},x}|\ \delta_{\cap_{\mathfrak{n}\in \mathcal{C}} \{S_{\mathfrak{n}}=0\}}
\end{equation}


Assigning a number $\sigma_{\mathfrak{n}}\in \mathbb{Z}_{T_{\text{max}}}$ for each node $\mathfrak{n}\in \mathcal{C}$, a number $\kappa_{\mathfrak{e}}\in \mathcal{D}(\alpha,1)\coloneqq\{2^{-K_{\mathfrak{e}}}:K_{\mathfrak{e}}\in  \mathbb{Z}\cap [0,ln\ \alpha^{-1}]\}$ for each edge $\mathfrak{e}$ and a number $k\in \mathbb{Z}^d_{L}$ for each fixed leg, we can define the associated equation $Eq(\mathcal{C}, \{\sigma_{\mathfrak{n}}\}_{\mathfrak{n}}, \{\kappa_{\mathfrak{e}}\}_{\mathfrak{e}},k)=Eq(\mathcal{C})$ as in \eqref{eq.diophantineeqpairedsigma'.threewave}. Then we have  
\begin{equation}\label{eq.termlemmaeq8.threewave}
    \sum_{\substack{k_1,\, \cdots,\, k_{l(T)+1},\, k'_1,\, \cdots,\, k'_{l(T)+1}\\\cap_{\mathfrak{n}\in \mathcal{C}} \{S_{\mathfrak{n}}=0\}}}=\sum_{\kappa_{\mathfrak{e}}\in \mathcal{D}(\alpha,1)}\sum_{\sigma_{\mathfrak{n}}\in \mathbb{Z}_{T_{\text{max}}}}\sum_{Eq(\mathcal{C}, \{\sigma_{\mathfrak{n}}\}_{\mathfrak{n}}, \{\kappa_{\mathfrak{e}}\}_{\mathfrak{e}},k)},
\end{equation}
which implies that
\begin{equation}\label{eq.termlemmaeq6.threewave}
\begin{split}
    |Term(T, p)_k|\lesssim \sum_{\kappa_{\mathfrak{e}}\in \mathcal{D}(\alpha,1)}\sum_{\sigma_{\mathfrak{n}}\in \mathbb{Z}_{T_{\text{max}}}}\sum_{Eq(\mathcal{C}, \{\sigma_{\mathfrak{n}}\}_{\mathfrak{n}}, \{\kappa_{\mathfrak{e}}\}_{\mathfrak{e}},k)} \sum_{c\in \mathscr{M}(\mathcal{C}) }\prod_{\mathfrak{n}\in \mathcal{C}}\frac{1}{|c(\Omega)_{\mathfrak{n}}|+T^{-1}_{\text{max}}} \prod_{\mathfrak{e}\in \mathcal{C}_{\text{norm}}}|k_{\mathfrak{e},x}|
\end{split}
\end{equation}
% \begin{equation}
% \begin{split}
%     |Term(T, p)_k|\lesssim& \sum_{\kappa_{\mathfrak{e}}\in \mathcal{D}(\alpha,1)}\sum_{\sigma_{\mathfrak{n}}\in \mathbb{Z}_{T_{\text{max}}}}\sum_{Eq(\mathcal{C}, \{\sigma_{\mathfrak{n}}\}_{\mathfrak{n}}, \{\kappa_{\mathfrak{e}}\}_{\mathfrak{e}},k)}
%     \\
%     &\sum_{c\in \mathscr{M}(T) }\prod_{\mathfrak{n}\in T_{\text{in}}}\frac{1}{\sqrt{(|c(\Omega)_{\mathfrak{n}}|+T^{-1}_{\text{max}})^2+|r_{\mathfrak{n}}|^2}} \sum_{c'\in \mathscr{M}(T)}\prod_{\mathfrak{n}'\in T_{\text{in}}}\frac{1}{\sqrt{(|c(\Omega)_{\mathfrak{n}'}|+T^{-1}_{\text{max}})^2+|r_{\mathfrak{n}'}|^2}} .
% \end{split}
% \end{equation}
%Since $Eq(\mathcal{C}, \{\sigma_{\mathfrak{n}}\}_{\mathfrak{n}}, \{\kappa_{\mathfrak{e}}\}_{\mathfrak{e}},k)$ contains equations in $\delta_{\cap_{\mathfrak{n}\in T_{\text{in}}} \{S_{\mathfrak{n}}=0\}}$, $\delta_{\cap_{\mathfrak{n}'\in T_{\text{in}}} \{S_{\mathfrak{n}'}=0\}}$ and $\delta_{p}$, so we may remove the three indicator functions from the above sum. 

Remember in the definition of $Eq(\mathcal{C})$, we have conditions that $|k_{\mathfrak{e}}|\lesssim 1$. These conditions come from the fact that on the support of $\delta_{\cap_{\mathfrak{n}\in T_{\text{in}}} \{S_{\mathfrak{n}}=0\}}$, if $k_1,\cdots,k_{l(T)+1}$ are bounded, then $|k_{\mathfrak{e}}|$ are also bounded. Recall that by \eqref{eq.defnS_n.threewave}, $S_{\mathfrak{n}}=\iota_{\mathfrak{e}_1}k_{\mathfrak{e}_1}+\iota_{\mathfrak{e}_2}k_{\mathfrak{e}_2}+\iota_{\mathfrak{e}}k_{\mathfrak{e}}$. The conditions that $S_{\mathfrak{n}}=0$ imply that the variables $k_{\mathfrak{e}}$ of the parent edge $\mathfrak{e}$ are linear combinations of the variables $k_{\mathfrak{e}_1}$, $k_{\mathfrak{e}_2}$ of children edges $\mathfrak{e}_1$, $\mathfrak{e}_2$. The variables of children edge $\mathfrak{e}_j$ is again a linear combination of the variables of their children. We may iterate this argument to show that all variables $k_{\mathfrak{e}}$ are linear combinations of variables of leaves $k_1,\cdots,k_{l(T)+1}$. Because $k_1,\cdots,k_{l(T)+1}$ are bounded, then their linear combinations $k_{\mathfrak{e}}$ are also bounded.

In the definition of $Eq(\mathcal{C}, \{\sigma_{\mathfrak{n}}\}_{\mathfrak{n}}, \{\kappa_{\mathfrak{e}}\}_{\mathfrak{e}},k)$,
$|\Omega_{\mathfrak{n}}-\sigma_{\mathfrak{n}}|=O(T^{-1}_{\text{max}})$ and $|k_{\mathfrak{e}x}| \sim \kappa_{\mathfrak{e}}$. Denote the constant in $O(T^{-1}_{\text{max}})$ by $\delta$ and then we have $|\Omega_{\mathfrak{n}}-\sigma_{\mathfrak{n}}|\le \delta T^{-1}_{\text{max}}$. Therefore, we have $|c(\Omega)_{\mathfrak{n}}-c(\{\sigma_{\mathfrak{n}}\})_{\mathfrak{n}}|\lesssim \delta T^{-1}_{\text{max}}$. We have the freedom of choosing $\delta$ in the definition and we take it sufficiently small so that $|c(\Omega)_{\mathfrak{n}}-c(\{\sigma_{\mathfrak{n}}\})_{\mathfrak{n}}|\le \frac{1}{2}T^{-1}_{\text{max}}$. This implies that $|c(\Omega)_{\mathfrak{n}}|+T^{-1}_{\text{max}}\gtrsim |c(\{\sigma_{\mathfrak{n}}\})_{\mathfrak{n}}|+T^{-1}_{\text{max}}$. Since we also have $|k_{\mathfrak{e}x}| \sim \kappa_{\mathfrak{e}}$, by \eqref{eq.termlemmaeq6.threewave} we get
\begin{equation}\label{eq.lemboundtermTp.threewave}
\begin{split}
    &Term(T, p)_k
    \\
    \lesssim& \sum_{\kappa_{\mathfrak{e}}\in \mathcal{D}(\alpha,1)}\sum_{\sigma_{\mathfrak{n}}\in \mathbb{Z}_{T_{\text{max}}}}\sum_{Eq(\mathcal{C}, \{\sigma_{\mathfrak{n}}\}_{\mathfrak{n}}, \{\kappa_{\mathfrak{e}}\}_{\mathfrak{e}},k)} \sum_{c\in \mathscr{M}(\mathcal{C}) }\prod_{\mathfrak{n}\in \mathcal{C}}\frac{1}{|c(\Omega)_{\mathfrak{n}}|+T^{-1}_{\text{max}}} \prod_{\mathfrak{e}\in \mathcal{C}_{\text{norm}}}|k_{\mathfrak{e},x}|
    \\
    \lesssim &\sum_{c\in \mathscr{M}(\mathcal{C}) }\sum_{\kappa_{\mathfrak{e}}\in \mathcal{D}(\alpha,1)}\sum_{\substack{\sigma_{\mathfrak{n}}\in \mathbb{Z}_{T_{\text{max}}}\\ |\sigma_{\mathfrak{n}}|\lesssim 1}}\prod_{\mathfrak{n}\in \mathcal{C}}\frac{1}{|c(\{\sigma_{\mathfrak{n}}\})_{\mathfrak{n}}|+T^{-1}_{\text{max}}} \sum_{Eq(\mathcal{C}, \{\sigma_{\mathfrak{n}}\}_{\mathfrak{n}}, \{\kappa_{\mathfrak{e}}\}_{\mathfrak{e}},k)} \prod_{\mathfrak{e}\in \mathcal{C}_{\text{norm}}}|k_{\mathfrak{e},x}|
    \\
    \lesssim &\sum_{c\in \mathscr{M}(\mathcal{C}) }\sum_{\kappa_{\mathfrak{e}}\in \mathcal{D}(\alpha,1)}\sum_{\substack{\sigma_{\mathfrak{n}}\in \mathbb{Z}_{T_{\text{max}}}\\ |\sigma_{\mathfrak{n}}|\lesssim 1}}\prod_{\mathfrak{n}\in \mathcal{C}}\frac{1}{|c(\{\sigma_{\mathfrak{n}}\})_{\mathfrak{n}}|+T^{-1}_{\text{max}}} \left(\prod_{\mathfrak{e}\in \mathcal{C}_{\text{norm}}}\kappa_{\mathfrak{e}}\right)\#Eq(\mathcal{C}, \{\sigma_{\mathfrak{n}}\}_{\mathfrak{n}}, \{\kappa_{\mathfrak{e}}\}_{\mathfrak{e}},k) 
    \\
    \lesssim &\sum_{c\in \mathscr{M}(\mathcal{C}) }\sum_{\kappa_{\mathfrak{e}}\in \mathcal{D}(\alpha,1)}\sum_{\substack{\sigma_{\mathfrak{n}}\in \mathbb{Z}_{T_{\text{max}}}\\ |\sigma_{\mathfrak{n}}|\lesssim 1}}\prod_{\mathfrak{n}\in \mathcal{C}}\frac{1}{|c(\{\sigma_{\mathfrak{n}}\})_{\mathfrak{n}}|+T^{-1}_{\text{max}}} \left(\prod_{\mathfrak{e}\in \mathcal{C}_{\text{norm}}}\kappa_{\mathfrak{e}}\right)L^{O(n\theta)} Q^{\frac{n}{2}}\ \prod_{\mathfrak{e}\in \mathcal{C}_{\text{norm}}} \kappa^{-1}_{\mathfrak{e}}
\end{split}
\end{equation}
Here in the last inequality, we applied \eqref{eq.countingbd0.threewave} in Proposition \ref{prop.counting}.

After simplification, \eqref{eq.lemboundtermTp.threewave} gives us 
\begin{equation}\label{eq.lemboundtermTpsimplify.threewave}
\begin{split}
    |Term(T, p)_k|\lesssim &L^{O(n\theta)} Q^{\frac{n}{2}}\sum_{c\in \mathscr{M}(\mathcal{C}) }\sum_{\kappa_{\mathfrak{e}}\in \mathcal{D}(\alpha,1)}\sum_{\substack{\sigma_{\mathfrak{n}}\in \mathbb{Z}_{T_{\text{max}}}\\ |\sigma_{\mathfrak{n}}|\lesssim 1}} \prod_{\mathfrak{n}\in \mathcal{C}}\frac{1}{|c(\{\sigma_{\mathfrak{n}}\})_{\mathfrak{n}}|+T^{-1}_{\text{max}}}
    \\
    \lesssim &L^{O(n\theta)} Q^{\frac{n}{2}} \left(\sum_{\kappa_{\mathfrak{e}}\in \mathcal{D}(\alpha,1)} 1\right) \sum_{c\in \mathscr{M}(\mathcal{C})}\sum_{\substack{\sigma_{\mathfrak{n}}\in \mathbb{Z}_{T_{\text{max}}}\\ |\sigma_{\mathfrak{n}}|\lesssim 1}} \prod_{\mathfrak{n}\in \mathcal{C}}\frac{1}{|c(\{\sigma_{\mathfrak{n}}\})_{\mathfrak{n}}|+T^{-1}_{\text{max}}}
    \\
    \lesssim & L^{O(n\theta)} Q^{\frac{n}{2}} \sum_{c\in \mathscr{M}(\mathcal{C})}\sum_{\substack{\sigma_{\mathfrak{n}}\in \mathbb{Z}_{T_{\text{max}}}\\ |\sigma_{\mathfrak{n}}|\lesssim 1}} \prod_{\mathfrak{n}\in \mathcal{C}}\frac{1}{|c(\{\sigma_{\mathfrak{n}}\})_{\mathfrak{n}}|+T^{-1}_{\text{max}}}
\end{split}
\end{equation}
% \begin{equation}
% \begin{split}
%     &Term(T, p)_k
%     \\
%     \lesssim&\sum_{c,c'\in \mathscr{M}(T) } \sum_{\sigma_{\mathfrak{n}}\in \mathbb{Z}_{T_{\text{max}}}} \prod_{\mathfrak{n}\in T_{\text{in}}}\frac{t\alpha}{|c(\{\sigma_{\mathfrak{n}}\})_{\mathfrak{n}}|+\alpha} \prod_{\mathfrak{n}\in T_{\text{in}}}\frac{t\alpha}{|c'(\{\sigma_{\mathfrak{n}}\})_{\mathfrak{n}}|+\alpha} (\#Eq(\mathcal{C}, \{\sigma_{\mathfrak{n}}\}_{\mathfrak{n}},k)+O(L^{-8d\, l(T)-8d}))
%     \\
%     \lesssim & L^{O(n\theta)} Q^{n} L^{\frac{1}{2} dn_d}\sum_{c,c'\in \mathscr{M}(T) } \sum_{\substack{\sigma_{\mathfrak{n}}\in \mathbb{Z}_{T_{\text{max}}}\\ |\sigma_{\mathfrak{n}}|\lesssim 1}} \prod_{\mathfrak{n}\in T_{\text{in}}}\frac{t\alpha}{|c(\{\sigma_{\mathfrak{n}}\})_{\mathfrak{n}}|+\alpha} \prod_{\mathfrak{n}\in T_{\text{in}}}\frac{t\alpha}{|c'(\{\sigma_{\mathfrak{n}}\})_{\mathfrak{n}}|+\alpha}  + O(L^{-6d\, l(T)-6d})
% \end{split}
% \end{equation}
Here in the last step we use the fact that $\#\mathcal{D}(\alpha,1)\lesssim ln(\alpha^{-1})$. $|\sigma_{\mathfrak{n}}|\lesssim 1$ in the sum of the first two inequalities comes from the fact that $\#Eq(\mathcal{C}, \{\sigma_{\mathfrak{n}}\}_{\mathfrak{n}},k)=0$ if some $|\sigma_{\mathfrak{n}}|\gtrsim 1$. This fact is true because in $Eq(\mathcal{C}, \{\sigma_{\mathfrak{n}}\}_{\mathfrak{n}},k)$, all $|k_{\mathfrak{e}}|\lesssim 1$, which implies that $|\Omega_{\mathfrak{n}}|\lesssim 1$ and therefore $|\sigma_{\mathfrak{n}}|\gtrsim 1$ (notice that $|\Omega_{\mathfrak{n}}-\sigma_{\mathfrak{n}}|=O(T^{-1}_{\text{max}})$).

We claim that 
\begin{equation}\label{eq.lemTpvarianceclaim.threewave}
     \sup_{c}\sum_{\substack{\sigma_{\mathfrak{n}}\in \mathbb{Z}_{T_{\text{max}}}\\ |\sigma_{\mathfrak{n}}|\lesssim 1}} \prod_{\mathfrak{n}\in \mathcal{C}}\frac{1}{|c(\{\sigma_{\mathfrak{n}}\})_{\mathfrak{n}}|+T^{-1}_{\text{max}}}\lesssim L^{O(n\theta)} T^{n}_{\text{max}}
\end{equation}

Since there are only bounded many matrices in $\mathscr{M}(\mathcal{C})$.  Given the above claim, we know that 
\begin{equation}
    |Term(T, p)_k|\lesssim L^{O(n\theta)} Q^{\frac{n}{2}} T^{n}_{\text{max}},
\end{equation}
which proves the lemma.

Now prove the claim. Notice that there are $n$ nodes in $\mathcal{C}$. We label these nodes by $h=1,\cdots,n$ and denote $\sigma_{\mathfrak{n}}$ by $\sigma_{h}$ if $\mathfrak{n}$ is labelled by $h$. Since $\sigma_{h}\in \mathbb{Z}_{T_{\text{max}}}$, there exists $m_{h}\in \mathbb{Z}$ such that $\sigma_{h}=\frac{m_{h}}{T_{\text{max}}}$. \eqref{eq.lemTpvarianceclaim.threewave} is thus equivalent to 
\begin{equation}\label{eq.lemTpvarianceclaim1.threewave}
    \sum_{\substack{m_{h}\in \mathbb{Z}\\ |m_{h}|\lesssim  T_{\text{max}}}} \prod_{h=1}^{n}\frac{T_{\text{max}}}{|c(\{m_{h}\})_{h}|+1}\lesssim L^{O(n\theta)}T^{n}_{\text{max}}
\end{equation}

Before proving \eqref{eq.lemTpvarianceclaim1.threewave}, let's first look at one of its special cases. If $c=Id$, then $c(\{m_{h}\})_{h}=m_h$. The right hand side of \eqref{eq.lemTpvarianceclaim1.threewave} becomes
\begin{equation}
\begin{split}
    T^{n}_{\text{max}}\sum_{\substack{m_{h}\in \mathbb{Z}\\ |m_{h}|\lesssim  T_{\text{max}}}} \prod_{h=1}^{n}\frac{1}{|m_{h}|+1} = T^{n}_{\text{max}}\prod_{h=1}^{n}\left(\sum_{\substack{m_{h}\in \mathbb{Z}\\ |m_{h}|\lesssim  T_{\text{max}}}} \frac{1}{|m_{h}|+1}\right)
    \lesssim T^{n}_{\text{max}} (ln(\alpha^{-1}))^{n}= L^{O(n\theta)}T^{n}_{\text{max}}.
\end{split}
\end{equation}
Here we use the fact that $\sum_{\substack{j\in \mathbb{Z}\\ |j|\lesssim  T_{\text{max}}}} \frac{1}{|j|+1}=O(ln(\alpha^{-1}))$.

Now we prove \eqref{eq.lemTpvarianceclaim1.threewave}. We just need to show that 
\begin{equation}\label{eq.lemTpvarianceEulerMac.threewave}
    \sum_{\substack{m_{h}\in \mathbb{Z}\\ |m_{h}|\lesssim  T_{\text{max}}}} \prod_{h=1}^{n}\frac{1}{|c(\{m_{h}\})_{h}|+1}\lesssim L^{O(n\theta)}
\end{equation}

By Euler-Maclaurin formula \eqref{eq.EulerMaclaurin.threewave} and change of variable formula, we get
\begin{equation}
\begin{split}
    \sum_{\substack{m_{h}\in \mathbb{Z}\\ |m_{h}|\lesssim  T_{\text{max}}}} \prod_{h=1}^{n}\frac{1}{|c(\{m_{h}\})_{h}|+1}\le& \int_{|m_{h}|\lesssim  T_{\text{max}}}  \prod_{h=1}^{n}\frac{1}{|c(\{m_{h}\})_{h}|+1}\prod_{h=1}^{n} dm_{h}
    \\
    =& \int_{c(\{|m_{h}|\lesssim  T_{\text{max}}\})}  \prod_{h=1}^{n}\frac{1}{|m_{h}|+1}|\text{det}\ c|\prod_{h=1}^{n}  dm_{h}
    \\
    \lesssim &\prod_{h=1}^{n}\int_{|m_{h}|\lesssim  T_{\text{max}}}  \frac{1}{|m_{h}|+1}  dm_{h}
    \\
    \lesssim & (ln(\alpha^{-1}))^{n}\lesssim L^{O(n\theta)}
\end{split}
\end{equation}

We complete the proof of the claim and thus the proof of the lemma.
\end{proof}

Now we prove Proposition \ref{prop.treetermsupperbound}. To start with, recall the large deviation estimate for the Gaussian polynomial.

\begin{lem}[Large deviation for Gaussian polynomial]\label{lem.largedev}
Let $\{\eta_k(\omega)\}$ be i.i.d. complex Gaussian variables with mean $0$ and variance $1$. Let $F=F(\omega)$ be an degree $n$ polynomial of $\{\eta_k(\omega)\}$ defined by \begin{equation}\label{indp}
F(\omega)=\sum_{k_1,\cdots,k_n}a_{k_1\cdots k_n}\prod_{j=1}^n\eta_{k_j}^{\iota_j},
\end{equation} 
where $a_{k_1\cdots k_n}$ are constants, then we have 
\begin{equation}\label{largedevest}\mathbb{P}\left(|F(\omega)|\geq A\cdot \left(\mathbb{E}|F(\omega)|^2\right)^{\frac{1}{2}}\right)\leq Ce^{-cA^{\frac{2}{n}}}
\end{equation} 
\end{lem}
\begin{proof} This is a corollary of the hypercontractivity of Ornstein-Uhlenbeck semigroup. A good reference for this topic is \cite{oh2015ornstein}. \eqref{largedevest} is equivalent to (B.9) in \cite{oh2015ornstein}.
\end{proof}

Proposition \ref{prop.treetermsupperbound} is a corollary of the above large deviation estimate and Proposition \ref{prop.treetermsvariance}.

\begin{proof}[Proof of Proposition \ref{prop.treetermsupperbound}] Because $(\mathcal{J}_{T})_{k}$ are Gaussian polynomials, we can take $F(\omega)=(\mathcal{J}_{T})_{k}$ in Lemma \ref{lem.largedev}. Then we obtained
\begin{equation}
|(\mathcal{J}_{T})_{k}(t)|\lesssim L^{\frac{n}{2}\theta} \sqrt{\mathbb{E}|(\mathcal{J}_T)_k|^2}
\end{equation} 
with probability less than $e^{-c(L^{\frac{n}{2}\theta})^{\frac{2}{n}}}=e^{-cL^{\theta}}$. By definition \ref{def.Lcertainly}, the above inequality holds true $L$-certainly.

Since by Proposition \ref{prop.treetermsvariance}, $\mathbb{E}|(\mathcal{J}_T)_k|^2\lesssim L^{O(l(T)\theta)} \rho^{2l(T)}$, then we get 
\begin{equation}\label{eq.proptreetermsupperbound1.threewave}
|(\mathcal{J}_{T})_{k}(t)|\lesssim L^{O(l(T)\theta)} \rho^{l(T)},\qquad \textit{L-certainly}
\end{equation} 

\eqref{eq.proptreetermsupperbound1.threewave} is very similar to the final goal \eqref{eq.treetermsupperbound.threewave}, except for the $\sup_t$ and $\sup_k$ in front. In what follows we apply the standard epsilon net and union bound method to remove these two $\sup$.

Assume that $|t-t'|\lesssim \rho^{l(T)}L^{-M}$, it's not hard to show that $|(\mathcal{J}_{T})_{k}(t)-(\mathcal{J}_{T})_{k}(t')|\lesssim \rho^{l(T)}$. Therefore, if $\sup_{i} \sup_{k} |(\mathcal{J}_{T})_{k}(i\rho^{l(T)}L^{-M})|\lesssim L^{O(l(T)\theta)} \rho^{l(T)}$, then $\sup_{t} \sup_{k} |(\mathcal{J}_{T})_{k}(t)|\lesssim L^{O(l(T)\theta)} \rho^{l(T)}$ and the proof is completed.
%Consider the decomposition $\left[0, T_{\text{max}}\right]=\bigcup_{i=0}^{L^{M} T_{\text{max}}}\left[\frac{i}{L^M},\frac{i+1}{L^M}\right]$.

Notice that 
\begin{equation}\label{eq.unionbound.threewave}
\begin{split}
    &\mathbb{P}\left(\sup_{i\in \mathbb{Z}} \sup_{k} |(\mathcal{J}_{T})_{k}(i\rho^{l(T)}L^{-M})|\gtrsim L^{O(l(T)\theta)} \rho^{l(T)}\right)
    \\
    =&\mathbb{P}\left(\bigcup_{i\in \mathbb{Z}\cap [0, T_{\text{max}}\rho^{-l(T)}L^{M}]}\bigcup_{k\in \mathbb{Z}_{L}\cap [0,1]}\{ |(\mathcal{J}_{T})_{k}(i\rho^{l(T)}L^{-M})|\gtrsim L^{O(l(T)\theta)} \rho^{l(T)}\}\right)
    \\
    \lesssim & \sum_{i\in \mathbb{Z}\cap [0, T_{\text{max}}\rho^{-l(T)}L^{M}]}\sum_{k\in \mathbb{Z}_{L}\cap [0,1]}\mathbb{P}\left( |(\mathcal{J}_{T})_{k}(i\rho^{l(T)}L^{-M})|\gtrsim L^{O(l(T)\theta)} \rho^{l(T)}\right)
    \\
    \lesssim & \sum_{i\in \mathbb{Z}\cap [0, T_{\text{max}}\rho^{-l(T)}L^{M}]}\sum_{k\in \mathbb{Z}_{L}\cap [0,1]} e^{-O(L^{\theta})}= L^{2M}e^{-O(L^{\theta})}=e^{-O(L^{\theta})}
\end{split}
\end{equation}
Here in the second inequality the two ranges $[0, T_{\text{max}}\rho^{-l(T)}L^{M}]$ and $[0,1]$ of $i$ and $k$ come from the fact that $t=i\rho^{l(T)}L^{-M}\lesssim  T_{\text{max}}$ and $(\mathcal{J}_{T})_{k}=0$ for $|k|\gtrsim 1$. In the last line we can replace the probability by $e^{-O(L^{\theta})}$ because the estimate $|(\mathcal{J}_{T})_{k}(t)|\lesssim L^{O(l(T)\theta)} \rho^{l(T)}$ holds true $L$-certainly. 

Now we complete the proof of Proposition \ref{prop.treetermsupperbound}.
\end{proof}


\subsection{Norm estimate of random matrices} \label{sec.randommatrices} In this section, we prove Proposition \ref{prop.operatorupperbound}. 

Remember that by definition of $\mathcal{P}_{T}$ and $\mathcal{T}$, 
\begin{equation}\label{eq.formulaP_T.threewave}
\mathcal{P}_{T}(w)=\mathcal{T}(\mathcal{J}_{T},w)=\frac{i\lambda}{L^{d}} \sum\limits_{S(k_1,k_2,k)=0}\int^{t}_0k_{x}\mathcal{J}_{T,k_1} w_{k_2}e^{i s\Omega(k_1,k_2,k)- \nu|k|^2(t-s)} ds.
\end{equation}    

\subsubsection{Dyadic decomposition of $\mathcal{P}_{T}$}
Remember that we have the dyadic decomposition $[0,1]= \bigcup_{\tau=0}^{\infty}[2^{-\tau},2^{-\tau-1}]$.

We can then construct a dyadic decomposition  $\mathcal{P}_{T}=\sum_{l=0}^{\infty} \mathcal{P}^l_{T}$. Here $\mathcal{P}_T^l$ is defined by the following formula. 
\begin{equation}
\mathcal{P}_T^l(w)=\frac{i\lambda}{L^{d}} \sum\limits_{S(k_1,k_2,k)=0}\int_{(t-s)/T_{\text{max}}\in [2^{-\tau},2^{-\tau-1}]}k_{x}\mathcal{J}_{T,k_1} w_{k_2}e^{i s\Omega(k_1,k_2,k)- \nu|k|^2(t-s)} ds
\end{equation}

We also introduce the bilinear operator $\mathcal{T}^l(\phi,\phi)_k$
\begin{equation}
\mathcal{T}^l(\phi,\phi)_k=\frac{i\lambda}{L^{d}} \sum\limits_{S(k_1,k_2,k)=0}\int_{(t-s)/T_{\text{max}}\in [2^{-\tau},2^{-\tau-1}]}k_{x}\phi_{k_1} \phi_{k_2}e^{i s\Omega(k_1,k_2,k)- \nu|k|^2(t-s)} ds
\end{equation}


Proposition \ref{prop.operatorupperbound} is a corollary of the following proposition.

\begin{prop}\label{prop.operatorupperbound'}
Let $\rho=\alpha\, T^{\frac{1}{2}}_{\text{max}}$ and $\mathcal{P}^l_{T}$ be the operator defined above. Define the space $X^{p}_{L^{2M}}=\{w\in X^p: w_k=0\text{ if }|k|\gtrsim L^{2M}\}$ and the norm $||w||_{X^{p}_{L^{2M}}}=\sup_{|k|\lesssim L^{2M}} \langle k\rangle^{p} |w_k|$. Then for any sequence of trees and numbers $\{T_1,\cdots,T_K\}$ and $\{\tau_1,\cdots,\tau_K\}$, we have $L$-certainly the operator bound
\begin{equation}\label{eq.operatornormtau.threewave}
    \left|\left|\sum_{\tau_1,\cdots,\tau_K}\prod_{j=1}^K\mathcal{P}^{\tau_j}_{T_j}\right|\right|_{L_t^{\infty}X^{p}_{L^{2M}}\rightarrow L_t^{\infty}X^{p}}\le L^{O\left(1+\theta \sum_{j=1}^K l(T_j)\right)} \rho^{\sum_{j=1}^K l(T_j)}.
\end{equation}
for any $T_j$ with $l(T_j)\le N$. 
\end{prop}

\begin{proof}[Proof of Proposition \ref{prop.operatorupperbound}:] By Proposition \ref{prop.operatorupperbound'}, we have
\begin{equation}
    \left|\left|\prod_{j=1}^K\mathcal{P}_{T_j}\right|\right|_{L_t^{\infty}X^p_{L^{2M}}\rightarrow L_t^{\infty}X^p}=\left|\left|\sum_{\tau_1,\cdots,\tau_K}\prod_{j=1}^K\mathcal{P}^{\tau_j}_{T}\right|\right|_{L_t^{\infty}X^{p}_{L^{2M}}\rightarrow L_t^{\infty}X^{p}}\le L^{O\left(1+\theta \sum_{j=1}^K l(T_j)\right)} \rho^{\sum_{j=1}^K l(T_j)}.
\end{equation}

Define $\left(X^{p}_{L^{2M}}\right)^{\perp}=\{w\in X^p: w_k=0\text{ if }|k|\lesssim L^{2M}\}$. To prove Proposition \ref{prop.operatorupperbound}, it suffices to show that for all $w\in \left(X^{p}_{L^{2M}}\right)^{\perp}$,
\begin{equation}\label{eq.prop2.8eq1.threewave}
    \left|\left|\prod_{j=1}^K\mathcal{P}_{T_j}w\right|\right|_{L_t^{\infty}X^p}\lesssim L^{O\left(1+\theta \sum_{j=1}^K l(T_j)\right)} \rho^{\sum_{j=1}^K l(T_j)} \left|\left|w\right|\right|_{L_t^{\infty}X^p}.
\end{equation}

By \eqref{eq.formulaP_T.threewave}, if $w\in \left(X^{p}_{L^{2M}}\right)^{\perp}$, then $\mathcal{P}_{T_j} w\in \left(X^{p}_{L^{2M}}\right)^{\perp}$ if we enlarge the constant in $|k|\lesssim L^{2M}$ in the definition of $\left(X^{p}_{L^{2M}}\right)^{\perp}$. Therefore, to prove \eqref{eq.prop2.8eq1.threewave}, it suffices to prove
\begin{equation}\label{eq.prop2.8eq2.threewave}
    \left|\left|\mathcal{P}_{T_j}w\right|\right|_{L_t^{\infty}X^p}\lesssim L^{O(1+l(T_j)\theta)} \rho^{l(T_j)} \left|\left|w\right|\right|_{L_t^{\infty}X^p},
\end{equation}
for all $w\in \left(X^{p}_{L^{2M}}\right)^{\perp}$.

By \eqref{eq.formulaP_T.threewave},
\begin{equation}\label{eq.prop2.8eq3.threewave}
\begin{split}
    |\mathcal{P}_{T_j}(w)_k|\le &\frac{\lambda}{L^{d}} \sum\limits_{S(k_1,k_2,k)=0}\int^{t}_0|k_{x}||\mathcal{J}_{T_j,k_1}| |w_{k_2}|e^{- \nu|k|^2(t-s)} ds
    \\
    \lesssim& L^{O(l(T_j)\theta)} \rho^{l(T_j)}\frac{\lambda}{L^{d}}|k_{x}| \int^{t}_0e^{- \nu|k|^2(t-s)} ds \sum_{k_2:|k_2-k|\lesssim 1} \langle k_2\rangle^{-p} 
    \\
    \lesssim& L^{O(l(T_j)\theta)} \rho^{l(T_j)}\frac{\lambda}{L^{d}} |k_{x}| \nu^{-1} \langle k\rangle^{-2} \sum_{k_2:|k_2-k|\lesssim 1} \langle k_2\rangle^{-p} 
    \\
    \lesssim& L^{O(l(T_j)\theta)} \rho^{l(T_j)}\frac{\lambda}{L^{d}} \nu^{-1} \langle k\rangle^{-1}  \langle k\rangle^{-p} \lesssim L^{O(l(T_j)\theta)} \rho^{l(T_j)}\frac{\lambda}{L^{d}} \nu^{-1} L^{-2M}  \langle k\rangle^{-p} 
    \\
    \lesssim& L^{-M} \rho^{l(T_j)} \langle k\rangle^{-p}
\end{split}
\end{equation} 
Here in the second inequality, we apply Proposition \ref{prop.treetermsupperbound}. In the third line we use the fact that $\int^{t}_0e^{- \nu|k|^2(t-s)} ds\le \nu^{-1} \langle k\rangle^{-2}$. In the fourth line we use the fact that $\sum_{k_2:|k_2-k|\lesssim 1} \langle k_2\rangle^{-p}\le \langle k\rangle^{-p}$ and $|k|\gtrsim L^{2M}$ (since if $|k|\lesssim L^{2M}$ then $|\mathcal{P}_{T_j}(w)_k|$ vanishes and there is nothing to prove).

\eqref{eq.prop2.8eq3.threewave} implies that $ \left|\left|\mathcal{P}_{T_j}w\right|\right|_{L_t^{\infty}X^p}\lesssim L^{-M} \rho^{l(T_j)} \left|\left|w\right|\right|_{L_t^{\infty}X^p}\lesssim L^{O(1+l(T_j)\theta)} \rho^{l(T_j)} \left|\left|w\right|\right|_{L_t^{\infty}X^p}$, so we have proved \eqref{eq.prop2.8eq2.threewave} and thus \eqref{eq.operatornorm'.threewave}.

Now we prove \eqref{eq.operatornorm.threewave}. Because $L=\sum_{1\le l(T)\le N} \mathcal{P}_{T}$ and $L^K=\sum_{1\le l(T_1),\cdots l(T_K)\le N} \mathcal{P}_{T_1}\cdots\mathcal{P}_{T_K}$, by \eqref{eq.operatornorm'.threewave} 
\begin{equation}
     \left|\left|L^K\right|\right|_{L_t^{\infty}X^p\rightarrow L_t^{\infty}X^p}\lesssim L^{O(1)} \sum_{1\le l(T_1),\cdots l(T_K)\le N} (L^{O(\theta)}\rho)^{\sum_{j=1}^K l(T_j)}\le L^{O\left(1+\theta \sum_{j=1}^K l(T_j)\right)} \rho^{K}
\end{equation}
Here in the last step, we use the fact that $l(T_j)\ge 1$ for all $j$ and there are bounded many trees satisfy $1\le l(T_1),\cdots l(T_K)\le N$.

Therefore, we complete the proof of Proposition \ref{prop.operatorupperbound}.
\end{proof}



\subsubsection{Formulas for product of random matrices $\mathcal{P}^l_{T_j}$} In order to prove Proposition \ref{prop.operatorupperbound'}, it is very helpful to find a good formula of  $\prod_{j=1}^K\mathcal{P}^{\tau_j}_{T_j}$, which is the main goal of this section.

$\mathcal{P}^{l}_{T}$ is almost the same as $\mathcal{P}_{T}$ except for the limits in the time integral, so in the rest part of this section we do not stress their difference. Now let's find a tree representation for $\mathcal{P}^{l}_{T}$ or $\mathcal{P}_{T}$.

%as in section \ref{sec.connection}.

By \eqref{eq.treeterm.threewave}, we know that $\mathcal{J}_{T}=\mathcal{T}(\mathcal{J}_{T_{\mathfrak{n}_1}}, \mathcal{J}_{T_{\mathfrak{n}_2}})$ corresponds to the tree $T$ in which the two subtrees of the root nodes are $T_{\mathfrak{n}_1}$ and $T_{\mathfrak{n}_2}$. Taking $T_{\mathfrak{n}_2}$ to be a one node tree, as the left tree in Figure \ref{fig.T(J,xi)andT(J,w)}, then this graph represents a term $\mathcal{T}(\mathcal{J}_{T_{\mathfrak{n}_1}}, \xi)$. The right tree in Figure \ref{fig.T(J,xi)andT(J,w)} represents the term $\mathcal{T}(\mathcal{J}_{T_{\mathfrak{n}_1}}, w)$, because as in section \ref{sec.connection}, the $\Box$ node represents a function $w$ (in section \ref{sec.connection} the function is $\phi$).

 \begin{figure}[H]
    \centering
    \scalebox{0.5}{
    \begin{tikzpicture}[level distance=80pt, sibling distance=100pt]
        \node[] at (0,0) (1) {} 
            child {node[fillcirc] (2) {} 
                child {node[draw, circle, minimum size=1cm, scale=2] (3) {$T_{\mathfrak{n}_1}$}}
                child {node[fillstar] (4) {}}
            };
        \draw[-{Stealth[length=5mm, width=3mm]}] (1) -- (2);
        \draw[-{Stealth[length=5mm, width=3mm]}] (2) -- (3);
        \draw[-{Stealth[length=5mm, width=3mm]}] (2) -- (4);
        
        \node[] at (12,0) (1) {} 
            child {node[fillcirc] (2) {} 
                child {node[draw, circle, minimum size=1cm, scale=2] (3) {$T_{\mathfrak{n}_1}$}}
                child {node[draw, minimum size=0.4cm] (4) {}}
            };
        \draw[-{Stealth[length=5mm, width=3mm]}] (1) -- (2);
        \draw[-{Stealth[length=5mm, width=3mm]}] (2) -- (3);
        \draw[-{Stealth[length=5mm, width=3mm]}] (2) -- (4);
    \end{tikzpicture}
    }
        \caption{Graphical representations of $\mathcal{T}(\mathcal{J}_{T_{\mathfrak{n}_1}}, \xi)$ and $\mathcal{T}(\mathcal{J}_{T_{\mathfrak{n}_1}}, w)$.}
        \label{fig.T(J,xi)andT(J,w)}
    \end{figure}

Let's find the tree representation for $\prod_{j=1}^K\mathcal{P}^{\tau_j}_{T_j}$ or $\prod_{j=1}^K\mathcal{P}_{T_j}$. Recall the expansion process in section \ref{sec.connection}: the replacement of $\Box$ by a branching node indicates the substitution of $\phi$ by $\mathcal{T}(\xi, \xi)$. The action of composition $\mathcal{P}_{T_1}\circ \mathcal{P}_{T_2}(w)=\mathcal{T}(\mathcal{J}_{T_{1}}, \mathcal{T}(\mathcal{J}_{T_{2}}, w))$ is the substitution of $\cdot$ by $\mathcal{T}(\mathcal{J}_{T_{2}}, w)$ in $\mathcal{T}(\mathcal{J}_{T_{1}},\cdot)$. As in Figure \ref{fig.substitution}, if $\mathcal{P}_{T_1}=\mathcal{T}(\mathcal{J}_{T_{1}},\cdot)$ is represented by the left tree, then as an analog that a $\Box$ node is replaced by a branching node, the substitution of $\cdot$ by $\mathcal{T}(\mathcal{J}_{T_{2}}, w)$ should correspond to the operation that the $\Box$ node in the left tree is replaced by the middle tree corresponding to $\mathcal{T}(\mathcal{J}_{T_{2}}, w)$, and the final resulting right tree should correspond to $\mathcal{P}_{T_1}\circ \mathcal{P}_{T_2}(w)$.

\begin{figure}[H]
    \centering
    \scalebox{0.5}{
    \begin{tikzpicture}[level distance=80pt, sibling distance=100pt]
        \node[] at (0,0) (1) {} 
            child {node[fillcirc] (2) {} 
                child {node[draw, circle, minimum size=1cm, scale=2] (3) {$T_{1}$}}
                child {node[draw, minimum size=0.4cm] (4) {}}
            };
        \draw[-{Stealth[length=5mm, width=3mm]}] (1) -- (2);
        \draw[-{Stealth[length=5mm, width=3mm]}] (2) -- (3);
        \draw[-{Stealth[length=5mm, width=3mm]}] (2) -- (4);
        
        \node[] at (8,0) (1) {} 
            child {node[fillcirc] (2) {} 
                child {node[draw, circle, minimum size=1cm, scale=2] (3) {$T_{2}$}}
                child {node[draw, minimum size=0.4cm] (4) {}}
            };
        \draw[-{Stealth[length=5mm, width=3mm]}] (1) -- (2);
        \draw[-{Stealth[length=5mm, width=3mm]}] (2) -- (3);
        \draw[-{Stealth[length=5mm, width=3mm]}] (2) -- (4);
        
        \node[] at (16,1.5) (1) {} 
            child {node[fillcirc] (2) {} 
                child {node[draw, circle, minimum size=1cm, scale=2] (3) {$T_{1}$}}
                child {node[fillcirc] (4) {} 
                child {node[draw, circle, minimum size=1cm, scale=2] (5) {$T_{2}$}}
                child {node[draw, minimum size=0.4cm] (6) {}}
            }
            };
        \draw[-{Stealth[length=5mm, width=3mm]}] (1) -- (2);
        \draw[-{Stealth[length=5mm, width=3mm]}] (2) -- (3);
        \draw[-{Stealth[length=5mm, width=3mm]}] (2) -- (4);
        \draw[-{Stealth[length=5mm, width=3mm]}] (4) -- (5);
        \draw[-{Stealth[length=5mm, width=3mm]}] (4) -- (6);
    \end{tikzpicture}
    }
        \caption{The substitution process.}
        \label{fig.substitution}
    \end{figure}

In conclusion, $\mathcal{P}_{T_1}\circ \mathcal{P}_{T_2}(w)$ corresponds to the right tree in Figure \ref{fig.substitution} and more generally, $\prod_{j=1}^K\mathcal{P}_{T_j}(w)$ (or $\prod_{j=1}^K\mathcal{P}^{\tau_j}_{T_j}(w)$) corresponds to the tree in in Figure \ref{fig.productformula}.

\begin{figure}[H]
    \centering
    \scalebox{0.4}{
    \begin{tikzpicture}[level distance=80pt, sibling distance=100pt]
        
        \node[] at (0,0) (1) {} 
            child {node[fillcirc] (2) {} 
                child {node[draw, circle, minimum size=1cm, scale=2] (3) {$T_{1}$}}
                child {node[fillcirc] (4) {} 
                child {node[draw, circle, minimum size=1cm, scale=2] (5) {$T_{2}$}}
                child {node[] (6) {}}
            }
            };
            
        \node[scale =3] at (3,-10.5) {$\cdots$};
        \draw[-{Stealth[length=5mm, width=3mm]}] (1) -- (2);
        \draw[-{Stealth[length=5mm, width=3mm]}] (2) -- (3);
        \draw[-{Stealth[length=5mm, width=3mm]}] (2) -- (4);
        \draw[-{Stealth[length=5mm, width=3mm]}] (4) -- (5);
        \draw[-{Stealth[length=5mm, width=3mm]}] (4) -- (6);
        
        
        \node[fillcirc] at (5,-12) (11) {} 
            child {node[draw, circle, minimum size=1cm, scale=2] (12) {$T_{K}$}}
            child {node[draw, minimum size=0.4cm] (13) {}
            };
        \node[scale =2] at (0.7,-2.8) {$s_1$};
        \node[scale =2] at (2.5,-5.6) {$s_2$};
        \node[scale =2] at (5.8,-12) {$s_K$};
    \end{tikzpicture}
    }
        \caption{Picture of $T_1\circ \cdots \circ T_{K}$}
        \label{fig.productformula}
    \end{figure}

We introduce the following definition.

\begin{defn}
\begin{enumerate}
    \item \textbf{Definition of $T_1\circ \cdots \circ T_{K}$:} We define  $T_1\circ \cdots \circ T_{K}$ to be the tree in Figure \ref{fig.productformula}.
    \item \textbf{Substitution nodes:} A node in a tree $T$ with $\Box$ nodes is defined to be a \underline{substitution node} if it is an ancestor of some $\Box$ nodes. For a substitution node, we assign a number $\tau$ to it, called its index. In Figure \ref{fig.productformula}, $s_1,\cdots,s_{K}$ are all the substitution nodes in $T_1\circ \cdots \circ T_{K}$ and we assign index $\tau_1,\cdots,\tau_{K}$ to them. Notice that $s_1$ is the root $\mathfrak{r}$.
\end{enumerate}
\end{defn}

Because the tree in this section contains $\Box$ nodes and substitution nodes, we propose the following generalization of Definition \ref{def.treeterms} of tree terms.

\begin{defn}\label{def.treetermsoperator} Given a binary tree $T$ with $\Box$ nodes and substitution nodes, we inductively define the quantity $\mathcal{J}_T$ by:
\begin{equation}\label{eq.treetermoperator.threewave}
    \mathcal{J}_T=
    \begin{cases}
    \xi, \qquad\qquad\quad\ \   \textit{ if $T$ has only one node $\star$.}
    \\
    w, \qquad\qquad\quad\ \textit{ if $T$ has only one node $\Box$.}
    \\
    \mathcal{T}^l(\mathcal{J}_{T_{\mathfrak{n}_1}}, \mathcal{J}_{T_{\mathfrak{n}_2}}), \textit{ if the root of $T$ is a substitution node with index $\tau$.}
    \\
    \mathcal{T}(\mathcal{J}_{T_{\mathfrak{n}_1}}, \mathcal{J}_{T_{\mathfrak{n}_2}}),\  \textit{ otherwise.}
    \end{cases}
\end{equation}
Here $\mathfrak{n}_1$, $\mathfrak{n}_2$ are two children of the root node $\mathfrak{r}$ and $T_{\mathfrak{n}_1}$, $T_{\mathfrak{n}_2}$ are the subtrees of $T$ rooted at above nodes.
\end{defn}

Then we have the following formula of $\prod_{j=1}^K\mathcal{P}^{\tau_j}_{T_j}(w)$

\begin{lem}
With above definition of $T_1\circ \cdots \circ T_{K}$ and $\mathcal{J}_T$, we have 
\begin{equation}\label{eq.operatoreqsimple.threewave}
    \prod_{j=1}^K\mathcal{P}^{\tau_j}_{T_j}(w)=\mathcal{J}_{T_1\circ \cdots \circ T_{K}}
\end{equation}
\end{lem}
\begin{proof}
This lemma follows from the above explanation.
\end{proof}

To get a good upper bound, we need a better formula of $\prod_{j=1}^K\mathcal{P}^{\tau_j}_{T_j}(w)$ which is an analog of Lemma \ref{lem.treeterms}. 

\begin{lem}\label{lem.treetermsoperator} (1) Using the same notation as Lemma \ref{lem.treeterms}. Given a tree $T$ with $\Box$ nodes and substitution nodes. Assume that the root $\mathfrak{r}$ is a substitution node of index $\tau_{1}$. Let $\mathcal{J}_T$ be terms defined in Definition \ref{def.treetermsoperator}, then their Fourier coefficients $\mathcal{J}_{T,k}$ are degree $l$ polynomials of $\xi$ and $w$ given by the following formula

\begin{equation}\label{eq.coeftermoperator.threewave}
\mathcal{J}_{T,k}=\left(\frac{i\lambda}{L^{d}}\right)^l\sum_{k_1,\, k_2,\, \cdots,\, k_{l+1}} \int_{\cup_{\mathfrak{n}\in T_{\text{in}}} A_{\mathfrak{n}}} e^{\sum_{\mathfrak{n}\in T_{\text{in}}} it_{\mathfrak{n}}\Omega_{\mathfrak{n}}-\nu(t_{\widehat{\mathfrak{n}}}-t_{\mathfrak{n}})|k_{\mathfrak{e}}|^2} \prod_{j=1}^{l+1} (\xi|w)_{k_j} %(\xi|w)_{k_1}(\xi|w)_{k_2}\cdots(\xi|w)_{k_{l+1}}    
\prod_{\mathfrak{n}\in T_{\text{in}}} dt_{\mathfrak{n}} 
\ \delta_{\cap_{\mathfrak{n}\in T_{\text{in}}} \{S_{\mathfrak{n}}=0\}}\ \prod_{\mathfrak{e}\in T_{\text{in}}}\iota_{\mathfrak{e}}k_{\mathfrak{e},x}
\end{equation}
Here $\iota$, $(\xi|w)_{k_j}$, $A_{\mathfrak{n}}$, $S_{\mathfrak{n}}$, $\Omega_{\mathfrak{n}}$ are defined by 
\begin{equation}
    \iota_{\mathfrak{e}}=\begin{cases}
        +1 \qquad \textit{if $\mathfrak{e}$ pointing inwards to $\mathfrak{n}$}
        \\
        -1 \qquad  \textit{if $\mathfrak{e}$ pointing outwards from $\mathfrak{n}$}
    \end{cases}
\end{equation}
\begin{equation}
    (\xi|w)_{k_j}=\begin{cases}
        \xi \qquad\  \textit{if $j$-th leaf node is a $\star$ node}
        \\
        w \qquad  \textit{if $j$-th leaf node is a $\Box$ node}
    \end{cases}
\end{equation}
\begin{equation}
    A_{\mathfrak{n}}=\left\{
    \begin{aligned}
        &\{t_{\mathfrak{r}}\le t, (t-t_{\mathfrak{r}})/T_{\text{max}}\in [2^{-\tau_{1}},2^{-\tau_{1}-1}]\} && \textit{if $\mathfrak{n}$ is the root $\mathfrak{r}$ }
        \\
        &\{t_{\mathfrak{n}_1},\, t_{\mathfrak{n}_2},\, t_{\mathfrak{n}_3}\le t_{\mathfrak{n}}\} && \textit{if $\mathfrak{n}\ne \mathfrak{r}$ and is not a substitution node}
        \\
        &\{t_{\mathfrak{n}_1},\, t_{\mathfrak{n}_2},\, t_{\mathfrak{n}_3}\le t_{\mathfrak{n}}, (t_{\widehat{\mathfrak{n}}}-t_{\mathfrak{n}})/T_{\text{max}}\in [2^{-\tau},2^{-\tau-1}]\}  &&\textit{if $\mathfrak{n}\ne \mathfrak{r}$ is an index $\tau$ substitution node}
    \end{aligned}\right.
\end{equation}

\begin{equation}\label{eq.defnS_noperator.threewave}
    S_{\mathfrak{n}}=\iota_{\mathfrak{e}_1}k_{\mathfrak{e}_1}+\iota_{\mathfrak{e}_2}k_{\mathfrak{e}_2}+\iota_{\mathfrak{e}}k_{\mathfrak{e}}
\end{equation}

\begin{equation}
    \Omega_{\mathfrak{n}}=\iota_{\mathfrak{e}_1}\Lambda_{k_{\mathfrak{e}_1}}+\iota_{\mathfrak{e}_2}\Lambda_{k_{\mathfrak{e}_2}}+\iota_{\mathfrak{e}}\Lambda_{k_{\mathfrak{e}}}
\end{equation}
For root node $\mathfrak{r}$, we impose the constrain that $k_{\mathfrak{r}}=k$. We also define $t_{\widehat{\mathfrak{r}}}=t$ to fix the problem that $\widehat{\mathfrak{r}}$ is not well defined because $\mathfrak{r}$ does not have a parent. 

(2) Define $T=T_1\circ \cdots \circ T_{K}$ and by definition there are only one $\Box$ leaf. Assume that there are $l+1$ leaves in $T$ and label all $\star$ leaves by $1$, $\cdots$, $l$, then $l=\sum_{j=1}^K l(T_j)$. As a corollary of (1), we have the following formula for Fourier coefficients of $\prod_{j=1}^K\mathcal{P}^{\tau_j}_{T_j}$.
\begin{equation}
    \left(\prod_{j=1}^K\mathcal{P}^{\tau_j}_{T_j}(w)\right)_{k}(t)=\sum_{k'}\int_0^t H^{\tau_1\cdots \tau_{K}}_{Tkk'}(t,s) w_{k'}(s) ds
\end{equation}
and the kernel $H^{\tau_1\cdots \tau_{K}}_{Tkk'}$ is given by
\begin{equation}
\begin{split}
H^{\tau_1\cdots \tau_{K}}_{Tkk'}(t,s)=\left(\frac{i\lambda}{L^{d}}\right)^l\sum_{k_1,\, k_2,\, \cdots,\, k_{l}} H^{\tau_1\cdots \tau_{K}}_{Tk_1\cdots k_{l}kk'} \xi_{k_1}\cdots \xi_{k_{l}}
\end{split}
\end{equation}
% \begin{equation}
% \begin{split}
% &H^{\tau_1\cdots \tau_{K}}_{Tkk'}(t,s)=
% \\
% &\left(\frac{i\lambda}{L^{d}}\right)^l\sum_{k_1,\, k_2,\, \cdots,\, k_{l}} \int_{\cup_{\mathfrak{n}\in T_{\text{in}}} B_{\mathfrak{n}}} e^{\sum_{\mathfrak{n}\in T_{\text{in}}} it_{\mathfrak{n}}\Omega_{\mathfrak{n}}-\nu(t_{\widehat{\mathfrak{n}}}-t_{\mathfrak{n}})|k_{\mathfrak{e}}|^2}  %(\xi|w)_{k_1}(\xi|w)_{k_2}\cdots(\xi|w)_{k_{l}}    
% \prod_{\mathfrak{n}\in T_{\text{in}}} dt_{\mathfrak{n}} 
% \ \delta_{\cap_{\mathfrak{n}\in T_{\text{in}}} \{S_{\mathfrak{n}}=0\}}\ \prod_{\mathfrak{e}\in T_{\text{in}}}\iota_{\mathfrak{e}}k_{\mathfrak{e},x}\prod_{j=1}^{l} \xi_{k_j}    
% \end{split}
% \end{equation}
and the coefficients $H^{\tau_1\cdots \tau_{K}}_{Tk_1\cdots k_{l}kk'}$ of the kernel is given by 
\begin{equation}
H^{\tau_1\cdots \tau_{K}}_{Tk_1\cdots k_{l}kk'}(t,s)=\int_{\cup_{\mathfrak{n}\in T_{\text{in}}} B_{\mathfrak{n}}} e^{\sum_{\mathfrak{n}\in T_{\text{in}}} it_{\mathfrak{n}}\Omega_{\mathfrak{n}}-\nu(t_{\widehat{\mathfrak{n}}}-t_{\mathfrak{n}})|k_{\mathfrak{e}}|^2}  %(\xi|w)_{k_1}(\xi|w)_{k_2}\cdots(\xi|w)_{k_{l}}    
\prod_{\mathfrak{n}\in T_{\text{in}}} dt_{\mathfrak{n}} 
\ \delta_{\cap_{\mathfrak{n}\in T_{\text{in}}} \{S_{\mathfrak{n}}=0\}}\ \prod_{\mathfrak{e}\in T_{\text{in}}}\iota_{\mathfrak{e}}k_{\mathfrak{e},x}
\end{equation}
Here $k_1,\cdots,k_{l}$ are all variables corresponding to $\star$ leaves, $k'$ is the variable corresponding to the $\Box$ node and $B_{\mathfrak{n}}$ is defined by 
\begin{equation}
    B_{\mathfrak{n}}=\left\{
    \begin{aligned}
        &\{t_{\mathfrak{r}}\le t, (t-t_{\mathfrak{r}})/T_{\text{max}}\in [2^{-\tau_{1}},2^{-\tau_{1}-1}]\} && \textit{if $\mathfrak{n}$ is the root $\mathfrak{r}$ }
        \\
        &\{t_{\mathfrak{n}}\ge s\} && \textit{if $\mathfrak{n}$ is a parent of the $\Box$ nodes}
        \\
        &\  &&\textit{and is not a substitution nodes}
        \\
        &\{t_{\mathfrak{n}}\ge s, (t_{\widehat{\mathfrak{n}}}-t_{\mathfrak{n}})/T_{\text{max}}\in [2^{-\tau},2^{-\tau-1}]\} && \textit{if $\mathfrak{n}$ is a parent of the $\Box$ nodes}
        \\
        &\  &&\textit{and is a substitution nodes of index $\tau$}
        \\
        &\{t_{\mathfrak{n}_1},\, t_{\mathfrak{n}_2},\, t_{\mathfrak{n}_3}\le t_{\mathfrak{n}}, (t_{\widehat{\mathfrak{n}}}-t_{\mathfrak{n}})/T_{\text{max}}\in [2^{-\tau},2^{-\tau-1}]\}  &&\textit{if not in the first three cases}
        \\
        &\  &&\textit{and $\mathfrak{n}$ is a substitution node of index $\tau$}
        \\
        &\{t_{\mathfrak{n}_1},\, t_{\mathfrak{n}_2},\, t_{\mathfrak{n}_3}\le t_{\mathfrak{n}}\} && \textit{otherwise}
    \end{aligned}\right.
\end{equation}
\end{lem}
\begin{proof}
Lemma \ref{lem.treetermsoperator} (1) can be proved by the same method as Lemma \ref{lem.treeterms}. We can check that $\mathcal{J}_T$ is defined by \eqref{eq.coefterm.threewave} and \eqref{eq.coef.threewave} satisfies the recursive formula \eqref{eq.treeterm.threewave} by direct substitution, so they are the unique solution of that recursive formula, and this proves (1).

(2) is a corollary of (1).
\end{proof}

We can also calculate $\prod_{j=1}^K\mathcal{P}^{\tau_j}_{T_j}(\mathcal{J}_{T})$, replacing $w$ by $\mathcal{J}_{T}$ corresponds to replacing a $\Box$ node by a tree $T$, so $\prod_{j=1}^K\mathcal{P}^{\tau_j}_{T_j}(\mathcal{J}_{T})=\mathcal{J}_{T_1\circ T_2\circ \cdots\circ T_K\circ T}$. Since $T$ does not contain $\Box$ node, so does $T_1\circ T_2\circ \cdots\circ T_K\circ T$ and $\mathcal{J}_{T_1\circ T_2\circ \cdots\circ T_K\circ T}$ is a polynomial of Gaussian variables $\xi_k$. Then we have the following lemma
\begin{lem}
We have 
\begin{equation}\label{eq.operatoreqsimpleJ_T.threewave}
    \prod_{j=1}^K\mathcal{P}^{\tau_j}_{T_j}(\mathcal{J}_{T})=\mathcal{J}_{T_1\circ T_2\circ \cdots\circ T_K\circ T}
\end{equation}
\end{lem}
\begin{proof}
This lemma follows from the above explanation.
\end{proof}



\subsubsection{The upper bound for coefficients} In this section, we prove the Lemma \ref{lem.boundcoefoperator} which gives an upper bound for $H^{\tau_1\cdots \tau_{K}}_{Tk_1\cdots k_{l}kk'}$. This lemma is an analog of Lemma \ref{lem.boundcoef}.

\begin{lem}\label{lem.boundcoefoperator}
Assume that $|k_1|, \cdots, |k_{l}|\lesssim 1$, then for $t\le  T_{\text{max}}$, we have the following upper bound for $H^{\tau_1\cdots \tau_{K}}_{Tk_1\cdots k_{l}kk'}$,
\begin{equation}\label{eq.boundcoefoperator.threewave}
    |H^{\tau_1\cdots \tau_{K}}_{Tk_1\cdots k_{l}kk'}|\lesssim \sum_{\{d_{\mathfrak{n}}\}_{\mathfrak{n}\in T_{\text{in}}}\in\{0,1\}^{l(T)}}\prod_{\mathfrak{n}\in T_{\text{in}}}\frac{2^{-\frac{\tau_{\mathfrak{n}}}{2}}}{|q_{\mathfrak{n}}|+T^{-1}_{\text{max}}}\ \delta_{\cap_{\mathfrak{n}\in T_{\text{in}}} \{S_{\mathfrak{n}}=0\}} \prod_{\mathfrak{e}\in T_{\text{in}}} p_{\mathfrak{e}}.
\end{equation}
Here $\tau_{\mathfrak{n}}$ is defined by 
\begin{equation}
   \tau_{\mathfrak{n}}=\left\{
    \begin{aligned}
        &0 && \textit{if $\mathfrak{n}$ is not a substitution node}
        \\
        &\tau_j  &&\textit{if $\mathfrak{n}$ is the $j$-th substitution node}
    \end{aligned}\right.
\end{equation}
Fix a sequence $\{d_{\mathfrak{n}}\}_{\mathfrak{n}\in T_{\text{in}}}$ whose elements $d_{\mathfrak{n}}$ takes boolean values $\{0,1\}$. We define the sequences $\{p_{\mathfrak{n}}\}_{\mathfrak{n}\in T_{\text{in}}}$, $\{q_{\mathfrak{n}}\}_{\mathfrak{n}\in T_{\text{in}}}$, $\{r_{\mathfrak{n}}\}_{\mathfrak{n}\in T_{\text{in}}}$ by following formulas
\begin{equation}\label{eq.p_noperator.threewave}
    p_{\mathfrak{e}}=\frac{|k_{\mathfrak{e},x}|}{|k_{\mathfrak{e},x}|+1}
\end{equation}
\begin{equation}\label{eq.q_noperator.threewave}
    q_{\mathfrak{n}}=
    \begin{cases}
    \Omega_{\mathfrak{r}}, \qquad\qquad \textit{ if $\mathfrak{n}=$ the root $\mathfrak{r}$.}
    \\
    \Omega_{\mathfrak{n}}+d_{\mathfrak{n}}q_{\mathfrak{n}'},\ \ \textit{ if $\mathfrak{n}\neq\mathfrak{r}$ and $\mathfrak{n}'$ is the parent of $\mathfrak{n}$.}
    \end{cases}
\end{equation}
\begin{equation}\label{eq.r_noperator.threewave}
    r_{\mathfrak{n}}=
    \begin{cases}
    |k_{\mathfrak{r}}|^2, \qquad\qquad \textit{ if $\mathfrak{n}=$ the root $\mathfrak{r}$.}
    \\
    |k_{\mathfrak{n}}|^2+d_{\mathfrak{n}}q_{\mathfrak{n}'},\ \ \textit{ if $\mathfrak{n}\neq\mathfrak{r}$ and $\mathfrak{n}'$ is the parent of $\mathfrak{n}$.}
    \end{cases}
\end{equation}


\end{lem}
\begin{proof} By definition
\begin{equation}
H^{\tau_1\cdots \tau_{K}}_{Tk_1\cdots k_{l}kk'}(t,s)=\int_{\cup_{\mathfrak{n}\in T_{\text{in}}} B_{\mathfrak{n}}} e^{\sum_{\mathfrak{n}\in T_{\text{in}}} it_{\mathfrak{n}}\Omega_{\mathfrak{n}}-\nu(t_{\widehat{\mathfrak{n}}}-t_{\mathfrak{n}})|k_{\mathfrak{e}}|^2}  %(\xi|w)_{k_1}(\xi|w)_{k_2}\cdots(\xi|w)_{k_{l}}    
\prod_{\mathfrak{n}\in T_{\text{in}}} dt_{\mathfrak{n}} 
\ \delta_{\cap_{\mathfrak{n}\in T_{\text{in}}} \{S_{\mathfrak{n}}=0\}}\ \prod_{\mathfrak{e}\in T_{\text{in}}}\iota_{\mathfrak{e}}k_{\mathfrak{e},x}
\end{equation}

For any edge $\mathfrak{e}$, assume that the two end points of $\mathfrak{e}$ are $\mathfrak{n}_1$ and $\mathfrak{n}_2$. If neither $\mathfrak{n}_1$ and $\mathfrak{n}_2$ is a substitution node, then we claim that $|k_{\mathfrak{e}}|\lesssim 1$. 

This is because if no endpoint of $\mathfrak{e}$ is a substitution node, then the subtree $T_{\mathfrak{e}}$ rooted at the upper endpoints of $\mathfrak{e}$ does not contain the $\Box$ node as its leaf. Therefore, the momentum conservation (Lemma \ref{lem.freeleg}) is applicable which implies that $k_{\mathfrak{e}}$ is a linear combination of $k_1,\cdots,k_{l}$. By $|k_1|, \cdots, |k_{l}|\lesssim 1$, we get $|k_{\mathfrak{e}}|\lesssim 1$. 

Since $|k_{\mathfrak{e}}|\lesssim 1$, we get $\left|\prod_{\mathfrak{e}\in T_{\text{in}}}\iota_{\mathfrak{e}}k_{\mathfrak{e},x}\right|\lesssim \prod^K_{j=1}|k_{s_j,x}|$. Here $s_1, \cdots, s_K$ are all substitution nodes and $k_{s_j}$ are corresponding variables of edges pointing towards $s_j$.


As in section \ref{sec.uppcoef}, we define 
\begin{equation}\label{eq.defF_Toperator.threewave}
F_{T}(t,\{a_{\mathfrak{n}}\}_{\mathfrak{n}\in T_{\text{in}}},\{b_{\mathfrak{n}}\}_{\mathfrak{n}\in T_{\text{in}}})=\int_{\cup_{\mathfrak{n}\in T_{\text{in}}} B_{\mathfrak{n}}} e^{\sum_{\mathfrak{n}\in T_{\text{in}}} it_{\mathfrak{n}} a_{\mathfrak{n}} - \nu(t_{\widehat{\mathfrak{n}}}-t_{\mathfrak{n}})b_{\mathfrak{n}}} \prod_{\mathfrak{n}\in T_{\text{in}}} dt_{\mathfrak{n}} \prod^K_{j=1}|k_{s_j,x}|
\end{equation}

Keeping notation the same as Lemma \ref{lem.boundcoef'}, if we can show that
\begin{equation}\label{eq.lemboundop1.threewave}
    |F_{T}(t,\{a_{\mathfrak{n}}\}_{\mathfrak{n}\in T_{\text{in}}},\{b_{\mathfrak{n}}\}_{\mathfrak{n}\in T_{\text{in}}})|\lesssim\sum_{\{d_{\mathfrak{n}}\}_{\mathfrak{n}\in T_{\text{in}}}\in\{0,1\}^{l(T)}}\prod_{\mathfrak{n}\in T_{\text{in}}}\frac{2^{-\frac{\tau_{\mathfrak{n}}}{2}}}{|q_{\mathfrak{n}}|+T^{-1}_{\text{max}}}\prod_{\mathfrak{e}\in T_{\text{in}}} p_{\mathfrak{e}}
,
\end{equation}
then this lemma can be proved by taking $a_{\mathfrak{n}}=\Omega_{\mathfrak{n}}$, $b_{\mathfrak{n}}=|k_{\mathfrak{e}}|^2$ in \eqref{eq.lemboundop1.threewave}.

Therefore, it suffices to prove \eqref{eq.lemboundop1.threewave}.

We run a similar inductive integration by parts argument of Lemma \ref{lem.boundcoef'}. If the roots $\mathfrak{r}$ is not a substitution node, then the same argument of Lemma \ref{lem.boundcoef'} works. Therefore, we just consider the case that the roots $\mathfrak{r}$ is a substitution node. 

Using the same calculation as \eqref{eq.lemboundcoef'1.threewave}, we get 
\begin{equation}
\begin{split}
    F_{T}(t)=\int_{\cup_{\mathfrak{n}\in T_{\text{in}}} B_{\mathfrak{n}}}&e^{it_{\mathfrak{r}}(a_{\mathfrak{r}}+T^{-1}_{\text{max}}\, \text{sgn}(a_{\mathfrak{r}}))- \nu(t-t_{\mathfrak{r}})b_{\mathfrak{r}}} e^{-iT^{-1}_{\text{max}}t_{\mathfrak{r}} \text{sgn}(a_{\mathfrak{r}})} 
    \\
    &e^{\sum_{\mathfrak{n}\in T_{\text{in},1}\cup T_{\text{in},2}} it_{\mathfrak{n}} a_{\mathfrak{n}} - \nu(t_{\widehat{\mathfrak{n}}}-t_{\mathfrak{n}})b_{\mathfrak{n}}}  \left(dt_{\mathfrak{r}}\prod_{j=1}^2\prod_{\mathfrak{n}\in T_{\text{in},j}}dt_{\mathfrak{n}}  \right)|k_{s_1,x}|\prod^K_{j=2}|k_{s_j,x}|
\end{split}
\end{equation}



We do integration by parts in the above integrals using the Stokes formula. Notice that for $t_{\mathfrak{r}}$, there are four inequality constrains, $t_{\mathfrak{r}}\le t-2^{-\tau_{1}}T_{\text{max}}$, $t_{\mathfrak{r}}\ge t-2^{-\tau_{1}-1}T_{\text{max}}$ and $t_{\mathfrak{r}}\ge t_{\mathfrak{n}_1},t_{\mathfrak{n}_2}$. Notice that the first two come from $(t-t_{\mathfrak{r}})/T_{\text{max}}\in [2^{-\tau_{1}},2^{-\tau_{1}-1}]$.


\begin{equation}\label{eq.lemboundcoefexpandop.threewave}
\begin{split}
    F_{T}(t)=&\frac{|k_{s_1,x}|}{ia_{\mathfrak{r}}+iT^{-1}_{\text{max}} \text{sgn}(a_{\mathfrak{r}})+\nu b_{\mathfrak{r}} }\int_{\cup_{\mathfrak{n}\in T_{\text{in}}} B_{\mathfrak{n}}} \frac{d}{dt_{\mathfrak{r}}}e^{it_{\mathfrak{r}}(a_{\mathfrak{r}}+T^{-1}_{\text{max}}\, \text{sgn}(a_{\mathfrak{r}}))- \nu(t-t_{\mathfrak{r}})b_{\mathfrak{r}}}  
    \\
    &\qquad\qquad\qquad\ \  e^{-iT^{-1}_{\text{max}}t_{\mathfrak{r}} \text{sgn}(a_{\mathfrak{r}})} e^{\sum_{\mathfrak{n}\in T_{\text{in},1}\cup T_{\text{in},2}} it_{\mathfrak{n}} a_{\mathfrak{n}} - \nu(t_{\widehat{\mathfrak{n}}}-t_{\mathfrak{n}})b_{\mathfrak{n}}}  \left(dt_{\mathfrak{r}}\prod_{j=1}^2\prod_{\mathfrak{n}\in T_{\text{in},j}}dt_{\mathfrak{n}}  \right)\prod^K_{j=2}|k_{s_j,x}|
\end{split}
\end{equation}
\begin{flalign*}
\hspace{1.3cm}
=&\frac{|k_{s_1,x}|}{ia_{\mathfrak{r}}+iT^{-1}_{\text{max}} \text{sgn}(a_{\mathfrak{r}})+\nu b_{\mathfrak{r}} }&&
\\
&\left(\int_{\cup_{\mathfrak{n}\in T_{\text{in}}} B_{\mathfrak{n}},\ t_{\mathfrak{r}}=t-2^{-\tau_{1}}T_{\text{max}}}
-\int_{\cup_{\mathfrak{n}\in T_{\text{in}}} B_{\mathfrak{n}},\ t_{\mathfrak{r}}=t-2^{-\tau_{1}-1}T_{\text{max}}}
-\int_{\cup_{\mathfrak{n}\in T_{\text{in}}} B_{\mathfrak{n}},\ t_{\mathfrak{r}}=t_{\mathfrak{n}_1}}
-\int_{\cup_{\mathfrak{n}\in T_{\text{in}}} B_{\mathfrak{n}},\ t_{\mathfrak{r}}=t_{\mathfrak{n}_2}}\right) &&
\\
& e^{it_{\mathfrak{r}}(a_{\mathfrak{r}}+T^{-1}_{\text{max}}\, \text{sgn}(a_{\mathfrak{r}}))- \nu(t-t_{\mathfrak{r}})b_{\mathfrak{r}}} e^{-iT^{-1}_{\text{max}}t_{\mathfrak{r}} \text{sgn}(a_{\mathfrak{r}})} e^{\sum_{\mathfrak{n}\in T_{\text{in},1}\cup T_{\text{in},2}} it_{\mathfrak{n}} a_{\mathfrak{n}} - \nu(t_{\widehat{\mathfrak{n}}}-t_{\mathfrak{n}})b_{\mathfrak{n}}} \left(dt_{\mathfrak{r}}\prod_{j=1}^2\prod_{\mathfrak{n}\in T_{\text{in},j}}dt_{\mathfrak{n}}  \right)\prod^K_{j=2}|k_{s_j,x}| &&
\end{flalign*}
\begin{flalign*}
\hspace{1.3cm}
-&\frac{|k_{s_1,x}|}{ia_{\mathfrak{r}}+iT^{-1}_{\text{max}} \text{sgn}(a_{\mathfrak{r}})+\nu b_{\mathfrak{r}} }\int_{\cup_{\mathfrak{n}\in T_{\text{in}}} B_{\mathfrak{n}}}e^{it_{\mathfrak{r}}(a_{\mathfrak{r}}+T^{-1}_{\text{max}}\, \text{sgn}(a_{\mathfrak{r}}))- \nu(t-t_{\mathfrak{r}})b_{\mathfrak{r}}} &&
    \\
    &\qquad\qquad \frac{d}{dt_{\mathfrak{r}}}(e^{-iT^{-1}_{\text{max}}t_{\mathfrak{r}} \text{sgn}(a_{\mathfrak{r}})}) e^{\sum_{\mathfrak{n}\in T_{\text{in},1}\cup T_{\text{in},2}} it_{\mathfrak{n}} a_{\mathfrak{n}} - \nu(t_{\widehat{\mathfrak{n}}}-t_{\mathfrak{n}})b_{\mathfrak{n}}}  \left(dt_{\mathfrak{r}}\prod_{j=1}^2\prod_{\mathfrak{n}\in T_{\text{in},j}}dt_{\mathfrak{n}}  \right)\prod^K_{j=2}|k_{s_j,x}| &&
\end{flalign*}
\begin{flalign*}
\hspace{1.3cm}
= \frac{|k_{s_1,x}|}{ia_{\mathfrak{r}}+iT^{-1}_{\text{max}} \text{sgn}(a_{\mathfrak{r}})+\nu b_{\mathfrak{r}} }(F_{I}-F_{I'}-\widetilde{F}_{T^{(1)}}-\widetilde{F}_{T^{(2)}}-F_{II}) &&
\end{flalign*}

We now derive upper bounds for $F_{I}$, $F_{I'}$, $\widetilde{F}_{T^{(1)}}$, $\widetilde{F}_{T^{(2)}}$, $F_{II}$.

The argument of $F_{I}$ and $F_{I'}$ is very similar, so we just consider $F_{I}$. By a direct calculation, we know that $\frac{|k_{s_1,x}|}{ia_{\mathfrak{r}}+iT^{-1}_{\text{max}} \text{sgn}(a_{\mathfrak{r}})+\nu b_{\mathfrak{r}} }F_{I}(t)$ equals to
\begin{equation}
    \begin{split}
        &\frac{|k_{s_1,x}|}{ia_{\mathfrak{r}}+iT^{-1}_{\text{max}} \text{sgn}(a_{\mathfrak{r}})+\nu b_{\mathfrak{r}} } \int_{\cup_{\mathfrak{n}\in T_{\text{in}}} B_{\mathfrak{n}},\ t_{\mathfrak{r}}=t-2^{-\tau_{1}}T_{\text{max}}} e^{it_{\mathfrak{r}}(a_{\mathfrak{r}}+T^{-1}_{\text{max}}\, \text{sgn}(a_{\mathfrak{r}}))- \nu(t-t_{\mathfrak{r}})b_{\mathfrak{r}}} e^{-iT^{-1}_{\text{max}}t_{\mathfrak{r}} \text{sgn}(a_{\mathfrak{r}})} 
        \\
        &e^{\sum_{\mathfrak{n}\in T_{\text{in},1}\cup T_{\text{in},2}} it_{\mathfrak{n}} a_{\mathfrak{n}} - \nu(t_{\widehat{\mathfrak{n}}}-t_{\mathfrak{n}})b_{\mathfrak{n}}} \left(dt_{\mathfrak{r}}\prod_{j=1}^2\prod_{\mathfrak{n}\in T_{\text{in},j}}dt_{\mathfrak{n}}  \right)\prod^K_{j=2}|k_{s_j,x}|
        \\
        =&\frac{|k_{s_1,x}|}{ia_{\mathfrak{r}}+iT^{-1}_{\text{max}} \text{sgn}(a_{\mathfrak{r}})+\nu b_{\mathfrak{r}} }  e^{i(t-2^{-\tau_{1}}T_{\text{max}})a_{\mathfrak{r}}- \nu T_{\text{max}} 2^{-\tau_{1}}b_{\mathfrak{r}}}\int_{\cup_{\mathfrak{n}\in T_{\text{in},1}\cup T_{\text{in},2}} B_{\mathfrak{n}},\ t_{\mathfrak{r}_1}, t_{\mathfrak{r}_2}\lesssim t-2^{-\tau_{1}}T_{\text{max}}}
        \\
        &  e^{\sum_{\mathfrak{n}\in T_{\text{in},1}\cup T_{\text{in},2}} it_{\mathfrak{n}} a_{\mathfrak{n}} - \nu(t_{\widehat{\mathfrak{n}}}-t_{\mathfrak{n}})b_{\mathfrak{n}}}\left(dt_{\mathfrak{r}}\prod_{j=1}^2\prod_{\mathfrak{n}\in T_{\text{in},j}}dt_{\mathfrak{n}}  \right)\prod^K_{j=2}|k_{s_j,x}|
        \\
        =&O\left(\left|\frac{|k_{s_1,x}|}{ia_{\mathfrak{r}}+iT^{-1}_{\text{max}} \text{sgn}(a_{\mathfrak{r}})+\nu b_{\mathfrak{r}} }  e^{i(t-2^{-\tau_{1}}T_{\text{max}})a_{\mathfrak{r}}- \nu T_{\text{max}} 2^{-\tau_{1}}b_{\mathfrak{r}}}\right||F_{T_1}(t)| |F_{T_2}(t)|\right)
        \\
        =&O\left(\frac{(|k_{s_1,x}|+1)e^{- 2^{-\tau_{1}}|k_{s_1}|^2} }{|q_{\mathfrak{r}}|+T^{-1}_{\text{max}}} |F_{T_1}(t)| |F_{T_2}(t)|\frac{|k_{s_1,x}|}{|k_{s_1,x}|+1}\right)=O\left(\frac{2^{-\frac{\tau_{1}}{2}}|F_{T_1}(t)| |F_{T_2}(t)|}{|q_{\mathfrak{r}}|+T^{-1}_{\text{max}}}p_{\mathfrak{e}}\right)
    \end{split}
\end{equation}
Here in the last line we use the fact that $b_{\mathfrak{r}}=|k_{s_1}|^2$, $\nu T_{\text{max}}\gtrsim 1$ \eqref{eq.conditionnu.threewave} and $(|k_{s_1,x}|+1)e^{- 2^{-\tau_{1}}|k_{s_1}|^2} \lesssim 2^{-\frac{\tau_{1}}{2}}$. By the above equation and the induction assumption, we know that $\frac{|k_{s_1,x}|}{ia_{\mathfrak{r}}+iT^{-1}_{\text{max}} \text{sgn}(a_{\mathfrak{r}})+\nu b_{\mathfrak{r}} }F_{I}(t)$ can be bounded by the right hand side of \eqref{eq.boundcoefoperator.threewave}.

Now we find upper bounds of $\widetilde{F}_{T^{(1)}}$ and $\widetilde{F}_{T^{(2)}}$. Let $T^{(j)}$, $j=1,2$ are trees that are obtained by deleting the root $\mathfrak{r}$, adding edges connecting $\mathfrak{n}_j$ with another node and defining $\mathfrak{n}_j$ to be the new root. For $T^{(j)}$, we can define the term $F_{T^{(j)}}$ by \eqref{eq.defF_Toperator.threewave}. By a direct calculation, we know that $\frac{|k_{s_1,x}|}{ia_{\mathfrak{r}}+iT^{-1}_{\text{max}} \text{sgn}(a_{\mathfrak{r}})+\nu b_{\mathfrak{r}} }\widetilde{F}_{T^{(1)}}(t)$ equals to
\begin{equation}
    \begin{split}
        &\frac{|k_{s_1,x}|}{ia_{\mathfrak{r}}+iT^{-1}_{\text{max}} \text{sgn}(a_{\mathfrak{r}})+\nu b_{\mathfrak{r}} } \int_{\cup_{\mathfrak{n}\in T_{\text{in}}} B_{\mathfrak{n}},\ t_{\mathfrak{r}}=t_{\mathfrak{n}_j}} e^{it_{\mathfrak{r}}(a_{\mathfrak{r}}+T^{-1}_{\text{max}}\, \text{sgn}(a_{\mathfrak{r}}))- \nu(t-t_{\mathfrak{r}})b_{\mathfrak{r}}} e^{-iT^{-1}_{\text{max}}t_{\mathfrak{r}} \text{sgn}(a_{\mathfrak{r}})} 
        \\
        &e^{\sum_{\mathfrak{n}\in T_{\text{in},1}\cup T_{\text{in},2}} it_{\mathfrak{n}} a_{\mathfrak{n}} - \nu(t_{\widehat{\mathfrak{n}}}-t_{\mathfrak{n}})b_{\mathfrak{n}}} \left(dt_{\mathfrak{r}}\prod_{j=1}^2\prod_{\mathfrak{n}\in T_{\text{in},j}}dt_{\mathfrak{n}}  \right)\prod^K_{j=2}|k_{s_j,x}|
        \\
        =&\frac{|k_{s_1,x}|}{ia_{\mathfrak{r}}+iT^{-1}_{\text{max}} \text{sgn}(a_{\mathfrak{r}})+\nu b_{\mathfrak{r}} }  e^{- \nu T_{\text{max}} 2^{-\tau_{1}}b_{\mathfrak{r}}}\int_{\cup_{\mathfrak{n}\in T_{\text{in}}} B_{\mathfrak{n}}} e^{it_{\mathfrak{n}_j}(a_{\mathfrak{r}}+T^{-1}_{\text{max}}\, \text{sgn}(a_{\mathfrak{r}}))- \nu((t-T_{\text{max}} 2^{-\tau_{1}})-t_{\mathfrak{n}_j})b_{\mathfrak{r}}}
        \\
        &  e^{\sum_{\mathfrak{n}\in T_{\text{in},1}\cup T_{\text{in},2}} it_{\mathfrak{n}} a_{\mathfrak{n}} - \nu(t_{\widehat{\mathfrak{n}}}-t_{\mathfrak{n}})b_{\mathfrak{n}}}\left(dt_{\mathfrak{r}}\prod_{j=1}^2\prod_{\mathfrak{n}\in T_{\text{in},j}}dt_{\mathfrak{n}}  \right)\prod^K_{j=2}|k_{s_j,x}|
        \\
        =&O\left(\frac{|k_{s_1,x}|e^{- 2^{-\tau_{1}}|k_{s_1}|^2} }{|q_{\mathfrak{r}}|+T^{-1}_{\text{max}}} |F_{T^{(j)}}(t)|\right)=O\left(\frac{2^{-\frac{\tau_{1}}{2}}|F_{T^{(j)}}(t)| }{|q_{\mathfrak{r}}|+T^{-1}_{\text{max}}}p_{\mathfrak{e}}\right)
    \end{split}
\end{equation}
We can apply the induction assumption to $F_{T^{(j)}}$ and show that $\frac{|k_{s_1,x}|}{ia_{\mathfrak{r}}+iT^{-1}_{\text{max}} \text{sgn}(a_{\mathfrak{r}})+\nu b_{\mathfrak{r}} } F_{T^{(j)}}$ can be bounded by the right hand side of \eqref{eq.boundcoefoperator.threewave}.


Another direct calculation gives that 
\begin{equation}
    F_{II}(t)=\int_{(t-t_{\mathfrak{r}})/T_{\text{max}}\in [2^{-\tau_{1}},2^{-\tau_{1}-1}]}  e^{it_{\mathfrak{r}}(a_{\mathfrak{r}}+T^{-1}_{\text{max}}\, \text{sgn}(a_{\mathfrak{r}}))- \nu(t-t_{\mathfrak{r}})b_{\mathfrak{r}}} \frac{d}{dt_{\mathfrak{r}}}(e^{-iT^{-1}_{\text{max}}t_{\mathfrak{r}} \text{sgn}(a_{\mathfrak{r}})})  F_{T_1}(t_{\mathfrak{r}})F_{T_2}(t_{\mathfrak{r}}) dt_{\mathfrak{r}}.
\end{equation}
Apply the induction assumption
\begin{equation}
\begin{split}
    &\left| \frac{|k_{s_1,x}|}{ia_{\mathfrak{r}}+iT^{-1}_{\text{max}} \text{sgn}(a_{\mathfrak{r}})+\nu b_{\mathfrak{r}} } F_{II}(t)\right|
    \\
    \le &\frac{T^{-1}_{\text{max}}|k_{s_1,x}|}{|q_{\mathfrak{r}}|+T^{-1}_{\text{max}}}\int_{(t-t_{\mathfrak{r}})/T_{\text{max}}\in [2^{-\tau_{1}},2^{-\tau_{1}-1}]}  e^{- \nu(t-t_{\mathfrak{r}})b_{\mathfrak{r}}}   |F_{T_1}(t_{\mathfrak{r}})| |F_{T_2}(t_{\mathfrak{r}})| dt_{\mathfrak{r}}
    \\
    \le& \frac{T^{-1}_{\text{max}} |k_{s_1,x}| e^{- \nu T_{\text{max}} 2^{-\tau_{1}}|k_{s_1}|^2}}{|q_{\mathfrak{r}}|+T^{-1}_{\text{max}}}\prod_{j=1}^2\left(\sum_{\{d_{\mathfrak{n}}\}_{\mathfrak{n}\in T_{\text{in},j}}\in\{0,1\}^{l(T_j)}}\prod_{\mathfrak{n}\in T_{\text{in},j}}\frac{2^{-\frac{\tau_{\mathfrak{n}}}{2}}}{|q_{\mathfrak{n}}|+T^{-1}_{\text{max}}}\prod_{\mathfrak{e}\in T_{\text{in},j}} p_{\mathfrak{e}}\right)
    \\
    \le& \sum_{\{d_{\mathfrak{n}}\}_{\mathfrak{n}\in T_{\text{in}}}\in\{0,1\}^{l(T)}}\prod_{\mathfrak{n}\in T_{\text{in}}}\frac{2^{-\frac{\tau_{\mathfrak{n}}}{2}}}{|q_{\mathfrak{n}}|+T^{-1}_{\text{max}}}\prod_{\mathfrak{e}\in T_{\text{in}}} p_{\mathfrak{e}}.
\end{split}
\end{equation}

Therefore, we get an upper bound for $F_{II}$.

Combining the bounds of $F_{I}$, $F_{I'}$, $\widetilde{F}_{T^{(1)}}$, $\widetilde{F}_{T^{(2)}}$, $F_{II}$, we conclude that $F_T$ can be bounded by the right hand side of \eqref{eq.boundcoefoperator.threewave} and thus complete the proof of Lemma \ref{lem.boundcoef'}.
\end{proof}

\subsubsection{The upper bound for expectation of entries} In this section, we prove the Proposition \ref{prop.treetermsvarianceoperator} which gives an upper bound for $\mathbb{E}|H^{\tau_1\cdots \tau_{K}}_{Tkk'}|^2$. This lemma is an analog of Proposition \ref{prop.treetermsvariance}.

\begin{prop}\label{prop.treetermsvarianceoperator}
Assume that $\alpha$ satisfies \eqref{eq.conditionalpha.threewave}. For any $\theta>0$, we have
\begin{equation}
    \sup_k\, \mathbb{E}|H^{\tau_1\cdots \tau_{K}}_{Tkk'}|^2\lesssim L^{O(l(T)\theta)} 2^{-\frac{1}{2}\sum_{j=1}^K \tau_{j}} \rho^{2l(T)}.
\end{equation}
and $\mathbb{E}|H^{\tau_1\cdots \tau_{K}}_{Tkk'}|^2=0$ if $|k-k'|\gtrsim 1$.
\end{prop}

\begin{proof} We first find a formula of $\mathbb{E}|H^{\tau_1\cdots \tau_{K}}_{Tkk'}|^2$ which is similar to \eqref{eq.termTp.threewave} and \eqref{eq.termexp.threewave}.

A direct calculation gives
\begin{equation}\label{eq.termexp1op.threewave}
\begin{split}
    \mathbb{E}|H^{\tau_1\cdots \tau_{K}}_{Tkk'}|^2=&\mathbb{E}(H^{\tau_1\cdots \tau_{K}}_{Tkk'}\overline{H^{\tau_1\cdots \tau_{K}}_{Tkk'}})=\left(\frac{\lambda}{L^{d}}\right)^{2l(T)}
    \sum_{k_1,\, k_2,\, \cdots,\, k_{l(T)}}\sum_{k'_1,\, k'_2,\, \cdots,\, k'_{l(T)}}
    \\[0.5em]
    & H^{\tau_1\cdots \tau_{K}}_{Tk_1\cdots k_{l(T)}kk'} \overline{H^{\tau_1\cdots \tau_{K}}_{Tk_1'\cdots k_{l(T)}'kk'}}  \mathbb{E}\Big(\xi_{k_1}\xi_{k_2}\cdots\xi_{k_{l(T)}}\xi_{k'_1}\xi_{k'_2}\cdots\xi_{k'_{l(T)}}\Big)
\end{split}
\end{equation}

Applying Wick theorem (Lemma \ref{th.wick}) to \eqref{eq.termexp1op.threewave}, we get
\begin{equation}\label{eq.termexpop.threewave}
\mathbb{E}|H^{\tau_1\cdots \tau_{K}}_{Tkk'}|^2=\left(\frac{\lambda}{L^{d}}\right)^{2l(T)}
    \sum_{p\in \mathcal{P}(\{k_1,\cdots, k_{l(T)}, k'_1,\cdots, k'_{l(T)}\})} Term(T, p)_{op,k,k
    '}.
\end{equation}
and
\begin{equation}\label{eq.termTpop.threewave}
\begin{split}
    &Term(T, p)_{op,k,k
    '}
    \\
    =&\sum_{k_1,\, k_2,\, \cdots,\, k_{l(T)}}\sum_{k'_1,\, k'_2,\, \cdots,\, k'_{l(T)}} H^{\tau_1\cdots \tau_{K}}_{Tk_1\cdots k_{l(T)}kk'} \overline{H^{\tau_1\cdots \tau_{K}}_{Tk_1'\cdots k_{l(T)}'kk'}} \delta_{p}(k_1,\cdots, k_{l(T)}, k'_1,\cdots, k'_{l(T)})\sqrt{n_{\textrm{in}}(k_1)}\cdots.
\end{split}
\end{equation}

Since $\alpha=\frac{\lambda}{L^{\frac{d}{2}}}$, $\frac{\lambda}{L^{d}}=\alpha L^{-\frac{d}{2}}$. Since the number of elements in $\mathcal{P}$ can be bounded by a constant, by Lemma \ref{lem.Tpvarianceop} proved below, we get
\begin{equation}
\begin{split}
    \mathbb{E}|H^{\tau_1\cdots \tau_{K}}_{Tkk'}|^2\lesssim& (\alpha L^{-\frac{d}{2}})^{2l(T)}
    L^{O(l(T)\theta)}2^{-\frac{1}{2}\sum_{j=1}^K \tau_{j}} (L^dT^{-1}_{\text{max}})^{l(T)} T_{\text{max}}^{2l(T)}
    \\
    =& L^{O(l(T)\theta)} 2^{-\frac{1}{2}\sum_{j=1}^K \tau_{j}} \rho^{2l(T)}
\end{split}
\end{equation}

By Lemma \ref{lem.Tpvariance} below $Term(T, p)_{op,k,k
    '}=0$ if $|k-k'|\gtrsim 1$, we know that the same is true for $\mathbb{E}|H^{\tau_1\cdots \tau_{K}}_{Tkk'}|^2=0$. 

Therefore, we complete the proof of this proposition.
\end{proof}


\begin{lem}\label{lem.Tpvarianceop} Let $Q=L^dT^{-1}_{\text{max}}$ be the same as in Proposition \ref{prop.counting}. Assume that $\alpha$ satisfies \eqref{eq.conditionalpha.threewave} and $n_{\mathrm{in}} \in C^\infty_0(\mathbb{R}^d)$ is compactly supported. Then for any $\theta>0$, we have
\begin{equation}
    \sup_k\, |Term(T, p)_{op,k,k
    '}|\le L^{O(l(T)\theta)}2^{-\frac{1}{2}\sum_{j=1}^K \tau_{j}} Q^{l(T)} T_{\text{max}}^{2l(T)}.
\end{equation}
and $Term(T, p)_{op,k,k
    '}=0$ if $|k-k'|\gtrsim 1$.
\end{lem}
\begin{proof} By \eqref{eq.termTpop.threewave}, we get
\begin{equation}
\begin{split}
    &Term(T, p)_{op,k,k
    '}
    \\
    =&\sum_{k_1,\, k_2,\, \cdots,\, k_{l(T)}}\sum_{k'_1,\, k'_2,\, \cdots,\, k'_{l(T)}} H^{\tau_1\cdots \tau_{K}}_{Tk_1\cdots k_{l(T)}kk'} \overline{H^{\tau_1\cdots \tau_{K}}_{Tk_1'\cdots k_{l(T)}'kk'}} \delta_{p}(k_1,\cdots, k_{l(T)}, k'_1,\cdots, k'_{l(T)})\sqrt{n_{\textrm{in}}(k_1)}\cdots.
\end{split}
\end{equation}

Since $n_{\mathrm{in}}$ are compactly supported and there are bounded many of them in $Term(T, p)_{op,k,k'}$, by $k_1 + k_2 + \cdots + k_{l(T)}=k-k'$, we know that $Term(T, p)_{op,k,k'}=0$ if $|k-k'|\gtrsim 1$.

By \eqref{eq.boundcoefoperator.threewave}, we get  %$\frac{1}{\sqrt{(|c(\Omega)_{\mathfrak{n}}|+T^{-1}_{\text{max}})^2+|r_{\mathfrak{n}}|^2}}\lesssim \frac{1}{|c(\Omega)_{\mathfrak{n}}|+T^{-1}_{\text{max}}}$
\begin{equation}\label{eq.termlemmaeq1op.threewave}
    |H^{\tau_1\cdots \tau_{K}}_{Tk_1\cdots k_{l}kk'}|\lesssim \sum_{\{d_{\mathfrak{n}}\}_{\mathfrak{n}\in T_{\text{in}}}\in\{0,1\}^{l(T)}}\prod_{\mathfrak{n}\in T_{\text{in}}}\frac{2^{-\frac{\tau_{\mathfrak{n}}}{2}}}{|q_{\mathfrak{n}}|+T^{-1}_{\text{max}}}\ \delta_{\cap_{\mathfrak{n}\in T_{\text{in}}} \{S_{\mathfrak{n}}=0\}}\prod_{\mathfrak{e}\in T_{\text{in}}} p_{\mathfrak{e}}.
\end{equation}

Define $[c_{\mathfrak{n},\widetilde{\mathfrak{n}}}]$, $\mathscr{M}(T)$, $c(\Omega)$ in the same way as the proof of Lemma \ref{lem.Tpvariance}. We can apply the same derivation of \eqref{eq.termlemmaeq3.threewave} to obtain
\begin{equation}\label{eq.termlemmaeq3op.threewave}
\begin{split}
    &|Term(T, p)_{op,k,k'}|\lesssim \sum_{\substack{k_1,\, \cdots,\, k_{l(T)},\, k'_1,\, \cdots,\, k'_{l(T)}\\ |k_{j}|, |k'_j|\lesssim 1, \forall j}} \sum_{c\in \mathscr{M}(T) }\prod_{\mathfrak{n}\in T_{\text{in}}}\frac{2^{-\frac{\tau_{\mathfrak{n}}}{2}}}{|c(\Omega)_{\mathfrak{n}}|+T^{-1}_{\text{max}}}    \ \delta_{\cap_{\mathfrak{n}\in T_{\text{in}}} \{S_{\mathfrak{n}}=0\}}\prod_{\mathfrak{e}\in T_{\text{in}}} p_{\mathfrak{e}} 
    \\
    &\sum_{c'\in \mathscr{M}(T)}\prod_{\mathfrak{n}'\in T_{\text{in}}}\frac{2^{-\frac{\tau_{\mathfrak{n}}}{2}}}{|c'(\Omega)_{\mathfrak{n}'}|+T^{-1}_{\text{max}}} \prod_{\mathfrak{e}'\in T_{\text{in}}} p_{\mathfrak{e}'}  \ \delta_{\cap_{\mathfrak{n}'\in T_{\text{in}}} \{S_{\mathfrak{n}'}=0\}} \delta_{p}(k_1,\cdots, k_{l(T)}, k'_1,\cdots, k'_{l(T)})
\end{split}
\end{equation}

We obviously have the following inequality
\begin{equation}\label{eq.termlemmaeq4op.threewave}
\begin{split}
    &|Term(T, p)_{op,k,k'}|\lesssim \sum_{k'}|Term(T, p)_{op,k,k'}|
    \\
    \lesssim& \sum_{\substack{k_1,\, \cdots,\, k_{l(T)},\, k'_1,\, \cdots,\, k'_{l(T)}, k'\\ |k_{j}|, |k'_j|\lesssim 1, \forall j,\ |k'-k|\lesssim 1}} \sum_{c\in \mathscr{M}(T) }\prod_{\mathfrak{n}\in T_{\text{in}}}\frac{2^{-\frac{\tau_{\mathfrak{n}}}{2}}}{|c(\Omega)_{\mathfrak{n}}|+T^{-1}_{\text{max}}}    \ \delta_{\cap_{\mathfrak{n}\in T_{\text{in}}} \{S_{\mathfrak{n}}=0\}} \prod_{\mathfrak{e}\in T_{\text{in}}} p_{\mathfrak{e}}
    \\
    &\sum_{c'\in \mathscr{M}(T)}\prod_{\mathfrak{n}'\in T_{\text{in}}}\frac{2^{-\frac{\tau_{\mathfrak{n}}}{2}}}{|c'(\Omega)_{\mathfrak{n}'}|+T^{-1}_{\text{max}}}  \prod_{\mathfrak{e}'\in T_{\text{in}}} p_{\mathfrak{e}'}  \ \delta_{\cap_{\mathfrak{n}'\in T_{\text{in}}} \{S_{\mathfrak{n}'}=0\}} \delta_{p}(k_1,\cdots, k_{l(T)}, k'_1,\cdots, k'_{l(T)})
\end{split}
\end{equation}

Switch the order of summations and products in \eqref{eq.termlemmaeq3.threewave}, then we get
\begin{equation}\label{eq.termlemmaeq2op.threewave}
\begin{split}
    |Term(T, p)_{op,k,k'}|\lesssim& \sum_{\substack{k_1,\, \cdots,\, k_{l(T)},\, k'_1,\, \cdots,\, k'_{l(T)}, k'\\ |k_{j}|, |k'_j|\lesssim 1, \forall j,\ |k'-k|\lesssim 1}} \sum_{c, c'\in \mathscr{M}(T) }\prod_{\mathfrak{n}, \mathfrak{n}'\in T_{\text{in}}}\frac{2^{-\frac{\tau_{\mathfrak{n}}}{2}}}{|c(\Omega)_{\mathfrak{n}}|+T^{-1}_{\text{max}}}\frac{2^{-\frac{\tau_{\mathfrak{n}}}{2}}}{|c'(\Omega)_{\mathfrak{n}'}|+T^{-1}_{\text{max}}}
    \\
    & \prod_{\mathfrak{e},\mathfrak{e}'\in T_{\text{in}}} (p_{\mathfrak{e}}p_{\mathfrak{e}'})\ \delta_{\cap_{\mathfrak{n},\mathfrak{n}'\in T_{\text{in}}} \{S_{\mathfrak{n}}=0, S_{\mathfrak{n}'}=0\}} \delta_{p}(k_1,\cdots, k_{l(T)}, k'_1,\cdots, k'_{l(T)}).
\end{split}
\end{equation}

Consider a tree $T$ with a $\Box$ nodes and a pairing $p\in \mathcal{P}(\{k_1,\cdots, k_{l(T)}, k'_1,\cdots, k'_{l(T)}\})$. $p$ can be viewed as a pairing of all star nodes of two copies of $T$. A couple $\mathcal{C}$ can be constructed by merging all paired star nodes according to $p$ and merging the two $\Box$ nodes. As in \eqref{eq.termlemmaeq4.threewave}, we can show that
\begin{equation}\label{eq.termlemmaeq4op'.threewave}
\sum_{c, c'\in \mathscr{M}(T) }=\sum_{c\in \mathscr{M}(\mathcal{C}) },\qquad \prod_{\mathfrak{n}, \mathfrak{n}'\in T_{\text{in}}}=\prod_{\mathfrak{n}\in \mathcal{C}}, \qquad \prod_{\mathfrak{e},\mathfrak{e}'\in T_{\text{in}}}=\prod_{\mathfrak{e}\in \mathcal{C}_{\text{norm}}},\qquad \cap_{\mathfrak{n},\mathfrak{n}'\in T_{\text{in}}}=\cap_{\mathfrak{n}\in \mathcal{C}}.    
\end{equation}


The analog of \eqref{eq.termlemmaeq5.threewave} is 
\begin{equation}\label{eq.termlemmaeq5op.threewave}
|Term(T, p)_{op,k,k'}|\lesssim \sum_{\substack{k_1,\, \cdots,\, k_{l(T)},\, k'_1,\, \cdots,\, k'_{l(T)}, k'\\ |k_{j}|, |k'_j|\lesssim 1, \forall j,\ |k'-k|\lesssim 1}} \sum_{c\in \mathscr{M}(\mathcal{C}) }\prod_{\mathfrak{n}\in \mathcal{C}}\frac{2^{-\frac{\tau_{\mathfrak{n}}}{2}}}{|c(\Omega)_{\mathfrak{n}}|+T^{-1}_{\text{max}}} \prod_{\mathfrak{e}\in \mathcal{C}_{\text{norm}}} p_{\mathfrak{e}}\  \delta_{\cap_{\mathfrak{n}\in \mathcal{C}} \{S_{\mathfrak{n}}=0\}}
\end{equation}

The rest part of the proof is exactly the same as the proof of Lemma \ref{lem.Tpvariance} after \eqref{eq.termlemmaeq5.threewave}. For completeness, we include a sketch.

As \eqref{eq.termlemmaeq8.threewave}, we have  
\begin{equation}
    \sum_{\substack{k_1,\, \cdots,\, k_{l(T)+1},\, k'_1,\, \cdots,\, k'_{l(T)+1}\\\cap_{\mathfrak{n}\in \mathcal{C}} \{S_{\mathfrak{n}}=0\}}}=\sum_{\kappa_{\mathfrak{e}}\in \mathcal{D}(\alpha,1)}\sum_{\sigma_{\mathfrak{n}}\in \mathbb{Z}_{T_{\text{max}}}}\sum_{Eq(\mathcal{C}, \{\sigma_{\mathfrak{n}}\}_{\mathfrak{n}}, \{\kappa_{\mathfrak{e}}\}_{\mathfrak{e}},k)},
\end{equation}
which implies that
\begin{equation}\label{eq.termlemmaeq6op.threewave}
\begin{split}
    |Term(T, p)_{op,k,k'}|\lesssim \sum_{\kappa_{\mathfrak{e}}\in \mathcal{D}(\alpha,1)}\sum_{\sigma_{\mathfrak{n}}\in \mathbb{Z}_{T_{\text{max}}}}\sum_{Eq(\mathcal{C}, \{\sigma_{\mathfrak{n}}\}_{\mathfrak{n}}, \{\kappa_{\mathfrak{e}}\}_{\mathfrak{e}},k)} \sum_{c\in \mathscr{M}(\mathcal{C}) }\prod_{\mathfrak{n}\in \mathcal{C}}\frac{2^{-\frac{\tau_{\mathfrak{n}}}{2}}}{|c(\Omega)_{\mathfrak{n}}|+T^{-1}_{\text{max}}} \prod_{\mathfrak{e}\in \mathcal{C}_{\text{norm}}} p_{\mathfrak{e}}
\end{split}
\end{equation}


As \eqref{eq.lemboundtermTp.threewave}, we get
\begin{equation}\label{eq.lemboundtermTpop.threewave}
\begin{split}
    &Term(T, p)_k
    \\
    \lesssim& \sum_{\kappa_{\mathfrak{e}}\in \mathcal{D}(\alpha,1)}\sum_{\sigma_{\mathfrak{n}}\in \mathbb{Z}_{T_{\text{max}}}}\sum_{Eq(\mathcal{C}, \{\sigma_{\mathfrak{n}}\}_{\mathfrak{n}}, \{\kappa_{\mathfrak{e}}\}_{\mathfrak{e}},k)} \sum_{c\in \mathscr{M}(\mathcal{C}) }\prod_{\mathfrak{n}\in \mathcal{C}}\frac{2^{-\frac{\tau_{\mathfrak{n}}}{2}}}{|c(\Omega)_{\mathfrak{n}}|+T^{-1}_{\text{max}}} \prod_{\mathfrak{e}\in \mathcal{C}_{\text{norm}}} p_{\mathfrak{e}}
    \\
    \lesssim &\sum_{c\in \mathscr{M}(\mathcal{C}) }\sum_{\kappa_{\mathfrak{e}}\in \mathcal{D}(\alpha,1)}\sum_{\substack{\sigma_{\mathfrak{n}}\in \mathbb{Z}_{T_{\text{max}}}\\ |\sigma_{\mathfrak{n}}|\lesssim 1}}\prod_{\mathfrak{n}\in \mathcal{C}}\frac{2^{-\frac{\tau_{\mathfrak{n}}}{2}}}{|c(\{\sigma_{\mathfrak{n}}\})_{\mathfrak{n}}|+T^{-1}_{\text{max}}} \sum_{Eq(\mathcal{C}, \{\sigma_{\mathfrak{n}}\}_{\mathfrak{n}}, \{\kappa_{\mathfrak{e}}\}_{\mathfrak{e}},k)} 1 \prod_{\mathfrak{e}\in \mathcal{C}_{\text{norm}}} \frac{\kappa_{\mathfrak{e}}}{\kappa_{\mathfrak{e}}+1}
    \\
    \lesssim &\sum_{c\in \mathscr{M}(\mathcal{C}) }\sum_{\kappa_{\mathfrak{e}}\in \mathcal{D}(\alpha,1)}\sum_{\substack{\sigma_{\mathfrak{n}}\in \mathbb{Z}_{T_{\text{max}}}\\ |\sigma_{\mathfrak{n}}|\lesssim 1}}\prod_{\mathfrak{n}\in \mathcal{C}}\frac{2^{-\frac{\tau_{\mathfrak{n}}}{2}}}{|c(\{\sigma_{\mathfrak{n}}\})_{\mathfrak{n}}|+T^{-1}_{\text{max}}} \#Eq(\mathcal{C}, \{\sigma_{\mathfrak{n}}\}_{\mathfrak{n}}, \{\kappa_{\mathfrak{e}}\}_{\mathfrak{e}},k) \prod_{\mathfrak{e}\in \mathcal{C}_{\text{norm}}} \frac{\kappa_{\mathfrak{e}}}{\kappa_{\mathfrak{e}}+1}
    \\
    \lesssim &\sum_{c\in \mathscr{M}(\mathcal{C}) }\sum_{\kappa_{\mathfrak{e}}\in \mathcal{D}(\alpha,1)}\sum_{\substack{\sigma_{\mathfrak{n}}\in \mathbb{Z}_{T_{\text{max}}}\\ |\sigma_{\mathfrak{n}}|\lesssim 1}}\prod_{\mathfrak{n}\in \mathcal{C}}\frac{2^{-\frac{\tau_{\mathfrak{n}}}{2}}}{|c(\{\sigma_{\mathfrak{n}}\})_{\mathfrak{n}}|+T^{-1}_{\text{max}}} L^{O(n\theta)} Q^{\frac{n}{2}} \prod_{\mathfrak{e}\in \mathcal{C}_{\text{norm}}} \frac{\kappa_{\mathfrak{e}}}{\kappa_{\mathfrak{e}}+1} \prod_{\mathfrak{e}\in \mathcal{C}_{\text{norm}}} \kappa_{\mathfrak{e}}^{-1}
\end{split}
\end{equation}
Here in the last inequality, we applied \eqref{eq.countingbd0.threewave} in Proposition \ref{prop.counting}.

After simplification, \eqref{eq.lemboundtermTpop.threewave} gives us 
\begin{equation}
\begin{split}
    |Term(T, p)_{op,k,k'}|\lesssim &L^{O(n\theta)} Q^{\frac{n}{2}}\sum_{c\in \mathscr{M}(\mathcal{C}) }\sum_{\kappa_{\mathfrak{e}}\in \mathcal{D}(\alpha,1)}\sum_{\substack{\sigma_{\mathfrak{n}}\in \mathbb{Z}_{T_{\text{max}}}\\ |\sigma_{\mathfrak{n}}|\lesssim 1}} \prod_{\mathfrak{n}\in \mathcal{C}}\frac{2^{-\frac{\tau_{\mathfrak{n}}}{2}}}{|c(\{\sigma_{\mathfrak{n}}\})_{\mathfrak{n}}|+T^{-1}_{\text{max}}}
    \\
    \lesssim &L^{O(n\theta)} Q^{\frac{n}{2}} \left(\sum_{\kappa_{\mathfrak{e}}\in \mathcal{D}(\alpha,1)} 1\right) \sum_{c\in \mathscr{M}(\mathcal{C})}\sum_{\substack{\sigma_{\mathfrak{n}}\in \mathbb{Z}_{T_{\text{max}}}\\ |\sigma_{\mathfrak{n}}|\lesssim 1}} \prod_{\mathfrak{n}\in \mathcal{C}}\frac{2^{-\frac{\tau_{\mathfrak{n}}}{2}}}{|c(\{\sigma_{\mathfrak{n}}\})_{\mathfrak{n}}|+T^{-1}_{\text{max}}}
    \\
    \lesssim & L^{O(n\theta)} Q^{\frac{n}{2}} \sum_{c\in \mathscr{M}(\mathcal{C})}\sum_{\substack{\sigma_{\mathfrak{n}}\in \mathbb{Z}_{T_{\text{max}}}\\ |\sigma_{\mathfrak{n}}|\lesssim 1}} \prod_{\mathfrak{n}\in \mathcal{C}}\frac{2^{-\frac{\tau_{\mathfrak{n}}}{2}}}{|c(\{\sigma_{\mathfrak{n}}\})_{\mathfrak{n}}|+T^{-1}_{\text{max}}}
\end{split}
\end{equation}
% \begin{equation}
% \begin{split}
%     &|Term(T, p)_{op,k,k'}|
%     \\
%     \lesssim&\sum_{c,c'\in \mathscr{M}(T) } \sum_{\sigma_{\mathfrak{n}}\in \mathbb{Z}_{T_{\text{max}}}} \prod_{\mathfrak{n}\in T_{\text{in}}}\frac{t\alpha}{|c(\{\sigma_{\mathfrak{n}}\})_{\mathfrak{n}}|+\alpha} \prod_{\mathfrak{n}\in T_{\text{in}}}\frac{t\alpha}{|c'(\{\sigma_{\mathfrak{n}}\})_{\mathfrak{n}}|+\alpha} (\#Eq(\mathcal{C}, \{\sigma_{\mathfrak{n}}\}_{\mathfrak{n}},k)+O(L^{-8d\, l(T)-8d}))
%     \\
%     \lesssim & L^{O(n\theta)} Q^{n} L^{\frac{1}{2} dn_d}\sum_{c,c'\in \mathscr{M}(T) } \sum_{\substack{\sigma_{\mathfrak{n}}\in \mathbb{Z}_{T_{\text{max}}}\\ |\sigma_{\mathfrak{n}}|\lesssim 1}} \prod_{\mathfrak{n}\in T_{\text{in}}}\frac{t\alpha}{|c(\{\sigma_{\mathfrak{n}}\})_{\mathfrak{n}}|+\alpha} \prod_{\mathfrak{n}\in T_{\text{in}}}\frac{t\alpha}{|c'(\{\sigma_{\mathfrak{n}}\})_{\mathfrak{n}}|+\alpha}  + O(L^{-6d\, l(T)-6d})
% \end{split}
% \end{equation}
Here in the first step we use the fact that $\prod_{\mathfrak{e}\in \mathcal{C}_{\text{norm}}} \frac{\kappa_{\mathfrak{e}}}{\kappa_{\mathfrak{e}}+1} \prod_{\mathfrak{e}\in \mathcal{C}_{\text{norm}}} \kappa_{\mathfrak{e}}^{-1}=\prod_{\mathfrak{e}\in \mathcal{C}_{\text{norm}}} \frac{1}{\kappa_{\mathfrak{e}}+1}\le 1$. The reason for other steps can be found in the derivation of \eqref{eq.lemboundtermTpsimplify.threewave}.

We claim that 
\begin{equation}\label{eq.lemTpvarianceclaimop.threewave}
     \sup_{c}\sum_{\substack{\sigma_{\mathfrak{n}}\in \mathbb{Z}_{T_{\text{max}}}\\ |\sigma_{\mathfrak{n}}|\lesssim 1}} \prod_{\mathfrak{n}\in \mathcal{C}}\frac{2^{-\frac{\tau_{\mathfrak{n}}}{2}}}{|c(\{\sigma_{\mathfrak{n}}\})_{\mathfrak{n}}|+T^{-1}_{\text{max}}}\lesssim L^{O(l(T)\theta)}2^{-\frac{1}{2}\sum_{j=1}^K \tau_{j}} T^{2l(T)}_{\text{max}}
\end{equation}

Since there are only bounded many matrices in $\mathscr{M}(\mathcal{C})$.  Given the above claim, we know that 
\begin{equation}
    |Term(T, p)_{op,k,k'}|\lesssim L^{O(l(T)\theta)} 2^{-\frac{1}{2}\sum_{j=1}^K \tau_{j}} Q^{\frac{n}{2}} T^{2l(T)}_{\text{max}},
\end{equation}
which proves the lemma since $n=2l(T)$.

Now prove the claim. In a tree $T$, there are $l(T)$ branching nodes, so there are $l(T)$ nodes in $T_{\text{in}}$. Since all nodes of $\mathcal{C}$ comes the two copies of $T_{\text{in}}$, so there are $2l(T)$ nodes in $\mathcal{C}$. Label these nodes by $h=1,\cdots,2l(T)$ and denote $\sigma_{\mathfrak{n}}$ by $\sigma_{h}$ if $\mathfrak{n}$ is labelled by $h$. Since $\sigma_{h}\in \mathbb{Z}_{T_{\text{max}}}$, there exists $m_{h}\in \mathbb{Z}$ such that $\sigma_{h}=T^{-1}_{\text{max}} m_{h}$. \eqref{eq.lemTpvarianceclaimop.threewave} is thus equivalent to 
\begin{equation}\label{eq.lemTpvarianceclaim1op.threewave}
    T^{2l(T)}_{\text{max}}\sum_{\substack{m_{h}\in \mathbb{Z}\\ |m_{h}|\lesssim T_{\text{max}}}} \prod_{h=1}^{2l(T)}\frac{2^{-\frac{\tau_{\mathfrak{n}}}{2}}}{|c(\{m_{h}\})_{h}|+1}\lesssim L^{O(l(T)\theta)}2^{-\frac{1}{2}\sum_{j=1}^K \tau_{j}} T^{2l(T)}_{\text{max}}
\end{equation}

To prove \eqref{eq.lemTpvarianceclaim1op.threewave}. We just need to show that 
\begin{equation}
    \sum_{\substack{m_{h}\in \mathbb{Z}\\ |m_{h}|\lesssim T_{\text{max}}}} \prod_{h=1}^{2l(T)}\frac{1}{|c(\{m_{h}\})_{h}|+1}\lesssim L^{O(l(T)\theta)}
\end{equation}
This can be proved by Euler-Maclaurin formula \eqref{eq.EulerMaclaurin.threewave} as \eqref{eq.lemTpvarianceEulerMac.threewave}.

Now we complete the proof of the claim and thus the proof of the lemma.
\end{proof}


\subsubsection{Proof of the operator norm bound} In this subsection, we finish the proof of Proposition \ref{prop.operatorupperbound'}.

\begin{proof}[Proof of Proposition \ref{prop.operatorupperbound'}]
By Lemma \ref{lem.treetermsoperator}, we have 

\begin{equation}
    \left(\prod_{j=1}^K\mathcal{P}^{\tau_j}_{T_j}(w)\right)_{k}(t)=\sum_{k'}\int_0^t H^{\tau_1\cdots \tau_{K}}_{Tkk'}(t,s) w_{k'}(s) ds
\end{equation}
and the kernel $H^{\tau_1\cdots \tau_{K}}_{Tkk'}$ is a polynomial of Gaussian variables given by
\begin{equation}
\begin{split}
H^{\tau_1\cdots \tau_{K}}_{Tkk'}(t,s)=\left(\frac{i\lambda}{L^{d}}\right)^l\sum_{k_1,\, k_2,\, \cdots,\, k_{l}} H^{\tau_1\cdots \tau_{K}}_{Tk_1\cdots k_{l}kk'} \xi_{k_1}\cdots \xi_{k_{l}}.
\end{split}
\end{equation}

By Proposition \ref{prop.treetermsvarianceoperator}, we have
\begin{equation}
    \sup_k\, \mathbb{E}|H^{\tau_1\cdots \tau_{K}}_{Tkk'}|^2\lesssim L^{O(l(T)\theta)}2^{-\frac{1}{2}\sum_{j=1}^K \tau_{j}} \rho^{2l(T)}.
\end{equation}

Then the large deviation estimate Lemma \ref{lem.largedev} gives 
\begin{equation}
|H^{\tau_1\cdots \tau_{K}}_{Tkk'}(t,s)|\lesssim L^{\frac{n}{2}\theta} \sqrt{\mathbb{E}|H^{\tau_1\cdots \tau_{K}}_{Tkk'}|^2}\lesssim L^{O(l(T)\theta)} 2^{-\frac{1}{4}\sum_{j=1}^K \tau_{j}} \rho^{l(T)},\qquad \textit{L-certainly}.
\end{equation} 

Summing over all $\tau_1,\cdots, \tau_{K}$ gives
\begin{equation}
\sum_{\tau_1,\cdots, \tau_{K}}|H^{\tau_1\cdots \tau_{K}}_{Tkk'}(t,s)|\lesssim L^{O(l(T)\theta)} \rho^{l(T)},\qquad \textit{L-certainly}.
\end{equation} 

Applying the epsilon net and union bound method as in \eqref{eq.unionbound.threewave}, we obtain
\begin{equation}
    \sup_{t,s}\sup_{|k|,|k'|\lesssim L^{2M}}\sum_{\tau_1,\cdots, \tau_{K}}|H^{\tau_1\cdots \tau_{K}}_{Tkk'}(t,s)|\lesssim L^{O(l(T)\theta)}  \rho^{l(T)},\qquad \textit{L-certainly}.
\end{equation}

For $w\in X^{p}_{L^{2M}}$, we have $|w_k(t)|\le \sup_{t}||w(t)||_{X^{p}_{L^{2M}}} \langle k\rangle^{-p}$. Since $w_{k'}=0$ if $|k'|\gtrsim L^{2M}$ and $H^{\tau_1\cdots \tau_{K}}_{Tkk'}=0$ if $|k-k'|\gtrsim 1$, we know that
\begin{equation}
\begin{split}
    \left|\left(\sum_{\tau_1,\cdots, \tau_{K}}\prod_{j=1}^K\mathcal{P}^{\tau_j}_{T_j}(w)\right)_{k}(t)\right|\le&\sum_{|k'|\lesssim L^{2M}}\int_0^t \sum_{\tau_1,\cdots, \tau_{K}}|H^{\tau_1\cdots \tau_{K}}_{Tkk'}(t,s)| |w_{k'}(s)| ds
    \\
    \lesssim&  L^{O(l(T)\theta)}  \rho^{l(T)}t \sup_{t}||w(t)||_{X^{p}_{L^{2M}}} \sum_{\substack{|k'|\lesssim L^{2M}\\ |k'-k|\lesssim 1}} \langle k'\rangle^{-p}
    \\
    \lesssim& L^{O(1+l(T)\theta)}  \rho^{l(T)} \sup_{t}||w(t)||_{X^{p}_{L^{2M}}}  \langle k\rangle^{-p}.
\end{split}
\end{equation}

Since $l(T)=\sum_{j=1}^K l(T_j)$, $L$-certainly we have 
\begin{equation}
    \left|\left|\sum_{\tau_1,\cdots,\tau_K}\prod_{j=1}^K\mathcal{P}^{\tau_j}_{T_j}\right|\right|_{L_t^{\infty}X^{p}}\le L^{O\left(1+\theta\sum_{j=1}^K l(T_j)\right)} \rho^{\sum_{j=1}^K l(T_j)}||w||_{L_t^{\infty}X^{p}_{L^{2M}}}.
\end{equation}

Therefore, we complete the proof of Proposition \ref{prop.operatorupperbound'}.
\end{proof}



\subsection{Asymptotics of the main terms} In this section, we prove \eqref{eq.n1.threewave} in Theorem \ref{th.main} which characterize the asymptotic behavior of $n^{(1)}(k)$.

\begin{prop}\label{prop.mainterms} Using the same notation as Theorem \ref{th.main} (2), then we have 
\begin{equation}
        n^{(1)}(k)=\left\{
\begin{aligned}
    &\frac{t}{T_{\mathrm{kin}}}\mathcal K(n_{\mathrm{in}})(k)+O_{\ell^\infty_k}\left(L^{-\theta}\frac{T_{\text{max}}}{T_{\mathrm {kin}}}\right)+\widetilde{O}_{\ell^\infty_k}\left(\epsilon_1\text{Err}_{D}(k_x)\frac{T_{\text{max}}}{T_{\mathrm {kin}}}\right)
    && \text{for any } |k|\le \epsilon_1 l_{d}^{-1},
    \\
    &0, && \text{for any } |k|\ge l_{d}^{-1}
\end{aligned}\right.
    \end{equation}
    Here 
    \begin{equation}
        \text{Err}_{D}(k_x)=\left\{\begin{aligned}
    &D^{d+1}, && \text{if } |k_x|\le D,
    \\
    &D^{d-1}(|k_x|^2+D|k_x|), && \text{if } |k_x|\ge D.
\end{aligned}
    \right.
    \end{equation}
\end{prop}
\begin{proof} The second case of \eqref{eq.n1.threewave} is obvious. We divide the proof of the first case into several steps.

\textbf{Step 1.} (Calculation of $\psi_{app,k}$ and $n^{(1)}(k)$) The first three terms in the tree expansion \eqref{eq.approxsol.threewave} can be calculated explicitly. 
\begin{equation}
\begin{split}
    \psi_{app,k}=\psi^{(0)}_{app,k}+\psi^{(1)}_{app,k}+\psi^{(2)}_{app,k}+\cdots
\end{split}
\end{equation}
% \begin{equation}
% \begin{split}
%     \psi_{app,k}=\xi_k+\frac{i\lambda}{L^{d}} \sum\limits_{k_1+k_2=k} k_{x}\xi_{k_1} \xi_{k_2} \int^{t}_0e^{i s\Omega(k_1,k_2,k)- \nu|k|^2(t-s)} ds
%     \\
%     -&2\left(\frac{\lambda}{L^{d}}\right)^2 \sum\limits_{k_1+k_2+k_3=k} k_{x}(k_{2x}+k_{3x})\xi_{k_1} \xi_{k_2}\xi_{k_3} \times
%     \\
%     &\int_{0\le r<s\le t}e^{i s\Omega(k_1,k_2+k_3,k)- \nu|k|^2(t-s)} e^{i r\Omega(k_2,k_3,k_2+k_3)- \nu|k_2+k_3|^2(s-r)} dsdr
%     \\
%     +&\cdots
% \end{split}
% \end{equation}
where $\psi^{(0)}_{app,k}$,$\psi^{(1)}_{app,k}$, $\psi^{(2)}_{app,k}$ are given by
\begin{equation}
    \psi^{(0)}_{app,k}=\xi_k
\end{equation}
\begin{equation}
    \psi^{(1)}_{app,k}=\frac{i\lambda}{L^{d}} \sum\limits_{k_1+k_2=k} k_{x}\xi_{k_1} \xi_{k_2} \int^{t}_0e^{i s\Omega(k_1,k_2,k)- \nu|k|^2(t-s)} ds
\end{equation}
\begin{equation}
\begin{split}
    \psi^{(2)}_{app,k}=&-2\left(\frac{\lambda}{L^{d}}\right)^2 \sum\limits_{k_1+k_2+k_3=k} k_{x}(k_{2x}+k_{3x})\xi_{k_1} \xi_{k_2}\xi_{k_3}\times
    \\
    &\int_{0\le r<s\le t}e^{i s\Omega(k_1,k_2+k_3,k)- \nu|k|^2(t-s)} e^{i r\Omega(k_2,k_3,k_2+k_3)- \nu|k_2+k_3|^2(s-r)} dsdr
\end{split}
\end{equation}

By \eqref{eq.n(j).threewave}, we know that
\begin{equation}\label{eq.n(1).threewave}
    n^{(1)}(k)=\mathbb E \left|\psi^{(1)}_{app,k}\right|^2+ 2\text{Re}\  \mathbb E \left(\psi^{(2)}_{app,k}\overline{\xi_k}\right).
\end{equation}

\textbf{Step 2.} (Decomposition of $\psi_{app,k}$) $e^{- \nu|k|^2(t-s)}$ is supposed to be close to $1$, so we have the decomposition $e^{- \nu|k|^2(t-s)}=1+(e^{- \nu|k|^2(t-s)}-1)=1-\left(\int_{0}^1 e^{-a\nu|k|^2(t-s)} da\right) \nu|k|^2(t-s)$. We can also decompose $\psi^{(1)}_{app,k}=\psi^{(11)}_{app,k}+\psi^{(12)}_{app,k}$ and $\psi^{(2)}_{app,k}=\psi^{(21)}_{app,k}+\psi^{(22)}_{app,k}$ into $\psi^{(1)}_{app,k}$ accordingly.
\begin{equation}
    \psi^{(11)}_{app,k}=\frac{i\lambda}{L^{d}} \sum\limits_{k_1+k_2=k} k_{x}\xi_{k_1} \xi_{k_2} \int^{t}_0e^{i s\Omega(k_1,k_2,k)} ds
\end{equation}
\begin{equation}
    \psi^{(12)}_{app,k}=-\nu|k|^2\frac{i\lambda}{L^{d}} \int_{0}^1\Bigg(\sum\limits_{k_1+k_2=k} k_{x}\xi_{k_1} \xi_{k_2} \int^{t}_0e^{i s\Omega(k_1,k_2,k)- a\nu|k|^2(t-s)} (t-s)ds\Bigg) da
\end{equation}
\begin{equation}
\psi^{(21)}_{app,k}=-2\left(\frac{\lambda}{L^{d}}\right)^2 \sum\limits_{k_1+k_2+k_3=k} k_{x}(k_{2x}+k_{3x})\xi_{k_1} \xi_{k_2}\xi_{k_3}\int_{0\le r<s\le t}e^{i s\Omega(k_1,k_2+k_3,k)} e^{i r\Omega(k_2,k_3,k_2+k_3)} dsdr
\end{equation}
\begin{equation}
\begin{split}
    \psi^{(22)}_{app,k}=&2\left(\frac{\lambda}{L^{d}}\right)^2 \int_{0}^1\Bigg(\sum\limits_{k_1+k_2+k_3=k} k_{x}(k_{2x}+k_{3x})\int_{0\le r<s\le t}(\nu|k|^2(t-s)+\nu|k_2+k_3|^2(s-r))
    \\
    & e^{i s\Omega(k_1,k_2+k_3,k)- a\nu|k|^2(t-s)} e^{i r\Omega(k_2,k_3,k_2+k_3)- a\nu|k_2+k_3|^2(s-r)} dsdr\xi_{k_1}\xi_{k_2}\xi_{k_3}\Bigg)da
\end{split}
\end{equation}

$n^{(1)}(k)$ is also decomposed into $n^{(1)}(k)=n^{(11)}(k)+n^{(12)}(k)$
\begin{equation}\label{eq.n(11).threewave}
    n^{(11)}(k)=\mathbb E \left|\psi^{(11)}_{app,k}\right|^2+ 2\text{Re}\  \mathbb E \left(\psi^{(21)}_{app,k}\overline{\xi_k}\right).
\end{equation}
\begin{equation}
    n^{(12)}(k)=\mathbb E \left|\psi^{(12)}_{app,k}\right|^2+2\text{Re}\ \mathbb E \left(\psi^{(11)}_{app,k}\overline{\psi^{(12)}_{app,k}}\right)+ 2\text{Re}\  \mathbb E \left(\psi^{(22)}_{app,k}\overline{\xi_k}\right).
\end{equation}

\textbf{Step 3.} (Estimate of $n^{(12)}(k)$ for $|k|\le \epsilon_1 l_{d}^{-1}$) In this step, all constants in $\lesssim$ depend only on the dimension $d$. Define 
\begin{equation}\label{eq.H(11).threewave}
    H^{(11)}_{k_1k_2k}=k_{x} \int^{t}_0e^{i s\Omega(k_1,k_2,k)} ds
\end{equation}
\begin{equation}\label{eq.H(12).threewave}
    H^{(12)}_{k_1k_2k}=-\nu|k|^2k_{x} \int^{t}_0e^{i s\Omega(k_1,k_2,k)- a\nu|k|^2(t-s)} (t-s)ds
\end{equation}
\begin{equation}\label{eq.H(21).threewave}
    H^{(21)}_{k_1k_2k_3k}=2k_{x}(k_{2x}+k_{3x})\int_{0\le r<s\le t}e^{i s\Omega(k_1,k_2+k_3,k)} e^{i r\Omega(k_2,k_3,k_2+k_3)} dsdr
\end{equation}
\begin{equation}\label{eq.H(22).threewave}
\begin{split}
    H^{(22)}_{k_1k_2k_3k}=&-2k_{x}(k_{2x}+k_{3x})\int_{0\le r<s\le t}(\nu|k|^2(t-s)+\nu|k_2+k_3|^2(s-r))
    \\
    & e^{i s\Omega(k_1,k_2+k_3,k)- a\nu|k|^2(t-s)} e^{i r\Omega(k_2,k_3,k_2+k_3)- a\nu|k_2+k_3|^2(s-r)} dsdr
\end{split}
\end{equation}

Then we have
\begin{equation}\label{eq.psi(11)app.threewave}
    \psi^{(11)}_{app,k}=\frac{i\lambda}{L^{d}} \sum\limits_{k_1+k_2=k} H^{(11)}_{k_1k_2k}\xi_{k_1} \xi_{k_2} 
\end{equation}
\begin{equation}\label{eq.psi(12)app.threewave}
    \psi^{(12)}_{app,k}=\int^1_{0}\frac{i\lambda}{L^{d}} \sum\limits_{k_1+k_2=k} H^{(12)}_{k_1k_2k}\xi_{k_1} \xi_{k_2} da 
\end{equation}
\begin{equation}\label{eq.psi(21)app.threewave}
    \psi^{(21)}_{app,k}=\left(\frac{i\lambda}{L^{d}}\right)^2 \sum\limits_{k_1+k_2+k_3=k} H^{(11)}_{k_1k_2k_3k}\xi_{k_1} \xi_{k_2}\xi_{k_3} 
\end{equation}
\begin{equation}\label{eq.psi(22)app.threewave}
    \psi^{(22)}_{app,k}=\int^1_{0}\left(\frac{i\lambda}{L^{d}}\right)^2 \sum\limits_{k_1+k_2+k_3=k} H^{(12)}_{k_1k_2k_3k}\xi_{k_1} \xi_{k_2}\xi_{k_3} da 
\end{equation}

To derive an upper bound for $n^{(12)}(k)$, it suffices to consider $\mathbb E \left|\psi^{(12)}_{app,k}\right|^2$, $\text{Re}\ \mathbb E \left(\psi^{(11)}_{app,k}\overline{\psi^{(12)}_{app,k}}\right)$ and $\text{Re}\  \mathbb E \left(\psi^{(22)}_{app,k}\overline{\xi_k}\right)$ separately.

\textbf{Step 3.1.} (Upper bounds of $\mathbb E \left|\psi^{(12)}_{app,k}\right|^2$ and $\text{Re}\ \mathbb E \left(\psi^{(11)}_{app,k}\overline{\psi^{(12)}_{app,k}}\right)$) We first derive upper bounds for $H^{(1j)}$. For $H^{(11)}$, we have
\begin{equation}
\begin{split}
    |H^{(11)}_{k_1k_2k}|\lesssim& \left|\int^{t}_0\frac{k_x}{i(\Omega+T^{-1}_{\text{max}}\text{sgn}(\Omega))}e^{-isT^{-1}_{\text{max}}\text{sgn}(\Omega)}\frac{d}{ds}e^{i s\Omega+is\, \text{sgn}(\Omega)/T_{\text{max}}} ds\right|
\end{split}
\end{equation}

Integration by parts we get 
\begin{equation}\label{eq.H(11)bound.threewave}
\begin{split}
    |H^{(11)}_{k_1k_2k}|\lesssim&\left|\int^{t}_0\frac{T_{\text{max}}^{-1}k_x}{\Omega+T^{-1}_{\text{max}}\text{sgn}(\Omega)}e^{i s\Omega} ds\right|+\left|\left[\frac{k_x}{i(\Omega+T^{-1}_{\text{max}}\text{sgn}(\Omega)}e^{i s\Omega}\right]_{0}^t\right|
    \\
    \lesssim& \frac{|k_x|}{|\Omega|+T^{-1}_{\text{max}}}
\end{split}
\end{equation}

By a similar integration by parts method, we get
\begin{equation}\label{eq.H(12)bound.threewave}
    |H^{(12)}_{k_1k_2k}|\lesssim \frac{|k|^2|k_x|\nu T_{\text{max}}}{|\Omega|+T^{-1}_{\text{max}}}.
\end{equation}

By \eqref{eq.psi(12)app.threewave}, we know that 
\begin{equation}\label{eq.psi(12)bound.threewave}
\begin{split}
    \mathbb E \left|\psi^{(12)}_{app,k}\right|^2\le& \int^1_{0}\frac{\lambda^2}{L^{2d}} \mathbb E\left|\sum\limits_{k_1+k_2=k} H^{(12)}_{k_1k_2k}\xi_{k_1} \xi_{k_2}\right|^2 da 
    \\
    \le& \frac{2\lambda^2}{L^{2d}} \sup_{a\in[0,1]}\sum\limits_{k_1+k_2=k} \left|H^{(12)}_{k_1k_2k}\right|^2n(k_1) n(k_2)
    \\
    \le& \frac{2\lambda^2}{L^{2d}} (|k|^2|k_x|\nu T_{\text{max}})^2 \sum\limits_{k_1} \frac{n(k_1) n(k-k_1)}{(|\Omega(k_1,k-k_1,k)|+T^{-1}_{\text{max}})^2}
    \\
    \le& \frac{2\epsilon_1^2\lambda^2}{L^{2d}} \max(|k_x|^2,D^2) T_{\text{max}} L^dD^{d-1}=2\epsilon_1^2 \max(|k_x|^2,D^2) T_{\text{max}} T^{-1}_{\text{kin}}D^{d-1} 
\end{split}   
\end{equation}
Here in the second inequality, we apply the Wick theorem to calculate the expectation. In the third inequality, we apply \eqref{eq.H(12)bound.threewave}. In the last line we apply \eqref{eq.asymptoticsbound.threewave} in Theorem \ref{th.numbertheory.threewave} by taking $t=T_{\text{max}}$, $g(x)=\frac{1}{(1+|x|)^2}$ and $F(k_1)=n(k_1) n(k-k_1)$. In the last line we also use the facts that $|k|\le \epsilon_1 l_{d}^{-1}=\epsilon_1 (\nu T_{\text{max}})^{-\frac{1}{2}}$, $T_{\text{kin}}=\frac{1}{8\pi\alpha^2}=\frac{L^{d}}{8\pi\lambda^2}$ and $|k_x|\le |k_{x1}|+|k_{x2}|\lesssim D$.

By \eqref{eq.H(11)bound.threewave} and \eqref{eq.H(12)bound.threewave}, we know that 
\begin{equation}\label{eq.psi(11)(12)bound.threewave}
\begin{split}
    \left|\text{Re}\ \mathbb E \left(\psi^{(11)}_{app,k}\overline{\psi^{(12)}_{app,k}}\right)\right|
    \le& \int^1_{0}\frac{2\lambda^2}{L^{2d}} \text{Re}\,\mathbb E\left(\sum\limits_{k_1+k_2=k} H^{(11)}_{k_1k_2k}\xi_{k_1} \xi_{k_2}  \sum\limits_{k_1+k_2=k} \overline{H^{(12)}_{k_1k_2k}\xi_{k_1} \xi_{k_2}}\right) da 
    \\
    \le& \frac{2\lambda^2}{L^{2d}} \sup_{a\in[0,1]}\sum\limits_{k_1+k_2=k} \left|H^{(11)}_{k_1k_2k}\right|\left|H^{(12)}_{k_1k_2k}\right|n(k_1) n(k_2)
    \\
    \le& \frac{2\lambda^2}{L^{2d}} |k|^2|k_x|^2\nu T_{\text{max}} \sum\limits_{k_1} \frac{n(k_1) n(k-k_1)}{(|\Omega(k_1,k-k_1,k)|+T^{-1}_{\text{max}})^2}D^{d-1}
    \\
    \le& \frac{2\epsilon_1\lambda^2}{L^{2d}} \max(|k_x|^2,D^2) T_{\text{max}} L^d=\epsilon_1 \max(|k_x|^2,D^2) T_{\text{max}} T^{-1}_{\text{kin}} D^{d-1}
\end{split}   
\end{equation}
Here in the second inequality we apply the Wick theorem to calculate the expectation. In the third inequality we apply \eqref{eq.H(11)bound.threewave} and \eqref{eq.H(12)bound.threewave}. In the last line we apply \eqref{eq.asymptoticsbound.threewave} in Theorem \ref{th.numbertheory.threewave} by taking $t=T_{\text{max}}$, $g(x)=\frac{1}{(1+|x|)^2}$ and $F(k_1)=n(k_1) n(k-k_1)$. In the last line we also use the facts that $|k|\le \epsilon_1 l_{d}^{-1}=\epsilon_1 (\nu T_{\text{max}})^{-\frac{1}{2}}$, $T_{\text{kin}}=\frac{1}{8\pi\alpha^2}=\frac{L^{d}}{8\pi\lambda^2}$ and $|k_x|\le |k_{x1}|+|k_{x2}|\lesssim D$.

\eqref{eq.psi(12)bound.threewave} and \eqref{eq.psi(11)(12)bound.threewave} give desire upper bounds of $\mathbb E \left|\psi^{(12)}_{app,k}\right|^2$ and $\text{Re}\ \mathbb E \left(\psi^{(11)}_{app,k}\overline{\psi^{(12)}_{app,k}}\right)$.




\textbf{Step 3.2.} (Upper bound of $\text{Re}\  \mathbb E \left(\psi^{(22)}_{app,k}\overline{\xi_k}\right)$) By \eqref{eq.psi(22)app.threewave}, we have 
\begin{equation}
\text{Re}\  \mathbb E \left(\psi^{(22)}_{app,k}\overline{\xi_k}\right)=\int^1_{0}\left(\frac{i\lambda}{L^{d}}\right)^2 \text{Re}\,\sum\limits_{k_1+k_2+k_3=k} H^{(22)}_{k_1k_2k_3k}\mathbb E\left(\xi_{k_1} \xi_{k_2}\xi_{k_3}\overline{\xi_k}\right) da 
\end{equation}

By Wick theorem, $\mathbb E\left(\xi_{k_1} \xi_{k_2}\xi_{k_3}\overline{\xi_k}\right)=\delta_{k_1=-k_2}\delta_{k_3=k}+\delta_{k_1=-k_3}\delta_{k_2=k}+\delta_{k_1=k}\delta_{k_2=-k_3}$. Therefore we get 
\begin{equation}\label{eq.H(22)wick.threewave}
\begin{split}
    \text{Re}\  \mathbb E \left(\psi^{(22)}_{app,k}\overline{\xi_k}\right)=&\int^1_{0}\left(\frac{i\lambda}{L^{d}}\right)^2 \text{Re}\,\sum\limits_{k_1+k_2+k_3=k} H^{(22)}_{k_1k_2k_3k}(\delta_{k_1=-k_2}\delta_{k_3=k}+\delta_{k_1=-k_3}\delta_{k_2=k}+0) da
    \\
    =&\int^1_{0}2\left(\frac{i\lambda}{L^{d}}\right)^2 \sum\limits_{k_1+k_2+k_3=k} \text{Re}\,\left(H^{(22)}_{k_1,-k_1,k,k}\right) da
\end{split}
\end{equation}
Here in the first equality the term corresponding to $\delta_{k_1=k}\delta_{k_2=-k_3}$ vanishes because $H^{(22)}_{k,k_2,-k_2,k}=0$ and the two terms corresponding to $\delta_{k_1=-k_2}\delta_{k_3=k}$, $\delta_{k_1=-k_3}\delta_{k_2=k}$ are equal.

By \eqref{eq.H(22).threewave}, we get
\begin{equation}
\begin{split}
    H^{(22)}_{k_1,-k_1,k,k}=-2k_{x}(k_{x}-k_{1x})\int_{0\le r<s\le t}&(\nu|k|^2(t-s)+\nu|k-k_1|^2(s-r))
    \\
    & e^{i (s-r)\Omega(k_1,k-k_1,k)- a\nu|k|^2(t-s)-a\nu|k-k_1|^2(s-r)}  dsdr.
\end{split}
\end{equation}

We find upper bound of $\text{Re}\,\left(H^{(22)}_{k_1,-k_1,k,k}\right)$ using integration by parts.
\begin{equation}\label{eq.ReH(22)bound.threewave}
\begin{split}
    &\text{Re}\,\left(H^{(22)}_{k_1,-k_1,k,k}\right)=2k_{x}(k_{x}-k_{1x})\int_{0\le r<s\le t}(\nu|k|^2(t-s)+\nu|k-k_1|^2(s-r))
    \\
    & \frac{e^{irT^{-1}_{\text{max}}sgn\, \Omega}}{i(\Omega+T^{-1}_{\text{max}}sgn\, \Omega)}\frac{d}{dr}e^{i (s-r)\Omega-irT^{-1}_{\text{max}}sgn\, \Omega}e^{- a\nu|k|^2(t-s)-a\nu|k-k_1|^2(s-r)}  dsdr
    \\
    =&\text{Re}\,\frac{2k_{x}(k_{x}-k_{1x})}{i(\Omega+T^{-1}_{\text{max}}sgn\, \Omega)}\int_{0\le s\le t}\nu|k|^2(t-s) e^{- a\nu|k|^2(t-s)}  ds
    \\
    -&\text{Re}\,\frac{2k_{x}(k_{x}-k_{1x})}{i(\Omega+T^{-1}_{\text{max}}sgn\, \Omega)}\int_{0\le s\le t}(\nu|k|^2(t-s)+\nu|k-k_1|^2s) e^{- a\nu|k|^2(t-s)-a\nu|k-k_1|^2s} e^{is\Omega} ds
    \\
    -&\text{Re}\,\frac{2k_{x}(k_{x}-k_{1x})}{i(\Omega+T^{-1}_{\text{max}}sgn\, \Omega)}\int_{0\le r<s\le t}\big[\nu|k-k_1|^2(\nu|k|^2(t-s)+\nu|k-k_1|^2(s-r)-1)
    \\
    & -iT^{-1}_{\text{max}}sgn\, \Omega\big]e^{i (s-r)\Omega- a\nu|k|^2(t-s)-a\nu|k-k_1|^2(s-r)}  dsdr.
\end{split}
\end{equation}

The first term on the right hand side equals $0$ after taking the real part. Using the same integration by parts argument as in \eqref{eq.H(11)bound.threewave}, the second term can be bounded by 
\begin{equation}\label{eq.H(22)secondterm.threewave}
    \frac{|k|^2|k_x|(|k_{1x}|+|k_x|)\nu T_{\text{max}}}{(|\Omega(k_1,k-k_1,k)|+T^{-1}_{\text{max}})^2}.
\end{equation}

The last term can be bounded by integration by parts in the following integral
\begin{equation}\label{eq.asymstep3.2.threewave}
\begin{split}
    \int_{0\le r<s\le t}\big[&\nu|k-k_1|^2(\nu|k|^2(t-s)+\nu|k-k_1|^2(s-r)-1)
    \\
    & -iT^{-1}_{\text{max}}sgn\, \Omega\big]\frac{e^{irT^{-1}_{\text{max}}sgn\, \Omega}}{i(\Omega+T^{-1}_{\text{max}}sgn\, \Omega)}\frac{d}{dr}e^{i (s-r)\Omega}e^{- a\nu|k|^2(t-s)-a\nu|k-k_1|^2(s-r)}  dsdr
\end{split}
\end{equation}

Integration by parts and bound the three resulting integral by taking the absolute value of the integrand, then we get 
\begin{equation}
|\eqref{eq.asymstep3.2.threewave}|\le  \frac{|k|^2}{|\Omega|+T^{-1}_{\text{max}}}
\end{equation}

Therefore, the last term in \eqref{eq.ReH(22)bound.threewave} can also be bounded by \eqref{eq.H(22)secondterm.threewave}.

Then we get 
\begin{equation}
    \text{Re}\,\left(H^{(22)}_{k_1,-k_1,k,k}\right)\lesssim \nu T_{\text{max}}|k|^2|k_x|\frac{|k_{1x}|+|k_x|}{(|\Omega(k_1,k-k_1,k)|+T^{-1}_{\text{max}})^2}.
\end{equation}

Substitute into \eqref{eq.H(22)wick.threewave}, then we have
\begin{equation}\label{eq.psi(22)bound.threewave}
\begin{split}
    \text{Re}\  \mathbb E \left(\psi^{(22)}_{app,k}\overline{\xi_k}\right)&\lesssim\frac{\lambda^2}{L^{2d}} |k|^2|k_x|\nu T_{\text{max}}n(k)\sum\limits_{k_1} \frac{|k_{1x}|+|k_x|}{(|\Omega(k_1,k-k_1,k)|+T^{-1}_{\text{max}})^2}n(k_1)
    \\
    &\lesssim \frac{\epsilon_1\lambda^2}{L^{2d}}|k_x|\sum\limits_{k_1} \frac{D}{(|\Omega(k_1,k-k_1,k)|+T^{-1}_{\text{max}})^2}n(k_1)
    \\
    &+\frac{\epsilon_1\lambda^2}{L^{2d}}|k_x|^2\sum\limits_{k_1} \frac{1}{(|\Omega(k_1,k-k_1,k)|+T^{-1}_{\text{max}})^2}n(k_1)
    \\
    &\lesssim \epsilon_1\frac{T_{\text{max}}}{T_{\text{kin}}}\max(D^{d+1},|k_x|^2D^{d-1}+|k_x|D^{d})
\end{split}
\end{equation}
In the second inequality, we also use the facts that $|k|\le \epsilon_1 l_{d}^{-1}=\epsilon_1 (\nu T_{\text{max}})^{-\frac{1}{2}}$ and $T_{\text{kin}}=\frac{1}{8\pi\alpha^2}=\frac{L^{d}}{8\pi\lambda^2}$. In the last inequality we apply \eqref{eq.asymptoticsbound.threewave} in Theorem \ref{th.numbertheory.threewave} by taking $t=T_{\text{max}}$, $g(x)=\frac{1}{(1+|x|)^2}$, $F(k_1)=n(k_1) n(k-k_1)$ and $|k_x|\le |k_{x1}|+|k_{x2}|\lesssim D$. 


Combining \eqref{eq.psi(12)bound.threewave}, \eqref{eq.psi(11)(12)bound.threewave} and \eqref{eq.psi(22)bound.threewave}, we get the following upper bound
\begin{equation}\label{eq.n(12)final.threewave}
    n^{(12)}(k)\lesssim \epsilon_1\frac{T_{\text{max}}}{T_{\text{kin}}}\underbrace{\max(D^{d+1},|k_x|^2D^{d-1}+|k_x|D^{d})}_{\text{Err}_{D}(k_x)}.
\end{equation}
% The derivation of upper bounds of $H^{(21)}$ and $H^{(22)}$ is similar to that in the proof of Lemma \ref{lem.boundcoef'}. We just present the derivation for $H^{(21)}$. Define $\Omega_1=\Omega(k_1,k_2+k_3,k)$ and $\Omega_2=\Omega(k_2,k_3,k_2+k_3)$, then we have
% \begin{equation}
% \begin{split}
%     &|H^{(21)}_{k_1k_2k}|\lesssim \left|k_{x}(k_{2x}+k_{3x})\int_{0\le r<s\le t}e^{i s\Omega_1} e^{i r\Omega_2} dsdr\right|
%     \\
%     \le&|k_{x}|(|k_{2x}|+|k_{3x}|)\left|\int_{0\le r<s\le t} \frac{1}{i(\Omega_1+T^{-1}_{\text{max}}\text{sgn}(\Omega_1))}e^{-isT^{-1}_{\text{max}}\text{sgn}(\Omega_1)}\frac{d}{ds}e^{i s\Omega_1+isT^{-1}_{\text{max}}\text{sgn}(\Omega_1)} e^{i r\Omega_2} dsdr\right|
% \end{split}
% \end{equation}

% Integration by parts we get 
% \begin{equation}
% \begin{split}
%     &|H^{(21)}_{k_1k_2k}|
%     \\
%     \lesssim&|k_{x}|(|k_{2x}|+|k_{3x}|)\left| \frac{T^{-1}_{\text{max}}}{i(\Omega_1+T^{-1}_{\text{max}}\text{sgn}(\Omega_1))}\int_{0\le r<s\le t}e^{i s\Omega_1} e^{i r\Omega_2} dsdr\right|
%     \\
%     +&|k_{x}|(|k_{2x}|+|k_{3x}|)\left| \frac{1}{i(\Omega_1+T^{-1}_{\text{max}}\text{sgn}(\Omega_1))}\int_{0\le r\le t}e^{i t\Omega_1} e^{i r\Omega_2} dr\right|
%     \\
%     +&|k_{x}|(|k_{2x}|+|k_{3x}|)\left| \frac{T^{-1}_{\text{max}}\text{sgn}(\Omega_1))}{i(\Omega_1+T^{-1}_{\text{max}}\text{sgn}(\Omega_1))} \int_{0\le r\le t}e^{i r(\Omega_1+\Omega_2)} dr\right|
% \end{split}
% \end{equation}

% We can do integration by parts with respect to $r$. Then the three terms in above equation can be bounded using the same method as $|H^{(11)}_{k_1k_2k}|$. Finally we get 
% \begin{equation}
% |H^{(21)}_{k_1k_2k}|\lesssim|k_{x}|(|k_{2x}|+|k_{3x}|)\left(\frac{1}{|\Omega_1|+T^{-1}_{\text{max}}}\frac{1}{|\Omega_2|+T^{-1}_{\text{max}}}+\frac{1}{|\Omega_1|+T^{-1}_{\text{max}}}\frac{1}{|\Omega_1+\Omega_2|+T^{-1}_{\text{max}}}\right)
% \end{equation}

% Similarly we get
% \begin{equation}
% \begin{split}
%     |H^{(22)}_{k_1k_2k}|\lesssim&|k_{x}|(|k_{2x}|+|k_{3x}|)(\nu T_{\text{max}}(|k_1|^2+|k_2|^2+|k|^2))^2
%     \\
%     &\left(\frac{1}{|\Omega_1|+T^{-1}_{\text{max}}}\frac{1}{|\Omega_2|+T^{-1}_{\text{max}}}+\frac{1}{|\Omega_1|+T^{-1}_{\text{max}}}\frac{1}{|\Omega_1+\Omega_2|+T^{-1}_{\text{max}}}\right)
% \end{split}
% \end{equation}


\textbf{Step 4.} (Asymptotics of $n^{(11)}(k)$) By \eqref{eq.n(11).threewave} and Wick theorem, we get 
\begin{equation}\label{eq.n(11)asym.threewave}
\begin{split}
    n^{(11)}(k)=&\frac{2\lambda^2}{L^{2d}} |k_x|^2\sum\limits_{k_1+k_2=k}n(k_1) n(k_2) \left|\int^{t}_0e^{i s\Omega(k_1,k_2,k)} ds\right|^2
    \\
    -&\frac{8\lambda^2}{L^{2d}}\sum_{k_1}k_x(k_x-k_{1x})n(k_1) n(k)\text{Re}\left(\int_{0\le r<s\le t} e^{i (s-r)\Omega(k_1,k-k_1,k)}  dsdr\right)
    \\
    =&\frac{2\lambda^2}{L^{2d}} |k_x|^2\sum\limits_{k_1+k_2=k}n(k_1) n(k_2) \frac{4\sin^2 \left(\frac{t}{2}\Omega(k_1,k_2,k\right))}{\Omega^2(k_1,k_2,k)}
    \\
    -&\frac{8\lambda^2}{L^{2d}}\sum_{k_1}k_x(k_x-k_{1x})n(k_1) n(k) \frac{2\sin^2 \left(\frac{t}{2}\Omega(k_1,k-k_1,k\right))}{\Omega^2(k_1,k-k_1,k)}
    \\
    =&8\pi \alpha^2t|k_x|^2\int_{\substack{(k_1, k_2)\in \mathbb{R}^{2d}\\k_1+k_2=k}}n(k_1) n(k_2)\delta(|k_1|^2k_{1x}+|k_2|^2k_{2x}-|k|^2k_{x})\, dk_1 dk_2
    \\
    -& 16\pi \alpha^2t\, n(k)\int_{\mathbb{R}^d}k_x(k_x-k_{1x})n(k_1) \delta(|k_1|^2k_{1x}+|k_2|^2k_{2x}-|k|^2k_{x})\, dk_1+O\left(L^{-\theta}\frac{T_{\text{max}}}{T_{\mathrm {kin}}}\right)
    \\
    =&\frac{t}{T_{\text{kin}}}\mathcal{K}(n)(k)+O\left(L^{-\theta}\frac{T_{\text{max}}}{T_{\mathrm {kin}}}\right)
\end{split}
\end{equation}
Here in the third equality we apply \eqref{eq.numbertheory1.threewave} in Theorem \ref{th.numbertheory.threewave} by taking $t\rightarrow\frac{t}{2}$, $g(x)=\frac{\sin^2(x)}{x^2}$ and $F(k_1)=n(k_1) n(k-k_1)$ or $F(k_1)=n(k_1)$. In the third equality we also use the facts that  $T_{\text{kin}}=\frac{1}{8\pi\alpha^2}=\frac{L^{d}}{8\pi\lambda^2}$.

Combining \eqref{eq.n(12)final.threewave} and \eqref{eq.n(11)asym.threewave}, we complete the prove of the first case in \eqref{eq.n1.threewave}.
\end{proof}
% \textbf{Step 5.} (Upper bound for $|k|\ge  l_{d}^{-1}$) By \eqref{eq.n(1).threewave} and Wick theorem, we get 
% \begin{equation}
% \begin{split}
%     n^{(1)}(k)=&\frac{2\lambda^2}{L^{2d}} |k_x|^2\sum\limits_{k_1+k_2=k}n(k_1) n(k_2) \left|\int^{t}_0e^{i s\Omega(k_1,k_2,k)- \nu|k|^2(t-s)} ds\right|^2
%     \\
%     -&\frac{8\lambda^2}{L^{2d}}\sum_{k_1}k_x(k_x-k_{1x})n(k_1) n(k)\text{Re}\left(\int_{0\le r<s\le t} e^{i (s-r)\Omega(k_1,k-k_1,k)- \nu|k|^2(t-s)- \nu|k-k_1|^2(s-r)}  dsdr\right)
%     \\
%     =&\frac{2\lambda^2}{L^{2d}} |k_x|^2\sum\limits_{k_1+k_2=k}n(k_1) n(k_2) \frac{4\sin^2 \left(\frac{t}{2}\Omega(k_1,k_2,k\right))}{\Omega^2(k_1,k_2,k)}
%     \\
%     -&\frac{8\lambda^2}{L^{2d}}\sum_{k_1}k_x(k_x-k_{1x})n(k_1) n(k) \frac{2\sin^2 \left(\frac{t}{2}\Omega(k_1,k-k_1,k\right))}{\Omega^2(k_1,k-k_1,k)}
%     \\
%     =&8\pi \alpha^2t|k_x|^2\int_{\substack{(k_1, k_2)\in \mathbb{R}^{2d}\\k_1+k_2=k}}n(k_1) n(k_2)\delta(|k_1|^2k_{1x}+|k_2|^2k_{2x}-|k|^2k_{x})\, dk_1 dk_2
%     \\
%     -& 16\pi \alpha^2t\, n(k)\int_{\mathbb{R}^d}k_x(k_x-k_{1x})n(k_1) \delta(|k_1|^2k_{1x}+|k_2|^2k_{2x}-|k|^2k_{x})\, dk_1+O\left(L^{-\theta}\frac{T_{\text{max}}}{T_{\mathrm {kin}}}\right)
%     \\
%     =&\frac{t}{T_{\text{kin}}}\mathcal{K}(n)(k)+O\left(L^{-\theta}\frac{T_{\text{max}}}{T_{\mathrm {kin}}}\right)
% \end{split}
% \end{equation}
