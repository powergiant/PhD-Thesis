\section{Lattice points counting and convergence results}
In this section, we give an incomplete proof of Proposition \ref{prop.treetermsupperbound} which gives upper bounds for some tree terms $\mathcal{J}_{T,k}$. The proof is divided into several steps.

In section \ref{sec.refexp}, we calculate the coefficients of $\mathcal{J}_{T,k}$ as polynomials of Gaussian random variables.

Large deviation theory suggests that an upper bound of a Gaussian polynomial can be derived from an upper bound of its expectation and variance.

In section \ref{sec.coupwick}, we introduce the concept of couples which is a graphical method of calculating the expectation of Gaussian polynomials. In this section, we also introduce and prove the formal version of our renormalized Wick theorem.

In section \ref{sec.numbertheory}, we use couples to establish a lattice point counting result. This result says gives the upper bound of numbers of the lattice points near the resonance surface and this bound is strong enough to derive the wave kinetic equation up to epsilon loss, \textbf{if there is no long bad chain of circle nodes in the renormalization forest}.

In section \ref{sec.heuristic}, we introduce a cancellation identity and explain \textbf{heuristically} why the Wick theorem is sufficient to exclude all bad couples.

% In the last section of \ref{sec.uppcoef}, we give an upper bounds for the coefficients of these Gaussian polynomials.





\subsection{Refined expression of coefficients}\label{sec.refexp} In this section, we calculate the coefficients of $\mathcal{J}_{T,k}$ using the definition \eqref{eq.treeterm} of them.



From \eqref{eq.intrenorm} and \eqref{eq.treeterm}, it is easy to show that $\mathcal{J}_{T,k}$ are polynomials of $\xi$ if we remove the renormalization symbol $:\cdot:$. Given this renormalization symbol, we know that $\mathcal{J}_{T,k}$ should be a renormalized polynomial of $\xi$ that will be defined in the following definition. An example of renormalized polynomial is $(:(:\xi_{k_1}\bar{\xi}_{k_2}:)\xi_{k_3}\bar{\xi}_{k_4}:)\bar{\xi}_{k_5}\xi_{k_6}$.

\begin{defn} \label{def.renorm}
\begin{enumerate}
 \item \textbf{The renormalization symbol.} For any random variable $X$, define the \underline{renormalization symbol} by $:X:=X-\mathbb{E}X$.
 \item \textbf{Some definitions about forests.} Given a \underline{forest} $F$, a node $\mathfrak{r}$ is a \underline{root} if it does not have a parent. Notice that a forest may possess multiple roots. Given a node $\mathfrak{n}\in T$, its \underline{subtree} $T_{\mathfrak{n}}$ is defined in the same way as the subtrees of trees. Assume that all the \textbf{non-leaf} children of $\mathfrak{n}$ are $\mathfrak{n}_1,\cdots, \mathfrak{n}_k$, then the \underline{subforest} $F_{\mathfrak{n}}$ of $\mathfrak{n}$ is defined as the union of all subtrees $T_{\mathfrak{n}_1}$, $\cdots$, $T_{\mathfrak{n}_k}$. The \underline{set of leaves} of $F_{\mathfrak{n}}$ (resp. $T_{\mathfrak{n}}$) is denoted by $L(F_{\mathfrak{n}})$ (resp. $L(T_{\mathfrak{n}})$). Each leaf of the forest is labeled by a number, called the \underline{label} of this leaf. The example of roots, subtrees, and subforests can be found in Figure \ref{fig.forests}.
 \begin{figure}[H]
 \centering
 \scalebox{0.3}{
 \begin{tikzpicture}[level distance=80pt, sibling distance=60pt]
 \node[circ] (1) {} 
 child {node[fillstar, xshift = -5cm] (11) {}}
 child {node[fillstar, xshift = -3cm] (12){}}
 child {node[circ, xshift = 2cm] (13) {}
 child {node[circ, xshift = -1cm] (131) {}
 child {node[fillstar] (1311) {}}
 child {node[fillstar] (1312) {}}
 child {node[fillstar] (1313) {}}
 }
 child {node[fillstar] (132) {}}
 child {node[circ, xshift = 1cm] (133) {}
 child {node[fillstar] (1331) {}}
 child {node[fillstar] (1332) {}}
 child {node[fillstar] (1333) {}}
 }
 };
 \node[scale=2.0] at ($(1)+(1.5,0)$) {root};
 %\node[scale=2.0] at (1.5,0) {root}; 
 % \node[scale=2.0] at (3.6,-2.8) {$\mathfrak{n}$}; 
 \node[scale=2.0] at ($(13)+(0.5,0)$) {$\mathfrak{n}$}; 
 
 \node[rectangle, draw, minimum width = 11.8cm, minimum height = 7cm] at (4.3,-5.7) {};
 \node[scale=2.0] at (4.3,-10) {subtree $F_\mathfrak{n}$}; 

 \node[circ] (2) at (20,0) {} 
 child {node[circ, xshift = -1cm] (21) {}
 child {node[circ, xshift = -1cm] (211) {}
 child {node[fillstar] (2111) {}}
 child {node[fillstar] (2112) {}}
 child {node[fillstar] (2113) {}}
 }
 child {node[fillstar, yshift = 1cm] (212) {}}
 child {node[circ, xshift = 1cm] (213) {}
 child {node[fillstar] (2131) {}}
 child {node[fillstar] (2132) {}}
 child {node[fillstar] (2133) {}}
 }
 }
 child {node[fillstar] (22) {}}
 child {node[fillstar, xshift = 1cm] (23) {}};
 
 \node[scale=2.0] at ($(2)+(1.5,0)$) {root};
 % \node[scale=2.0] at (21.5,0) {root}; 
 % \node[scale=2.0] at (17.6,-2.8) {$\mathfrak{m}$}; 
 \node[scale=2.0] at ($(21)+(0.6,0)$) {$\mathfrak{m}$}; 
 \node[rectangle, draw, minimum width = 11.8cm, minimum height = 4cm] at (17.2,-7) {};
 \node[scale=2.0] at (17.6,-10) {subforest $F_\mathfrak{m}$}; 
 
 \node[circ] (3) at (30, 0) {}
 child {node[circ] (31) {}
 child {node[fillstar] (311) {}}
 child {node[fillstar] (312) {}}
 child {node[fillstar] (313) {}}
 }
 child {node[fillstar] (32) {}}
 child {node[fillstar] (33) {}};

 % \node[scale=2.0] at (31.5,0) {root}; 
 \node[scale=2.0] at ($(3)+(1.5,0)$) {root};

 \node[scale=2.0] at ($(11)+(0.5,-0.1)$) {$1$};
 \node[scale=2.0] at ($(12)+(0.5,-0.1)$) {$2$};
 \node[scale=2.0] at ($(1311)+(0.5,-0.1)$) {$3$};
 \node[scale=2.0] at ($(1312)+(0.5,-0.1)$) {$4$};
 \node[scale=2.0] at ($(1313)+(0.5,-0.1)$) {$5$};
 \node[scale=2.0] at ($(132)+(0.5,-0.1)$) {$6$};
 \node[scale=2.0] at ($(1331)+(0.5,-0.1)$) {$7$};
 \node[scale=2.0] at ($(1332)+(0.5,-0.1)$) {$8$};
 \node[scale=2.0] at ($(1333)+(0.5,-0.1)$) {$9$};

 \node[scale=2.0] at ($(2111)+(0.6,-0.1)$) {$10$};
 \node[scale=2.0] at ($(2112)+(0.6,-0.1)$) {$11$};
 \node[scale=2.0] at ($(2113)+(0.6,-0.1)$) {$12$};
 \node[scale=2.0] at ($(212)+(0.6,-0.1)$) {$13$};
 \node[scale=2.0] at ($(2131)+(0.6,-0.1)$) {$14$};
 \node[scale=2.0] at ($(2132)+(0.6,-0.1)$) {$15$};
 \node[scale=2.0] at ($(2133)+(0.6,-0.1)$) {$16$};
 \node[scale=2.0] at ($(22)+(0.6,-0.1)$) {$17$};
 \node[scale=2.0] at ($(23)+(0.6,-0.1)$) {$18$};

 \node[scale=2.0] at ($(311)+(0.6,-0.1)$) {$19$};
 \node[scale=2.0] at ($(312)+(0.6,-0.1)$) {$20$};
 \node[scale=2.0] at ($(313)+(0.6,-0.1)$) {$21$};
 \node[scale=2.0] at ($(32)+(0.6,-0.1)$) {$22$};
 \node[scale=2.0] at ($(33)+(0.6,-0.1)$) {$23$};

 \end{tikzpicture}
 }
 \caption{Example of roots, subtrees, and subforests}
 \label{fig.forests}
 \end{figure}
 \item \textbf{Renormalization by a forest.} Given a forest $F$ and a monomial $\xi_{k_1}\cdots \xi_{k_{2m}}$, we define $[\xi_{k_1}\cdots \xi_{k_{2m}}]_{F}$, the \underline{renormalization} of $\xi_{k_1}\cdots \xi_{k_{2m}}$ by $F$, inductively. If $F=\emptyset$, we define $[\xi_{k_1}\cdots \xi_{k_{2m}}]_{F}=\xi_{k_1}\cdots \xi_{k_{2m}}$. If the root nodes of $F$ are $\mathfrak{r}_1,\cdots, \mathfrak{r}_r$ and their subforests are $F_{\mathfrak{r}_1},\cdots, F_{\mathfrak{r}_r}$, assume that the renormalizations by these subforests have been defined, then we define $[\xi_{k_1}\cdots \xi_{k_{2m}}]_{F}=\prod_{i\notin L(F)} \xi_{k_i} \prod_{j=1}^r :\left(\left[\prod_{i_j\in L(T_{\mathfrak{r}_j})}\xi_{k_{i_j}}\right]_{F_{\mathfrak{r}_j}}\right):$, where the number $i_j\in L(T_{\mathfrak{r}_j}$ if and only if the leaf labeled by $i_j$ belongs to $L(T_{\mathfrak{r}_j})$.
 
 We say $[\xi_{k_1}\cdots \xi_{k_{2m}}]_{F}$ to be a \underline{renormalized monomial} and a sum of renormalized monomial to be a \underline{renormalized polynomial}. 
 \item \textbf{Renormalization forests.} Given a ternary tree $T$ whose nodes are decorated by $\bullet$, $\circ$, $\star$, then the \underline{associated renormalization forest} $R(T)$ of $T$ is defined to be the unique forest determined by these conditions. First, it is formed by all $\circ$, $\star$ nodes in $T$. Second, a node $\mathfrak{n}$ in $R(T)$ is a parent of another node $\mathfrak{n}'$ if and only if $\mathfrak{n}$ is the closest one in the ancestors of $\mathfrak{n}'$ that satisfies the property that the unique path from $\mathfrak{n}$ to $\mathfrak{n}'$ contains at least one primary or secondary child edge of a $\circ$ node. (Actually, in this case, this child edge is unique and is at the beginning of this path.) Third, remove all nodes whose path to the root does not contain any primary or secondary child edge. 
 
 One example of this construction can be found in Figure \ref{fig.forestsr}.
 
 \begin{figure}[H]
 \centering
 \scalebox{0.36}{
 \begin{tikzpicture}[level distance=80pt, sibling distance=60pt]
 \draw node[fillcirc](1) at (0,0) {}
 child {node[circ, xshift = -2cm] (11) {}
 child {node[circ, xshift = -1cm] (111) {}
 child {node[fillstar] (1111) {}}
 child {node[fillstar] (1112) {}}
 child {node[fillstar] (1113) {}}
 }
 child {node[fillstar] (112) {}}
 child {node[fillstar, xshift = 1cm] (113) {}}
 }
 child {node[fillstar] (12) {}}
 child {node[fillcirc, xshift = 2cm] (13) {}
 child {node[fillstar, xshift = -1cm] (311) {}}
 child {node[fillstar] (312) {}}
 child {node[circ, xshift = 1cm] (313) {}
 child {node[fillstar] (3131) {}}
 child {node[fillstar] (3132) {}}
 child {node[fillstar] (3133) {}}
 }
 }; 
 \node[scale=1.5] at (-8.9,-8.5) {$1$};
 \node[scale=1.5] at (-6.7,-8.5) {$2$};
 \node[scale=1.5] at (-4.6,-8.5) {$3$};
 \node[scale=1.5] at (-3.6,-5.65) {$4$}; 
 \node[scale=1.5] at (-0.5,-5.65) {$5$};
 \node[scale=1.5] at (0.5,-2.8) {$6$};
 \node[scale=1.5] at (1.5,-5.65) {$7$};
 \node[scale=1.5] at (4.6,-5.65) {$8$}; 
 \node[scale=1.5] at (5.6,-8.5) {$9$};
 \node[scale=1.5] at (7.7,-8.5) {$10$};
 \node[scale=1.5] at (9.9,-8.5) {$11$};
 
 \node[draw, single arrow,
 minimum height=66mm, minimum width=16mm,
 single arrow head extend=2mm,
 anchor=west, rotate=0] at (10,-3) {}; 
 \node[circ, xshift = -2cm] (2) at (25, 0) {}
 child {node[circ, xshift = -1cm] (21) {}
 child {node[fillstar, xshift = -1cm] (211) {}}
 child {node[fillstar, xshift = 1cm] (212) {}}
 }
 child {node[fillstar] (22) {}}
 child {node[fillstar, xshift = 1cm] (23) {}};
 \node[scale=1.5] at (18.2,-5.65) {$1$};
 \node[scale=1.5] at (22.5,-5.65) {$2$};
 \node[scale=1.5] at (23.5,-2.8) {$3$}; 
 \node[scale=1.5] at (26.6,-2.8) {$4$};
 \node[circ, xshift = -1cm] (3) at (31, 0) {}
 child {node[fillstar, xshift = -1cm] (31) {}}
 child {node[fillstar, xshift = 1cm] (32) {}}; 
 \node[scale=1.5] at (28.5,-2.8) {$9$}; 
 \node[scale=1.5] at (32.6,-2.8) {$10$};
 \end{tikzpicture}
 }
 \caption{Construction of renormalization forests}
 \label{fig.forestsr}
 \end{figure}
\end{enumerate}
\end{defn}

Now we show that $\mathcal{J}_{T,k}$ is a renormalized polynomial and calculate the coefficients of $\mathcal{J}_{T,k}$.

\begin{lem}\label{lem.treeterms} Given a tree $T$ of depth $l=l(T)$, denote by $T_{\text{in}}$ the tree formed by all non-leaf nodes $\mathfrak{n}$, then associate each node $\mathfrak{n}\in T_{\text{in}}$ and edge $\mathfrak{l}\in T$ with variables $t_{\mathfrak{n}}$ and $k_{\mathfrak{l}}$ respectively. Given a labelling of all leaves by $1$, $2$, $\cdots$, $2l+1$, we identify $k_{\mathfrak{e}}$ with $k_j$ if $\mathfrak{e}$ connects a leaf labelled by $j$. Given a non-leaf node $\mathfrak{n}$, let $\mathfrak{e}_1$, $\mathfrak{e}_2$, $\mathfrak{e}_3$, $\mathfrak{e}$ be the four edges from or pointing to $\mathfrak{n}$ ($\mathfrak{e}$ is the parent of $\mathfrak{e}_1$, $\mathfrak{e}_2$, $\mathfrak{e}_3$), $\mathfrak{n}_1$, $\mathfrak{n}_2$, $\mathfrak{n}_3$ be children of $\mathfrak{n}$ and $\hat{\mathfrak{n}}$ be the parent of $\mathfrak{n}$.

Let $\mathcal{J}_T$ be terms defined in Definition \ref{def.treeterms}, then their Fourier coefficients $\mathcal{J}_{T,k}$ are degree $2l+1$ renormalized polynomials of $\xi$ given by the following formula

\begin{equation}\label{eq.coefterm}
\mathcal{J}_{T,k}=\left(\frac{-i\lambda^2}{L^{2d}}\right)^l\sum_{k_1,\, k_2,\, \cdots,\, k_{2l+1}} H^T_{k_1\cdots k_{2l+1}} [\xi_{k_1}\bar{\xi}_{k_2}\cdots\xi_{k_{2l+1}}]_{R(T)} 
\end{equation}
where $H^T_{k_1\cdots k_{2l+1}}$ is given by
\begin{equation}\label{eq.coef}
H^T_{k_1\cdots k_{2l+1}}=\int_{\cup_{\mathfrak{n}\in T_{\text{in}}} A_{\mathfrak{n}}} e^{-i\sum_{\mathfrak{n}\in T_{\text{in}}} %\iota_{\mathfrak{n}}
(t_{\mathfrak{n}}\Omega_{\mathfrak{n}}+\widetilde{\Omega}_{\mathfrak{n}})} \prod_{\mathfrak{n}\in T_{\text{in}}} dt_{\mathfrak{n}} %\prod_{\mathfrak{n}\in T_{\text{in}} %\textit{with children $\mathfrak{n}_1$, $\mathfrak{n}_2$, $\mathfrak{n}_3$ }}
\ \delta_{\cap_{\mathfrak{n}\in T_{\text{in}}} S_{\mathfrak{n}}}\ \prod_{%\mathfrak{e}\neq \text{leg }\mathfrak{l}
}\Lambda^{-1}(k_{\mathfrak{e}}) ,
\end{equation}
and $ \iota_{\mathfrak{e}}$ is defined by Definition \ref{def.tree} (12), $A_{\mathfrak{n}}$, $S_{\mathfrak{n}}$, $\Omega_{\mathfrak{n}}$, $\widetilde{\Omega}_{\mathfrak{n}}$ are defined by 
% \begin{equation}\label{eq.iotadef}
% \iota_{\mathfrak{e}}=\begin{cases}
% +1 \qquad \textit{if $\mathfrak{e}$ pointing inwards to $\mathfrak{n}$}
% \\
% -1 \qquad \textit{if $\mathfrak{e}$ pointing outwards from $\mathfrak{n}$}
% \end{cases}
% \end{equation}
\begin{equation}
 A_{\mathfrak{n}}=
 \begin{cases}
 \{t_{\mathfrak{n}_1},\, t_{\mathfrak{n}_2},\, t_{\mathfrak{n}_3}\le t_{\mathfrak{n}}\} \qquad \textit{if }\mathfrak{n}\ne\textit{ the root }\mathfrak{r}
 \\
 \{t_{\mathfrak{r}}\le t\} \qquad\qquad\qquad\  \textit{if }\mathfrak{n}= \mathfrak{r}
 \end{cases}
\end{equation}
\begin{equation}
 S_{\mathfrak{n}}=
 \begin{cases}
 \{k_{\mathfrak{e}_1}-k_{\mathfrak{e}_2}+k_{\mathfrak{e}_3}-k_{\mathfrak{e}}=0,\ k_{\mathfrak{e}_1}\ne k_{\mathfrak{e}_2}\ne k_{\mathfrak{e}_3} \qquad \ \  \textit{if }\mathfrak{n}\textit{ is decorated by }\bullet
 \\
 \qquad\qquad\qquad\qquad\text{or}\ k_{\mathfrak{e}_1}= k_{\mathfrak{e}_2}= k_{\mathfrak{e}_3}=k\}, 
 \\
 \{k_{\mathfrak{e}_1}=k_{\mathfrak{e}_2},\ k_{\mathfrak{e}_3}=k_{\mathfrak{e}}\} \qquad\qquad\qquad\qquad\qquad\quad\ \textit{if }\mathfrak{n}\textit{ is decorated by }\circ
 \end{cases}
\end{equation}
\begin{equation}
 \Omega_{\mathfrak{n}}=
 \begin{cases}
 \iota_{\mathfrak{e}_1}\Lambda_{k_{\mathfrak{e}_1}}+\iota_{\mathfrak{e}_2}\Lambda_{k_{\mathfrak{e}_2}}+\iota_{\mathfrak{e}_3}\Lambda_{k_{\mathfrak{e}_3}}+\iota_{\mathfrak{e}}\Lambda_{k_{\mathfrak{e}}} \qquad \textit{if }\mathfrak{n}\textit{ is decorated by }\bullet
 \\
 0 \qquad\qquad\qquad\qquad\qquad\qquad \qquad\quad\ \ \textit{if }\mathfrak{n}\textit{ is decorated by }\circ
 \end{cases}
\end{equation}
\begin{equation}
 \widetilde{\Omega}_{\mathfrak{n}}=
 \begin{cases}
 2\alpha \int^t_{0}m(s) ds\left(\iota_{\mathfrak{e}_1}\Lambda_{k_{\mathfrak{e}_1}}^{-1}+\iota_{\mathfrak{e}_2}\Lambda_{k_{\mathfrak{e}_2}}^{-1}+\iota_{\mathfrak{e}_3}\Lambda_{k_{\mathfrak{e}_3}}^{-1}+\iota_{\mathfrak{e}}\Lambda_{k_{\mathfrak{e}}}^{-1}\right) \qquad \textit{if }\mathfrak{n}\textit{ is decorated by }\bullet
 \\
 0 \qquad\qquad\qquad\qquad\qquad\qquad\qquad\qquad\qquad\qquad\quad\ \ \ \ \  \textit{if }\mathfrak{n}\textit{ is decorated by }\circ
 \end{cases}
\end{equation}
For root node $\mathfrak{r}$, we impose the constrain that $k_{\mathfrak{r}}=k$ and $t_{\widehat{\mathfrak{r}}}=t$ (notice that $\mathfrak{r}$ does not have a parent so $\widehat{\mathfrak{r}}$ is not well defined). 
\end{lem}


\begin{proof}
We can check that $\mathcal{J}_T$ defined by \eqref{eq.coefterm} and \eqref{eq.coef} satisfies the recursive formula \eqref{eq.treeterm} by direct substitution, so they are the unique solution of that recursive formula, and this proves Lemma \ref{lem.treeterms}.
\end{proof}


The following table provides a dictionary between concepts in Definition \ref{def.tree} in the previous section and expressions in tree terms $\mathcal{J}_{T,k}$ in Lemma \ref{lem.treeterms}.

\begin{table}[H]
\begin{center}
 \begin{longtable}{|c|c|}
 \hline
 & \\
 Concepts & Corresponding expressions \\
 & \\
 \hline
 & \\
 Ternary tree $T$ & Tree term $\mathcal{J}_{T,k}$\\
 & \\
 \hline
 & \\
 Edge $\mathfrak{e}$ & Summation index $k_{\mathfrak{e}}$\\
 & \\
 \hline
 & \\
 Leaf edges $\mathfrak{l}$ labelled by $j$ & Summation index $k_{j}$\\
 & \\
 \hline
 & \\
 $\star$ node $\mathfrak{n}$ & Two expressions $\widetilde{\Omega}_{\mathfrak{n}}$ and $k_{\mathfrak{e}_1}-k_{\mathfrak{e}_2}+k_{\mathfrak{e}_3}-k_{\mathfrak{e}}=0$\\
 & \\
 \hline
 & \\
 $\circ$ node $\mathfrak{n}$ & Expression $k_{\mathfrak{e}_1}=k_{\mathfrak{e}_2},\ k_{\mathfrak{e}_3}=k_{\mathfrak{e}}$\\
 & \\
 \hline
 & \\
 Leaf $\mathfrak{n}_{\mathfrak{l}}$ labelled by $j$ with a sign & $\xi_{k_j}$ or $\bar{\xi}_{k_j}$ depending on the sign\\
 & \\
 \hline
 & \\
 Direction of edges & $\iota_{\mathfrak{e}}$\\
 & \\
 \hline
 & \\
 Leg & The fixed index $k$\\
 & \\
 \hline
 & \\
 Conjugated tree $\overline{T}$ & Tree term $\overline{\mathcal{J}_{T,k}}$ \\
 & \\
 \hline
 \end{longtable}
\end{center}
\caption{\label{tab.dict} A dictionary between concepts of trees and expressions in tree terms.}
\end{table}

Lemma \ref{lem.treeterms} suggests that the coefficients is small when $|\Omega_{\mathfrak{n}}|\gg \omega$, $\omega$ is supposed to be $\frac{1}{T_{\text{kin}}}$. Therefore, in order to bound $\mathcal{J}_{T,k}$, we should count the lattice points on $|\Omega_{\mathfrak{n}}|\lesssim \omega$
\begin{equation}\label{eq.diophantineeq'}
 \{k_{\mathfrak{e}}\in \mathbb{Z}^d,\ |k_{\mathfrak{e}}|\lesssim 1,\ \forall \mathfrak{e}: |\Omega_{\mathfrak{n}}|\lesssim \omega,\ \forall \mathfrak{n}. \ \{k_{\mathfrak{e}}\}_{\mathfrak{e}}\in \cap_{\mathfrak{n}\in T_{\text{in}}} S_{\mathfrak{n}}.\ k_{\mathfrak{l}}=k\}
\end{equation}


\eqref{eq.diophantineeq'} can be read from the tree diagrams $T$. As in Figure \ref{fig.equations}, each edge corresponds to a variable $k_{\mathfrak{e}}$. The leg $\mathfrak{l}$ corresponds to equation $k_{\mathfrak{l}}=k$. Each node $\mathfrak{n}$ is connected with four edges $\mathfrak{e}_1$, $\mathfrak{e}_2$, $\mathfrak{e}_3$, $\mathfrak{e}$ whose corresponding variables $k_{\mathfrak{e}_1}$, $k_{\mathfrak{e}_2}$, $k_{\mathfrak{e}_3}$, $k_{\mathfrak{e}}$ satisfy the momentum conservation equation
\begin{equation}
\begin{split}
k_{\mathfrak{e}_1}-&k_{\mathfrak{e}_2}+k_{\mathfrak{e}_3}-k_{\mathfrak{e}}=0,\ k_{\mathfrak{e}_1}\ne k_{\mathfrak{e}_2}\ne k_{\mathfrak{e}_3},\ \text{or}\ k_{\mathfrak{e}_1}= k_{\mathfrak{e}_2}= k_{\mathfrak{e}_3}=k_{\mathfrak{e}}
\end{split}
\end{equation}
and the energy conservation equation (if the node is decorated by $\bullet$ or $\circ$)
\begin{equation}
 \begin{split}
 &\Lambda_{k_{\mathfrak{e}_1}}-\Lambda_{k_{\mathfrak{e}_2}}+\Lambda_{k_{\mathfrak{e}_3}}-\Lambda_{k_{\mathfrak{e}}} =0.
 \\
 \text{or } &\text{$k_{\mathfrak{e}_1}=k_{\mathfrak{e}_2},\ k_{\mathfrak{e}_3}=k_{\mathfrak{e}}$ (depending on decoration of $\mathfrak{n}$)} 
 \end{split}
\end{equation}



\begin{figure}[H]
 \centering
 \scalebox{0.5}{
 \begin{tikzpicture}[level distance=80pt, sibling distance=80pt]
 \node[scale=2.0] at (-0.7,0) {$k_{\mathfrak{e}}$};
 \node[scale=2.0] at (-0.5,-1.3) {$\mathfrak{n}$};
 \node[scale=2.0] at (-2.2,-2.7) {$k_{\mathfrak{e}_1}$};
 \node[scale=2.0] at (-0.5,-2.7) {$k_{\mathfrak{e}_2}$};
 \node[scale=2.0] at (2.2,-2.7) {$k_{\mathfrak{e}_3}$};
 \node[] at (0,1.5) (0) {} 
 child {node[fillcirc] (1) {} 
 child {node[fillstar] (11) {}
 }
 child {node[fillstar] (12) {}}
 child {node[fillstar] (13) {}}
 };
 \draw[-{Stealth[length=5mm, width=3mm]}] (0) -- (1);
 \draw[-{Stealth[length=5mm, width=3mm]}] (1) -- (11);
 \draw[-{Stealth[length=5mm, width=3mm]}] (12) -- (1);
 \draw[-{Stealth[length=5mm, width=3mm]}] (1) -- (13);
 \node[scale=2.0] at (0,-5.5) {$\iota_{\mathfrak{e}_1}=\iota_{\mathfrak{e}_3}=1, \iota_{\mathfrak{e}}=\iota_{\mathfrak{e}_2}=-1$};
 \node[scale=2.0] at (0,-6.5) {$k_{\mathfrak{e}_1}-k_{\mathfrak{e}_2}+k_{\mathfrak{e}_3}-k_{\mathfrak{e}}=0$, $\cdots$};
 \node[scale=2.0] at (0,-7.5) {$\Lambda_{k_{\mathfrak{e}_1}}-\Lambda_{k_{\mathfrak{e}_2}}+\Lambda_{k_{\mathfrak{e}_3}}-\Lambda_{k_{\mathfrak{e}}}=O(\omega)$};


 \node[scale=2.0] at (14.3,0) {$k_{\mathfrak{e}}$};
 \node[scale=2.0] at (14.5,-1.3) {$\mathfrak{n}$};
 \node[scale=2.0] at (12.8,-2.7) {$k_{\mathfrak{e}_1}$};
 \node[scale=2.0] at (14.5,-2.7) {$k_{\mathfrak{e}_2}$};
 \node[scale=2.0] at (17.2,-2.7) {$k_{\mathfrak{e}_3}$};
 \node[] at (15,1.5) (0) {} 
 child {node[circ] (1) {} 
 child {node[fillstar] (11) {}
 }
 child {node[fillstar] (12) {}}
 child {node[fillstar] (13) {}}
 };
 \draw[-{Stealth[length=5mm, width=3mm]}] (0) -- (1);
 \draw[-{Stealth[length=5mm, width=3mm]}] (1) -- (11);
 \draw[-{Stealth[length=5mm, width=3mm]}] (12) -- (1);
 \draw[-{Stealth[length=5mm, width=3mm]}] (1) -- (13);
 \node[scale=2.0] at (15,-5.5) {$\iota_{\mathfrak{e}_1}=\iota_{\mathfrak{e}_3}=1, \iota_{\mathfrak{e}}=\iota_{\mathfrak{e}_2}=-1$};
 \node[scale=2.0] at (15,-6.5) {$k_{\mathfrak{e}_1}-k_{\mathfrak{e}_2}+k_{\mathfrak{e}_3}-k_{\mathfrak{e}}=0$, $\cdots$};
 \node[scale=2.0] at (15,-7.5) {$k_{\mathfrak{e}_1}=k_{\mathfrak{e}_2},\ k_{\mathfrak{e}_3}=k_{\mathfrak{e}}$ };
 \end{tikzpicture}
 }
 \caption{Equations of a node $\mathfrak{n}$}
 \label{fig.equations}
 \end{figure}
 


 \subsection{Couples and the renormalized Wick theorem} \label{sec.coupwick} In this section, we introduce the renormalized Wick theorem and calculate $\mathbb{E}|\mathcal{J}_{T,k}|^2$ using this theorem. We also introduce another type of diagram, the couple diagrams, to represent the result.


 By the upper bound in the last section, the coefficients $H^T_{k_1\cdots k_{2l+1}}$ concentrate near the surface $\Omega_{\mathfrak{n}}=0$, $\forall \mathfrak{n}$. But to get an upper bound of $\mathcal{J}_{T,k}$, we need the upper bound of their variance $\mathbb{E}|\mathcal{J}_{T,k}|^2$. The coefficients of $\mathbb{E}|\mathcal{J}_{T,k}|^2$ also concentrate near a surface whose expression is similar to \eqref{eq.diophantineeq'}. 
 
 Let's derive the expression of the coefficients of $\mathbb{E}|\mathcal{J}_{T,k}|^2$ and its concentration surface.
 
 %$\mathbb{E}(\mathcal{J}_{T,k}\overline{\mathcal{J}_{T_{\text{in}},k'}})$
 
 %In this section, we derive the formula \eqref{eq.couples} for calculating expectations $\mathbb{E}(\mathcal{J}_{T,k}\overline{\mathcal{J}_{T_{\text{in}},k'}})$ in terms of couples. 
 
 By Lemma \ref{lem.treeterms}, we know that $\mathcal{J}_{T,k}$ is a polynomial of $\xi$ which are proportional to i.i.d Gaussians. Therefore, 
 \begin{equation}\label{eq.termexp1}
 \begin{split}
  \mathbb{E}|\mathcal{J}_{T,k}|^2=&\mathbb{E}(\mathcal{J}_{T,k}\overline{\mathcal{J}_{T,k}})=\left(\frac{\lambda^2}{L^{2d}}\right)^{2l(T)}
  \sum_{k_1,\, k_2,\, \cdots,\, k_{2l(T)+1}}\sum_{k'_1,\, k'_2,\, \cdots,\, k'_{2l(T)+1}}
  \\[0.5em]
  & H^T_{k_1\cdots k_{2l(T)+1}} \overline{H^{T}_{k'_1\cdots k'_{2l(T)+1}}} \mathbb{E}\Big([\xi_{k_1}\bar{\xi}_{k_2}\cdots\xi_{k_{2l(T)+1}}]_{R(T)}
  [\bar{\xi}_{k'_1}\xi_{k'_2}\cdots\bar{\xi}_{k'_{2l(T)+1}}]_{R(T)}\Big)
 \end{split}
 \end{equation}
 % \begin{equation}\label{eq.termexp1}
 % \begin{split}
 % \mathbb{E}(\mathcal{J}_{T,k}\overline{\mathcal{J}_{T',k'}})&=\left(\frac{-i\lambda^2}{L^{2d}}\right)^{l(T)+l(T')}
 % \sum_{k_1,\, k_2,\, \cdots,\, k_{2l(T)+1}}\sum_{k'_1,\, k'_2,\, \cdots,\, k'_{2l(T')+1}}
 % \\[0.5em]
 % & H^T_{k_1\cdots k_{2l(T)+1}} H^{T'}_{k'_1\cdots k'_{2l(T')+1}} \mathbb{E}\Big([\xi_{k_1}\bar{\xi}_{k_2}\cdots\xi_{k_{2l(T)+1}}]_{R(T)}
 % [\xi_{k'_1}\bar{\xi}_{k'_2}\cdots\xi_{k'_{2l(T')+1}}]_{R(T')}\Big)
 % \end{split}
 % \end{equation}
 
 % Temporarily ignore the renormalization, then we just need to calculate 
 % \begin{equation}\label{eq.expectation''}
 % \mathbb{E}\Big(\xi_{k_1}\bar{\xi}_{k_2}\cdots\xi_{k_{2l(T)+1}}
 % \bar{\xi}_{k'_1}\xi_{k'_2}\cdots\bar{\xi}_{k'_{2l(T)+1}}\Big).
 % \end{equation}
 % Notice that $\xi_k=\sqrt{n_{\textrm{in}}(k)} \, \eta_{k}(\omega)$ and $\eta_{k}$ are i.i.d Gaussians. We can apply the Wick theorem to calculate above expectations. 
 
 We just need to calculate 
 \begin{equation}\label{eq.expectationr}
  \mathbb{E}\Big([\xi_{k_1}\bar{\xi}_{k_2}\cdots\xi_{k_{2l(T)+1}}]_{R(T)}
  [\bar{\xi}_{k'_1}\xi_{k'_2}\cdots\bar{\xi}_{k'_{2l(T)+1}}]_{R(T)}\Big).
 \end{equation}
 Notice that $\xi_k=\sqrt{n_{\textrm{in}}(k)} \, \eta_{k}(\omega)$ and $\eta_{k}$ are i.i.d Gaussians. We can apply the Wick theorem to calculate above expectations. 
 
 To introduce the renormalized Wick theorem, we need the following definition.
 
 \begin{defn}\label{def.pairing}
 \begin{enumerate}
  \item \textbf{Sign map.} Given a set $A$, a map $\tau:A\rightarrow \{-1,1\}$ is said to be a \underline{sign map} of $A$. $\tau(a)\in \{-1,1\}$ is said to be the \underline{sign} of $a$. 
  %In what follows, we put a bar $\bar{a}$ on an element $a\in A$ to indicate that it has a negative sign, and elements without a bar have a positive sign. Using this notation, $A$ can be written as $A=\{a_1,\cdots,a_{m}, \bar{b}_1, \cdots,\bar{b}_n\}$.
  \item \textbf{Balanced set.} A set $A$ with sign map $\tau$ is \underline{balanced} if $A$ has equal amount of elements of positive ($\tau(a)=1$) and negative sign ($\tau(a)=-1$).
  \item \textbf{Pairing.} Suppose that we have a balanced set $A=\{a_1,\cdots,a_{m}, b_1, \cdots,b_m\}$. Let $a_1$, $\cdots$, $a_{m}$ (resp. $b_1$, $\cdots$, $b_{m}$) be elements that have a positive sign (resp. negative sign). A \underline{balanced pairing} or simply \underline{balanced pairing} is a partition of $A=\{a_{i_1},b_{j_1}\}\cup\cdots\cup \{a_{i_{m}},b_{j_m}\}$ into $m$ subsets which have exactly two elements of different sign. Given a pairing $p$, elements $a_{i_{k}}$, $b_{j_{k}}$ in the same subset of $p$ are called \underline{paired} with each other, which is denoted by $a_{i_{k}}\sim_{p} b_{j_{k}}$.
  \item \textbf{The pairing sets $\mathcal{P}(A)$ and $\mathcal{P}(B,A)$.} Denote by $\mathcal{P}(A)$ the set of all pairings of $A$. If $B$ is a subset of $A$, let $\mathcal{P}(B,A)=\{p\in \mathcal{P}(A):\forall a\in B,\ b\sim_{p}a \Rightarrow b\in B\}$, i.e. set of pairings that pair elements in $B$ with elements in $B$. 
  \item \textbf{The renormalized pairing sets $\mathcal{P}_F(A)$.} Given a forest $F$ and a set $A$ that contains the leaf set $L(F)$ as its subset, we define the \underline{renormalized pairing set} $\mathcal{P}_F(A)$ by $\mathcal{P}_F(A)=\{p\in\mathcal{P}:\forall \mathfrak{n}\in F,p\notin \mathcal{P}(L(T_{\mathfrak{n}}),A)\}$. $\mathcal{P}_F(A)$ is the set of pairings that do not pair all leaves of any subtree with leaves of the same subtree.
  \item \textbf{Concatenation of two pairings.} Given two pairing $p_1\in\mathcal{P}(A_1)$ and $p_2\in\mathcal{P}(A_2)$, (see Definition \ref{def.pairing} (4) for the definition of $\mathcal{P}(A,B)$) we can define their \underline{concatenation} $p_1\vee p_2\in \mathcal{P}(A_1)\cup \mathcal{P}(A_2)$ as the following. 
  
  Assume that $A_1=\{a^{(1)}_1,\cdots,a^{(1)}_{m}, b^{(1)}_1, \cdots,b^{(1)}_m\}$, $A_2=\{a^{(2)}_1,\cdots,a^{(2)}_{m}, b^{(2)}_1, \cdots,b^{(2)}_m\}$, $p_1$ is given by the partition $\{a^{(1)}_{i_1},b^{(1)}_{j_1}\}\cup\cdots\cup \{a^{(1)}_{i_{m}},b^{(1)}_{j_m}\}$ and $p_2$ is given by the partition $\{a^{(2)}_{i_1},b^{(2)}_{j_1}\}\cup\cdots\cup \{a^{(2)}_{i_{m}},b^{(2)}_{j_m}\}$, then $p_1\vee p_2$ is given by the partition $\{a^{(1)}_{i_1},b^{(1)}_{j_1}\}\cup\cdots\cup \{a^{(1)}_{i_{m}},b^{(1)}_{j_m}\}\cup \{a^{(2)}_{i_1},b^{(2)}_{j_1}\}\cup\cdots\cup \{a^{(2)}_{i_{m}},b^{(2)}_{j_m}\}$.
 
  Similarly given pairings $p_1$, $\cdots$, $p_n$, we can define the concatenation $p_1\vee p_2\vee\cdots\vee p_n$
  %Suppose that we have a set $A=\{a_1,\cdots,a_{m}, \bar{b}_1, \cdots,\bar{b}_m\}$ which has an equal amount of elements of positive and negative sign. A \underline{pairing} is a partition of $A=\{a_{i_1},\bar{b}_{j_1}\}\cup\cdots\cup \{a_{i_{m}},\bar{b}_{j_m}\}$ into $m$ subsets which have exactly two elements of different sign. Given a pairing $p$, elements $a_{i_{k}}$, $\bar{b}_{j_{k}}$ in the same subset of $p$ are called \underline{paired} with each other, which is denoted by $a_{i_{k}}\sim_{p} \bar{b}_{j_{k}}$. Denote by $\mathcal{P}(A)$ the set of all pairings of $A$ and $\mathcal{P}$ the set of all pairings of $\{k_1,k_2,\cdots,k_{m}, ,\bar{k}'_1, \bar{k}'_2, \cdots, ,\bar{k}'_{m}\}$. %$\{k_1,\bar{k}_2,\cdots,\bar{k}_{2l}, k_{2l+1},\bar{k}'_1, k'_2, \cdots,k_{2l },\bar{k}_{2l+1}\}$. 
 \end{enumerate}
 
 % Pairing: $\wick{\c i_1 \c i_{2} \c i_3 \c i_{4} \cdots \c i_{2m-1} \c i_{2m}}$ (representation of a pairing is unique if $i_1<i_3<\cdots<i_{2m-1}$).
 
 % components of a pairing: $p_{k}=i_k$
 
 
 \end{defn}
 
 The original Wick theorem is important in the proof of the renormalized Wick theorem.
 
 
 \begin{thm}[Wick theorem]\label{th.wick}
 Let $\{\eta_k\}_{k\in\mathbb{Z}^d}$ be i.i.d complex Gaussian random variable. Assume that $\tau$ is a sign map of a balanced set $A=\{k_1,k_2,\cdots,k_{2m}\}$. Define $\tau_{j}=\tau(k_j)$. Let $\eta_{k_j}^{\tau_j}=\eta_{k_j}$ if $\tau_j=1$ and $\eta_{k_j}^{\tau_j}=\bar{\eta}_{k_j}$ if $\tau_j=-1$. %Assign $+1$ to $k_1,k_2,\cdots,k_{m}$ and $-1$ to $k'_1, k'_2, \cdots, ,k'_{m}$ and 
 Let $\mathcal{P}$ be the set of all pairings of $\{k_1,k_2,\cdots,k_{2m}\}$, then
 \begin{equation}
  %\mathbb{E}(\xi_{k_1}\cdots \xi_{k_{2m}})=\sum_{p\in \mathcal{P}} \prod_{i=1}^{m} \delta_{k_{p_{2i-1}}=k_{p_{2i}}}
  \mathbb{E}(\eta^{\tau_1}_{k_1}\cdots \eta^{\tau_{2m}}_{k_{2m}})=\sum_{p\in \mathcal{P}} \delta_{p}(k_1,\cdots,k_{2m}), 
 \end{equation}
 where 
 \begin{equation}\label{eq.deltapairing}
 \delta_{p}=\begin{cases}
 1\qquad \textit{if $k_{i}=k_{j}$ for all $k_{i}\sim_{p}k_{j}$,}
 \\
 0\qquad \textit{otherwise.}
 \end{cases}
 \end{equation}
 \end{thm}
 \begin{proof}
 This Wick theorem is a direct corollary of the complex Isserlis' theorem proved in Lemma A.2 of \cite{deng2021full}.
 \end{proof}
 
 \begin{thm}[Renormalized Wick theorem]\label{th.wickr} In addition to assumptions in Theorem \ref{th.wick}, define $\mathcal{P}_F=\mathcal{P}_F(A)$ (remember that $A=\{k_1,k_2,\cdots,k_{2m}\}$ and $\mathcal{P}_F(A)$ is defined in Definition \ref{def.pairing} (5)), then 
 \begin{equation}
  \mathbb{E}([\eta^{\tau_1}_{k_1}\cdots \eta^{\tau_{2m}}_{k_{2m}}]_F)=\sum_{p\in \mathcal{P}_F} \delta_{p}(k_1,\cdots,k_{2m})
 \end{equation}
 \end{thm}
 % \begin{rem}\label{rem.renormwick} The \textit{renormalized pairing set} $p\in \mathcal{P}_F$ is the set of pairings that do not pair all leaves of any subtree with leaves of the same subtree.
 % \end{rem}
 
 \begin{proof} Before starting the formal proof, we give proof for an example to demonstrate the idea.
  
 \textbf{Proof for a special case.} Consider the following forest $F$ in Figure \ref{fig.forestexample}.
 
 \begin{figure}[H]
  \centering
  \scalebox{0.36}{
  \begin{tikzpicture}[level distance=80pt, sibling distance=60pt]
  \node[circ] (1) at (0, 0) {}
  child {node[circ, xshift = -1cm] (11){}
  child {node[circ, xshift = -1cm] (111) {}
  child {node[fillstar, xshift = -1.5cm] (1111) {}}
  child {node[fillstar, xshift = 1.5cm] (1112) {}}
  }
  child {node[fillstar] (112) {}}
  child {node[fillstar, xshift = 1cm] (113) {}}
  }
  child {node[fillstar] (12) {}}
  child {node[fillstar, xshift = 1cm] (13) {}};
  \node[scale=1.5] at ($(1111)+(0.5,0)$) {$1$};
  \node[scale=1.5] at ($(1112)+(0.5,0)$) {$2$};
  \node[scale=1.5] at ($(112)+(0.5,0)$) {$3$}; 
  \node[scale=1.5] at ($(113)+(0.5,0)$) {$4$}; 
  \node[scale=1.5] at ($(12)+(0.5,0)$) {$9$}; 
  \node[scale=1.5] at ($(13)+(0.5,0)$) {$10$};
  \end{tikzpicture}
  }
  \caption{An example of forests}
  \label{fig.forestexample}
 \end{figure}
 
 Let us calculate $\mathbb{E}([\xi_{k_1}\bar{\xi}_{k_2}\xi_{k_3}\bar{\xi}_{k_4}\bar{\xi}_{k_5}\xi_{k_6})]_{F})=\mathbb{E}((:(:\xi_{k_1}\bar{\xi}_{k_2}:)\xi_{k_3}\bar{\xi}_{k_4}:)\bar{\xi}_{k_5}\xi_{k_6})$.
 
 Expanding the renormalization symbol gives
 \begin{equation}
  \begin{split}
  &\mathbb{E}((:(:\xi_{k_1}\bar{\xi}_{k_2}:)\xi_{k_3}\bar{\xi}_{k_4}:)\bar{\xi}_{k_5}\xi_{k_6})
  \\
  =&\mathbb{E}(\xi_{k_1}\bar{\xi}_{k_2}\xi_{k_3}\bar{\xi}_{k_4}\bar{\xi}_{k_5}\xi_{k_6})-\mathbb{E}(\xi_{k_1}\bar{\xi}_{k_2})\mathbb{E}( \xi_{k_3}\bar{\xi}_{k_4}\bar{\xi}_{k_5}\xi_{k_6})-
  \\
  &\mathbb{E}(\xi_{k_1}\bar{\xi}_{k_2}\xi_{k_3}\bar{\xi}_{k_4})\mathbb{E}( \bar{\xi}_{k_5}\xi_{k_6}) + \mathbb{E}(\xi_{k_1}\bar{\xi}_{k_2})\mathbb{E}(\xi_{k_3}\bar{\xi}_{k_4})\mathbb{E}( \bar{\xi}_{k_5}\xi_{k_6})
  \\
  =&\sum_{\text{all pairings}} \delta_p - \sum_{A=\{p \text{ pairs } (12)\text{ and }(3456)\}} \delta_p-\sum_{B = \{p \text{ pairs } (1234)\text{ and }(56)\}} \delta_p+\sum_{A\cap B} \delta_p
  \\
  = & \sum_{A^c\cap B^c} \delta_p = \sum_{p\in \mathcal{P}_F} \delta_p
  \end{split}
 \end{equation}
 
 Here in the second equality, we have applied the Theorem \ref{th.wick} and the inclusion-exclusion principle. The last equality follows from the definition of $\mathcal{P}_F$.
 
 \textbf{Proof for the general case.} The proof for the general case also uses an analogous inclusion-exclusion argument. Since renormalization is defined by recursion in Definition \ref{def.renorm} (3), we will prove the theorem by induction. 
 
 \textbf{Step 1.} (The induction assumption) For the ease of notation let $\eta_{k_j}=\eta^{\tau_j}_{k_j}$. Let $A=\{k_1,k_2,\cdots,$ $k_{2m}\}$. Then we need to show that $\mathbb{E}([\eta_{k_1}\cdots \eta_{k_{2m}}]_F)=\sum_{p\in \mathcal{P}_F}\delta_p$.
 
 We prove the theorem by induction on $Depth(F)$.
 
 If $Depth(F)=0$, i.e. $F=\emptyset$, then $\mathbb{E}([\eta_{k_1}\cdots \eta_{k_{2m}}]_F)=\mathbb{E}(\eta_{k_1}\cdots \eta_{k_{2m}})$ and the theorem follows from the standard Wick theorem (Theorem \ref{th.wick}).
 
 Assuming that the theorem is true for $Depth(F)<l$, we show that it is true for $Depth(F)=l$ in the next steps.
 
 \textbf{Step 2.} (Expanding the product) Let $\{\mathfrak{r}_1,\cdots, \mathfrak{r}_r\}$ be roots of $F$ and $F_{\mathfrak{r}_1},\cdots,F_{\mathfrak{r}_r}$ be their subforests, then by Definition \ref{def.renorm} (3)
 \begin{equation}
  \mathbb{E}([\eta_{k_1}\cdots \eta_{k_{2m}}]_F)=\mathbb{E}\left(\prod_{k_i\notin L(F)} \eta_{k_i} \prod_{j=1}^r \Bigg(\Bigg[\prod_{k_{i_j}\in L(T_{\mathfrak{r}_j})}\eta_{k_{i_j}}\Bigg]_{F_{\mathfrak{r}_j}}-\mathbb{E}\Bigg[\prod_{k_{i_j}\in L(T_{\mathfrak{r}_j})}\eta_{k_{i_j}}\Bigg]_{F_{\mathfrak{r}_j}}\Bigg)\right)
 \end{equation}
 
 
 Expanding the product $\prod_{j=1}^r$ gives
 
 \begingroup
  \allowdisplaybreaks
  \begin{align}\label{eq.lemrenorm1}
  &\mathbb{E}([\eta_{k_1}\cdots \eta_{k_{2m}}]_F)\notag
  \\
  =&\mathbb{E}\left(\sum_{S\subseteq\{1,\cdots,r\}}(-1)^{|S|}\prod_{k_i\notin L(F)} \eta_{k_i} \prod_{j'\notin S} \Bigg[\prod_{k_{i_{j'}}\in L(T_{\mathfrak{r}_{j'}})}\eta_{k_{i_{j'}}}\Bigg]_{F_{\mathfrak{r}_{j'}}}
  \prod_{j\in S}\mathbb{E}\Bigg[\prod_{k_{i_{j}}\in L(T_{\mathfrak{r}_j})}\eta_{k_{i_j}}\Bigg]_{F_{\mathfrak{r}_j}}\right)
  \\
  =&\sum_{S\subseteq\{1,\cdots,r\}}(-1)^{|S|}\underbrace{\mathbb{E}\left(\prod_{k_{i}\notin L(F)} \eta_{k_i} \prod_{j'\notin S} \Bigg[\prod_{k_{i_{j'}}\in L(T_{\mathfrak{r}_{j'}})}\eta_{k_{i_{j'}}}\Bigg]_{F_{\mathfrak{r}_{j'}}}\right)}_{E}
  \prod_{j\in S}\underbrace{\mathbb{E}\Bigg[\prod_{k_{i_{j}}\in L(T_{\mathfrak{r}_j})}\eta_{k_{i_j}}\Bigg]_{F_{\mathfrak{r}_j}}}_{E_{F_{\mathfrak{r}_j}}}\notag
  \end{align}
 \endgroup
 Here in the second line, we have used the identity 
 \begin{equation}
  \prod_{j=1}^r (a_j-b_j)=\sum_{S\subseteq\{1,\cdots,r\}}(-1)^{|S|} \prod_{j\in S}a_j\prod_{j'\notin S}b_j
 \end{equation}
 
 
 % \begin{flalign*}
 % \hspace{1.1cm}
 % =\mathbb{E}\left(\sum_{S\subseteq\{1,\cdots,r\}}(-1)^{|S|}\prod_{k_i\notin L(F)} \eta_{k_i} \prod_{j'\notin S} \Bigg[\prod_{k_{i_{j'}}\in L(F_{\mathfrak{r}_{j'}})}\eta_{k_{i_{j'}}}\Bigg]_{F_{\mathfrak{r}_{j'}}}
 % \prod_{j\in S}\mathbb{E}\Bigg[\prod_{k_{i_{j}}\in L(F_{\mathfrak{r}_j})}\eta_{k_{i_j}}\Bigg]_{F_{\mathfrak{r}_j}}\right)&&
 % \end{flalign*}
 
 % \begin{flalign*}
 % \hspace{1.1cm}
 % =\sum_{S\subseteq\{1,\cdots,r\}}(-1)^{|S|}\mathbb{E}\left(\prod_{k_{i}\notin L(F)} \eta_{k_i} \prod_{j'\notin S} \Bigg[\prod_{k_{i_{j'}}\in L(F_{\mathfrak{r}_{j'}})}\eta_{k_{i_{j'}}}\Bigg]_{F_{\mathfrak{r}_{j'}}}\right)
 % \prod_{j\in S}\mathbb{E}\Bigg[\prod_{k_{i_{j}}\in L(F_{\mathfrak{r}_j})}\eta_{k_{i_j}}\Bigg]_{F_{\mathfrak{r}_j}}&&
 % \end{flalign*}
 
 %For any subset $A$ of $\{1,\cdots,r\}$, define $\mathcal{P}(A)$ to be set of pairing of elements of $A$ and 
 \textbf{Step 3.} (Applying the induction assumption) Notice that $E$ in the last line of \eqref{eq.lemrenorm1} is the expectation of the polynomial $\prod_{k_i\notin L(F)\text{ or }k_i\in \cup_{j'\notin S}L(T_{\mathfrak{r}_j})}\eta_{k_i}$ renormalized by the forest $\cup_{j'\notin S}F_{\mathfrak{r}_j}$. $E_{F_{\mathfrak{r}_j}}$ is the expectation of the polynomial $\prod_{k_{i_{j}}\in L(T_{\mathfrak{r}_j})}\eta_{k_{i_j}}$ renormalized by the forest $F_{\mathfrak{r}_j}$. 
 
 
 In other words, 
 \begin{equation}
  E = \mathbb{E}\left(\left[\prod_{k_i\notin L(F)\text{ or }k_i\in \cup_{j'\notin S}L(T_{\mathfrak{r}_j})}\eta_{k_i}\right]_{\cup_{j'\notin S}F_{\mathfrak{r}_j}}\right), 
  \qquad 
  E_{F_{\mathfrak{r}_j}} = \mathbb{E}\left(\Bigg[\prod_{k_{i_{j}}\in L(T_{\mathfrak{r}_j})}\eta_{k_{i_j}}\Bigg]_{F_{\mathfrak{r}_j}}\right)
 \end{equation}
 
 Since $Depth(F_{\mathfrak{r}_j})<l$ and $Depth(\cup_{j'\notin S}F_{\mathfrak{r}_j})<l$ for any $j$, we can thus apply the induction assumption to calculate $E$ and $E_{F_{\mathfrak{r}_j}}$.
 \begin{equation}\label{eq.lemrenorm2}
  \begin{split}
  E &= \sum_{p\in \mathcal{P}_{\cup_{j'\notin S} F_{\mathfrak{r}_{j'}}}(\{k_i\notin L(F)\text{ or }k_i\in \cup_{j'\notin S}L(T_{\mathfrak{r}_j})\})}\delta_p
  \\
  &= \sum_{p\in \mathcal{P}_{\cup_{j'\notin S} F_{\mathfrak{r}_{j'}}}((\cup_{j\in S}L(T_{\mathfrak{r}_j}))^c)}\delta_p,
  \end{split}
 \end{equation}
 and
 \begin{equation}\label{eq.lemrenorm3}
  E_{F_{\mathfrak{r}_j}} = \prod_{j\in S}\sum_{q\in \mathcal{P}_{F_{\mathfrak{r}_j}}(T_{\mathfrak{r}_j})} \delta_{q_j},
 \end{equation}
 where in \eqref{eq.lemrenorm2}, we have used the fact that $\{k_i\notin L(F)\text{ or }k_i\in \cup_{j'\notin S}L(T_{\mathfrak{r}_j})\} = (\cup_{j\in S}L(T_{\mathfrak{r}_j}))^c$, and in \eqref{eq.lemrenorm2} and \eqref{eq.lemrenorm3}, we haved used the notation $\mathcal{P}_{F}(B)$ in Definition \ref{def.pairing} (5). In \eqref{eq.lemrenorm2}, $\mathcal{P}_{\cup_{j'\notin S} F_{\mathfrak{r}_{j'}}}((\cup_{j\in S}L(T_{\mathfrak{r}_j}))^c)$ is of the form $\mathcal{P}_{F}(B)$, in which $F = \cup_{j'\notin S} F_{\mathfrak{r}_{j'}}$ and $B = (\cup_{j\in S}L(T_{\mathfrak{r}_j}))^c$.
 
 Substituting \eqref{eq.lemrenorm2} and \eqref{eq.lemrenorm3} into \eqref{eq.lemrenorm1}, we get 
 \begin{equation}
  \mathbb{E}([\eta_{k_1}\cdots \eta_{k_{2m}}]_F) =\sum_{S\subseteq\{1,\cdots,r\}}(-1)^{|S|}
  \left(\sum_{p\in \mathcal{P}_{\cup_{j'\notin S} F_{\mathfrak{r}_{j'}}}((\cup_{j\in S}L(T_{\mathfrak{r}_j}))^c)}\delta_p\right)
  \left(\prod_{j\in S}\sum_{q\in \mathcal{P}_{F_{\mathfrak{r}_j}}(T_{\mathfrak{r}_j})} \delta_{q_j}\right)
 \end{equation}
 
 
 \textbf{Step 4.} (Concatenating short pairings) Expanding the product, we get
 \begin{equation}\label{eq.lemrenorm4}
  \begin{split}
  \mathbb{E}([\eta_{k_1}\cdots \eta_{k_{2m}}]_F) &= \sum_{S\subseteq\{1,\cdots,r\}}(-1)^{|S|}
  \left(\sum_{p\in \mathcal{P}_{\cup_{j'\notin S} F_{\mathfrak{r}_{j'}}}((\cup_{j\in S}L(T_{\mathfrak{r}_j}))^c)}\delta_p\right)
  \left(\prod_{j\in S}\sum_{q\in \mathcal{P}_{F_{\mathfrak{r}_j}}(T_{\mathfrak{r}_j})} \delta_{q_j}\right)
  \\
  & =\sum_{S\subseteq\{1,\cdots,r\}}(-1)^{|S|}\sum_{\substack{p\in \mathcal{P}_{\cup_{j'\notin S} F_{\mathfrak{r}_{j'}}}((\cup_{j\in S}L(T_{\mathfrak{r}_j}))^c)
  \\q_j\in 
  \mathcal{P}_{F_{\mathfrak{r}_j}}(T_{\mathfrak{r}_j})\ \forall j}} \left(\delta_p
  \prod_{j\in S} \delta_{q_j}\right)
  \end{split}
 \end{equation}
 
 % \begin{flalign*}
 % \hspace{1.1cm}
 % =\sum_{S\subseteq\{1,\cdots,r\}}(-1)^{|S|}
 % \left(\sum_{p\in \mathcal{P}_{\cup_{j'\notin S} F_{\mathfrak{r}_{j'}}}((\cup_{j\in S}L(T_{\mathfrak{r}_j}))^c)}\delta_p\right)
 % \left(\prod_{j\in S}\sum_{q\in \mathcal{P}_{F_{\mathfrak{r}_j}}(T_{\mathfrak{r}_j})} \delta_{q_j}\right)&&
 % \end{flalign*} 
  
 % \begin{flalign*}
 % \hspace{1.1cm}
 % =\sum_{S\subseteq\{1,\cdots,r\}}(-1)^{|S|}\sum_{\substack{p\in \mathcal{P}((\cup_{j\in S}L(T_{\mathfrak{r}_j}))^c)\cap \mathcal{P}_{\cup_{j'\notin S} F_{\mathfrak{r}_{j'}}}
 % \\q_j\in 
 % \mathcal{P}(F_{\mathfrak{r}_j})\cap
 % \mathcal{P}_{F_{\mathfrak{r}_j}}\ \forall j}} \left(\delta_p
 % \prod_{j\in S} \delta_{q_j}\right)&&
 % \end{flalign*} 
  
 For any tree or forest $T$, define $\mathcal{P}[T]=\mathcal{P}(L(T),A)$ (see Definition \ref{def.pairing} (4) for the definition of $\mathcal{P}(A,B)$), where $A=\{k_1,k_2,\cdots,k_{2m}\}$. For trees or forests $\{T_j\}_{j\in S}$, define $\mathcal{P}[T_j,j\in S]=\cap_{j\in S}\mathcal{P}[L(F_j)]$. 
 
 \textit{Claim.} Given $p\in \mathcal{P}_{\cup_{j'\notin S} F_{\mathfrak{r}_{j'}}}((\cup_{j\in S}L(T_{\mathfrak{r}_j}))^c)$ and $q_j\in 
 \mathcal{P}_{F_{\mathfrak{r}_j}}(T_{\mathfrak{r}_j})\ \forall j$, then their concatenation satisfies $p\vee (\vee_{j\in S} q_j)\in \mathcal{P}[T_j,j\in S]\cap \mathcal{P}_{\cup_{j=1}^r F_{\mathfrak{r}_{j}}}$. Conversely, all pairings in $p\vee (\vee_{j\in S} q_j)\in \mathcal{P}[T_j,j\in S]\cap \mathcal{P}_{\cup_{j=1}^r F_{\mathfrak{r}_{j}}}$ can be generated in this way.
 
 \begin{proof}[Proof of the claim] $p\in \mathcal{P}_{\cup_{j'\notin S} F_{\mathfrak{r}_{j'}}}((\cup_{j\in S}L(T_{\mathfrak{r}_j}))^c)$ means that $p$ does not pair all leaves of any subtree of $\cup_{j'\notin S} F_{\mathfrak{r}_{j'}}$ with leaves of the same subtree. $q_j\in 
 \mathcal{P}_{F_{\mathfrak{r}_j}}(T_{\mathfrak{r}_j})\ \forall j$ means that $q_j$ does not pair all leaves of any subtree of $F_{\mathfrak{r}_j}$ with leaves of the same subtree. Therefore, $p\vee (\vee_{j\in S} q_j)$ does not pair all leaves of any subtree of $\cup_{j=1}^r F_{\mathfrak{r}_{j}}$ with leaves of the same subtree, which implies that $\mathcal{P}_{\cup_{j=1}^r F_{\mathfrak{r}_{j}}}$.
 
 It it not difficult to show that $p\vee (\vee_{j\in S} q_j)\in \mathcal{P}[T_j,j\in S]$, so we have $p\vee (\vee_{j\in S} q_j)\in \mathcal{P}[T_j,j\in S]\cap \mathcal{P}_{\cup_{j=1}^r F_{\mathfrak{r}_{j}}}$.
 
 Given a pairing $p'\in \mathcal{P}[T_j,j\in S]\cap \mathcal{P}_{\cup_{j=1}^r F_{\mathfrak{r}_{j}}}$, we can restrict $p'$ to subsets and define $p = p'|_{(\cup_{j\in S}L(T_{\mathfrak{r}_j}))^c}$ and $q_j = p|_{\cup_{j\in S}L(T_{\mathfrak{r}_j})}$. It is not hard to show that $p'=p\vee (\vee_{j\in S} q_j)$, $q_j\in 
 \mathcal{P}_{F_{\mathfrak{r}_j}}(T_{\mathfrak{r}_j})\ \forall j$ and $p\in \mathcal{P}_{\cup_{j'\notin S} F_{\mathfrak{r}_{j'}}}((\cup_{j\in S}L(T_{\mathfrak{r}_j}))^c)$.
 
 Therefore, we have finished the proof of the claim.
 \end{proof}
 
 From above claim and \eqref{eq.lemrenorm4}, we know that 
 \begin{equation}
  \mathbb{E}([\eta_{k_1}\cdots \eta_{k_{2m}}]_F)=\sum_{S\subseteq\{1,\cdots,r\}}(-1)^{|S|}\sum_{p\vee (\vee_{j\in S} q_j)\in \mathcal{P}[T_j,j\in S]\cap \mathcal{P}_{\cup_{j=1}^r F_{\mathfrak{r}_{j}}}} 
  \delta_{p\vee (\vee_{j\in S} q_j)}
 \end{equation}
  
 % \begin{flalign*}
 % \hspace{1.1cm}
 % =\sum_{S\subseteq\{1,\cdots,r\}}(-1)^{|S|}\sum_{p\vee (\vee_{j\in S} q_j)\in \mathcal{P}[T_j,j\in S]\cap \mathcal{P}_{\cup_{j=1}^r F_{\mathfrak{r}_{j}}}} 
 % \delta_{p\vee (\vee_{j\in S} q_j)}&&
 % \end{flalign*} 
 
 \textbf{Step 5.} (The inclusion-exclusion principle) Now we apply the following inclusion-exclusion principle.
 \begin{equation}\label{eq.inclu-exclu}
  \mathbbm{1}_{A_1^c\cap\cdots\cap A_m^c} = \sum_{S\subseteq\{1,\cdots,r\}}(-1)^{|S|} \mathbbm{1}_{\cap_{j\in S} A_j}
 \end{equation}
 
 
 Since $\mathcal{P}[T_j,j\in S]=\cap_{j\in S}\mathcal{P}[T_j]$, we get
 \begin{equation}
  \begin{split}
  &\mathbb{E}([\eta_{k_1}\cdots \eta_{k_{2m}}]_F)=\sum_{S\subseteq\{1,\cdots,r\}}(-1)^{|S|}\sum_{p} \mathbbm{1}_{\cap_{j\in S}\mathcal{P}[T_j]\cap \mathcal{P}_{\cup_{j=1}^r F_{\mathfrak{r}_{j}}}} 
  \delta_p
  \\
  =&\sum_{p} \left(\sum_{S\subseteq\{1,\cdots,r\}}(-1)^{|S|}\mathbbm{1}_{\cap_{j\in S}\mathcal{P}[T_j]\cap \mathcal{P}_{\cup_{j=1}^r F_{\mathfrak{r}_{j}}}} \right)\delta_p 
  \\
  =& \sum_{p} \left(\sum_{S\subseteq\{1,\cdots,r\}}(-1)^{|S|}\mathbbm{1}_{\cap_{j\in S}\mathcal{P}[T_j]} \right)\mathbbm{1}_{ \mathcal{P}_{\cup_{j=1}^r F_{\mathfrak{r}_{j}}}}\delta_p
  \\
  =&\sum_{p} \mathbbm{1}_{\cap_{j\in S}(\mathcal{P}[T_j])^c\cap \mathcal{P}_{\cup_{j=1}^r F_{\mathfrak{r}_{j}}}} 
  \delta_p=\sum_{p} \mathbbm{1}_{\mathcal{P}_F} 
  \delta_p
  \\
  =&\sum_{p\in \mathcal{P}_F} \delta_p
  \end{split}
 \end{equation}
 Here in the fourth line we have used \eqref{eq.inclu-exclu} and the following identity
 \begin{equation}\label{eq.lemrenorm5}
  \mathcal{P}_F = \cap_{j\in S}(\mathcal{P}[T_j])^c\cap \mathcal{P}_{\cup_{j=1}^r F_{\mathfrak{r}_{j}}}.
 \end{equation}
 
 Now we prove the above identity. By definition (Definition \ref{def.pairing} (5)), 
 \begin{equation}
  \begin{split}
  \mathcal{P}_{\cup_{j=1}^r F_{\mathfrak{r}_{j}}}=&\{p\in\mathcal{P}:\forall \mathfrak{n}\in \cup_{j=1}^r F_{\mathfrak{r}_{j}},p\notin \mathcal{P}(L(T_{\mathfrak{n}}),A)\}
  \\
  =&\{p\in\mathcal{P}:\forall \mathfrak{n}\in F\backslash \{\mathfrak{r}_1,\cdots,\mathfrak{r}_r\},p\notin \mathcal{P}(L(T_{\mathfrak{n}}),A)\},
  \end{split}
 \end{equation}
 and 
 \begin{equation}
  \mathcal{P}_F=\{p\in\mathcal{P}:\forall \mathfrak{n}\in F,p\notin \mathcal{P}(L(T_{\mathfrak{n}}),A)\}.
 \end{equation}
 
 By definition of $\mathcal{P}[T_j]$ (it is in the paragraph above the claim), 
 \begin{equation}
  (\mathcal{P}[T_j])^c = (\mathcal{P}(L(T_{\mathfrak{r}_j}),A))^c = \{p\in\mathcal{P}:\mathfrak{r}_j,p\notin \mathcal{P}(L(T_{\mathfrak{r}_j}),A)\}.
 \end{equation}
 
 Therefore, we have
 \begin{equation}
  \begin{split}
  &\cap_{j\in S}(\mathcal{P}[T_j])^c\cap \mathcal{P}_{\cup_{j=1}^r F_{\mathfrak{r}_{j}}}
  \\
  =&\cap_{j\in S} (\{p\in\mathcal{P}:\mathfrak{r}_j,p\notin \mathcal{P}(L(T_{\mathfrak{r}_j}),A)\}) \cap \{p\in\mathcal{P}:\forall \mathfrak{n}\in F\backslash \{\mathfrak{r}_1,\cdots,\mathfrak{r}_r\},p\notin \mathcal{P}(L(T_{\mathfrak{n}}),A)\}
  \\
  = & \{p\in\mathcal{P}:\forall \mathfrak{n}\in F,p\notin \mathcal{P}(L(T_{\mathfrak{n}}),A)\} = \mathcal{P}_F.
  \end{split}
 \end{equation}
 
 % \begin{flalign*}
 % \hspace{1.1cm}
 % =\sum_{S\subseteq\{1,\cdots,r\}}(-1)^{|S|}\sum_{p} \mathbbm{1}_{\cap_{j\in S}\mathcal{P}[T_j]\cap \mathcal{P}_{\cup_{j=1}^r F_{\mathfrak{r}_{j}}}} 
 % \delta_p&&
 % \end{flalign*} 
 
 % \begin{flalign*}
 % \hspace{1.1cm}
 % =\sum_{p} \left(\sum_{S\subseteq\{1,\cdots,r\}}(-1)^{|S|}\mathbbm{1}_{\cap_{j\in S}\mathcal{P}[T_j]\cap \mathcal{P}_{\cup_{j=1}^r F_{\mathfrak{r}_{j}}}} \right)
 % \delta_p&&
 % \end{flalign*}
 
 % Since $\mathcal{P}_F=\cap_{j\in S}(\mathcal{P}[T_j])^c\cap \mathcal{P}_{\cup_{j\in S} F_{n_{j}}}$, inclusion-exclusion principle implies that
 
 % \begin{flalign*}
 % \hspace{1.1cm}
 % =&\sum_{p} \mathbbm{1}_{\cap_{j\in S}(\mathcal{P}[T_j])^c\cap \mathcal{P}_{\cup_{j=1}^r F_{\mathfrak{r}_{j}}}} 
 % \delta_p=\sum_{p} \mathbbm{1}_{\mathcal{P}_F} 
 % \delta_p&&
 % \\
 % =&\sum_{p\in \mathcal{P}_F} \delta_p
 % \end{flalign*}
 
 
 Now we finished the proof of \eqref{eq.lemrenorm5} and thus the proof of the renormalized Wick theorem.
 \end{proof}
 
 Denote by $R(T)\cup R(T)$ the union of two copies of $R(T)$, then we have 
 \begin{equation}
  \begin{split}
  &\mathbb{E}\Big([\xi_{k_1}\bar{\xi}_{k_2}\cdots\xi_{k_{2l(T)+1}}]_{R(T)}
  [\bar{\xi}_{k'_1}\xi_{k'_2}\cdots\bar{\xi}_{k'_{2l(T)+1}}]_{R(T)}\Big) 
  \\
  =& \mathbb{E}\Big([\xi_{k_1}\bar{\xi}_{k_2}\cdots\xi_{k_{2l(T)+1}}\bar{\xi}_{k'_1}\xi_{k'_2}\cdots\bar{\xi}_{k'_{2l(T)+1}}]_{R(T)\cup R(T)}\Big)
  \end{split}
 \end{equation}
 
 Applying Wick theorem to \eqref{eq.termexp1}, we get
 \begin{equation}\label{eq.termexp'}
 \begin{split}
  &\mathbb{E}|\mathcal{J}_{T,k}|^2=\left(\frac{\lambda^2}{L^{2d}}\right)^{2l(T)}
  \sum_{p\in \mathcal{P}_F(\{k_1,\cdots, k_{2l(T)+1}, k'_1,\cdots, k'_{2l(T)+1}\})}
  \\[0.5em]
  & \underbrace{\sum_{\substack{k_1,\, k_2,\, \cdots,\, k_{2l(T)+1}\\k'_1,\, k'_2,\, \cdots,\, k'_{2l(T)+1}}}
  % \sum_{k'_1,\, k'_2,\, \cdots,\, k'_{2l(T)+1}} 
  H^T_{k_1\cdots k_{2l(T)+1}} H^{T}_{k'_1\cdots k'_{2l(T)+1}} \delta_{p}(k_1,\cdots, k_{2l(T)+1}, k'_1,\cdots, k'_{2l(T)+1})\sqrt{n_{\textrm{in}}(k_1)}\cdots}_{Term(T, p)}.
 \end{split}
 \end{equation}
 % \begin{equation}\label{eq.termexp1}
 % \begin{split}
 % &\mathbb{E}(\mathcal{J}_{T,k}\overline{\mathcal{J}_{T',k'}})=\left(\frac{-i\lambda^2}{L^{2d}}\right)^{l(T)+l(T')}
 % \sum_{p\in \mathcal{P}(\{k_1,\cdots, k_{2l(T)+1}, k'_1,\cdots, k'_{2l(T')+1}\})}
 % \\[0.5em]
 % & \underbrace{\sum_{k_1,\, k_2,\, \cdots,\, k_{2l(T)+1}}\sum_{k'_1,\, k'_2,\, \cdots,\, k'_{2l(T')+1}} H^T_{k_1\cdots k_{2l(T)+1}} H^{T'}_{k'_1\cdots k'_{2l(T')+1}} \delta_{p}(k_1,\cdots, k_{2l(T)+1}, k'_1,\cdots, k'_{2l(T')+1})\sqrt{n_{\textrm{in}}(k_1)}\cdots}_{Term(T, T', p)}.
 % \end{split}
 % \end{equation}
 
 We see that the correlation of two tree terms is a sum of smaller expressions $Term(T, p)$. By \eqref{eq.diophantineeq'}, the coefficients $H^T_{k_1\cdots k_{2l(T)+1}} H^{T}_{k'_1\cdots k'_{2l(T)+1}}$ of $Term(T, p)$ concentrate near the subset 
 \begin{equation}\label{eq.diophantineequnpaired}
  \{k_{\mathfrak{e}}, k'_{\mathfrak{e}}\in \mathbb{Z}^d,\ |k_{\mathfrak{e}}|, |k'_{\mathfrak{e}}|\lesssim 1,\ \forall \mathfrak{e}: |\Omega_{\mathfrak{n}}|,|\Omega'_{\mathfrak{n}}|\lesssim \omega,\ \forall \mathfrak{n}. \ \{k_{\mathfrak{e}}\}_{\mathfrak{e}}, \{k'_{\mathfrak{e}}\}_{\mathfrak{e}}\in \cap_{\mathfrak{n}\in T_{\text{in}}} S_{\mathfrak{n}}.\ k_{\mathfrak{l}}=k_{\mathfrak{l}}'=k\}.
 \end{equation}
 
 The pairing $p$ in Wick theorem introduces new equations $k_{i}=k'_{j}$ (defined in \eqref{eq.deltapairing}) and the coefficients $H^T_{k_1\cdots k_{2l(T)+1}} H^{T}_{k'_1\cdots k'_{2l(T)+1}} \delta_{p}$ concentrate near the subset 
 \begin{equation}\label{eq.diophantineeqpaired}
 \begin{split}
  \{k_{\mathfrak{e}}, k'_{\mathfrak{e}}\in \mathbb{Z}^d,\ &|k_{\mathfrak{e}}|, |k'_{\mathfrak{e}}|\lesssim 1,\ \forall \mathfrak{e}: |\Omega_{\mathfrak{n}}|,|\Omega'_{\mathfrak{n}}|\lesssim \omega,\ \forall \mathfrak{n}. \ \{k_{\mathfrak{e}}\}_{\mathfrak{e}}, \{k'_{\mathfrak{e}}\}_{\mathfrak{e}}\in \cap_{\mathfrak{n}\in T_{\text{in}}} S_{\mathfrak{n}}.\ k_{\mathfrak{l}}=k_{\mathfrak{l}}'=k. \\
  &\textit{$k_{i}=k'_{j}$ (and $k_{i}=k_{j}$, $k'_{i}=k'_{j}$) for all $k_{i}\sim_{p}k'_{j}$ (and $k_{i}\sim_{p}k_{j}$, $k'_{i}\sim_{p}k'_{j}$)}\}.
 \end{split}
 \end{equation}
 
 As in the case of \eqref{eq.diophantineeq'}, there is a graphical representation of \eqref{eq.diophantineeqpaired}. To explain this, we need the concept of couples.
 
 %It turns out we can find a very compact formula for them. To do this, we need to introduce the concept of couples.
 
 
 \begin{defn}[Construction of couples]\label{def.couple}Given two trees $T$ and $T'$, we flip the orientation of all edges in $T'$ (as in the two left trees in Figure \ref{fig.treepairing}). We also label their leaves by $1, 2, \cdots, 2l(T)+1$ and $1, 2, \cdots, 2l(T')+1$ so that the corresponding variables of these leaves are $k_1, k_2, \cdots, k_{2l(T)+1}$ and $k_1, k_2, \cdots, k_{2l(T')+1}$. Assume that we have a pairing $p$ of the set $\{k_1, k_2, \cdots, k_{2l(T)+1}, k_1, k_2, \cdots,$ $ k_{2l(T')+1}\}$, then this pairing induces a pairing between leaves (if $k_i\sim_p k_j$ then define $\textit{the $i$-th leaf}\sim_p \textit{the $j$-th leaf}$). Given this pairing of leaves, we define the following procedure which glues two trees $T$ and $T'$ into a couple $\mathcal{C}(T,T,p)$. Some examples of pairing can be found in Figure \ref{fig.treepairing}. 
 
  \begin{figure}[H]
  \centering
  \scalebox{0.3}{
  \begin{tikzpicture}[level distance=80pt, sibling distance=70pt]
  \node[] at (0,0) (1) {} 
  child {node[fillcirc] (11) {} 
  child {node[fillcirc] (111) {}
  child {node[fillstar] (1111) {}}
  child {node[fillstar] (1112) {}}
  child {node[fillstar] (1113) {}}
  }
  child {node[fillstar] (112) {}}
  child {node[fillstar] (113) {}}
  };
  \draw[-{Stealth[length=5mm, width=3mm]}] (1) -- (11);
  \draw[-{Stealth[length=5mm, width=3mm]}] (11) -- (111);
  \draw[{Stealth[length=5mm, width=3mm]}-] (11) -- (112);
  \draw[-{Stealth[length=5mm, width=3mm]}] (11) -- (113);
  \draw[-{Stealth[length=5mm, width=3mm]}] (111) -- (1111);
  \draw[{Stealth[length=5mm, width=3mm]}-] (111) -- (1112);
  \draw[-{Stealth[length=5mm, width=3mm]}] (111) -- (1113);
  \node[] at ($(1)+(8,0)$) (2) {} 
  child {node[fillcirc] (21) {} 
  child {node[fillcirc] (211) {}
  child {node[fillstar] (2111) {}}
  child {node[fillstar] (2112) {}}
  child {node[fillstar] (2113) {}}
  }
  child {node[fillstar] (212) {}}
  child {node[fillstar] (213) {}}
  };
  \draw[{Stealth[length=5mm, width=3mm]}-] (2) -- (21);
  \draw[{Stealth[length=5mm, width=3mm]}-] (21) -- (211);
  \draw[-{Stealth[length=5mm, width=3mm]}] (21) -- (212);
  \draw[{Stealth[length=5mm, width=3mm]}-] (21) -- (213);
  \draw[{Stealth[length=5mm, width=3mm]}-] (211) -- (2111);
  \draw[-{Stealth[length=5mm, width=3mm]}] (211) -- (2112);
  \draw[{Stealth[length=5mm, width=3mm]}-] (211) -- (2113);
  \draw[bend right =30, dashed] (1111) edge (2111);
  \draw[bend right =30, dashed] (1112) edge (2112);
  \draw[bend right =30, dashed] (1113) edge (2113);
  \draw[bend right =30, dashed] (112) edge (212);
  \draw[bend right =30, dashed] (113) edge (213);
 
  
  \node[] at ($(1)+(20,0)$) (3) {} 
  child {node[circ] (31) {} 
  child {node[fillcirc] (311) {}
  child {node[fillstar] (3111) {}}
  child {node[fillstar] (3112) {}}
  child {node[fillstar] (3113) {}}
  }
  child {node[fillstar] (312) {}}
  child {node[fillstar] (313) {}}
  };
  \draw[-{Stealth[length=5mm, width=3mm]}] (3) -- (31);
  \draw[-{Stealth[length=5mm, width=3mm]}] (31) -- (311);
  \draw[{Stealth[length=5mm, width=3mm]}-] (31) -- (312);
  \draw[-{Stealth[length=5mm, width=3mm]}] (31) -- (313);
  \draw[-{Stealth[length=5mm, width=3mm]}] (311) -- (3111);
  \draw[{Stealth[length=5mm, width=3mm]}-] (311) -- (3112);
  \draw[-{Stealth[length=5mm, width=3mm]}] (311) -- (3113);
  \node[] at ($(1)+(28,0)$) (4) {} 
  child {node[fillcirc] (41) {} 
  child {node[fillcirc] (411) {}
  child {node[fillstar] (4111) {}}
  child {node[fillstar] (4112) {}}
  child {node[fillstar] (4113) {}}
  }
  child {node[fillstar] (412) {}}
  child {node[fillstar] (413) {}}
  };
  \draw[{Stealth[length=5mm, width=3mm]}-] (4) -- (41);
  \draw[{Stealth[length=5mm, width=3mm]}-] (41) -- (411);
  \draw[-{Stealth[length=5mm, width=3mm]}] (41) -- (412);
  \draw[{Stealth[length=5mm, width=3mm]}-] (41) -- (413);
  \draw[{Stealth[length=5mm, width=3mm]}-] (411) -- (4111);
  \draw[-{Stealth[length=5mm, width=3mm]}] (411) -- (4112);
  \draw[{Stealth[length=5mm, width=3mm]}-] (411) -- (4113);
  \draw[bend right =30, dashed] (3111) edge (312);
  \draw[bend right =30, dashed] (3112) edge (4112);
  \draw[bend right =30, dashed] (3113) edge (4113);
  \draw[bend right =30, dashed] (313) edge (413);
  \draw[bend right =30, dashed] (4111) edge (412);
  \end{tikzpicture}
  }
  \caption{Example of pairings between trees.}
  \label{fig.treepairing}
  \end{figure}
 
 \begin{enumerate}
  \item \textbf{Merging edges connected to leaves.} Given two edges with opposite orientations connected to two paired leaves, these two edges can be \underline{merged} into one edge as in Figure \ref{fig.pairingleaves}. Since the pairing in this paper is a balanced pairing, only edges with opposite orientations can be paired.
  \begin{figure}[H]
  \centering
  \scalebox{0.5}{
  \begin{tikzpicture}[level distance=80pt, sibling distance=100pt]
  \node[draw, circle, minimum size=1cm, scale=2] at (0,0) (1) {$T_1$} [grow =300] 
  child {node[fillstar] (2) {}};
  \node[draw, circle, minimum size=1cm, scale=2] at (5,0) (3) {$T_2$} [grow =240] 
  child {node[fillstar] (4) {}};
  \draw[-{Stealth[length=5mm, width=3mm]}] (1) -- (2);
  \draw[{Stealth[length=5mm, width=3mm]}-] (3) -- (4);
  \draw[bend right =40, dashed] (2) edge (4);
  
  \node[draw, single arrow,
  minimum height=33mm, minimum width=8mm,
  single arrow head extend=2mm,
  anchor=west, rotate=0] at (7,-1.5) {}; 
  
  \node[draw, circle, minimum size=1cm, scale=2] at (13,-1.5) (5) {$T_1$}; 
  \node[draw, circle, minimum size=1cm, scale=2] at (18,-1.5) (6) {$T_2$};
  \draw[-{Stealth[length=5mm, width=3mm]}] (5) -- (6);
  
  \end{tikzpicture}
  }
  \caption{Pairing and merging of two edges}
  \label{fig.pairingleaves}
  \end{figure}
  We know that two edges connected to leaves correspond to two indices $k_i$, $k_j$. Merging two such edges is a graphical interpretation that $k_i=k_j=k$. 
  \item \textbf{Splitting of $\circ$ nodes: } Given a $\circ$ node $\mathfrak{n}$, we split it into two edges as in Figure \ref{fig.splitcirc}.
  \begin{figure}[H]
  \centering
  \scalebox{0.3}{
  \begin{tikzpicture}[level distance=80pt, sibling distance=70pt]
  \node[] at (0,7) (1) {}
  child{node[circ] (11) {} 
  child {node[fillstar] (111) {}}
  child {node[fillstar] (112) {}}
  child {node[fillstar] (113) {}}
  };
  \draw[-{Stealth[length=5mm, width=3mm]}] (1) -- (11);
  \draw[-{Stealth[length=5mm, width=3mm]}] (11) -- (111);
  \draw[{Stealth[length=5mm, width=3mm]}-] (11) -- (112);
  \draw[-{Stealth[length=5mm, width=3mm]}] (11) -- (113);
 
  \node[draw, single arrow,
  minimum height=44mm, minimum width=10mm,
  single arrow head extend=2mm,
  anchor=west, rotate=0] at (5,4) {};
 
  \node[fillstar] at (12,4) (2) {};
  \node[fillstar] at (15,4) (3) {};
  \node[] at (18,5.5) (4) {};
  \node[fillstar] at (18,2.5) (5) {};
  \draw[{Stealth[length=5mm, width=3mm]}-, dashed] (2) -- (3);
  \draw[-{Stealth[length=5mm, width=3mm]}] (4) -- (5);
 
  \node[scale=1.5] at ($(111)+(0.5,0)$) {$1$};
  \node[scale=1.5] at ($(112)+(0.5,0)$) {$2$};
  \node[scale=1.5] at ($(113)+(0.5,0)$) {$3$};
  \node[scale=1.5] at ($(1)+(0.5,-0.2)$) {$\mathfrak{n}$};
 
  \node[scale=1.5] at ($(2)+(0,-0.5)$) {$1$};
  \node[scale=1.5] at ($(3)+(0,-0.5)$) {$2$};
  \node[scale=1.5] at ($(5)+(0.5,0)$) {$3$};
  \node[scale=1.5] at ($(4)+(0.5,-0.2)$) {$\mathfrak{n}$};
  \end{tikzpicture}
  }
  \caption{Splitting of a $\circ$ node}
  \label{fig.splitcirc}
  \end{figure}
  One of these two edges is \underline{usual} and the other one is \underline{dotted} (called \underline{dotted edge}). If the parent of $\circ$ is $\mathfrak{n}$ and the children of $\circ$ are $\mathfrak{n}_1$, $\mathfrak{n}_2$, $\mathfrak{n}_3$ from left to right, then we require the usual edge connects $\mathfrak{n}$ and $\mathfrak{n}_3$ and the dotted edge connects
  $\mathfrak{n}_1$ and $\mathfrak{n}_2$. Figure \ref{fig.splitcircexample} is an example of splitting $\circ$ in a tree
  \begin{figure}[H]
  \centering
  \scalebox{0.3}{
  \begin{tikzpicture}[level distance=80pt, sibling distance=70pt]
  \node[] at (0,0) (1) {} 
  child {node[fillcirc] (11) {} 
  child {node[circ, xshift = -1cm] (111) {}
  child {node[fillcirc, xshift = -1cm] (1111) {}
  }
  child {node[fillcirc] (1112) {}
  child {node[fillstar, xshift = 1cm] (11121) {}}
  child {node[fillstar] (11122) {}}
  child {node[fillstar, xshift = -1cm] (11123) {}}
  }
  child {node[fillstar, xshift = 1cm] (1113) {}
  child {node[fillstar, xshift = 1cm] (11131) {}}
  child {node[fillstar] (11132) {}}
  child {node[fillstar, xshift = -1cm] (11133) {}}
  }
  }
  child {node[fillstar] (112) {}}
  child {node[fillstar, xshift = 1cm] (113) {}}
  };
  \draw[-{Stealth[length=5mm, width=3mm]}] (1) -- (11);
  \draw[-{Stealth[length=5mm, width=3mm]}] (11) -- (111);
  \draw[{Stealth[length=5mm, width=3mm]}-] (11) -- (112);
  \draw[-{Stealth[length=5mm, width=3mm]}] (11) -- (113);
  \draw[-{Stealth[length=5mm, width=3mm]}] (111) -- (1111);
  \draw[{Stealth[length=5mm, width=3mm]}-] (111) -- (1112);
  \draw[-{Stealth[length=5mm, width=3mm]}] (111) -- (1113);
  \draw[{Stealth[length=5mm, width=3mm]}-] (1112) -- (11121);
  \draw[-{Stealth[length=5mm, width=3mm]}] (1112) -- (11122);
  \draw[{Stealth[length=5mm, width=3mm]}-] (1112) -- (11123);
  \draw[-{Stealth[length=5mm, width=3mm]}] (1113) -- (11131);
  \draw[{Stealth[length=5mm, width=3mm]}-] (1113) -- (11132);
  \draw[-{Stealth[length=5mm, width=3mm]}] (1113) -- (11133);
 
  \node[draw, single arrow,
  minimum height=44mm, minimum width=10mm,
  single arrow head extend=2mm,
  anchor=west, rotate=0] at (5,-7) {};
  
  \node[] at (15,-2) (1) {} 
  child {node[fillcirc] (11) {} 
  child {node[fillcirc] (111) {}
  child {node[fillstar] (1111) {}}
  child {node[fillstar] (1112) {}}
  child {node[fillstar] (1113) {}}
  }
  child {node[fillstar] (112) {}}
  child {node[fillstar] (113) {}}
  };
  \draw[-{Stealth[length=5mm, width=3mm]}] (1) -- (11);
  \draw[{Stealth[length=5mm, width=3mm]}-] (11) -- (112);
  \draw[-{Stealth[length=5mm, width=3mm]}] (11) -- (113);
  \draw[-{Stealth[length=5mm, width=3mm]}] (111) -- (1111);
  \draw[{Stealth[length=5mm, width=3mm]}-] (111) -- (1112);
  \draw[-{Stealth[length=5mm, width=3mm]}] (111) -- (1113);
  
  \node[fillstar] at (23,-4) (1) {};
  \node[fillcirc] at (23,-6.5) (11) {} 
  child {node[fillstar] (111) {}}
  child {node[fillstar] (112) {}}
  child {node[fillstar] (113) {}};
  \draw[dashed] (1) edge (11);
  \draw[-{Stealth[length=5mm, width=3mm]}] (11) -- (113);
  \draw[{Stealth[length=5mm, width=3mm]}-] (11) -- (112);
  \draw[-{Stealth[length=5mm, width=3mm]}] (11) -- (113);
  \end{tikzpicture}
  }
  \caption{An example of splitting a $\circ$ node}
  \label{fig.splitcircexample}
  \end{figure}
  
  We know that the $\circ$ node corresponds to equation $S_{\mathfrak{n}}=\{k_{\mathfrak{e}_1}=k_{\mathfrak{e}_2},\ k_{\mathfrak{e}_3}=k_{\mathfrak{e}}\}$. The corresponding variable $k_{\mathfrak{e}_d}$ of the dotted edge $\mathfrak{e}_d$ is defined to be the common value of $k_{\mathfrak{e}_1}$ and $k_{\mathfrak{e}_2}$. The corresponding variable $k_{\mathfrak{e}_n}$ of the usual edge $\mathfrak{e}_n$ is defined to be the common value of $k_{\mathfrak{e}_3}$ and $k_{\mathfrak{e}}$. 
  \item \textbf{Pairing of trees and couples.} Given a pairing $p$ of the set of leaves in $T$, $T'$ we merge all edges paired by $p$ as in Figure \ref{fig.couple} and the resulting combinatorial structure is called a \underline{couple}. 
  \begin{figure}[H]
  \centering
  \scalebox{0.2}{
  \begin{tikzpicture}[level distance=80pt, sibling distance=70pt]
  \node[] at (0,0) (1) {} 
  child {node[fillcirc] (11) {} 
  child {node[fillstar] (111) {}}
  child {node[fillstar] (112) {}}
  child {node[fillstar] (113) {}}
  };
  \draw[-{Stealth[length=5mm, width=3mm]}] (1) -- (11);
  \draw[-{Stealth[length=5mm, width=3mm]}] (11) -- (111);
  \draw[{Stealth[length=5mm, width=3mm]}-] (11) -- (112);
  \draw[-{Stealth[length=5mm, width=3mm]}] (11) -- (113);
  
  \node[] at (0,-14.5) (2) {} [grow =90] 
  child {node[fillcirc] (21) {} 
  child {node[fillstar] (211) {}}
  child {node[fillstar] (212) {}}
  child {node[fillstar] (213) {}}
  }; 
  \draw[-{Stealth[length=5mm, width=3mm]}] (2) -- (21);
  \draw[-{Stealth[length=5mm, width=3mm]}] (21) -- (211);
  \draw[{Stealth[length=5mm, width=3mm]}-] (21) -- (212);
  \draw[-{Stealth[length=5mm, width=3mm]}] (21) -- (213); 
 
  \draw[dashed] (111) edge (213);
  \draw[dashed] (112) edge (212);
  \draw[dashed] (113) edge (211);
  
  
  \node[draw, single arrow,
  minimum height=66mm, minimum width=16mm,
  single arrow head extend=2mm,
  anchor=west, rotate=0] at (7,-7.5) {};
  
  \node[] at (20,0) (21) {}; 
  \node[fillcirc] at (20,-4) (22) {};
  \node[fillcirc] at (20,-11) (23) {};
  \node[] (24) at (20,-15) {};
  \draw[-{Stealth[length=5mm, width=3mm]}] (21) edge (22);
  \draw[-{Stealth[length=5mm, width=3mm]}, bend right =60] (22) edge (23);
  \draw[-{Stealth[length=5mm, width=3mm]}] (22) edge (23);
  \draw[-{Stealth[length=5mm, width=3mm]}, bend left =60] (22) edge (23);
  \draw[-{Stealth[length=5mm, width=3mm]}] (23) edge (24);
  \end{tikzpicture}
  }
  \caption{The construction of a couple}
  \label{fig.couple}
  \end{figure}
  
  We know that each edge connected to a leaf corresponds to a variable $k_i$. A pairing $p$ of $\{k_1,k_2,\cdots,k_{2m}\}$ in \eqref{eq.diophantineeqpaired} induces a pairing of edges connected to leaves. Merging paired edges corresponds to $k_{i}=k'_{j}$ for all $k_{i}\sim_{p}k'_{j}$ in \eqref{eq.diophantineeqpaired}. 
 \end{enumerate}
 \end{defn}
 
 
 \begin{prop}\label{prop.couple}
 \eqref{eq.diophantineeqpaired} can be read from a couple diagram $\mathcal{C}(T,T,p)$. Each edge corresponds to a variable $k_{\mathfrak{e}}$. The leg $\mathfrak{l}$ corresponds to equation $k_{\mathfrak{l}}=k$. Each node corresponds to a momentum conservation equation
 \begin{equation}
  k_{\mathfrak{e}_1}-k_{\mathfrak{e}_2}+k_{\mathfrak{e}_3}-k_{\mathfrak{e}}=0,\ k_{\mathfrak{e}_1}\ne k_{\mathfrak{e}_2}\ne k_{\mathfrak{e}_3},\ \text{or}\ k_{\mathfrak{e}_1}= k_{\mathfrak{e}_2}= k_{\mathfrak{e}_3}=k,
 \end{equation} 
 and an energy conservation equation 
 \begin{equation}
  \Lambda_{k_{\mathfrak{e}_1}}-\Lambda_{k_{\mathfrak{e}_2}}+\Lambda_{k_{\mathfrak{e}_3}}-\Lambda_{k_{\mathfrak{e}}} = O(\omega).
  % \begin{cases}
  % \Lambda_{k_{\mathfrak{e}_1}}-\Lambda_{k_{\mathfrak{e}_2}}+\Lambda_{k_{\mathfrak{e}_3}}-\Lambda_{k_{\mathfrak{e}}} = O(\alpha), \textit{ if the node is decorated by }\bullet.
  % \\
  % k_{\mathfrak{e}_1}=k_{\mathfrak{e}_2},\ k_{\mathfrak{e}_3}=k_{\mathfrak{e}}, \qquad\qquad\qquad\textit{ if the node is decorated by }\circ.
  % \end{cases}
 \end{equation} 
 \end{prop}
 \begin{rem}
 In a couple diagrams, we only have nodes decorated by $\bullet$. Nodes decorated by $\circ$ and $\star$ have been removed in (2), (3) of Definition \ref{def.couple}.
 \end{rem}
 \begin{rem}
 Through the process of (2), (3) in Definition \ref{def.couple}, a couple diagram can automatically encode the equation $k_{i}=k'_{j}$ for all $k_{i}\sim_{p}k'_{j}$ and $k_{\mathfrak{e}_1}=k_{\mathfrak{e}_2},\ k_{\mathfrak{e}_3}=k_{\mathfrak{e}}$. Therefore, they do not appear in Proposition \ref{prop.couple}.
 \end{rem}
 
 \begin{proof}
 This directly follows from the definition of couples. 
 \end{proof}
 
 The calculations of this section are summarized in the following proposition. 
 
 \begin{prop}\label{prop.termcouple} 
 (1) Define $Term(T,p)$ in the same way as in \eqref{eq.termexp'},
 \begin{equation}\label{eq.termTp}
 \begin{split}
  Term(T, p)=&\sum_{k_1,\, k_2,\, \cdots,\, k_{2l(T)+1}}\sum_{k'_1,\, k'_2,\, \cdots,\, k'_{2l(T)+1}}
  \\
  &H^T_{k_1\cdots k_{2l(T)+1}} H^{T}_{k'_1\cdots k'_{2l(T)+1}} \delta_{p}(k_1,\cdots, k_{2l(T)+1}, k'_1,\cdots, k'_{2l(T)+1})\sqrt{n_{\textrm{in}}(k_1)}\cdots\sqrt{n_{\textrm{in}}(k'_1)}\cdots
 \end{split}
 \end{equation}
 then $\mathbb{E}|\mathcal{J}_{T,k}|^2$ is a sum of $Term(T,p)$ for all $p\in \mathcal{P}_F$, (in \eqref{eq.termexp'} the sum is over set of all possible pairing $\mathcal{P}$)
 \begin{equation}\label{eq.termexp}
 \begin{split}
  \mathbb{E}|\mathcal{J}_{T,k}|^2=\left(\frac{\lambda^2}{L^{2d}}\right)^{2l(T)}
  \sum_{p\in \mathcal{P}_F(\{k_1,\cdots, k_{2l(T)+1}, k'_1,\cdots, k'_{2l(T)+1}\})} Term(T, p).
 \end{split}
 \end{equation}
 
 (2) Since the definition \eqref{eq.termTp} of $Term(T,p)$ do not change, it still concentrates near the subset \eqref{eq.diophantineeqpaired} which has a simple graphical representation given by Proposition \ref{prop.couple}. 
 \end{prop}
 
 \begin{proof} The proof of (1), (2) is easy and thus skipped.
 \end{proof}
 % \begin{lem}
 % Formula for $\mathbb{E}(\mathcal{J}_{T,k}\overline{\mathcal{J}_{T',k'}})$ in terms of couple $C$
 
 % \begin{equation}\label{eq.couples}
 % 1
 % \end{equation}
 % \end{lem}
 
 
\subsection{Counting lattice points}\label{sec.numbertheory} In this section, we apply the connection between couple and concentration subset \eqref{eq.diophantineeqpaired} to count the number of solutions of a generalized version of \eqref{eq.diophantineeqpaired},
\begin{equation}\label{eq.diophantineeqpairedsigma}
\begin{split}
    \{k_{\mathfrak{e}}, k'_{\mathfrak{e}}\in \mathbb{Z}^d,\ &|k_{\mathfrak{e}}|, |k'_{\mathfrak{e}}|\lesssim 1,\ \forall \mathfrak{e}: |\Omega_{\mathfrak{n}}-\sigma_{\mathfrak{n}}|,|\Omega'_{\mathfrak{n}}-\sigma'_{\mathfrak{n}}|\lesssim \omega,\ \forall \mathfrak{n}. \ \{k_{\mathfrak{e}}\}_{\mathfrak{e}}, \{k'_{\mathfrak{e}}\}_{\mathfrak{e}}\in \cap_{\mathfrak{n}\in T'} S_{\mathfrak{n}}.\ k_{\mathfrak{l}}=k_{\mathfrak{l}}'=k. \\
    &\textit{$k_{i}=k'_{j}$ (and $k_{i}=k_{j}$, $k'_{i}=k'_{j}$) for all $k_{i}\sim_{p}k'_{j}$ (and $k_{i}\sim_{p}k_{j}$, $k'_{i}\sim_{p}k'_{j}$)}\}.
\end{split}
\end{equation}

\eqref{eq.diophantineeqpairedsigma} is obtained by replacing $\Omega_{\mathfrak{n}}$, $\Omega'_{\mathfrak{n}}$ by $\Omega_{\mathfrak{n}}-\sigma_{\mathfrak{n}}$, $\Omega'_{\mathfrak{n}}-\sigma'_{\mathfrak{n}}$ in \eqref{eq.diophantineeqpaired}. $\sigma_{\mathfrak{n}}$ and $\sigma'_{\mathfrak{n}}$ are some given constants. The counterpart of Proposition \ref{prop.couple} in this case is

\begin{prop}\label{prop.couple'}
\eqref{eq.diophantineeqpairedsigma} can be read from a couple diagram $\mathcal{C}=\mathcal{C}(T,T,p)$. Each edge corresponds to a variable $k_{\mathfrak{e}}$. The leg $\mathfrak{l}$ corresponds to equation $k_{\mathfrak{l}}=k$. Each node corresponds to a momentum conservation equation
\begin{equation}\label{eq.momentumconservationunit}
    k_{\mathfrak{e}_1}-k_{\mathfrak{e}_2}+k_{\mathfrak{e}_3}-k_{\mathfrak{e}}=0,\ k_{\mathfrak{e}_1}\ne k_{\mathfrak{e}_2}\ne k_{\mathfrak{e}_3},\ \text{or}\ k_{\mathfrak{e}_1}= k_{\mathfrak{e}_2}= k_{\mathfrak{e}_3}=k,
\end{equation} 
and a energy conservation equation 
\begin{equation}\label{eq.energyconservationunit}
    \Lambda_{k_{\mathfrak{e}_1}}-\Lambda_{k_{\mathfrak{e}_2}}+\Lambda_{k_{\mathfrak{e}_3}}-\Lambda_{k_{\mathfrak{e}}} =\sigma_{\mathfrak{n}}+O(\omega).
\end{equation}  
Denote the momentum and energy conservation equations by $MC_{\mathfrak{n}}$ and $EC_{\mathfrak{n}}$ respectively, then \eqref{eq.diophantineeqpairedsigma} can be rewritten as 
\begin{equation}\label{eq.diophantineeqpairedsigma'}
    \text{\eqref{eq.diophantineeqpairedsigma}}=\{k_{\mathfrak{e}}\in \mathbb{Z}^d,\ |k_{\mathfrak{e}}| \lesssim 1,\ \forall \mathfrak{e}\in \mathcal{C}:MC_{\mathfrak{n}},\  EC_{\mathfrak{n}},\ \forall \mathfrak{n}\in \mathcal{C}.\ k_{\mathfrak{l}}=k_{\mathfrak{l}}'=k.\}
\end{equation}
\end{prop}
\begin{proof}
This directly follows from Proposition \ref{prop.couple}. 
\end{proof}

\subsubsection{Basic ideas for counting lattice points} In this section, we explain the basic idea used in this paper to count lattice points. To do this, we need the following definitions related to couples.

\begin{defn}\label{def.morecouple}
\begin{enumerate}
 % \item \textbf{Admissible couple.} A couple $\mathcal{C}$ is an \underline{admissible couple} if all connected components of $\mathcal{C}$ without legs contain at least two dotted edges.
 \item \textbf{Connected couples.} A couple $\mathcal{C}$ is a \underline{connected couple} if it is connected as a graph. The \underline{connected components} is also defined in the same way as the graph theory. 
 %In what follows, \textit{a connected components of a couple is also called as \underline{couple}}. 
 \item \textbf{Equations of a couple $Eq(\mathcal{C})$:} Given a couple $\mathcal{C}$ and constants $k$, $\sigma_{\mathfrak{n}}$, let $Eq(\mathcal{C},\{\sigma_{\mathfrak{n}}\}_{\mathfrak{n}}, k)$ (or simply $Eq(\mathcal{C})$) be the system of equation \eqref{eq.diophantineeqpairedsigma'} constructed in Proposition \ref{prop.couple'}. For any system of equations $Eq$, let $\#(Eq)$ be its number of solutions.
\end{enumerate}
\end{defn}

The main goal of this section is to prove an upper bound of $\#Eq(\mathcal{C})$. The main idea of proving this is to decompose a large couple $\mathcal{C}$ into smaller pieces and then prove this for the smaller piece using the induction hypothesis. To explain the idea, let us first focus on an example. Let $\mathcal{C}$ be the left couple in the following picture. (The corresponding variables of each edge are labeled near these edges.)
\begin{figure}[H]
 \centering
 \scalebox{0.4}{
 \begin{tikzpicture}[level distance=80pt, sibling distance=100pt]
 \node[] at (0,0) (1) {}; 
 \node[fillcirc] at (3,0) (2) {}; 
 \node[fillcirc] at (6,-2) (3) {}; 
 \node[fillcirc] at (9,-2) (4) {}; 
 \node[fillcirc] at (12,0) (5) {}; 
 \node[] at (15,0) (6) {}; 
 \draw[-{Stealth[length=5mm, width=3mm]}] (1) edge (2);
 \draw[-{Stealth[length=5mm, width=3mm]}] (2) edge (3);
 \draw[-{Stealth[length=5mm, width=3mm]}, bend left =40] (3) edge (4);
 \draw[-{Stealth[length=5mm, width=3mm]}, bend right =40] (3) edge (4);
 \draw[{Stealth[length=5mm, width=3mm]}-] (3) edge (4);
 \draw[-{Stealth[length=5mm, width=3mm]}] (4) edge (5);
 \draw[-{Stealth[length=5mm, width=3mm]}] (5) edge (6);
 \draw[-{Stealth[length=5mm, width=3mm]}, bend left =60] (2) edge (5);
 \draw[{Stealth[length=5mm, width=3mm]}-] (2) edge (5);
 
 \node[scale=2.0] at (3,-0.7) {$\mathfrak{n}_{1}$};
 \node[scale=2.0] at (6,-2.7) {$\mathfrak{n}_{2}$};
 \node[scale=2.0] at (9,-2.7) {$\mathfrak{n}_{3}$};
 \node[scale=2.0] at (12,-0.7) {$\mathfrak{n}_{4}$};
 \node[scale=2.0] at (1.5,-0.5) {$k$};
 \node[scale=2.0] at (4.3,-1.4) {$a$};
 \node[scale=2.0] at (7.5,0.4) {$b$};
 \node[scale=2.0] at (7.5,2.8) {$c$};
 \node[scale=2.0] at (7.5,-1) {$d$};
 \node[scale=2.0] at (7.5,-1.8) {$e$};
 \node[scale=2.0] at (7.5,-3.1) {$f$};
 \node[scale=2.0] at (10.7,-1.4) {$g$};
 \node[scale=2.0] at (13.5,-0.5) {$k$};
 
 
 \node[draw, single arrow,
 minimum height=33mm, minimum width=8mm,
 single arrow head extend=2mm,
 anchor=west, rotate=0] at (16,0) {}; 
 
 
 \node[] at (20,0) (11) {}; 
 \node[fillcirc] at (23,0) (12) {}; 
 \node[fillstar] at (26,-2) (13) {};
 \node[fillstar] at (26,0) (15) {};
 \node[fillstar] at (26,2) (14) {};
 \draw[-{Stealth[length=5mm, width=3mm]}] (11) edge (12);
 \draw[-{Stealth[length=5mm, width=3mm]}] (12) edge (13);
 \draw[-{Stealth[length=5mm, width=3mm]}] (12) edge (14);
 \draw[{Stealth[length=5mm, width=3mm]}-] (12) edge (15);
 
 \node[scale=2.0] at (23,-0.7) {$\mathfrak{n}_{1}$};
 \node[scale=2.0] at (21.5,-0.5) {$k$};
 \node[scale=2.0] at (24.3,-1.4) {$a$};
 \node[scale=2.0] at (24.8,0.4) {$b$};
 \node[scale=2.0] at (24.3,1.4) {$c$};
 \node[scale=2.0] at (23,-1.8) {$A$};
 
 
 \node[] at (28,-2) (32) {}; 
 \node[fillcirc] at (31,-2) (33) {}; 
 \node[fillcirc] at (34,-2) (34) {}; 
 \node[fillcirc] at (37,0) (35) {}; 
 \node[] at (34,0) (38) {}; 
 \node[] at (40,0) (36) {}; 
 \node[] at (34,2) (37) {};
 \draw[-{Stealth[length=5mm, width=3mm]}] (32) edge (33);
 \draw[-{Stealth[length=5mm, width=3mm]}, bend left =40] (33) edge (34);
 \draw[-{Stealth[length=5mm, width=3mm]}, bend right =40] (33) edge (34);
 \draw[{Stealth[length=5mm, width=3mm]}-] (33) edge (34);
 \draw[-{Stealth[length=5mm, width=3mm]}] (34) edge (35);
 \draw[-{Stealth[length=5mm, width=3mm]}] (35) edge (36);
 \draw[-{Stealth[length=5mm, width=3mm]}] (37) edge (35);
 \draw[-{Stealth[length=5mm, width=3mm]}] (35) edge (38);
 
 \node[scale=2.0] at (31,-2.7) {$\mathfrak{n}_{2}$};
 \node[scale=2.0] at (34,-2.7) {$\mathfrak{n}_{3}$};
 \node[scale=2.0] at (37,-0.7) {$\mathfrak{n}_{4}$};
 \node[scale=2.0] at (29.5,-2.5) {$a$};
 \node[scale=2.0] at (32.5,-3.1) {$f$};
 \node[scale=2.0] at (32.5,-1) {$d$};
 \node[scale=2.0] at (32.5,-1.8) {$e$};
 \node[scale=2.0] at (35.7,-1.4) {$g$};
 \node[scale=2.0] at (35.7,1.4) {$c$};
 \node[scale=2.0] at (35.2,0.4) {$b$};
 \node[scale=2.0] at (38.5,-0.5) {$k$};
 \node[scale=2.0] at (37,-2.2) {$B_{a,b,c}$};
 \end{tikzpicture}
 }
 \caption{An example of decomposing a couple}
 \label{fig.exampleofcuttingidea}
 \end{figure}

By \eqref{eq.diophantineeqpairedsigma'}, we know that the couple $\mathcal{C}$ corresponds to the following equations.
\begin{equation}\label{eq.cuttingexmaple}
 \begin{split}
 \{&(a, b, c, d, e, f, g):\ (|a|\text{ to }|g|)\lesssim 1,
 \\
 &a - b + c = k,\ \Lambda(a) - \Lambda(b) + \Lambda(c) - \Lambda(k) =\sigma_{1} + O(\omega)
 \\
 &d - e + f = a,\ \Lambda(d) - \Lambda(e) + \Lambda(f) - \Lambda(a) =\sigma_{2} + O(\omega)
 \\
 &d - e + f = g,\ \Lambda(d) - \Lambda(e) + \Lambda(f) - \Lambda(g) =\sigma_{3} + O(\omega)
 \\
 &c - b + g = k,\ \Lambda(c) - \Lambda(b) + \Lambda(g) - \Lambda(k) =\sigma_{4} + O(\omega)\}
 \end{split}
\end{equation}

We know that \eqref{eq.cuttingexmaple} can be rewritten into the form $\bigcup_{a,e\in A} B_{a,e}$, where
\begin{equation}\label{eq.couplequationA}
 A=\{a, b, c:\ |a|,|b|,|c|\lesssim 1,\ a - b + c = k,\ \Lambda(a) - \Lambda(b) + \Lambda(c) - \Lambda(k) =\sigma_{1} + O(\omega)\}
\end{equation}
\begin{equation}\label{eq.couplequationB}
 \begin{split}
 B_{a,b,c}=\{&d, e, f, g:\ |d|,|e|,|f|,|g|\lesssim 1,
 \\
 &d - e + f = a,\ \Lambda(d) - \Lambda(e) + \Lambda(f) - \Lambda(a) =\sigma_{2} + O(\omega)
 \\
 &d - e + f = g,\ \Lambda(d) - \Lambda(e) + \Lambda(f) - \Lambda(g) =\sigma_{3} + O(\omega)
 \\
 &c - b + g = k,\ \Lambda(c) - \Lambda(b) + \Lambda(g) - \Lambda(k) =\sigma_{4} + O(\omega)\}
 \end{split} 
\end{equation}

Since an upper bound of $\# Eq(\mathcal{C})$ can be derived from upper bounds of $\# A$, $\# B_{a,b,c}$, we just need to consider $A$, $B_{a,b,c}$ which are systems of equations of smaller size. We can reduce the size of systems of equations in this way and prove upper bounds by induction.

One problem of applying an induction argument is that $A$, $B_{a,b,c}$ cannot be represented by couple defined by Definition \ref{def.couple} that can contain at most two legs (an edge just connected to one node). In Definition \ref{def.couple}, a leg is used to represent a variable that is fixed, as in the condition $k_{\mathfrak{l}} = k_{\mathfrak{l}}'= k$ in \eqref{eq.diophantineeqpairedsigma'}. The definition of $\# B_{a,b,c}$ contains four fixed variables $a$, $b$, $c$, $k$ which cannot be represented by just two legs. Therefore, we have to define a new type of couple that allows multiple legs.

Except for the lack of legs, we also have the problem of representing free variables. We know that the couple representation of $A$ should contain one node and four edges if we insist on the rule that a node corresponds to an equation and the variables in the equation correspond to edges connected to this node. All these edges are legs, but three of four edges correspond to variables $a$, $b$, $c$ which are not fixed. Therefore, we have to define a type of legs that can correspond to unfixed variables.

To solve the above problems, we introduce the following definition.

\begin{defn}\label{def.couplemultileg}
\begin{enumerate}
 \item \textbf{Couples with multiple legs:} A graph in which all nodes have degree $1$ or $4$ is called a \underline{couples with multiple legs}. The graph $A$ and $B_{a,b,c}$ in Figure \ref{fig.exampleofcuttingidea} are examples of this definition.
 \item \textbf{Legs:} In a couple with multiple legs, an edge connected to a degree one node is called a \underline{leg}. Remember that we have encountered this concept in the second paragraph of section \ref{sec.refexp} and in what follows, we call the leg defined there the \underline{root leg} of a tree. 
 \item \textbf{Free legs and fixed legs:} In a couple with multiple legs, we use two types of node decoration for degree $1$ nodes as in Figure \ref{fig.decorationdegreeone}. One is $\star$ and the other one is \underline{invisible}. 
 \begin{figure}[H]
 \centering
 \scalebox{0.5}{
 \begin{tikzpicture}[level distance=80pt, sibling distance=100pt]
 \node[draw, circle, minimum size=1cm, scale=2] at (0,0) (1) {$\mathcal{C}_1$} 
 child {node[fillstar] (2) {}};
 \node[draw, circle, minimum size=1cm, scale=2] at (5,0) (3) {$\mathcal{C}_2$} 
 child {node[] (4) {}};
 \draw[{Stealth[length=5mm, width=3mm]}-] (1) -- (2);
 \draw[{Stealth[length=5mm, width=3mm]}-] (3) -- (4);
 %\draw[bend right =40, dashed] (2) edge (4);
 
 \end{tikzpicture}
 }
 \caption{Node decoration of degree one nodes}
 \label{fig.decorationdegreeone}
 \end{figure}
 An edge connected to a $\star$ or invisible nodes is called a \underline{free leg} or \underline{fixed leg} respectively. They correspond to free variables or fixed variables $k_{\mathfrak{e}}$ respectively.
 
 
 \item \textbf{Equations of a couple $Eq(\mathcal{C},\{c_{\mathfrak{l}}\}_{\mathfrak{l}})$:} We define the corresponding equations for couples with multiple legs.
 \begin{equation}\label{eq.Eq(C,c)}
 Eq(\mathcal{C},\{c_{\mathfrak{l}}\}_{\mathfrak{l}})=\{k_{\mathfrak{e}}\in \mathbb{Z}^d_L,\ |k_{\mathfrak{e}}|\lesssim 1\ \forall \mathfrak{e}\in \mathcal{C}:\ MC_{\mathfrak{n}},\  EC_{\mathfrak{n}},\ \forall \mathfrak{n}\in \mathcal{C}.\ k_{\mathfrak{l}}=c_{\mathfrak{l}},\ \forall \mathfrak{l}.\} 
 \end{equation}
 In this representation, the corresponding variable of a fixed leg $\mathfrak{l}$ is fixed to be the constant $c_{\mathfrak{l}}$ and the corresponding variable of a free leg $\mathfrak{l}$ is not fixed.
\end{enumerate}
\end{defn}

With the above definition, it's easy to show that the couple $A$ and $B_{a,b,c}$ in Figure \ref{fig.exampleofcuttingidea} correspond to the system of equations \eqref{eq.couplequationA} and \eqref{eq.couplequationB} respectively.


\subsubsection{The cutting operation and its properties} In this section, we give the formal definition of cutting and explain how $\#Eq(\mathcal{C})$ changes after cutting.

\begin{defn}
 \begin{enumerate}
 \item \textbf{Cutting an edge:} Given an edge $\mathfrak{e}$, we can cut it into two edges (a fixed and a free leg) as in Figure \ref{fig.cutedge}.
 
 \begin{figure}[H]
 \centering
 \scalebox{0.5}{
 \begin{tikzpicture}[level distance=80pt, sibling distance=100pt]
 \node[draw, circle, minimum size=1cm, scale=2] at (0,0) (1) {$\mathcal{C}_1$}; 
 \node[draw, circle, minimum size=1cm, scale=2] at (5,0) (2) {$\mathcal{C}_2$};
 \draw[-{Stealth[length=5mm, width=3mm]}] (1) -- (2);
 \node[scale =3] at (2.5,0) {$\times$};
 
 
 \node[draw, single arrow,
 minimum height=33mm, minimum width=8mm,
 single arrow head extend=2mm,
 anchor=west, rotate=0] at (7.5,0) {}; 
 
 
 \node[draw, circle, minimum size=1cm, scale=2] at (13,0) (11) {$\mathcal{C}_1$}; 
 \node[fillstar] at (16,0) (12) {};
 \draw[-{Stealth[length=5mm, width=3mm]}] (11) -- (12);
 
 \node[] at (17,0) (13) {}; 
 \node[draw, circle, minimum size=1cm, scale=2] at (20,0) (14) {$\mathcal{C}_2$};
 \draw[-{Stealth[length=5mm, width=3mm]}] (13) -- (14); 
 \end{tikzpicture}
 }
 \caption{An example of cutting an edge}
 \label{fig.cutedge}
 \end{figure}
 
 \item \textbf{Cut:} A \underline{cut} $c$ of a couple $\mathcal{C}$ is a set of edges such that $\mathcal{C}$ is disconnected after cutting all edges in $c$, together with a map $\text{rc}:c\rightarrow \{\text{left}, \text{right}\}$. For each $\mathfrak{e}\in c$, if $\text{rc}(\mathfrak{e})=\text{left}$ (resp. right), then as in Figure \ref{fig.cutedge} the left node (resp. right node) produced by cutting $\mathfrak{e}$ is a $\star$ node (resp. invisible node). The map $\text{rc}$ describes which one should be the free or fixed leg in the two legs produced by cutting an edge. An \underline{admissible cut} is defined to be the cut such that the new legs after cutting are all free in one component and are fixed in another component. 
 \item \textbf{$c(\mathfrak{e})$, $c(\mathfrak{n})$ and $c(\mathfrak{l})$:} Given an edge $\mathfrak{e}$ that is not a leg, define $c(\mathfrak{e})$ to be the cut that contains only one edge $\mathfrak{e}$. Given a node $\mathfrak{n}\in \mathcal{C}$, let $\{\mathfrak{e}_{i}\}$ be edges that are connected to $\mathfrak{n}$, then define $c(\mathfrak{n})$ to be the cut that consists of edges $\{\mathfrak{e}_{i}\}$. Given a leg $\mathfrak{l}$, let $\mathfrak{n}$ be the unique node connected to it, then define $c(\mathfrak{l})$ to be the cut $c(\mathfrak{n})$. An example of cutting $c(\mathfrak{e})$ is given by Figure \ref{fig.cutedge}. The following picture gives an example of cutting $c(\mathfrak{n})$ or $c(\mathfrak{l})$ (in this picture $\mathfrak{n}=\mathfrak{n}_1$ and $\mathfrak{l}$ is the leg labelled by $k$.)
 
 \begin{figure}[H]
 \centering
 \scalebox{0.4}{
 \begin{tikzpicture}[level distance=80pt, sibling distance=100pt]
 \node[] at (0,0) (1) {}; 
 \node[fillcirc] at (3,0) (2) {}; 
 \node[fillcirc] at (6,-2) (3) {}; 
 \node[fillcirc] at (9,-2) (4) {}; 
 \node[fillcirc] at (12,0) (5) {}; 
 \node[] at (15,0) (6) {}; 
 \draw[-{Stealth[length=5mm, width=3mm]}] (1) edge (2);
 \draw[-{Stealth[length=5mm, width=3mm]}] (2) edge (3);
 \draw[-{Stealth[length=5mm, width=3mm]}, bend left =40] (3) edge (4);
 \draw[-{Stealth[length=5mm, width=3mm]}, bend right =40] (3) edge (4);
 \draw[{Stealth[length=5mm, width=3mm]}-] (3) edge (4);
 \draw[-{Stealth[length=5mm, width=3mm]}] (4) edge (5);
 \draw[-{Stealth[length=5mm, width=3mm]}] (5) edge (6);
 \draw[-{Stealth[length=5mm, width=3mm]}, bend left =60] (2) edge (5);
 \draw[{Stealth[length=5mm, width=3mm]}-] (2) edge (5);
 
 \node[scale=2.0] at (3,-0.7) {$\mathfrak{n}_{1}$};
 \node[scale=2.0] at (6,-2.7) {$\mathfrak{n}_{2}$};
 \node[scale=2.0] at (9,-2.7) {$\mathfrak{n}_{3}$};
 \node[scale=2.0] at (12,-0.7) {$\mathfrak{n}_{4}$};
 \node[scale=2.0] at (1.5,-0.5) {$k$};
 \node[scale=2.0] at (4.3,-1.4) {$a$};
 \node[scale=2.0] at (7.5,0.5) {$b$};
 \node[scale=2.0] at (7.5,2.8) {$c$};
 \node[scale=2.0] at (7.5,-1) {$d$};
 \node[scale=2.0] at (7.5,-1.8) {$e$};
 \node[scale=2.0] at (7.5,-3.1) {$f$};
 \node[scale=2.0] at (10.7,-1.4) {$g$};
 \node[scale=2.0] at (13.5,-0.5) {$k$};
 \node[scale=3.0, rotate =45] at (4.6,-1.12) {$\times$};
 \node[scale=3.0, rotate = 0] at (7.5,2.4) {$\times$};
 \node[scale=3.0, rotate = 0] at (7.5,0) {$\times$};
 
 
 \node[draw, single arrow,
 minimum height=33mm, minimum width=8mm,
 single arrow head extend=2mm,
 anchor=west, rotate=0] at (16,0) {}; 
 
 
 \node[] at (20,0) (11) {}; 
 \node[fillcirc] at (23,0) (12) {}; 
 \node[fillstar] at (26,-2) (13) {};
 \node[fillstar] at (26,0) (15) {};
 \node[fillstar] at (26,2) (14) {};
 \draw[-{Stealth[length=5mm, width=3mm]}] (11) edge (12);
 \draw[-{Stealth[length=5mm, width=3mm]}] (12) edge (13);
 \draw[-{Stealth[length=5mm, width=3mm]}] (12) edge (14);
 \draw[{Stealth[length=5mm, width=3mm]}-] (12) edge (15);
 
 \node[scale=2.0] at (23,-0.7) {$\mathfrak{n}_{1}$};
 \node[scale=2.0] at (21.5,-0.5) {$k$};
 \node[scale=2.0] at (24.3,-1.4) {$a$};
 \node[scale=2.0] at (24.8,0.4) {$b$};
 \node[scale=2.0] at (24.3,1.4) {$c$};
 \node[scale=2.0] at (23,-1.8) {$A$};
 
 
 \node[] at (28,-2) (32) {}; 
 \node[fillcirc] at (31,-2) (33) {}; 
 \node[fillcirc] at (34,-2) (34) {}; 
 \node[fillcirc] at (37,0) (35) {}; 
 \node[] at (34,0) (38) {}; 
 \node[] at (40,0) (36) {}; 
 \node[] at (34,2) (37) {};
 \draw[-{Stealth[length=5mm, width=3mm]}] (32) edge (33);
 \draw[-{Stealth[length=5mm, width=3mm]}, bend left =40] (33) edge (34);
 \draw[-{Stealth[length=5mm, width=3mm]}, bend right =40] (33) edge (34);
 \draw[{Stealth[length=5mm, width=3mm]}-] (33) edge (34);
 \draw[-{Stealth[length=5mm, width=3mm]}] (34) edge (35);
 \draw[-{Stealth[length=5mm, width=3mm]}] (35) edge (36);
 \draw[-{Stealth[length=5mm, width=3mm]}] (37) edge (35);
 \draw[-{Stealth[length=5mm, width=3mm]}] (35) edge (38);
 
 \node[scale=2.0] at (31,-2.7) {$\mathfrak{n}_{2}$};
 \node[scale=2.0] at (34,-2.7) {$\mathfrak{n}_{3}$};
 \node[scale=2.0] at (37,-0.7) {$\mathfrak{n}_{4}$};
 \node[scale=2.0] at (29.5,-2.5) {$a$};
 \node[scale=2.0] at (32.5,-3.1) {$f$};
 \node[scale=2.0] at (32.5,-1) {$d$};
 \node[scale=2.0] at (32.5,-1.8) {$e$};
 \node[scale=2.0] at (35.7,-1.4) {$g$};
 \node[scale=2.0] at (35.7,1.4) {$c$};
 \node[scale=2.0] at (35.2,0.4) {$b$};
 \node[scale=2.0] at (38.5,-0.5) {$k$};
 \node[scale=2.0] at (37,-2.2) {$B_{a,b,c}$};
 \end{tikzpicture}
 }
 \caption{An example of cuts, $c(\mathfrak{n})$ and $c(\mathfrak{l})$}
 \label{fig.c(n)c(e)}
 \end{figure}
 \end{enumerate}
\end{defn}
\begin{rem}
Explicitly writing down the full definition of $\text{rc}$ is often complicated, so in what follows, when defining $\text{rc}$, we will only describe which one should be the free or fixed leg in the two legs produced by cutting an edge.
%Explicitly writing down the full definition of $\text{rc}$ is often complicated, so I what follows, when defining $\text{rc}$, we will just describe which node is $\star$ nodes or invisible nodes in the two nodes after cutting one edge.
\end{rem}




Let us explain how does $Eq(\mathcal{C})$ and $\#Eq(\mathcal{C})$ changes after cutting. The result is summarized in the following lemma.
\begin{lem}\label{lem.Eq(C)cutting}
Let $c$ be an admissible cut of $\mathcal{C}$ that consists of edges $\{\mathfrak{e}_{i}\}$ and $\mathcal{C}_1$, $\mathcal{C}_2$ be two components after cutting. Let $\mathfrak{e}_{i}^{(1)}\in \mathcal{C}_1$, $\mathfrak{e}_{i}^{(2)}\in \mathcal{C}_2$ be two edges obtained by cutting $\mathfrak{e}_{i}$. The $\text{rc}$ map is defined by assigning $\{\mathfrak{e}_{i}^{(1)}\}$ to be free legs and $\{\mathfrak{e}_{i}^{(2)}\}$ to be fixed legs. Then we have 
\begin{equation}\label{eq.Eq(C)cutting}
 Eq(\mathcal{C},\{c_{\mathfrak{l}}\}_{\mathfrak{l}})=\left\{(k_{\mathfrak{e}_1},k_{\mathfrak{e}_{2}}):\ k_{\mathfrak{e}_1}\in Eq(\mathcal{C}_1,\{c_{\mathfrak{l}_1}\}),\  k_{\mathfrak{e}_{2}}\in Eq\left(\mathcal{C}_{2}, \{c_{\mathfrak{l}_2}\}, \left\{k_{\mathfrak{e}_{i}^{(1)}}\right\}_{i}\right)\right\}.
\end{equation}
and
\begin{equation}\label{eq.Eq(C)cuttingcounting}
 \sup_{\{c_{\mathfrak{l}}\}_{\mathfrak{l}}}\#Eq(\mathcal{C},\{c_{\mathfrak{l}}\}_{\mathfrak{l}})\le
 \sup_{\{c_{\mathfrak{l}_1}\}_{\mathfrak{l}_1\in \text{leg}(\mathcal{C}_1)} } \# Eq(\mathcal{C}_1,\{c_{\mathfrak{l}_1}\}) \sup_{\{c_{\mathfrak{l}_2}\}_{\mathfrak{l}_2\in \text{leg}(\mathcal{C}_2)} }\# Eq(\mathcal{C}_{2}, \{c_{\mathfrak{l}_2}\}).
\end{equation}
Here $\text{leg}(\mathcal{C})$ is the set of fixed legs in $\mathcal{C}$ (not the set of all legs!).
\end{lem}
\begin{proof} By definition \eqref{eq.Eq(C,c)}, we have
\begin{equation}
\begin{split}
 Eq(\mathcal{C},\{c_{\mathfrak{l}}\}_{\mathfrak{l}})=&\{k_{\mathfrak{e}}\in \mathbb{Z}^d,\ |k_{\mathfrak{e}}| \lesssim 1,\ \forall \mathfrak{e}\in \mathcal{C}:MC_{\mathfrak{n}},\  EC_{\mathfrak{n}},\ \forall \mathfrak{n}\in \mathcal{C}.\ k_{\mathfrak{l}}=c_{\mathfrak{l}},\ \forall \mathfrak{l}\in \text{leg}(\mathcal{C}).\} \\
 =&\{(k_{\mathfrak{e}_1},k_{\mathfrak{e}_2}):\ |k_{\mathfrak{e}_1}| \lesssim 1,\ MC_{\mathfrak{n}_1},\  EC_{\mathfrak{n}_1}.\ \forall \mathfrak{e}_1, \mathfrak{n}_1\in\mathcal{C}_1.\ k_{\mathfrak{l}_1}=c_{\mathfrak{l}_1},\ \forall \mathfrak{l}_1\in \text{leg}(\mathcal{C})\cap \text{leg}(\mathcal{C}_1)
 \\
 \ |k_{\mathfrak{e}_2}| \lesssim 1,\ &MC_{\mathfrak{n}_2},\  EC_{\mathfrak{n}_2}.\ \forall \mathfrak{e}_2, \mathfrak{n}_2\in\mathcal{C}_2.\ k_{\mathfrak{l}_2}=c_{\mathfrak{l}_2},\ \forall \mathfrak{l}_2\in \text{leg}(\mathcal{C})\cap \text{leg}(\mathcal{C}_2),\ k_{\mathfrak{e}_{i}^{(2)}}=k_{\mathfrak{e}_{i}^{(1)}},\ \forall\mathfrak{e}_{i}\in c\}
 \\
 =&\left\{(k_{\mathfrak{e}_1},k_{\mathfrak{e}_{2}}):\ k_{\mathfrak{e}_1}\in Eq(\mathcal{C}_1,\{c_{\mathfrak{l}_1}\}),\  k_{\mathfrak{e}_{2}}\in Eq\left(\mathcal{C}_{2}, \{c_{\mathfrak{l}_2}\}, \left\{k_{\mathfrak{e}_{i}^{(1)}}\right\}_{i}\right)\right\}
\end{split}
\end{equation}
Here in $Eq\left(\mathcal{C}_{2}, \{c_{\mathfrak{l}_2}\}, \left\{k_{\mathfrak{e}_{i}^{(1)}}\right\}_{i}\right)$, $k_{\mathfrak{e}_{i}^{(1)}}$ are view as a constant value and $k_{\mathfrak{e}_{i}^{(2)}}$ are fixed to be this constant value.
%Let $c$ be a cut of $\mathcal{C}$ that consists of edges $\{\mathfrak{e}_{i}\}$. Let $\mathcal{C}_1$ and $\mathcal{C}_2$ be two components after cutting and $\mathfrak{e}_{i}^{(1)}\in \mathcal{C}_1$, $\mathfrak{e}_{i}^{(2)}\in \mathcal{C}_2$ be two edges obtained by cutting $\mathfrak{e}_{i}$. We assume that $\{\mathfrak{e}_{i}^{(1)}\}$ are free legs and $\{\mathfrak{e}_{i}^{(2)}\}$ are legs. 
\begin{equation}
 \begin{split}
 Eq(\mathcal{C},\{c_{\mathfrak{l}}\}_{\mathfrak{l}})=&\{k_{\mathfrak{e}}\in \mathbb{Z}^d,\ |k_{\mathfrak{e}}| \lesssim 1,\ \forall \mathfrak{e}\in \mathcal{C}:MC_{\mathfrak{n}},\  EC_{\mathfrak{n}},\ \forall \mathfrak{n}\in \mathcal{C}.\ k_{\mathfrak{l}}=c_{\mathfrak{l}},\ \forall \mathfrak{l}\in \text{leg}(\mathcal{C}).\} \\
 =&\{(k_{\mathfrak{e}_1},k_{\mathfrak{e}_2}):\ |k_{\mathfrak{e}_1}| \lesssim 1,\ MC_{\mathfrak{n}_1},\  EC_{\mathfrak{n}_1}.\ \forall \mathfrak{e}_1, \mathfrak{n}_1\in\mathcal{C}_1.\ k_{\mathfrak{l}_1}=c_{\mathfrak{l}_1},\ \forall \mathfrak{l}_1\in \text{leg}(\mathcal{C})\cap \text{leg}(\mathcal{C}_1)
 \\
 \ |k_{\mathfrak{e}_2}| \lesssim 1,\ &MC_{\mathfrak{n}_2},\  EC_{\mathfrak{n}_2}.\ \forall \mathfrak{e}_2, \mathfrak{n}_2\in\mathcal{C}_2.\ k_{\mathfrak{l}_2}=c_{\mathfrak{l}_2},\ \forall \mathfrak{l}_2\in \text{leg}(\mathcal{C})\cap \text{leg}(\mathcal{C}_2),\ k_{\mathfrak{e}_{i}^{(2)}}=k_{\mathfrak{e}_{i}^{(1)}},\ \forall\mathfrak{e}_{i}\in c\}
 \\
 =&\left\{(k_{\mathfrak{e}_1},k_{\mathfrak{e}_{2}}):\ k_{\mathfrak{e}_1}\in Eq(\mathcal{C}_1,\{c_{\mathfrak{l}_1}\}),\  k_{\mathfrak{e}_{2}}\in Eq\left(\mathcal{C}_{2}, \{c_{\mathfrak{l}_2}\}, \left\{k_{\mathfrak{e}_{i}^{(1)}}\right\}_{i}\right)\right\}
 \end{split}
\end{equation}
Here in $Eq\left(\mathcal{C}_{2}, \{c_{\mathfrak{l}_2}\}, \left\{k_{\mathfrak{e}_{i}^{(1)}}\right\}_{i}\right)$, $k_{\mathfrak{e}_{i}^{(1)}}$ are view as a constant value and $k_{\mathfrak{e}_{i}^{(2)}}$ are fixed to be this constant value.

Therefore, we have the following identity of $Eq(\mathcal{C},\{c_{\mathfrak{l}}\}_{\mathfrak{l}})$
\begin{equation}
 Eq(\mathcal{C},\{c_{\mathfrak{l}}\}_{\mathfrak{l}})=\left\{(k_{\mathfrak{e}_1},k_{\mathfrak{e}_{2}}):\ k_{\mathfrak{e}_1}\in Eq(\mathcal{C}_1,\{c_{\mathfrak{l}_1}\}),\  k_{\mathfrak{e}_{2}}\in Eq\left(\mathcal{C}_{2}, \{c_{\mathfrak{l}_2}\}, \left\{k_{\mathfrak{e}_{i}^{(1)}}\right\}_{i}\right)\right\}.
\end{equation}
which proves \eqref{eq.Eq(C)cutting}.

We can also find the relation between $\#Eq(\mathcal{C}_1)$, $\#Eq(\mathcal{C}_2)$ and $\#Eq(\mathcal{C})$. Applying \eqref{eq.Eq(C)cutting},
\begin{equation}
\begin{split}
 \#Eq(\mathcal{C},\{c_{\mathfrak{l}}\}_{\mathfrak{l}})=&\sum_{(k_{\mathfrak{e}_1},k_{\mathfrak{e}_{2}})\in \#Eq(\mathcal{C},\{c_{\mathfrak{l}}\}_{\mathfrak{l}})} 1
 \\
 =&\sum_{\left\{(k_{\mathfrak{e}_1},k_{\mathfrak{e}_{2}}):\ k_{\mathfrak{e}_1}\in Eq(\mathcal{C}_1,\{c_{\mathfrak{l}_1}\}),\  k_{\mathfrak{e}_{2}}\in Eq\left(\mathcal{C}_{2}, \{c_{\mathfrak{l}_2}\}, \left\{k_{\mathfrak{e}_{i}^{(1)}}\right\}_{i}\right)\right\}} 1
 \\
 =&\sum_{k_{\mathfrak{e}_1}\in Eq(\mathcal{C}_1,\{c_{\mathfrak{l}_1}\})} \sum_{k_{\mathfrak{e}_{2}}\in Eq\left(\mathcal{C}_{2}, \{c_{\mathfrak{l}_2}\}, \left\{k_{\mathfrak{e}_{i}^{(1)}}\right\}_{i}\right)} 1
 \\
 =&\sum_{k_{\mathfrak{e}_1}\in Eq(\mathcal{C}_1,\{c_{\mathfrak{l}_1}\})} \# Eq\left(\mathcal{C}_{2}, \{c_{\mathfrak{l}_2}\}, \left\{k_{\mathfrak{e}_{i}^{(1)}}\right\}_{i}\right)
\end{split}
\end{equation}
Take $\sup$ in the above equation
\begin{equation}
\begin{split}
 &\sup_{\{c_{\mathfrak{l}}\}_{\mathfrak{l}}}\#Eq(\mathcal{C},\{c_{\mathfrak{l}}\}_{\mathfrak{l}})
 =\sup_{\{c_{\mathfrak{l}}\}_{\mathfrak{l}}}\sum_{k_{\mathfrak{e}_1}\in Eq(\mathcal{C}_1,\{c_{\mathfrak{l}_1}\})} \# Eq\left(\mathcal{C}_{2}, \{c_{\mathfrak{l}_2}\}, \left\{k_{\mathfrak{e}_{i}^{(1)}}\right\}_{i}\right)
 \\
 \le &\sup_{\{c_{\mathfrak{l}_1}\}_{\mathfrak{l}_1\in \text{leg}(\mathcal{C})\cap \text{leg}(\mathcal{C}_1)} }\sum_{k_{\mathfrak{e}_1}\in Eq(\mathcal{C}_1,\{c_{\mathfrak{l}_1}\})} \sup_{\{c_{\mathfrak{l}_2}\}_{\mathfrak{l}_2\in \text{leg}(\mathcal{C})\cap \text{leg}(\mathcal{C}_2)} }\# Eq\left(\mathcal{C}_{2}, \{c_{\mathfrak{l}_2}\}, \left\{k_{\mathfrak{e}_{i}^{(1)}}\right\}_{i}\right)
 \\
 \le &\sup_{\{c_{\mathfrak{l}_1}\}_{\mathfrak{l}_1\in \text{leg}(\mathcal{C}_1)} }\sum_{k_{\mathfrak{e}_1}\in Eq(\mathcal{C}_1,\{c_{\mathfrak{l}_1}\})} \sup_{\{c_{\mathfrak{l}_2}\}_{\mathfrak{l}_2\in \text{leg}(\mathcal{C}_2)} }\# Eq(\mathcal{C}_{2}, \{c_{\mathfrak{l}_2}\})
 \\
 = &\sup_{\{c_{\mathfrak{l}_1}\}_{\mathfrak{l}_1\in \text{leg}(\mathcal{C}_1)} } \# Eq(\mathcal{C}_1,\{c_{\mathfrak{l}_1}\}) \sup_{\{c_{\mathfrak{l}_2}\}_{\mathfrak{l}_2\in \text{leg}(\mathcal{C}_2)} }\# Eq(\mathcal{C}_{2}, \{c_{\mathfrak{l}_2}\})
\end{split}
\end{equation}

This proves \eqref{eq.Eq(C)cuttingcounting}.
\end{proof}



\subsubsection{Combinatorial properties of couples} In this section, we prove several combinatorial properties of couples. These properties are important in the counting argument.

The following lemma gives a relation between the numbers of non-leg edges, legs, and nodes.

\begin{lem}\label{lem.eulerchar}
For any couple $\mathcal{C}$, let $n(\mathcal{C})$ be the total number of nodes in $\mathcal{C}$ and $n_e(\mathcal{C})$ (resp. $n_{fx}(\mathcal{C})$, $n_{\textit{fr}}(\mathcal{C})$) be the total number of non-leg edges (resp. fixed legs, free legs). Then we have

(1) $n(\mathcal{C})$, $n_e(\mathcal{C})$, $n_{fx}(\mathcal{C})$ and $n_{\textit{fr}}(\mathcal{C})$ satisfies the following relation
\begin{equation}\label{eq.nnenlnle}
 2n_e(\mathcal{C})+n_{fx}(\mathcal{C})+n_{\textit{fr}}(\mathcal{C})=4n(\mathcal{C})
\end{equation}

(2) For any couple $\mathcal{C}$, define $\chi(\mathcal{C})=n_e(\mathcal{C})+n_{\textit{fr}}(\mathcal{C})-n(\mathcal{C})$. Let $c$, $\mathcal{C}_1$, $\mathcal{C}_2$ be the same as in Lemma \ref{lem.Eq(C)cutting} and also assume that $\{\mathfrak{e}_{i}^{(1)}\}$ are free legs and $\{\mathfrak{e}_{i}^{(2)}\}$ are fixed legs. Then 
\begin{equation}
 \chi(\mathcal{C})=\chi(\mathcal{C}_1)+\chi(\mathcal{C}_2).
\end{equation}

\end{lem}

\begin{proof}
We first prove (1). Consider the set $\mathcal{S}=\{(\mathfrak{n}, \mathfrak{e})\in \mathcal{C}: \mathfrak{n} \textit{ is an end point of }\mathfrak{e}\}$, then 
\begin{equation}
\#\mathcal{S}=\sum_{\substack{(\mathfrak{n}, \mathfrak{e})\in \mathcal{C}\\ \mathfrak{n} \textit{ is an end point of }\mathfrak{e}}} 1
\end{equation}

First sum over $\mathfrak{e}$ and then over $\mathfrak{n}$, we get 
\begin{equation}
\#\mathcal{S}=\sum_{\mathfrak{n}}\sum_{\substack{\mathfrak{e}\in \mathcal{C}\\ \mathfrak{n} \textit{ is an end point of }\mathfrak{e}}} 1=\sum_{\mathfrak{n}}\ 4=4n(\mathcal{C}).
\end{equation}
In the second equality, $\sum_{\mathfrak{e}\in \mathcal{C}: \mathfrak{n} \textit{ is an end point of }\mathfrak{e}} 1 =4$ because for each node $\mathfrak{n}$ there are $4$ edges connected to it.

Switch the order of summation, we get 
\begin{equation}
\begin{split}
\#\mathcal{S}=&\sum_{\mathfrak{e} \textit{ is a non-leg edges}}\sum_{\substack{\mathfrak{n}\in \mathcal{C}\\ \mathfrak{n} \textit{ is an end point of }\mathfrak{e}}} 1+\sum_{\mathfrak{e} \textit{ is a leg}}\sum_{\substack{\mathfrak{n}\in \mathcal{C}\\ \mathfrak{n} \textit{ is an end point of }\mathfrak{e}}} 1
\\
=&\sum_{\mathfrak{e} \textit{ is a non-leg edges}} 2+\sum_{\mathfrak{e} \textit{ is a leg}} 1
\\
=& 2n_e(\mathcal{C})+n_{fx}(\mathcal{C})+n_{\textit{fr}}(\mathcal{C})
\end{split}
\end{equation}
In the second equality, $\sum_{\substack{\mathfrak{n}\in \mathcal{C}\\ \mathfrak{n} \textit{ is an end point of }\mathfrak{e}}} 1$ equals to $1$ or $2$ because for each non-leg edge (resp. leg) there are $2$ (resp. 1) nodes connected to it.

Because the value of $\#\mathcal{S}$ does not depend on the order of summation, we conclude that $2n_e(\mathcal{C})+n_{fx}(\mathcal{C})+n_{\textit{fr}}(\mathcal{C})=4n(\mathcal{C})$, which proves (1).


We now prove (2). Since cutting does change the number of nodes, we have $n(\mathcal{C})=n(\mathcal{C}_1)+n(\mathcal{C}_2)$. Let $c$ be cut that consists of edges $\{\mathfrak{e}_{i}\}$ and $n(c)$ be the number of edges in $c$. 
%Given a couple $\mathcal{C}$, let $n(\mathcal{C})$, $n_e(\mathcal{C})$, $n_{\textit{fx}}(\mathcal{C})$ and $n_{\textit{fr}}(\mathcal{C})$ be the total number of the number of nodes, non-leg edges, fixed legs, and free legs respectively. 
When cutting $\mathcal{C}$ into $\mathcal{C}_1$ and $\mathcal{C}_2$, $n(c)$ non-leg edges are cut into pairs of free and fixed legs, so $n_e(\mathcal{C})=n(\mathcal{C}_1)+n(\mathcal{C}_2)+n(c)$. Because we have $n(c)$ additional free legs after cutting, so $n_{\textit{fr}}(\mathcal{C})=n_{\textit{fr}}(\mathcal{C}_1)+n_{\textit{fr}}(\mathcal{C}_2)-n(c)$. Therefore, we get
\begin{equation}
\begin{split}
 \chi(\mathcal{C})=&n_e(\mathcal{C})+n_{\textit{fr}}(\mathcal{C})-n(\mathcal{C})
 \\
 =&n(\mathcal{C}_1)+n(\mathcal{C}_2)+n(c)+n_{\textit{fr}}(\mathcal{C}_1)+n_{\textit{fr}}(\mathcal{C}_2)-n(c)-(n(\mathcal{C}_1)+n(\mathcal{C}_2))
 \\
 =&(n(\mathcal{C}_1)+n_{\textit{fr}}(\mathcal{C}_1)-n(\mathcal{C}_1))+(n(\mathcal{C}_2)+n_{\textit{fr}}(\mathcal{C}_2)-n(\mathcal{C}_2))
 \\
 =&\chi(\mathcal{C}_1)+\chi(\mathcal{C}_2).
\end{split}
\end{equation}

We complete the proof of (2).
\end{proof}

The following lemma gives a relation between variables of legs.

\begin{lem}\label{lem.freeleg} %Free leg moving lemma
Given a connected couple $\mathcal{C}$ with multiple legs, then we have the following conclusions.

(1) Let $\{k_{\mathfrak{l}_i}\}_{i=1,\cdots,n_{\text{leg}}}$ be the variables corresponding to legs in couple $\mathcal{C}$. Let $\iota_{\mathfrak{e}}$ be the same as Definition \ref{def.tree} (12), then $Eq(\mathcal{C})$ implies the following momentum conservation equation
\begin{equation}\label{eq.momentumconservation}
 \sum_{i=1}^{n_{\text{leg}}} \iota_{\mathfrak{l}_i}k_{\mathfrak{l}_i}=0,
\end{equation}

(2) Assume that there is exactly one free leg $\mathfrak{l}_{i_0}$ in $\mathcal{C}$ and all other variables $\{k_{\mathfrak{l}_{i}}\}_{i\ne i_0}$ corresponding to fix legs are fixed to be constants $\{c_{\mathfrak{l}_{i}}\}_{i\ne i_0}$. For any $i_{1}=1,\cdots,n_{\text{leg}}$, we can construct a new couple $\widehat{\mathcal{C}}$ by replacing the $i_0$ leg by a fixed leg and $i_1$ leg by a free leg. If $i\ne i_0, i_1$, fix $k_{\mathfrak{l}_{i}}$ to be the constant $c_{\mathfrak{l}_{i}}$, if $i=i_0$, fix $k_{\mathfrak{l}_{i_0}}$ to be the constant $-\iota_{\mathfrak{l}_{i_0}}\sum_{i\ne i_0} \iota_{\mathfrak{l}_i}k_{\mathfrak{l}_i}$. Under the above assumptions, we have
\begin{equation}
 Eq(\mathcal{C}, \{c_{\mathfrak{l}_{i}}\}_{i\ne i_0})=Eq\left(\widehat{\mathcal{C}}, \{c_{\mathfrak{l}_{i}}\}_{i\ne i_0, i_1}\cup \{-\iota_{\mathfrak{l}_{i_0}}\sum_{i\ne i_0} \iota_{\mathfrak{l}_i}k_{\mathfrak{l}_i}\}\right).
\end{equation}

(3) Assume that there is no free leg in $\mathcal{C}$ and all $\{k_{\mathfrak{l}_{i}}\}_{i\ne i_0}$ are fixed to be constants $\{c_{\mathfrak{l}_{i}}\}_{i\ne i_0}$. For any $i_{1}=1,\cdots,n_{\text{leg}}$, we can construct a new couple $\widehat{\mathcal{C}}$ by replacing the $i_0$ leg by a free leg. Then we have
\begin{equation}
 Eq(\mathcal{C}, \{c_{\mathfrak{l}_{i}}\}_{i})=Eq(\widehat{\mathcal{C}}, \{c_{\mathfrak{l}_{i}}\}_{i\ne i_0}).
\end{equation}

(4) If the couple $\mathcal{C}$ contains any leg, then it contains at least two legs.

\end{lem}
\begin{proof}
We first prove (1). Given a node $\mathfrak{n}$ and an edge $\mathfrak{e}$ connected to it, we define $\iota_{\mathfrak{e}}(\mathfrak{n})$ by the following rule
\begin{equation}
\iota_{\mathfrak{e}}(\mathfrak{n})=\begin{cases}
 +1 \qquad \textit{if $\mathfrak{e}$ pointing towards $\mathfrak{n}$}
 \\
 -1 \qquad \textit{if $\mathfrak{e}$ pointing outwards from $\mathfrak{n}$}
\end{cases}
\end{equation}
For a leg $\mathfrak{l}$, since it is connected to just one node, we may omit the $(\mathfrak{n})$ and just write $\iota_{\mathfrak{l}}$ as in the statement of the lemma.

For each node $\mathfrak{n}$, let $\mathfrak{e}_1(\mathfrak{n})$, $\mathfrak{e}_2(\mathfrak{n})$, $\mathfrak{e}(\mathfrak{n})$ be the three edges connected to it. For each edge $\mathfrak{e}$, let $\mathfrak{n}_1(\mathfrak{e})$, $\mathfrak{n}_2(\mathfrak{e})$ be the two nodes connected to it. Then we know that $\iota_{\mathfrak{e}}(\mathfrak{n}_1(\mathfrak{e}))+\iota_{\mathfrak{e}}(\mathfrak{n}_2(\mathfrak{e}))$, since $\mathfrak{n}_1(\mathfrak{e})$ and $\mathfrak{n}_2(\mathfrak{e})$ have the opposite direction. 

Since $k_{\mathfrak{e}}$ satisfy $Eq(\mathcal{C})$, by \eqref{eq.momentumconservationunit}, we get 
\begin{equation}
 \iota_{\mathfrak{e}_1(\mathfrak{n})}(\mathfrak{n})k_{\mathfrak{e}_1(\mathfrak{n})}+\iota_{\mathfrak{e}_2(\mathfrak{n})}(\mathfrak{n})k_{\mathfrak{e}_2(\mathfrak{n})}+\iota_{\mathfrak{e}_3(\mathfrak{n})}(\mathfrak{n})k_{\mathfrak{e}_3(\mathfrak{n})}+\iota_{\mathfrak{e}(\mathfrak{n})}(\mathfrak{n})k_{\mathfrak{e}(\mathfrak{n})}=0.
\end{equation}

Summing over $\mathfrak{n}$ gives 
\begin{equation}
\begin{split}
 0=&\sum_{\mathfrak{n}\in \mathcal{C}}\iota_{\mathfrak{e}_1(\mathfrak{n})}(\mathfrak{n})k_{\mathfrak{e}_1(\mathfrak{n})}+\iota_{\mathfrak{e}_2(\mathfrak{n})}(\mathfrak{n})k_{\mathfrak{e}_2(\mathfrak{n})}+\iota_{\mathfrak{e}_3(\mathfrak{n})}(\mathfrak{n})k_{\mathfrak{e}_3(\mathfrak{n})}+\iota_{\mathfrak{e}(\mathfrak{n})}(\mathfrak{n})k_{\mathfrak{e}(\mathfrak{n})}
 \\
 =& \sum_{\mathfrak{e}\text{ is not a leg}} 
 (\iota_{\mathfrak{e}}(\mathfrak{n}_1(\mathfrak{e}))+\iota_{\mathfrak{e}}(\mathfrak{n}_2(\mathfrak{e}))) k_{\mathfrak{e}}+ \sum_{\mathfrak{l}\text{ is a leg}} 
 \iota_{\mathfrak{l}} k_{\mathfrak{l}}
 \\
 =& \sum_{i=1}^{n_{\text{leg}}} \iota_{\mathfrak{l}_i}k_{\mathfrak{l}_i}
\end{split}
\end{equation}
This proves \eqref{eq.momentumconservation} and thus proves (1).

Now we prove (2). Since in $Eq\left(\widehat{\mathcal{C}}, \{c_{\mathfrak{l}_{i}}\}_{i\ne i_0, i_1}\cup \{-\iota_{\mathfrak{l}_{i_0}}\sum_{i\ne i_0} \iota_{\mathfrak{l}_i}k_{\mathfrak{l}_i}\}\right)$, $\{k_{\mathfrak{l}_{i}}\}_{i\ne i_0, i_1}$ are fixed to be constants $\{c_{\mathfrak{l}_{i}}\}_{i\ne i_0, i_1}$ and $k_{\mathfrak{l}_{i_0}}$ is fixed to be the constant $-\iota_{\mathfrak{l}_{i_0}}\sum_{i\ne i_0} \iota_{\mathfrak{l}_i}k_{\mathfrak{l}_i}$, by \eqref{eq.momentumconservation}, we know that 
\begin{equation}
 \iota_{\mathfrak{l}_{i_0}}\left(-\iota_{\mathfrak{l}_{i_0}}\sum_{i\ne i_0} \iota_{\mathfrak{l}_i}k_{\mathfrak{l}_i}\right)+ \iota_{\mathfrak{l}_{i_1}}k_{\mathfrak{l}_{i_1}}+\sum_{i\ne i_0, i_1} \iota_{\mathfrak{l}_i}c_{\mathfrak{l}_i}=0.
\end{equation}

This implies that $k_{\mathfrak{l}_{i_1}}=c_{\mathfrak{l}_{i_1}}$ in $Eq(\widehat{\mathcal{C}})$. Therefore, equations in $Eq(\widehat{\mathcal{C}})$ automatically imply $k_{\mathfrak{l}_{i_1}}=c_{\mathfrak{l}_{i_1}}$. Notice that whether or not containing $k_{\mathfrak{l}_{i_1}}=c_{\mathfrak{l}_{i_1}}$ is the only difference between $Eq(\mathcal{C})$ and $Eq(\widehat{\mathcal{C}})$. We conclude that $Eq(\mathcal{C})=Eq(\widehat{\mathcal{C}})$. We thus complete the proof of (2).

The proof of (3) is similar to (2). Whether or not containing $k_{\mathfrak{l}_{i_0}}=c_{\mathfrak{l}_{i_0}}$ is the only difference between $Eq(\mathcal{C})$ and $Eq(\widehat{\mathcal{C}})$. But if $\{k_{\mathfrak{l}_{i}}\}_{i\ne i_0}$ are fixed to be constants $\{c_{\mathfrak{l}_{i}}\}_{i\ne i_0}$, by momentum conservation we know that 
\begin{equation}
 k_{\mathfrak{l}_{i_0}}=-\iota_{\mathfrak{l}_{i_0}}\sum_{i\ne i_0} \iota_{\mathfrak{l}_i}c_{\mathfrak{l}_i}.
\end{equation}
Therefore, $k_{\mathfrak{l}_{i_0}}$ is fixed to be the constant $-\iota_{\mathfrak{l}_{i_0}}\sum_{i\ne i_0} \iota_{\mathfrak{l}_i}c_{\mathfrak{l}_i}$ in $Eq(\widehat{\mathcal{C}})$ and we conclude that $Eq(\mathcal{C})=Eq(\widehat{\mathcal{C}})$. We thus complete the proof of (3).

By Lemma \ref{lem.eulerchar}, we know that the total number of legs $n_{leg} = n_{fx}(\mathcal{C})+n_{\textit{fr}}(\mathcal{C})$ is an even number. This proves (4).
\end{proof}

After splitting all $\circ$ nodes in Definition \ref{def.couple}, a couple may have several connected components. The following properties of connected components are important in the counting argument.

\begin{defn}
 \begin{enumerate}
 \item \textbf{Open and closed couples.} A connected couple is said to be \underline{open} (resp. \underline{closed}) if it (resp. does not) contains legs. A connected component of a couple is a \underline{open component} (resp. \underline{closed component}) if it is open (resp. closed).
 \item \textbf{Good couples and bad couples.} A closed couple is a \underline{good couple} (resp. \underline{bad couple}) if it (resp. does not) contains at least two dotted edges (see Definition \ref{def.couple} (2)). 
 
 Figure \ref{fig.goodbad} provides some examples of couples that are good or bad.
 
 \begin{figure}[H]
 \centering
 \scalebox{0.4}{
 \begin{tikzpicture}
 \node[fillcirc] at (0,0) (1) {};
 \node[fillcirc] at (0,-4) (2) {};
 \node[fillcirc] at (4,0) (3) {};
 \node[fillcirc] at (4,-4) (4) {};

 \draw[-{Stealth[length=5mm, width=3mm]}] (2) edge (1);
 \draw[-{Stealth[length=5mm, width=3mm]}, bend left =40] (1) edge (2);
 \draw[-{Stealth[length=5mm, width=3mm]}, bend right =40] (1) edge (2);
 \draw[-{Stealth[length=5mm, width=3mm]}] (3) edge (4);
 \draw[-{Stealth[length=5mm, width=3mm]}, bend left =40] (4) edge (3);
 \draw[-{Stealth[length=5mm, width=3mm]}, bend right =40] (4) edge (3);
 \draw[-{Stealth[length=5mm, width=3mm]}, dashed] (3) -- (1);
 \draw[-{Stealth[length=5mm, width=3mm]}, dashed] (2) -- (4);

 \node[fillcirc] at (13,0) (1) {};
 \node[fillcirc] at (13,-4) (2) {};
 \node[fillcirc] at (17,0) (3) {};
 \node[fillcirc] at (17,-4) (4) {};

 \draw[-{Stealth[length=5mm, width=3mm]}] (2) edge (1);
 \draw[-{Stealth[length=5mm, width=3mm]}, bend left =40] (1) edge (2);
 \draw[-{Stealth[length=5mm, width=3mm]}, bend right =40] (1) edge (2);
 \draw[-{Stealth[length=5mm, width=3mm]}] (3) edge (4);
 \draw[-{Stealth[length=5mm, width=3mm]}, bend left =40] (4) edge (3);
 \draw[-{Stealth[length=5mm, width=3mm]}, bend right =40] (4) edge (3);
 \draw[-{Stealth[length=5mm, width=3mm]}, dashed] (3) -- (1);
 \draw[-{Stealth[length=5mm, width=3mm]}] (2) -- (4);
 
 
 \end{tikzpicture}
 }
 \caption{Example of a good couple (left) and bad couple (right)}
 \label{fig.goodbad}
 \end{figure}

 \item \textbf{Good nodes and bad nodes in renormalization forest.} Given a pairing $p$ and a node $\mathfrak{n}$ in a renormalization forest $F$, let $T_{\mathfrak{n}}$ be the subtree $F$ rooted at $\mathfrak{n}$ and $\text{Child}_{\mathfrak{n}}$ be the set of all children of $\mathfrak{n}$, then $\mathfrak{n}$ is said to be \underline{bad} if the pairing $p\notin \mathcal{P}(L(T_{\mathfrak{n}})\backslash\cup_{n'\in \text{Child}_{\mathfrak{n}}} L(T_{\mathfrak{n}'}),A)$. Otherwise, $\mathfrak{n}$ is said to be \underline{good}. Here $A=\{k_1,k_2,\cdots,k_{2l+1}\}$, $\mathcal{P}(B,A)$ is defined in Definition \ref{def.pairing} (4)
 
 In other word, a node is bad if $p$ does pair all leaves in $L(T_{\mathfrak{n}})\backslash\cup_{n'\in \text{Child}_{\mathfrak{n}}} L(T_{\mathfrak{n}'})$ with leaves in the this set. For example, consider the following renormalization forest (this forest is the same as the forest in Figure \ref{fig.forestsr}).
 \begin{figure}[H]
 \centering
 \scalebox{0.36}{
 \begin{tikzpicture}[level distance=80pt, sibling distance=60pt]
 
 \node[circ, xshift = -2cm] (2) at (25, 0) {}
 child {node[circ, xshift = -1cm] (21) {}
 child {node[fillstar, xshift = -1cm] (211) {}}
 child {node[fillstar, xshift = 1cm] (212) {}}
 }
 child {node[fillstar] (22) {}}
 child {node[fillstar, xshift = 1cm] (23) {}};
 \node[scale=1.5] at ($(211)+(0.5,0)$) {$1$};
 \node[scale=1.5] at ($(212)+(0.5,0)$) {$2$};
 \node[scale=1.5] at ($(22)+(0.5,0)$) {$3$};
 \node[scale=1.5] at ($(23)+(0.5,0)$) {$4$};
 % \node[scale=1.5] at (18.2,-5.65) {$1$};
 % \node[scale=1.5] at (22.5,-5.65) {$2$};
 % \node[scale=1.5] at (23.5,-2.8) {$3$}; 
 % \node[scale=1.5] at (26.6,-2.8) {$4$};
 \node[circ, xshift = -1cm] (3) at (31, 0) {}
 child {node[fillstar, xshift = -1cm] (31) {}}
 child {node[fillstar, xshift = 1cm] (32) {}}; 
 \node[scale=1.5] at ($(31)+(0.5,0)$) {$9$};
 \node[scale=1.5] at ($(32)+(0.5,0)$) {$10$};
 % \node[scale=1.5] at (28.5,-2.8) {$9$}; 
 % \node[scale=1.5] at (32.6,-2.8) {$10$};

 \draw[bend right =50, dashed] (22) edge (23);
 \draw[bend right =50, dashed] (211) edge (31);
 \draw[bend right =50, dashed] (212) edge (32);
 \end{tikzpicture}
 }
 \caption{Good nodes and bad nodes}
 \label{fig.goodbadnodes}
 \end{figure}
 Assume that the pairing is $p=\{1,9\}\cup\{2,10\}\cup\{3,4\}$, then the top left node $\mathfrak{n}_0$ is bad because $L(T_{\mathfrak{n}_0})\backslash\cup_{n'\in \text{Child}_{\mathfrak{n}_0}} L(T_{\mathfrak{n}'})=\{3,4\}$ are paired. The other two nodes are good nodes.

 \item \textbf{Chain of bad circle nodes.} Given a renormalization forest $F$, a \underline{chain of bad nodes} is a sequence of bad nodes $\mathfrak{n}_1$, $\mathfrak{n}_2$, $\cdots$, $\mathfrak{n}_{l}$ such that this sequence have the property that for each $1\le i\le l-1$, $\mathfrak{n}_{i}$ only have one child $\mathfrak{n}_{i+1}$ and this sequence is one of the maximal sequences satisfies this property. We say a chain of bad nodes is \underline{long} or \underline{short}, if $l>1$ or $l=1$ respectively.
 
 For example, consider the following renormalization forest, the pairing is $p=\{1,7\}\cup\{2,8\}\cup\{3,4\}\cup\{5,6\}$, then $n_1$, $n_2$ form a long chain of bad nodes, and $n_2$ or $n_3$ form two short chain of bad nodes

 \begin{figure}[H]
 \centering
 \scalebox{0.36}{
 \begin{tikzpicture}[level distance=80pt, sibling distance=60pt]
 
 \node[circ, xshift = -2cm] (1) at (0, 0) {}
 child {node[circ, xshift = -1cm] (11) {}
 child {node[circ, xshift = -1cm] (111) {}
 child {node[fillstar, xshift = -1cm] (1111) {}}
 child {node[fillstar, xshift = 1cm] (1112) {}}}
 child {node[fillstar] (112) {}}
 child {node[fillstar, xshift = 1cm] (113) {}}
 }
 child {node[fillstar] (12) {}}
 child {node[fillstar, xshift = 1cm] (13) {}};
 \node[scale=1.5] at ($(1111)+(0.5,0)$) {$1$};
 \node[scale=1.5] at ($(1112)+(0.5,0)$) {$2$};
 \node[scale=1.5] at ($(112)+(0.5,0)$) {$3$};
 \node[scale=1.5] at ($(113)+(0.5,0)$) {$4$};
 \node[scale=1.5] at ($(12)+(0.5,0)$) {$5$};
 \node[scale=1.5] at ($(13)+(0.5,0)$) {$6$};
 \node[circ, xshift = -1cm] (2) at (8, 0) {}
 child {node[fillstar, xshift = -1cm] (21) {}}
 child {node[fillstar, xshift = 1cm] (22) {}}; 
 \node[scale=1.5] at ($(21)+(0.5,0)$) {$7$};
 \node[scale=1.5] at ($(22)+(0.5,0)$) {$8$};
 \node[scale=1.5] at ($(1)+(0.7,0)$) {$\mathfrak{n}_1$};
 \node[scale=1.5] at ($(11)+(0.7,0)$) {$\mathfrak{n}_2$};
 \node[scale=1.5] at ($(111)+(0.7,0)$) {$\mathfrak{n}_3$};
 \node[scale=1.5] at ($(2)+(0.7,0)$) {$\mathfrak{n}_4$};

 \draw[bend right =50, dashed] (1111) edge (21);
 \draw[bend right =50, dashed] (1112) edge (22);
 \draw[bend right =50, dashed] (112) edge (113);
 \draw[bend right =50, dashed] (12) edge (13);
 \end{tikzpicture}
 }
 \caption{Chain of bad nodes}
 \label{fig.chainofbadnodes}
 \end{figure}
 %Let $n_{G}(\mathcal{C})$ and $n_{B}(\mathcal{C})$ be the number of good and bad component of $\mathcal{C}$ respectively. Let $n_{G}(F)$ and $n_{B}(F)$ be the good and bad nodes of $F$ respectively. Let $n_{BC}(\mathcal{C})$, $n_{BC}(\mathcal{F})$ chain of bad components and chain of bad nodes.
 
 % \begin{figure}[H]
 % \centering
 % %\includegraphics[scale=0.25]{components_cutted.png}
 % \caption{Example of good nodes and bad nodes in a renormalization forest}
 % \label{fig.goodbadr}
 % \end{figure}

 % \item \textbf{Two edge reducibility.} A couple is said to be \underline{two edge reducible} if there is a cut consisting of two edges that are all connected to the same node such that after cutting all edges in this cut, one of the resulting components has exactly two legs.

 % Figure \ref{fig.reducibility} provides some examples that do or do not satisfy this property.
 
 % \begin{figure}[H]
 % \centering
 % %\includegraphics[scale=0.25]{components_cutted.png}
 % \caption{Example of two edge reducible couples (left) and two edge reducible couples (right)}
 % \label{fig.reducibility}
 % \end{figure}
 \end{enumerate}
\end{defn}

The counting result in the next section says that we lose a factor for each bad node but gain a factor for each good node. We have the following lemma that gives the upper bounds of the number of chains of bad nodes in terms of good nodes.
% gives equalities and inequalities between the number of good and bad components and nodes. 

\begin{lem}\label{eq.ineqn_BCn_G} Given two trees $T$, $T'$ and a pairing $p\in\mathcal{P}_F$ of leaves in the renormalized pairing set $\mathcal{P}_F$, let $F$ be the forest which is the union of two forests $R(T)$ and $R(T')$. Then the couple $\mathcal{C} = \mathcal{C}(T,T',p)$ constructed from $T$, $T'$, $p$ have the following properties.

% Let $n_{G}(\mathcal{C})$ and $n_{B}(\mathcal{C})$ be the number of good and bad component of $\mathcal{C}$ respectively. 

Let $n_{G}(F)$ be the number of good and bad nodes of $F$ respectively. Let $n_{BC}(F)$ be the number of chains of bad nodes of $F$. Then we have the following inequality.
% \begin{equation}\label{eq.n_B}
% n_{B}(\mathcal{C}) = n_{B}(T)
% \end{equation}
% \begin{equation}\label{eq.n_G}
% n_{G}(\mathcal{C}) \le \frac{1}{2} n_{G}(T)
% \end{equation}
% \begin{equation}\label{eq.n_Bn_G}
% n_{B}(T) \le n_{G}(T)
% \end{equation}
\begin{equation}\label{eq.n_Bn_G}
 n_{BC}(F) + 1 \le 2n_{G}(F)
\end{equation}
\end{lem}
\begin{rem}
It is crucial that the couple is constructed from a pairing $p$ from $\mathcal{P}_F$. For a general pairing, the lemma is not true.
\end{rem}
\begin{proof} We first consider the following claim.
 
\textit{Claim 1.} If a circle node $\mathfrak{n}$ in $F$ has no circle child, then $\mathfrak{n}$ is good.

\begin{proof}
 Under this assumption, $\cup_{n'\in \text{Child}_{\mathfrak{n}}} L(T_{\mathfrak{n}'}) = \emptyset$ and $L(T_{\mathfrak{n}}) = L(T_{\mathfrak{n}})\backslash\cup_{n'\in \text{Child}_{\mathfrak{n}}} L(T_{\mathfrak{n}'})$. By the definition of $\mathcal{P}_F$, Definition \ref{def.pairing} (5), $p\notin\mathcal{P}(L(T_{\mathfrak{n}}), A)$. Therefore, $\mathfrak{n}$ is a good node.
\end{proof}


\textit{Claim 2.} The last nodes of a chain of bad circle nodes must either have a good circle child or have at least two bad children.

\begin{proof}
 If the claim is wrong, the last node has at most one circle child and this child is a bad circle node. If the last node does not have circle children, by claim 1, the last node must be good, which gives a contradiction. If the last node has exactly one circle child which is bad, then this contradicts the maximality of the chain with respect to its defining property.
\end{proof}


If a chain of bad nodes has no bad ancestors, we call it \underline{top chain of bad nodes}. Denote all top chain of bad nodes by $\mathfrak{c}_1, \cdots, \mathfrak{c}_k$. Denote by $T_{\mathfrak{c}_1}\cdots T_{\mathfrak{c}_k}$ the subtree of the first node of this chain. Then we know that 
\begin{equation}
 n_{BC}(F) = n_{BC}(T_{\mathfrak{c}_1}) + \cdots + n_{BC}(T_{\mathfrak{c}_k})
\end{equation}

\begin{equation}
 n_{G}(T_{\mathfrak{c}_1}) + \cdots + n_{G}(T_{\mathfrak{c}_k})\le n_{G}(F)
\end{equation}

By these two inequalities, we know that it suffices to prove \eqref{eq.n_Bn_G} for trees $T$ whose root is a bad circle node.

We prove \eqref{eq.n_Bn_G} for a tree $T$ with a bad root by induction on the total number $n$ of bad chains. 

If $n=1$, by claim 2, it has good circle children, so $n_{G}(T)\ge 1$. We get $n_{G}(T)+1\ge 2 =2n_{BC}(T) $.

Assume that the Lemma is true for $<n$. We prove the case $n$. 


For the top chain of bad nodes, by claim 2, it either have a good child or two bad children.

\underline{The one good child case.} In the first case, remove the top chain $\mathfrak{c}$ from $T$ to get a forest, then take the subtrees $T_{\mathfrak{c}_1'},\cdots, T_{\mathfrak{c}_{k'}'}$ of top chains of the new forest. We have 
\begin{equation}\label{eq.lemineqn_BCn_G1}
 n_{BC}(T) = n_{BC}(T_{\mathfrak{c}_1'}) + \cdots + n_{BC}(T_{\mathfrak{c}_{k'}'}) + 1
\end{equation}

\begin{equation}\label{eq.lemineqn_BCn_G2}
 n_{G}(T_{\mathfrak{c}_1'}) + \cdots + n_{G}(T_{\mathfrak{c}_{k'}'}) + 1\le n_{G}(T).
\end{equation}
Here we have the $+1$ in \eqref{eq.lemineqn_BCn_G1} for $n_{G}$ because in the first case, $\mathfrak{c}$ has good children.

By induction assumption, we have
\begin{equation}
 n_{BC}(T_{\mathfrak{c}_i'}) + 1 \le 2n_{G}(T_{\mathfrak{c}_i'}).
\end{equation}

Summing over $i$ and applying \eqref{eq.lemineqn_BCn_G1} and \eqref{eq.lemineqn_BCn_G2}, we get 
\begin{equation}
 n_{BC}(T) + k'-1 \le 2(n_{G}(T)-1).
\end{equation}

Since $k'\ge 0$, this implies \eqref{eq.n_Bn_G} in the first case.

\underline{The two bad child case.} Now we consider the second case. Remove the top chain $\mathfrak{c}$ from $T$ to get a forest, then take the subtrees $T_{\mathfrak{c}_1'},\cdots, T_{\mathfrak{c}_{k'}'}$ of top chains of the new forest. We have 
\begin{equation}\label{eq.lemineqn_BCn_G3}
 n_{BC}(T) = n_{BC}(T_{\mathfrak{c}_1'}) + \cdots + n_{BC}(T_{\mathfrak{c}_{k'}'}) + 1
\end{equation}

\begin{equation}\label{eq.lemineqn_BCn_G4}
 n_{G}(T_{\mathfrak{c}_1'}) + \cdots + n_{G}(T_{\mathfrak{c}_{k'}'})\le n_{G}(T).
\end{equation}

By induction assumption, we have
\begin{equation}
 n_{BC}(T_{\mathfrak{c}_i'}) + 1 \le 2n_{G}(T_{\mathfrak{c}_i'}).
\end{equation}

Summing over $i$ and applying \eqref{eq.lemineqn_BCn_G1} and \eqref{eq.lemineqn_BCn_G2}, we get 
\begin{equation}
 n_{BC}(T) + k' - 1 \le 2n_{G}(T).
\end{equation}

Since $k'\ge 2$, this implies \eqref{eq.n_Bn_G} in the second case, which completes the proof of the lemma.
\end{proof}


% For different chains of bad circle nodes, the good children are different, and then we can define a bijection map from the set of chains of bad nodes to good nodes.

% \begin{lem} Given two trees $T$, $T'$ and a pairing $p$ of leaves, let $F$ be the forest which is the union of two forests $R(T)$ and $R(T')$, we can construct a couple $\mathcal{C}(T,T',p)$ from $T$, $T'$, $p$. Let $n_{G}(\mathcal{C})$ and $n_{B}(\mathcal{C})$ be the number of good and bad component of $\mathcal{C}$ respectively. Let $n_{G}(F)$ and $n_{B}(F)$ be the good and bad nodes of $F$ respectively. Then we have the following equalities and inequalities.
% \begin{equation}\label{eq.n_B}
% n_{B}(\mathcal{C}) = n_{B}(T)
% \end{equation}
% \begin{equation}\label{eq.n_G}
% n_{G}(\mathcal{C}) \le \frac{1}{2} n_{G}(T)
% \end{equation}
% \begin{equation}\label{eq.n_Bn_G}
% n_{B}(T) \le n_{G}(T)
% \end{equation}
% \end{lem}
% \begin{proof}
% We first prove \eqref{eq.n_B}. From Definition \ref{def.couple} (2), we know that one $\circ$ node corresponds to one dotted edge. We need the following claim to continue the proof.

% \textit{Claim.} Given a tree $T$ and its renormalization forest $F=R(T)$, $\circ$ node and its children $n_1$, $n_2$ and $n_3$, then after splitting all $\circ$ nodes, the set of all leaves $L(T_{n_1})\cup L(T_{n_2})$ of $T$ in subtrees $T_{n_1}$, $T_{n_2}$ of $T$ equals to $L(T_{\mathfrak{n}})\backslash\cup_{n'\in \text{Child}_{\mathfrak{n}}} L(T_{\mathfrak{n}'})$.

% \begin{proof}
% This is obvious from the definition of splitting.
% \end{proof}

% Given a bad node, since all leaves in $L(T_{\mathfrak{n}})\backslash\cup_{n'\in \text{Child}_{\mathfrak{n}}} L(T_{\mathfrak{n}'})$ are paired by $p$ with leaves in the same set, by the above claim, we know that the same is true for $L(T_{n_1})\cup L(T_{n_2})$. Since all leaves in $L(T_{n_1})\cup L(T_{n_2})$ are paired with each other, we get a closed component with only one dotted edge which is bad. All bad components arise in this way, this gives \eqref{eq.n_B}.


% Then we prove \eqref{eq.n_G}. Since each good components contain at least two dotted edges, and these two dotted edges correspond to two good nodes, so we know that $2n_{G}(\mathcal{C}) \le n_{G}(T)$. This gives \eqref{eq.n_G}.

% Finally, we prove \eqref{eq.n_Bn_G}.

% \end{proof}


% We have the following lemma that gives an upper bound on the number of components in terms of the number of dotted edges in a couple and excludes the presence of two edge reducible components.


% \begin{lem} 

% \begin{enumerate}
% \item Given two trees $T$, $T'$ and a pairing $p$ of leaves, let $F$ be the forest which is the union of two forests $R(T)$ and $R(T')$, we can construct a couple $\mathcal{C}(T,T',p)$ from $T$, $T'$, $p$.
 
% Then all couples $\mathcal{C} = \mathcal{C}(T,T',p)$ constructed from a pairing $p\in \mathcal{P}_F$ have the property that the number $n_{G}(\mathcal{C})$ of good components of $\mathcal{C}$ is greater than the number $n_{B}(\mathcal{C})$ of bad components. In another word, we have the following inequality.
% \begin{equation}
% n_{G}(\mathcal{C})\ge n_{B}(\mathcal{C})
% \end{equation}

% \item Given a two edge reducible couple $\mathcal{C}$, we have $\# Eq(\mathcal{C}) = 0$.
% \end{enumerate}
% \end{lem}
% \begin{proof}
% We first prove (2). If $C$ is two edge reducible couple, then there is a cut consisting of two edges $\mathfrak{e}_1, \mathfrak{e}_2$ that are all connected to the same node $\mathfrak{n}$, such that after cutting all edges in this cut, one of the resulting components $\mathcal{C}_1$ has exactly two legs.

% Apply the Lemma \ref{lem.momentumconservation} to $\mathcal{C}_1$, then we know that $k_{\mathfrak{e}_1}=k_{\mathfrak{e}_2}$. By \ref{eq.momentumconservationunit}, we know that $k_{\mathfrak{e}_1}\ne k_{\mathfrak{e}_2}\ne k_{\mathfrak{e}_3}$. Therefore, the set of solutions must be empty. 

% Now we prove (1). 

% After splitting all $\circ$ nodes as in Definition \ref{def.couple} (3), $T$ is decomposed into $\{T_j\}$ and a tree $T_{\text{leg}}$, so that $T_{\text{leg}}$ contains the legs of $T$ and each $T_j$ contains exactly one dotted edge. It's easy to see that $\cup_{j} L(T_j)$ and $L(T_{\text{leg}})$ coincides with $L(F_{\mathfrak{n}})$ and $\{k_1,k_2,\cdots,k_{2m}\}\backslash L(F_{\mathfrak{n}})$ respectively. ($F=R(T)\cup R(T')$) Therefore, by Theorem \ref{th.wickr} and Definition \ref{def.pairing} (5), we know that there is no tree in $\{T_j\}$ and $T_{\text{leg}}$ whose leaves pair only with the leaves itself. Therefore, all trees in $\{T_j\}$ and $T_{\text{leg}}$ have at least one edge connected to another different tree. If a component does not have a leg, which means that $T_{\text{leg}}$ does not contain this component, then it contains at least two trees in $\{T_j\}$. This implies that this component has at least two dotted edges since each of these two $T_j$ trees contains a dotted edge.


% *****************************************************

% \textbf{Step 1.} In this step, we show that after splitting all $\circ$ nodes as in Definition \ref{def.couple} (3), $T$ is decomposed into $\{T^{\mathfrak{n}}\}_{\mathfrak{n}\text{ is decorated by } \circ}$ and a tree $T_{\text{leg}}$, so that $T_{\text{leg}}$ contains the legs of $T$ and each $T^{\mathfrak{n}}$ contains exactly one dotted edge.

% Before doing any proof, we refine Definition \ref{def.couple} (3) about splitting $\circ$ nodes when constructing couples. We eliminate its ambiguity by specifying the order of splitting $\circ$ nodes. 

% First split all $\circ$ nodes without $\circ$ descendants as in Definition \ref{def.couple} (2). After splitting one such $\circ$ nodes $\mathfrak{n}_{\circ}$, we get one tree $T_{\text{leg}}^{1}$ containing the leg and another tree $T^{\mathfrak{n}_{\circ}}$ containing one dotted edge connecting two subtrees $T_{\mathfrak{n}_1}$, $T_{\mathfrak{n}_2}$. Notice that there are no $\circ$ nodes in $T^{\mathfrak{n}_{\circ}}$.

% After splitting all $\circ$ descendant free $\circ$ nodes, there may be new such nodes in $T_{\text{leg}}^{1}$, then we split all such nodes and get $T_{\mathfrak{n}_2}$ and new $T^{\mathfrak{n}_{\circ}}$. Repeat this process until all $\circ$ nodes are split. 

% Finally, $T$ is decomposed into $\{T^{\mathfrak{n}}\}_{\mathfrak{n}\text{ is decorated by } \circ}$ and a tree $T_{\text{leg}}$ containing legs. Each $T^{\mathfrak{n}}$ contains exactly one dotted edge. Therefore, we finish the proof of step 1.


% \textbf{Step 2.} In this step, we prove that the set of leaves of $T^{\mathfrak{n}}$ is the same as the set of leaves $L(F_{\mathfrak{n}})$ if $F=R(T)\cup R(T')$.





% \textbf{Step 3.} In this step, we finish the proof of (3).

% By definition of $\mathcal{P}_{F}$ in Theorem \ref{th.wickr}, 

% When pairing leaves, several components of $T$, $T'$ can be connected into one component. The definition of $\mathcal{P}_F$ implies that any
% \end{proof}


\subsubsection{Counting results and their proof}
The following proposition gives an upper bound of the number of solutions of \eqref{eq.diophantineeqpairedsigma} (or \eqref{eq.diophantineeqpairedsigma'}).

\begin{prop}\label{prop.counting}
Let $\mathcal{C}=\mathcal{C}(T,T',p)$ be an open admissible couple, $n$ be the total number of nodes in $\mathcal{C}$, $n_c$ be the number of closed components of $\mathcal{C}$ and $Q=L^{d}\sqrt{\omega}$. We fix $k\in \mathbb{R}$ for the fixed legs $\mathfrak{l}$, $\mathfrak{l}'$ and $\sigma_{\mathfrak{n}}\in\mathbb{R}$ for each $\mathfrak{n}\in \mathcal{C}$. Assume that $\omega\ge L^{-1}$ (or $\omega\ge L^{-2}$ in case the ratio of periods is generic in the sense of Theorem \ref{th.numbertheory}). Then we have the following conclusions

(1) The number of solutions $M$ of \eqref{eq.diophantineeqpairedsigma} (or \eqref{eq.diophantineeqpairedsigma'}) is bounded by 
\begin{equation}\label{eq.countingbd0}
    M\leq L^{O(\theta)} Q^{n} L^{dn_c}.
\end{equation}

(2) We know that $n$ equals the total number of $\bullet$ nodes in $T$ and $T'$. Let $n_G$ and $n_B$ be the number of good and bad nodes in $T$ and $T'$ respectively. Then we have 
\begin{equation}\label{eq.ineqn_Gn_B}
    \left\{\begin{aligned}
        & 2n_G \ge n_{BC}+1
        \\
        & n_G \ge 2(n_c - n_B)
    \end{aligned}\right.
\end{equation}



% (3) Let $n_d$ be the total number of dotted edges in $\mathcal{C}$. Then $n_d$ equals to the total number of circle nodes in $T$ and $T'$. We also have the inequality
% \begin{equation}\label{eq.dottededgeclosedcomponent}
%     n_c\le \frac{1}{2}n_d
% \end{equation}

(3) Let $l$ and $l'$ be the total number of nodes in $T$ and $T'$ respectively. Assume that there is no long chain of bad circle nodes and $\omega \ge L^{-\frac{d}{3}}$. Then we also have 

\begin{equation}\label{eq.countingbd1}
    M/Q^{l+l'}\leq L^{O(\theta)}. 
\end{equation}

This inequality is important because the order of magnitude of $Term(T,T',p)$ is supposed to be  $M/Q^{l+l'}$ when $\omega=T_{\text{kin}}^{-1}$. As long as we can remove the condition that there is no long chain of bad circle nodes, up to epsilon loss the wave kinetic equation can be derived.

\end{prop}
\begin{proof} We first prove (2). The fact that $n$ equals the total number of $\bullet$ nodes in $T$ and $T'$ follows from the definition. The first inequality of \eqref{eq.ineqn_Gn_B} follows from Lemma \ref{eq.ineqn_BCn_G}. 
 
    Now we prove the second inequality of \eqref{eq.ineqn_Gn_B}. Remember from the definition of dotted edges, Definition \ref{def.couple} (2), each dotted edge corresponds to a circle node. By definition of bad circle nodes, components arising from them can only contain one dotted edge, so $n_c - n_B$ is the number of couples containing at least two dotted edges. All these couples correspond to at least two good circle nodes and thus the number of them is less than $n_G/2$. Therefore, we proved the desired inequality.
    
    Then we prove (3) from (1) and (2). Let $n_{\text{circ}}$ be the total number of circle nodes in $T$ and $T'$. Since $l+l'$ and $n$ are the total number of nodes and $\bullet$ nodes in $T$ and $T'$ respectively, we know that $l+l' = n + n_{\text{circ}}$. By definition of $n_G$ and $n_B$, we know that $n_{} = n_{G} + n_{B}$. 
    
    By the assumption that there is no long chain of bad circle nodes, we know that $n_{B} = n_{BC}$. Therefore, the first inequality of \eqref{eq.ineqn_Gn_B} implies 
    \begin{equation}\label{eq.n_Gn_B'}
     n_{G} \ge \frac{1}{2}n_{B}.
    \end{equation}
    
    Multiply this inequality by $\frac{2}{5}$, the second inequality of \eqref{eq.ineqn_Gn_B} $\frac{3}{5}$ and then sum them. We get the following inequality.
    
    \begin{equation}\label{eq.n_Gn_B''}
     n_{G} \ge \frac{6}{5}n_c - n_{B}\qquad \Leftrightarrow\qquad n_c \le \frac{5}{6} (n_{G} + n_{B}) 
    \end{equation}
    
    By (1), we get 
    \begin{equation}
     \begin{split}
     M/Q^{l+l'}\leq& L^{O(\theta)} Q^{n} L^{dn_c}/Q^{n+n_G+n_B}
     \\
     =& L^{O(\theta)} L^{dn} (\sqrt{\omega})^{n}L^{dn_c}/(L^{d(n+n_G+n_B)}(\sqrt{\omega})^{n+n_G+n_B})
     \\
     =& L^{O(\theta)} (\sqrt{\omega})^{-n_G-n_B}L^{d(n_c-n_G-n_B)}
     \\
     =& L^{O(\theta)} (L^{-\frac{d}{6}}\sqrt{\omega}^{-1})^{n_G+n_B}\le L^{O(\theta)}.
     \end{split}
    \end{equation}
    Here in the third inequality, we applied \eqref{eq.n_Gn_B''} and $\omega \ge L^{-\frac{d}{3}}$ in the hypothesis of (3).
    
    Therefore, we have finished the proof of (3)
    
    The proof of (1) is lengthy and therefore divided into several steps. To present its proof, let us first introduce several useful notations.
    
    Given a couple $\mathcal{C}$ and $k$, $\sigma_{\mathfrak{n}}$, let $Eq(\mathcal{C})$ be the system of equation \eqref{eq.diophantineeqpairedsigma'} constructed in Proposition \ref{prop.couple'}. For any system of equations $Eq$, let $\#(Eq)$ be its number of solutions.
    
    Now we prove (1).
    
    \textbf{Step 1.} In this step, we reduce Proposition \ref{prop.counting} to the case of connected couples.
    
    By Lemma \eqref{eq.Eq(C)cutting}, we know that 
    \begin{equation}\label{eq.conn}
     \#(Eq(\mathcal{C}))= \#(Eq(\widetilde{\mathcal{C}}))\prod_{j=1}^{n_c} \#(Eq(\mathcal{C}_{j})).
    \end{equation}
    Here $\widetilde{\mathcal{C}}$ is the open component of $\mathcal{C}$ (it is not difficult to show that $\mathcal{C}$ only has one open component), $\mathcal{C}_{j}$, $1\le j\le n_c$ are the closed components of $\mathcal{C}$ and we have $\mathcal{C}=\widetilde{\mathcal{C}}\cup(\cup_{j=1}^{n_c} \mathcal{C}_{j})$.
    
    Therefore, we just need to prove the following proposition which claims that Proposition \ref{prop.counting} is true for connected couples. Note that in the statement of the following proposition, we have an extra factor $L^d$ for each closed component. In total, we get a factor $L^{dn_c}$. This how we get the $L^{dn_c}$ in $M\leq L^{O(\theta)} Q^{n} L^{dn_c}$ of Proposition \ref{prop.counting}.
    
    
    \begin{prop}\label{prop.countingconn} Let $\mathcal{C}$ be a connected couple and $n$ be the total number of nodes in $\mathcal{C}$. We fix $\sigma_{\mathfrak{n}}\in\mathbb{R}$ for each $\mathfrak{n}\in \mathcal{C}$ and $k\in \mathbb{R}$ if $\mathcal{C}$ is open. Assume that $\omega\ge L^{-1}$ (or $\omega\ge L^{-2}$ in case the ratio of periods is generic in the sense of Theorem \ref{th.numbertheory}). Then (recall that $Q=L^{d}\sqrt{\omega}$)
    \begin{equation}\label{eq.countingbd2}\# Eq(\mathcal{C})\leq\left\{\begin{aligned}
    &L^{O(\theta)} Q^{n}, \qquad\quad\textit{if $C$ is open},\\
    & L^{d+O(\theta) }Q^{n},\qquad\textit{if $C$ is closed}.
    \end{aligned}
    \right.
    \end{equation}
    \end{prop}
    
    Steps 2-4 are devoted to the proof of Proposition \ref{prop.countingconn} using a cutting edge argument. 
    
    
    \textbf{Step 2.} Although all components can have at most two legs at the beginning, after cutting, the resulting couples can have more than two legs. In this step, we introduce a stronger version of Proposition \ref{prop.countingconn} that covers legs of more than two legs.
    
    \begin{lem}\label{prop.countingind}
    Let $\mathcal{C}$ be a connected couple with multiple legs, $n$ be the total number of nodes in $\mathcal{C}$ and $n_e$ (resp. $n_{\textit{fx}}$, $n_{\textit{fr}}$) be the total number of non-leg edges (resp. fixed legs, free legs). We fix $\sigma_{\mathfrak{n}}\in\mathbb{R}$ for each $\mathfrak{n}\in \mathcal{C}$ and $c_{\mathfrak{l}}\in \mathbb{R}$ for each fixed leg $\mathfrak{l}$. Assume that $\omega\ge L^{-1}$ (or $\omega\ge L^{-2}$ in case the ratio of periods is generic in the sense of Theorem \ref{th.numbertheory}). We assume that $n_{\textit{fx}},n_{\textit{fr}}\ne 0$. Then 
    
    \begin{equation}\label{eq.countingbd3}
    \sup_{\{c_{\mathfrak{l}}\}_{\mathfrak{l}}}\#Eq(\mathcal{C},\{c_{\mathfrak{l}}\}_{\mathfrak{l}})\leq L^{O(\theta)} Q^{\chi(\mathcal{C})} = L^{O(\theta)} Q^{n_e+n_{\textit{fr}}-n}.
    \end{equation}
    
    \end{lem}
    
    Proposition \ref{prop.countingind} will be proved in steps 3 and 4. In the rest part of this step, we show that Proposition \ref{prop.countingconn} is a corollary of Proposition \ref{prop.countingind}.
    
    Since the couples in Proposition \ref{prop.countingconn} and Proposition \ref{prop.counting} have no free leg, these two propositions are not obvious corollaries of Proposition \ref{prop.countingind}.
    
    \begin{proof}[Proof of Proposition \ref{prop.countingconn}]
    The closed couple case of Proposition \ref{prop.countingconn} can be derived from the open couple case by the following argument.
    
    Let $\mathcal{C}$ be a closed couple, choose an edge $\mathfrak{e}_{*}\in \mathcal{C}$ which connects $\mathfrak{n}_1$, $\mathfrak{n}_2$. Cut this edge into two legs $\mathfrak{l}_1$ and $\mathfrak{l}_2$, then the couple $\widetilde{\mathcal{C}}$ obtained in this way is an open couple. Therefore, we have
    
    \begin{equation}\label{eq.deriveclosedfromopen}
    \begin{split}
    \#Eq(\mathcal{C})=&\#\{k_{\mathfrak{e}}\in \mathbb{Z}^d,\ |k_{\mathfrak{e}}| \lesssim L^+,\ \forall \mathfrak{e}\in \mathcal{C}:MC_{\mathfrak{n}},\  EC_{\mathfrak{n}},\ \forall \mathfrak{n}\in \mathcal{C}\} 
    \\
    =&\#\{k_{\mathfrak{e}}\in \mathbb{Z}^d,\ |k_{\mathfrak{e}}| \lesssim L^+,\ \forall \mathfrak{e}\in \mathcal{C}:MC_{\mathfrak{n}},\  EC_{\mathfrak{n}},\ \forall \mathfrak{n}\in \mathcal{C}. MC_{n_1}(k_{\mathfrak{e}_*},\cdots),\ MC_{n_2}(k_{\mathfrak{e}_*},\cdots)\} 
    \\
    =&\#\sum_{k_{\mathfrak{e}_*}}\{k_{\mathfrak{e}}\in \mathbb{Z}^d,\ |k_{\mathfrak{e}}| \lesssim L^+,\ \forall \mathfrak{e}\in \mathcal{C}:MC_{\mathfrak{n}},\  EC_{\mathfrak{n}},\ \forall \mathfrak{n}\in \mathcal{C}. MC_{n_1}(k_{\mathfrak{l}_1},\cdots),
    \\
    &\qquad\quad MC_{n_2}(k_{\mathfrak{l}_2},\cdots),\ k_{\mathfrak{l}_1}=k_{\mathfrak{l}_2}=k_{\mathfrak{e}_*}\} 
    \\
    =&\sum_{|k_{\mathfrak{e}_*}|\lesssim 1} \#Eq(\widetilde{\mathcal{C}},\{k_{\mathfrak{e}_*},k_{\mathfrak{e}_*}\})
    \end{split}
    \end{equation}
    If we can prove the open couple case, we have $\#Eq(\widetilde{\mathcal{C}},\{k_{\mathfrak{e}_*},k_{\mathfrak{e}_*}\})\le L^{O(\theta)} Q^{n}$. Then by \eqref{eq.deriveclosedfromopen}, we can prove the closed couple case: 
    \begin{equation}
     \#Eq(\mathcal{C})=\sum_{|k_{\mathfrak{e}_*}|\lesssim 1} \#Eq(\widetilde{\mathcal{C}},\{k_{\mathfrak{e}_*},k_{\mathfrak{e}_*}\})\lesssim L^{d+O(\theta)} Q^{n}.
    \end{equation}
    
    The open couple case can be proved by Lemma \ref{lem.freeleg}.
    
    An open couple has two fixed legs $\mathfrak{l}$, $\mathfrak{l}'$ with opposite sign, so by momentum conservation equation \eqref{eq.momentumconservation}, we have $k_{\mathfrak{l}}=k_{\mathfrak{l}'}$, and $k_{\mathfrak{l}}=k_{\mathfrak{l}}'=k$ is a consequence of $k_{\mathfrak{l}}=k$. Therefore, we have 
    \begin{equation}\label{eq.stephatC}
    \begin{split}
     Eq(\mathcal{C})=&\{k_{\mathfrak{e}}\in \mathbb{Z}^d,\ |k_{\mathfrak{e}}| \lesssim L^+,\ \forall \mathfrak{e}\in \mathcal{C}:MC_{\mathfrak{n}},\  EC_{\mathfrak{n}},\ \forall \mathfrak{n}\in \mathcal{C}.\ k_{\mathfrak{l}}=k_{\mathfrak{l}}'=k.\} 
     \\
     =& \{k_{\mathfrak{e}}\in \mathbb{Z}^d,\ |k_{\mathfrak{e}}| \lesssim L^+,\ \forall \mathfrak{e}\in \mathcal{C}:MC_{\mathfrak{n}},\  EC_{\mathfrak{n}},\ \forall \mathfrak{n}\in \mathcal{C}.\ k_{\mathfrak{l}}=k.\} 
     \\
     =&Eq(\widehat{\mathcal{C}},k)
    \end{split}
    \end{equation}
    where $\widehat{\mathcal{C}}$ is the couple obtained by replacing a fixed leg with a free leg.
    
    In $\widehat{\mathcal{C}}$, $n_{\textit{fx}}=n_{\textit{fr}}=1\ne 0$, so by \eqref{eq.nnenlnle}, we have $n_e+1=2n$. Applying \eqref{eq.countingbd3}, we get
    \begin{equation}\label{eq.stephatC'}
     Eq(\mathcal{C})=Eq(\widehat{\mathcal{C}},k)\lesssim L^{O(\theta)} Q^{n_e+n_{\textit{fr}}-n} = L^{O(\theta)} Q^{n}.
    \end{equation}
    
    This proves Proposition \ref{prop.countingconn}.
    \end{proof}
    
    
    \textbf{Step 3.} In this step, we prove Proposition \ref{prop.countingind} for basic counting units.
    
    To prove Proposition \ref{prop.countingind}, we keep cutting nodes connected to legs from the couple $\mathcal{C}$. Suppose that $\mathcal{C}$ has $n$ nodes and we cut a node $\mathfrak{n}$ from $\mathcal{C}$ using cut $c(\mathfrak{n})$. Let $\mathcal{C}_{\mathfrak{n}}$ be the smaller components contains fixed leg $\mathfrak{l}$ with $1$ nodes $\mathfrak{n}$ and $\mathcal{C}' = \mathcal{C}\backslash \mathcal{C}_{\mathfrak{n}}$ be the larger components with $n-1$ nodes.
    
    As in Figure \ref{fig.3possibilities}, there are three possibilities of $\mathcal{C}_{\mathfrak{n}}$. We label them by $\mathcal{C}_{I}$, $\mathcal{C}_{II}$, $\mathcal{C}_{III}$.
    
    
    \begin{figure}[H]
     \centering
     \scalebox{0.3}{
     \begin{tikzpicture}[level distance=80pt, sibling distance=100pt]
     \node[] at (0,0) (1) {} 
     child {node[fillcirc] (11) {} 
     child {node[fillstar] (111) {}}
     child {node[fillstar] (112) {}}
     child {node[fillstar] (113) {}}
     };
     \draw[-{Stealth[length=5mm, width=3mm]}] (1) -- (11);
     \draw[-{Stealth[length=5mm, width=3mm]}] (11) -- (111);
     \draw[-{Stealth[length=5mm, width=3mm]}] (11) -- (112);
     \draw[-{Stealth[length=5mm, width=3mm]}] (11) -- (112);
     
     \node[] at (12,0) (2) {} 
     child {node[fillcirc] (21) {} 
     child {node[fillstar] (211) {}}
     child {node[fillstar] (212) {}}
     child {node[] (213) {}}
     };
     \draw[-{Stealth[length=5mm, width=3mm]}] (2) -- (21);
     \draw[-{Stealth[length=5mm, width=3mm]}] (21) -- (211);
     \draw[-{Stealth[length=5mm, width=3mm]}] (21) -- (212);
     \draw[-{Stealth[length=5mm, width=3mm]}] (21) -- (213);
    
     \node[] at (24,0) (3) {} 
     child {node[fillcirc] (31) {} 
     child {node[fillstar] (311) {}}
     child {node[] (312) {}}
     child {node[] (313) {}}
     };
     \draw[-{Stealth[length=5mm, width=3mm]}] (3) -- (31);
     \draw[-{Stealth[length=5mm, width=3mm]}] (31) -- (311);
     \draw[-{Stealth[length=5mm, width=3mm]}] (31) -- (312);
     \draw[-{Stealth[length=5mm, width=3mm]}] (31) -- (313);
     \end{tikzpicture}
     }
     \caption{Three possibilities of $\mathcal{C}_\mathfrak{l}$.}
     \label{fig.3possibilities}
    \end{figure}
     
     
    \begin{lem}\label{lem.countingbdunit}
    $\mathcal{C}_{I}$, $\mathcal{C}_{II}$, $\mathcal{C}_{III}$ satisfy the bound \eqref{eq.countingbd3} in Proposition \ref{prop.countingind}. In other words, fix $c_1$ (resp. $c_1$, $c_3$ and $c_1$, $c_2$, $c_3$) for the legs of $\mathcal{C}_{I}$ (resp. $\mathcal{C}_{II}$ and $\mathcal{C}_{III}$), then we have 
    \begin{equation}\label{eq.countingbdunit}
     \# Eq(\mathcal{C}_{I})\leq L^\theta Q^{2},\qquad \# Eq(\mathcal{C}_{II})\leq L^\theta Q,\qquad \# Eq(\mathcal{C}_{III})\leq Q^0=1.
    \end{equation}
    \end{lem}
    \begin{proof} Given $c_1$, $c_2$, $c_3$ the equation of $\mathcal{C}_{III}$ is 
     \begin{equation}
     \begin{cases}
     k_{\mathfrak{e}_1}-k_{\mathfrak{e}_2}+k_{\mathfrak{e}_3}-k_{\mathfrak{e}}=0,\ k_{\mathfrak{e}_2}=c_2,\ k_{\mathfrak{e}_3}=c_3,\ k_{\mathfrak{e}}=c_1,
     \\
     \Lambda_{k_{\mathfrak{e}_1}}-\Lambda_{k_{\mathfrak{e}_2}}+\Lambda_{k_{\mathfrak{e}_3}}-\Lambda_{k_{\mathfrak{e}}}=\sigma_{\mathfrak{n}}+O(\omega)
     \end{cases}
     \end{equation}
     It's obvious that there is at most one solution to this system of equations.
     
     Given $c_1$, the equation of $\mathcal{C}_{I}$ is 
     \begin{equation}
     \begin{cases}
     k_{\mathfrak{e}_1}-k_{\mathfrak{e}_2}+k_{\mathfrak{e}_3}-k_{\mathfrak{e}}=0,\ k_{\mathfrak{e}}=c_1,
     \\
     \Lambda_{k_{\mathfrak{e}_1}}-\Lambda_{k_{\mathfrak{e}_2}}+\Lambda_{k_{\mathfrak{e}_3}}-\Lambda_{k_{\mathfrak{e}}}=\sigma_{\mathfrak{n}}+O(\omega)
     \end{cases}
     \end{equation}
     By Theorem \ref{th.numbertheory1} or \ref{th.numbertheory} \eqref{eq.numbertheory1} or \eqref{eq.numbertheory}, corresponding to the general or generic case, the number of solutions of the above system of equations can be bounded by $ L^\theta Q^2$.
    
     Given $c_1$, $c_2$, the equation of $\mathcal{C}_{II}$ is 
     \begin{equation}
     \begin{cases}
     k_{\mathfrak{e}_1}-k_{\mathfrak{e}_2}+k_{\mathfrak{e}_3}-k_{\mathfrak{e}}=0,\ k_{\mathfrak{e}}=c_1,\ k_{\mathfrak{e}_3}=c_3,
     \\
     \Lambda_{k_{\mathfrak{e}_1}}-\Lambda_{k_{\mathfrak{e}_2}}+\Lambda_{k_{\mathfrak{e}_3}}-\Lambda_{k_{\mathfrak{e}}}=\sigma_{\mathfrak{n}}+O(\omega)
     \end{cases}
     \end{equation}
     By Theorem \ref{th.numbertheory1} or \ref{th.numbertheory} \eqref{eq.numbertheory1''} or \eqref{eq.numbertheory'}, corresponding to the general or generic case, the number of solutions of the above system of equations can be bounded by $ L^\theta Q$
     
     Therefore, we complete the proof of this lemma.
     \end{proof}
    
    \textbf{Step 4.} In this step, we apply the edge cutting argument to prove Proposition \ref{prop.countingind} by induction.
    
    If $\mathcal{C}$ has only one node ($n=1$), then $\mathcal{C}$ equals to $\mathcal{C}_{I}$, $\mathcal{C}_{II}$ or $\mathcal{C}_{III}$ and Proposition \ref{prop.countingind} in this case follows from Lemma \ref{lem.countingbdunit}.
    
    Suppose that Proposition \ref{prop.countingind} holds true for couples with the number of nodes $\le n-1$. We prove it for couples with the number of nodes $n$. 
    
    Since $n_{\textit{fx}}, n_{\textit{fr}}\ne 0$, there exists both fixed and free legs. We will cut a node $\mathfrak{n}_*\in\mathcal{C}$ which is connected to a fixed leg. Denoted by $\mathcal{N}_{leg}$ the set of all such nodes and by $\mathcal{C}_{\mathfrak{n}_*}$ the components containing $\mathfrak{n}_*$ after cutting. There are several different situations in this cutting.
    
    \textbf{Case 1.} Assume that for all $\mathfrak{n}\in \mathcal{N}_{leg}$, after cutting $\mathfrak{n}$, $\mathcal{C}\backslash \mathcal{C}_{\mathfrak{n}}$ is still connected.
    
    \textbf{Case 1.1.} Assume that not all legs are connected to one node. One example of this case is shown in Figure \ref{fig.examplecase1.1}.
    
    \begin{figure}[H]
     \centering
     \scalebox{0.5}{
     \begin{tikzpicture}[level distance=80pt, sibling distance=100pt]
     \node[] at (0,0) (1) {};
     \node[fillcirc] at (3, 0) (2) {};
     \node[draw, circle, minimum size=1cm, scale=2] at (7,0) (3) {$\mathcal{C}\backslash \mathcal{C}_{\mathfrak{n}}$};
     % \node[fillcirc] at (11, 0) (4) {};
     % \node[] at (14, 0) (5) {};
     \node[fillstar] at (11, 0) (6) {};
     \draw[-{Stealth[length=5mm, width=3mm]}] (1) edge (2);
     \draw[{Stealth[length=5mm, width=3mm]}-] (2) edge (3);
     \draw[-{Stealth[length=5mm, width=3mm]}, bend left =40] (2) edge (3);
     \draw[-{Stealth[length=5mm, width=3mm]}, bend right =40] (2) edge (3);
     % \draw[{Stealth[length=5mm, width=3mm]}-] (3) edge (4);
     % \draw[-{Stealth[length=5mm, width=3mm]}, bend left =40] (3) edge (4);
     % \draw[-{Stealth[length=5mm, width=3mm]}, bend right =40] (3) edge (4);
     % \draw[-{Stealth[length=5mm, width=3mm]}] (4) edge (5);
     \draw[-{Stealth[length=5mm, width=3mm]}] (3) edge (6);
    
     \node[scale=1.5] at ($(2)+(0,-0.5)$) {$\mathfrak{n}_*$};
     \node[scale =2] at ($(2)+(1.8,1.1)$) {$\times$};
     \node[scale =2] at ($(2)+(1.8,0)$) {$\times$};
     \node[scale =2] at ($(2)+(1.8,-1.1)$) {$\times$};
    
     
     % \node[] at (10,0) (1) {};
     % \node[fillcirc] at (10, -3) (2) {};
     % \node[draw, circle, minimum size=1cm, scale=1.5] at (7,-7) (3) {$(\mathcal{C}(k)')_1$};
     % \node[draw, circle, minimum size=1cm, scale=1.5] at (13,-7) (4) {$(\mathcal{C}(k)')_2$};
     % \draw[-{Stealth[length=5mm, width=3mm]}] (1) edge (2);
     % \draw[-{Stealth[length=5mm, width=3mm]}] (2) edge (3);
     % \draw[-{Stealth[length=5mm, width=3mm]}] (2) edge (4);
     \end{tikzpicture}
     }
     \caption{Example of case 1.1.}
     \label{fig.examplecase1.1}
    \end{figure}
    
    In this case, we have the following claim.
    
    \textit{Claim.} There exists a node $\mathfrak{n}_*\in \mathcal{N}_{leg}$ such that not all free legs are connected to this node.
    \begin{proof}
    If there is only one node in $\mathfrak{n}\in \mathcal{N}_{leg}$, then all fixed legs are connected to this node. By this, we know that not all free legs are connected to this node because, by the assumption of case 1.1, not all legs are connected to one node. We choose $\mathfrak{n}_*$ to be this node. 
     
    If there are two nodes $\mathfrak{n}_1, \mathfrak{n}_2\in \mathcal{N}_{leg}$, and if all free legs are connected to one of them, then choose $\mathfrak{n}_*$ to be the other one. Otherwise, choose $\mathfrak{n}_*$ to either one of $\mathfrak{n}_1, \mathfrak{n}_2$ works.
    \end{proof}
    
    We cut the node $\mathfrak{n}_*$ in this claim from $\mathcal{C}$. Let $\mathcal{C}_{\mathfrak{n}}$, $\mathcal{C}_1=\mathcal{C}\backslash \mathcal{C}_{\mathfrak{n}}$ be two components after cutting. As in Lemma \ref{lem.Eq(C)cutting}, let $\{\mathfrak{e}_{i}^{(1)}\}\subseteq \mathcal{C}_{\mathfrak{n}}$ be free legs and $\{\mathfrak{e}_{i}^{(2)}\}\subseteq \mathcal{C}_1$ be fixed legs, then applying Lemma \ref{lem.Eq(C)cutting} gives
    \begin{equation}
    \begin{split}
     \sup_{\{c_{\mathfrak{l}}\}_{\mathfrak{l}}}\#Eq(\mathcal{C},\{c_{\mathfrak{l}}\}_{\mathfrak{l}})\le&
     \sup_{\{c_{\mathfrak{l}_1}\}_{\mathfrak{l}_1\in \text{leg}(\mathcal{C}_{\mathfrak{n}})} } \# Eq(\mathcal{C}_{\mathfrak{n}},\{c_{\mathfrak{l}_1}\}) \sup_{\{c_{\mathfrak{l}_2}\}_{\mathfrak{l}_2\in \text{leg}(\mathcal{C}_1)} }\# Eq(\mathcal{C}_1, \{c_{\mathfrak{l}_2}\})
     \\
     \lesssim& L^{\theta} Q^{\chi(\mathcal{C}_{\mathfrak{n}})} L^{O(\theta)} Q^{\chi(\mathcal{C}_1)} = L^{O(\theta)} Q^{\chi(\mathcal{C}_{\mathfrak{n}})+\chi(\mathcal{C}_1)} \\
     =& L^{O(\theta)} Q^{\chi(\mathcal{C})}.
    \end{split}
    \end{equation}
    Here the second inequality follows from the induction assumption and Lemma \ref{lem.countingbdunit}. Because it can be verified that in both $\mathcal{C}_{\mathfrak{n}}$ and $\mathcal{C}_1$, $n_{\textit{fx}}$ and $n_{\textit{fr}}$ are not equal to $0$, the induction assumption and Lemma \ref{lem.countingbdunit} are applicable.
    
    Now we provide a detailed analysis of why in both $\mathcal{C}_{\mathfrak{n}}$ and $\mathcal{C}_1$, $n_{\textit{fx}}$ and $n_{\textit{fr}}$ are not equal to $0$. 
    
    $\mathcal{C}_{\mathfrak{n}}$ contains a fixed leg because $\mathfrak{n}_*\in \mathcal{N}_{leg}$ and by definition of $\mathcal{N}_{leg}$, $\mathfrak{n}_*$ is connected to a fixed leg. $\mathcal{C}_{\mathfrak{n}}$ contains a free leg because it contains free legs $\{\mathfrak{e}_{i}^{(1)}\}\subseteq \mathcal{C}_{\mathfrak{n}}$.
    
    By the claim, since not all free legs are connected to $\mathfrak{n}_*$, $\mathcal{C}_1$ contains a free leg. $\mathcal{C}_1$ contains a fixed leg because it contains fixed legs $\{\mathfrak{e}_{i}^{(2)}\}\subseteq \mathcal{C}_1$.
    
    \textbf{Case 1.2.} Assume that all legs are connected to one node $\mathfrak{n}_*$. One example of this case is shown in Figure \ref{fig.examplecase1.2}
    
    \begin{figure}[H]
     \centering
     \scalebox{0.4}{
     \begin{tikzpicture}[level distance=80pt, sibling distance=100pt]
     \node[] at (-2,0) (0) {};
     \node[fillstar] at (1.8,-0.2) (1) {};
     \node[fillcirc] at (0, -3) (2) {};
     \node[scale = 1.3, draw, circle, minimum size=1cm, scale=2] at (0,-7) (3) {$\mathcal{C}_1$}; 
     \draw[-{Stealth[length=5mm, width=3mm]}] (0) edge (2);
     \draw[{Stealth[length=5mm, width=3mm]}-] (1) edge (2);
     \draw[{Stealth[length=5mm, width=3mm]}-, bend left = 40] (2) edge (3);
     \draw[-{Stealth[length=5mm, width=3mm]}, bend right = 40] (2) edge (3);
     \node[scale =3] at (1.1,-4.5) {$\times$};
     \node[scale =3] at (-1.1,-4.5) {$\times$};
     \end{tikzpicture}
     }
     \caption{Example of case 1.2.}
     \label{fig.examplecase1.2}
    \end{figure}
    
    We cut $\mathfrak{n}_*$ from $\mathcal{C}$. Let $\mathcal{C}_{\mathfrak{n}}$, $\mathcal{C}_1=\mathcal{C}\backslash \mathcal{C}_{\mathfrak{n}}$ be two components after cutting.
    
    By \eqref{eq.nnenlnle}, we know that $n_{\textit{fx}}+n_{\textit{fr}}$ is an even number, so it must equal to $2$. Then we know that $n_{\textit{fx}}=n_{\textit{fr}}=1$ and $\mathcal{C}$ has one fixed leg $\mathfrak{l}$ and one free leg $\mathfrak{l}\mathfrak{e}$. The other two edges $\mathfrak{e}', \mathfrak{e}''$ from $\mathfrak{n}_*$ should be connected to $\mathcal{C}_1$.
    
    Unlike case 1.1 we cannot simply apply induction assumption to bound $\# Eq(\mathcal{C}_1, \{c_{\mathfrak{l}_2}\})$. Because in this case $\mathcal{C}_1$ does not have free legs ($n_{\textit{fx}}=2$, $n_{\textit{fr}}=0$). See Figure \ref{fig.examplecase1.2}.
    
    As in Lemma \ref{lem.Eq(C)cutting}, $\mathfrak{e}', \mathfrak{e}''$ are cut into $\mathfrak{e}'^{(1)}, \mathfrak{e}''^{(1)}$
    and $\mathfrak{e}'^{(2)}, \mathfrak{e}''^{(2)}$. $\mathfrak{e}'^{(1)}, \mathfrak{e}''^{(1)}\in \mathcal{C}_{\mathfrak{n}}$ are free legs and $\mathfrak{e}'^{(2)}, \mathfrak{e}''^{(2)}\in \mathcal{C}_1$ are fixed legs, then applying Lemma \ref{lem.Eq(C)cutting} \eqref{eq.Eq(C)cutting} gives
    \begin{equation}
     Eq(\mathcal{C},c_{\mathfrak{l}})=\left\{(k_{\mathfrak{e}_1},k_{\mathfrak{e}_{2}}):\ k_{\mathfrak{e}_1}\in Eq(\mathcal{C}_{\mathfrak{n}},c_{\mathfrak{l}}),\  k_{\mathfrak{e}_{2}}\in Eq\left(\mathcal{C}_1, \left\{k_{\mathfrak{e}'^{(1)}}, k_{\mathfrak{e}''^{(1)}}\right\}_{i}\right)\right\}.
    \end{equation}
    
    By \eqref{eq.momentumconservation}, we know that $k_{\mathfrak{e}'^{(1)}}=k_{\mathfrak{e}''^{(1)}}$. Therefore, we have 
    \begin{equation}\label{eq.case1.2expand}
     Eq(\mathcal{C},c_{\mathfrak{l}})=\left\{(k_{\mathfrak{e}_1},k_{\mathfrak{e}_{2}}):\ k_{\mathfrak{e}_1}\in Eq(\mathcal{C}_{\mathfrak{n}},c_{\mathfrak{l}})\cap \{k_{\mathfrak{e}'^{(1)}}=k_{\mathfrak{e}''^{(1)}}\},\  k_{\mathfrak{e}_{2}}\in Eq\left(\mathcal{C}_1, \left\{k_{\mathfrak{e}'^{(1)}}, k_{\mathfrak{e}''^{(1)}}\right\}_{i}\right)\right\}.
    \end{equation}
    
    By definition, we know that 
    \begin{equation}
    \begin{split}
     Eq(\mathcal{C}_{\mathfrak{n}},c_{\mathfrak{l}})=&\{(k_{\mathfrak{l}}, k_{\mathfrak{l}\mathfrak{f}}, k_{\mathfrak{e}'}, k_{\mathfrak{e}''}): |k_{\mathfrak{l}}|, |k_{\mathfrak{l}\mathfrak{f}}|, |k_{\mathfrak{e}'}|, |k_{\mathfrak{e}''}| \lesssim L^+,\ MC_{\mathfrak{n}_*},\  EC_{\mathfrak{n}_*},\ k_{\mathfrak{l}}=c_{\mathfrak{l}}\}
     \\
     =&\{(k_{\mathfrak{l}}, k_{\mathfrak{l}\mathfrak{f}}, k_{\mathfrak{e}'}, k_{\mathfrak{e}''}): |k_{\mathfrak{l}\mathfrak{f}}|, |k_{\mathfrak{e}'}|, |k_{\mathfrak{e}''}| \lesssim L^+,
     \\
     &\ k_{\mathfrak{e}'}-k_{\mathfrak{e}''}+k_{\mathfrak{l}\mathfrak{f}}-k_{\mathfrak{l}}=0,\ k_{\mathfrak{l}\mathfrak{f}}\ne k_{\mathfrak{e}'}\ne k_{\mathfrak{e}''},\ \text{or}\ k_{\mathfrak{e}'}= k_{\mathfrak{e}''}=k_{\mathfrak{l}\mathfrak{f}}=k_{\mathfrak{l}}
     \\
     &\  \Lambda_{k_{\mathfrak{e}'}}-\Lambda_{k_{\mathfrak{e}''}}+\Lambda_{k_{\mathfrak{l}\mathfrak{f}}}-\Lambda_{k_{\mathfrak{l}}} =\sigma_{\mathfrak{n}}+O(\omega),\ k_{\mathfrak{l}}=c_{\mathfrak{l}}\}
    \end{split}
    \end{equation}
    Therefore, we have
    \begin{equation}
     Eq(\mathcal{C}_{\mathfrak{n}},c_{\mathfrak{l}})\cap \{k_{\mathfrak{e}'^{(1)}}=k_{\mathfrak{e}''^{(1)}}\}=\{(k_{\mathfrak{l}}, k_{\mathfrak{l}\mathfrak{f}}, k_{\mathfrak{e}'}, k_{\mathfrak{e}''}):k_{\mathfrak{e}'}=k_{\mathfrak{e}''}=k_{\mathfrak{l}\mathfrak{f}}=k_{\mathfrak{l}}=c_{\mathfrak{l}}\}.
    \end{equation}
    
    By \eqref{eq.case1.2expand}, we have
    \begin{equation}
     \#Eq(\mathcal{C},c_{\mathfrak{l}})=\#Eq(\mathcal{C}_1, \{c_{\mathfrak{l}}, c_{\mathfrak{l}}\}).
    \end{equation}
    
    Let $\widehat{\mathcal{C}_1}$ be the couple obtained by replacing a fixed leg with a free leg in $\mathcal{C}_1$. Using a similar argument to \eqref{eq.stephatC} and \eqref{eq.stephatC'} and then applying the induction assumption, we get
    \begin{equation}\label{eq.case1.2expand'}
    \begin{split}
     \#Eq(\mathcal{C},c_{\mathfrak{l}})=&\#Eq(\mathcal{C}_1, \{c_{\mathfrak{l}}, c_{\mathfrak{l}}\})=\#Eq(\widehat{\mathcal{C}_1}, \{c_{\mathfrak{l}}\})
     \\
     \lesssim& L^{O(\theta)} Q^{\chi(\widehat{\mathcal{C}_1})}=L^{O(\theta)} Q^{\chi(\mathcal{C})-1}\le L^{O(\theta)} Q^{\chi(\mathcal{C})}
    \end{split}
    \end{equation}
    Here the last step is because $\chi(\mathcal{C})=\chi(\mathcal{C}_{\mathfrak{n}})+\chi(\mathcal{C}_1)=2+\chi(\mathcal{C}_1)=1+\chi(\widehat{\mathcal{C}_1})$.
    
    \textbf{Case 2.} Assume that for for some $\mathfrak{n}_*\in \mathcal{N}_{leg}$, after cutting $\mathfrak{n}_*$, $\mathcal{C}\backslash \mathcal{C}_{\mathfrak{n}}$ has exactly three components. One example of this case is shown in Figure \ref{fig.examplecase2}
    \begin{figure}[H]
     \centering
     \scalebox{0.4}{
     \begin{tikzpicture}[level distance=80pt, sibling distance=100pt]
     \node[] at (10,0) (1) {};% ($(1)+(0, -3))$)
     \node[fillcirc] at (10,-3) (2) {};
     \node[scale = 1.7, draw, circle, minimum size=1cm, scale=1.5] at (6,-7) (3) {$\mathcal{C}_1$};
     \node[scale = 1.7, draw, circle, minimum size=1cm, scale=1.5] at (10,-7) (5) {$\mathcal{C}_2$};
     \node[scale = 1.7, draw, circle, minimum size=1cm, scale=1.5] at (14,-7) (4) {$\mathcal{C}_3$};
     \draw[-{Stealth[length=5mm, width=3mm]}] (1) edge (2);
     \draw[-{Stealth[length=5mm, width=3mm]}] (2) edge (3);
     \draw[-{Stealth[length=5mm, width=3mm]}] (2) edge (4);
     \draw[{Stealth[length=5mm, width=3mm]}-] (2) edge (5);
    
     \node[scale =3] at (8.5,-4.7) {$\times_1$};
     \node[scale =3] at (10.2,-4.7) {$\times_2$};
     \node[scale =3] at (11.83,-4.7) {$\times_2$};
     \end{tikzpicture}
     }
     \caption{Example of case 2. $\times_1$ and $\times_2$ indicate the edges cut in the first and second time respectively.}
     \label{fig.examplecase2}
    \end{figure}
    
    We cut $\mathfrak{n}_*$ from $\mathcal{C}$. Let $\mathcal{C}_1$, $\mathcal{C}_2$, $\mathcal{C}_3$ be the three components of $\mathcal{C}\backslash \mathcal{C}_{\mathfrak{n}}$ after cutting.
    
    Since $\mathfrak{n}_*\in \mathcal{N}_{leg}$, there should be a fixed leg $\mathfrak{l}$ that is connected to $\mathfrak{n}_*$. The other three edges $\mathfrak{e}'$, $\mathfrak{e}''$, $\mathfrak{e}'''$ from $\mathfrak{n}_*$ should be connected to $\mathcal{C}_1$, $\mathcal{C}_2$, $\mathcal{C}_3$ respectively. 
    
    Since there exists at least one free leg in $\mathcal{C}$ ($n_{\textit{fr}}\ne 0$), at least one of $\mathcal{C}_1$, $\mathcal{C}_2$, $\mathcal{C}_3$ contain a free leg. Without loss of generality let us assume that $\mathcal{C}_1$ contains a free leg.
    
    Let us first cut $\mathfrak{e}'$ into $\mathfrak{e}'^{(1)}\in \mathcal{C}\backslash\mathcal{C}_1=\mathcal{C}_{\mathfrak{n}_*}\cup \mathcal{C}_2\cup \mathcal{C}_3$ and $\mathfrak{e}'^{(2)}\in \mathcal{C}_1$. Let $\mathfrak{e}'^{(1)}$ be a free leg and $\mathfrak{e}'^{(2)}$ be a fixed leg. Therefore, $\mathcal{C}_1$ contains both fixed and free legs. Applying Lemma \ref{lem.Eq(C)cutting} gives
    \begin{equation}\label{eq.case2expand}
    \begin{split}
     \sup_{\{c_{\mathfrak{l}}\}_{\mathfrak{l}}}\#Eq(\mathcal{C},\{c_{\mathfrak{l}}\}_{\mathfrak{l}})\le&
     \sup_{\{c_{\mathfrak{l}_1}\}_{\mathfrak{l}_1\in \text{leg}(\mathcal{C}\backslash\mathcal{C}_1)} } \# Eq(\mathcal{C}\backslash\mathcal{C}_1,\{c_{\mathfrak{l}_1}\}) \sup_{\{c_{\mathfrak{l}_2}\}_{\mathfrak{l}_2\in \text{leg}(\mathcal{C}_1)} }\# Eq(\mathcal{C}_{1}, \{c_{\mathfrak{l}_2}\})
     \\
     \lesssim& L^{O(\theta)} Q^{\chi(\mathcal{C}_1)}\sup_{\{c_{\mathfrak{l}_1}\}_{\mathfrak{l}_1\in \text{leg}(\mathcal{C}\backslash\mathcal{C}_1)} } \# Eq(\mathcal{C}\backslash\mathcal{C}_1,\{c_{\mathfrak{l}_1}\}).
    \end{split}
    \end{equation}
    Here in the second inequality, we can apply the induction assumption to $\mathcal{C}_1$ because it contains both fixed and free legs.
    
    Then we cut $\mathfrak{e}''$ into $\mathfrak{e}''^{(1)}\in \mathcal{C}_{\mathfrak{n}_*}\cup \mathcal{C}_3$ and $\mathfrak{e}''^{(2)}\in \mathcal{C}_2$. Let $\mathfrak{e}''^{(1)}$ be a free leg (or fix leg) and $\mathfrak{e}''^{(2)}$ be a fixed leg (or a free leg) if $\mathcal{C}_2$ contains a free leg (or a fixed leg). Therefore, $\mathcal{C}_2$ contains both fixed and free legs. Applying Lemma \ref{lem.Eq(C)cutting} and \eqref{eq.case2expand} gives
    \begin{equation}\label{eq.case2expand'}
    \begin{split}
     &\sup_{\{c_{\mathfrak{l}}\}_{\mathfrak{l}}}\#Eq(\mathcal{C},\{c_{\mathfrak{l}}\}_{\mathfrak{l}})
     \lesssim L^{O(\theta)} Q^{\chi(\mathcal{C}_1)}\sup_{\{c_{\mathfrak{l}_1}\}_{\mathfrak{l}_1\in \text{leg}(\mathcal{C}\backslash\mathcal{C}_1)} } \# Eq(\mathcal{C}\backslash\mathcal{C}_1,\{c_{\mathfrak{l}_1}\})
     \\
     \lesssim& L^{O(\theta)} Q^{\chi(\mathcal{C}_1)}\sup_{\{c_{\mathfrak{l}_1}\}_{\mathfrak{l}_1\in \text{leg}(\mathcal{C}_{\mathfrak{n}_*}\cup \mathcal{C}_3)} } \# Eq(\mathcal{C}_{\mathfrak{n}_*}\cup \mathcal{C}_3,\{c_{\mathfrak{l}_1}\}) \sup_{\{c_{\mathfrak{l}_2}\}_{\mathfrak{l}_2\in \text{leg}(\mathcal{C}_2)} }\# Eq(\mathcal{C}_{2}, \{c_{\mathfrak{l}_2}\})
     \\
     \lesssim& L^\theta Q^{\chi(\mathcal{C}_1)+\chi(\mathcal{C}_2)}\sup_{\{c_{\mathfrak{l}_1}\}_{\mathfrak{l}_1\in \text{leg}(\mathcal{C}_{\mathfrak{n}_*}\cup \mathcal{C}_3)} } \# Eq(\mathcal{C}_{\mathfrak{n}_*}\cup \mathcal{C}_3,\{c_{\mathfrak{l}_1}\}) 
    \end{split}
    \end{equation}
    Here in the third inequality, we can apply the induction assumption to $\mathcal{C}_2$ because it contains both fixed and free legs.
    
    The argument for $\mathcal{C}_3$ is similar to that of $\mathcal{C}_2$. We cut $\mathfrak{e}'''$ into $\mathfrak{e}'''^{(1)}\in \mathcal{C}_{\mathfrak{n}_*}$ and $\mathfrak{e}'''^{(2)}\in \mathcal{C}_3$. Let $\mathfrak{e}'''^{(1)}$ be a free leg (or a fixed leg) and $\mathfrak{e}'''^{(2)}$ be a fixed leg (or a free leg) if $\mathcal{C}_3$ contains a free leg (or a fixed leg). Therefore, $\mathcal{C}_3$ contains both fixed and free legs. Applying Lemma \ref{lem.Eq(C)cutting} and \eqref{eq.case2expand'} gives
    \begin{equation}
    \begin{split}
     \sup_{\{c_{\mathfrak{l}}\}_{\mathfrak{l}}}\#Eq(\mathcal{C},\{c_{\mathfrak{l}}\}_{\mathfrak{l}})
     \lesssim& L^{O(\theta)} Q^{\chi(\mathcal{C}_1)+\chi(\mathcal{C}_2)}\sup_{\{c_{\mathfrak{l}_1}\}_{\mathfrak{l}_1\in \text{leg}(\mathcal{C}_{\mathfrak{n}_*}\cup \mathcal{C}_3)} } \# Eq(\mathcal{C}_{\mathfrak{n}_*}\cup \mathcal{C}_3,\{c_{\mathfrak{l}_1}\}) 
     \\
     \lesssim& L^{O(\theta)} Q^{\chi(\mathcal{C}_1)+\chi(\mathcal{C}_2)}\sup_{\{c_{\mathfrak{l}_1}\}_{\mathfrak{l}_1\in \text{leg}(\mathcal{C}_{\mathfrak{n}_*}} } \# Eq(\mathcal{C}_{\mathfrak{n}_*},\{c_{\mathfrak{l}_1}\}) \sup_{\{c_{\mathfrak{l}_2}\}_{\mathfrak{l}_2\in \text{leg}(\mathcal{C}_3)} }\# Eq(\mathcal{C}_{3}, \{c_{\mathfrak{l}_2}\})
     \\
     \lesssim& L^{O(\theta)} Q^{\chi(\mathcal{C}_1)+\chi(\mathcal{C}_2)} L^{O(\theta)} Q^{\chi(\mathcal{C}_{\mathfrak{n}_*})} L^\theta Q^{\chi(\mathcal{C}_3)} 
    \end{split}
    \end{equation}
    Here in the third inequality, we can apply Lemma \ref{lem.countingbdunit} and the induction assumption to $\mathcal{C}_{\mathfrak{n}_*}$ and $\mathcal{C}_3$ because they contain both fixed and free legs. $\mathcal{C}_{\mathfrak{n}_*}$ contains fixed legs (or free legs) because $\mathfrak{n}_*\in\mathcal{N}_{leg}$ (or $\mathfrak{e}'^{(1)}$ is a free legs from $\mathfrak{n}_*$).
    
    \textbf{Case 3.} Assume that for all $\mathfrak{n}\in \mathcal{N}_{leg}$, after cutting $\mathfrak{n}$, $\mathcal{C}\backslash \mathcal{C}_{\mathfrak{n}}$ has exactly two components.
    
    Choose arbitrarily $\mathfrak{n}_*\in \mathcal{N}_{leg}$. We cut $\mathfrak{n}_*$ from $\mathcal{C}$. Let $\mathcal{C}_1$, $\mathcal{C}_2$ be the three components of $\mathcal{C}\backslash \mathcal{C}_{\mathfrak{n}}$ after cutting. Since $\mathfrak{n}_*\in \mathcal{N}_{leg}$, there should be a fixed leg $\mathfrak{l}$ that is connected to $\mathfrak{n}_*$. There are several possibilities of the distribution of other three edges $\mathfrak{e}'$, $\mathfrak{e}''$, $\mathfrak{e}'''$ from $\mathfrak{n}_*$. 
    
    \textbf{Case 3.1.} One of $\mathfrak{e}'$, $\mathfrak{e}''$, $\mathfrak{e}'''$ connects $\mathfrak{n}_*$ and $\mathcal{C}_1$ (w.l.o.g. assume that it's $\mathfrak{e}'$), another one connects $\mathfrak{n}_*$ and $\mathcal{C}_2$ (w.l.o.g. assume that it's $\mathfrak{e}''$) and the last one is a leg (w.l.o.g. assume that it's $\mathfrak{e}'''$). One example of this case is shown in Figure \ref{fig.examplecase3.1}
    \begin{figure}[H]
     \centering
     \scalebox{0.4}{
     \begin{tikzpicture}[level distance=80pt, sibling distance=100pt]
     \node[] at (-2.5,0) (0) {};
     \node[fillstar] at (2.5,0) (1) {};
     \node[fillcirc] at (0, -3) (2) {};
     \node[scale = 1.3, draw, circle, minimum size=1cm, scale=2] at (-3,-7) (3) {$\mathcal{C}_1$};
     \node[scale = 1.3, draw, circle, minimum size=1cm, scale=2] at (3,-7) (4) {$\mathcal{C}_2$};
     \draw[-{Stealth[length=5mm, width=3mm]}] (0) edge (2);
     \draw[{Stealth[length=5mm, width=3mm]}-] (1) edge (2);
     \draw[-{Stealth[length=5mm, width=3mm]}] (2) edge (3);
     \draw[-{Stealth[length=5mm, width=3mm]}] (2) edge (4);
     \node[scale =3, rotate = 44] at (-1.18,-4.5) {$\times$};
     \node[scale =3, rotate = 44] at (1.18,-4.5) {$\times$};
     \end{tikzpicture}
     }
     \caption{Example of case 3.1.}
     \label{fig.examplecase3.1}
    \end{figure}
     
    \textbf{Case 3.1.1.} Assume that $\mathfrak{e}'''$ is an fixed leg. Then one of $\mathcal{C}_1$ and $\mathcal{C}_2$ contains a free leg (w.l.o.g. assume that it's $\mathcal{C}_1$). 
    
    We apply the same argument as case 2 in this case. Cut $\mathfrak{e}'$ into $\mathfrak{e}'^{(1)}\in \mathcal{C}_{\mathfrak{n}_*}\cup \mathcal{C}_2$, and $\mathfrak{e}'^{(2)}\in \mathcal{C}_1$ and then cut $\mathfrak{e}''$ into $\mathfrak{e}''^{(1)}\in \mathcal{C}_{\mathfrak{n}_*}$ and $\mathfrak{e}''^{(2)}\in \mathcal{C}_2$. Let $\mathfrak{e}'^{(1)}$ be a free leg and $\mathfrak{e}'^{(2)}$ be a fixed leg. Let $\mathfrak{e}''^{(1)}$ be a free leg (or a fixed leg) and $\mathfrak{e}''^{(2)}$ be a fixed leg (or a free leg) if $\mathcal{C}_2$ contains a free leg (or a fixed leg). Cutting in this way, all couples contain both fixed and free legs.
    
    As in case 2, we have
    \begin{equation}
    \begin{split}
     \sup_{\{c_{\mathfrak{l}}\}_{\mathfrak{l}}}\#Eq(\mathcal{C},\{c_{\mathfrak{l}}\}_{\mathfrak{l}})
     \le&
     \sup_{\{c_{\mathfrak{l}_1}\}_{\mathfrak{l}_1\in \text{leg}(\mathcal{C}_{\mathfrak{n}_*}\cup \mathcal{C}_2)} } \# Eq(\mathcal{C}_{\mathfrak{n}_*}\cup \mathcal{C}_2,\{c_{\mathfrak{l}_1}\}) \sup_{\{c_{\mathfrak{l}_2}\}_{\mathfrak{l}_2\in \text{leg}(\mathcal{C}_1)} }\# Eq(\mathcal{C}_{1}, \{c_{\mathfrak{l}_2}\})
     \\
     \lesssim& L^{O(\theta)} Q^{\chi(\mathcal{C}_1)}\sup_{\{c_{\mathfrak{l}_1}\}_{\mathfrak{l}_1\in \text{leg}(\mathcal{C}_{\mathfrak{n}_*}\cup \mathcal{C}_2)} } \# Eq(\mathcal{C}_{\mathfrak{n}_*}\cup \mathcal{C}_2,\{c_{\mathfrak{l}_1}\})
     \\
     \lesssim& L^{O(\theta)} Q^{\chi(\mathcal{C}_1)} \sup_{\{c_{\mathfrak{l}_1}\}_{\mathfrak{l}_1\in \text{leg}(\mathcal{C}_{\mathfrak{n}_*})} } \# Eq(\mathcal{C}_{\mathfrak{n}_*},\{c_{\mathfrak{l}_1}\}) \sup_{\{c_{\mathfrak{l}_2}\}_{\mathfrak{l}_2\in \text{leg}(\mathcal{C}_2)} }\# Eq(\mathcal{C}_{2}, \{c_{\mathfrak{l}_2}\})
     \\
     \lesssim& L^{O(\theta)} Q^{\chi(\mathcal{C}_1)+\chi(\mathcal{C}_2)+\chi(\mathcal{C}_{\mathfrak{n}_*})}=L^{O(\theta)} Q^{\chi(\mathcal{C})}.
    \end{split}
    \end{equation}
    
    \textbf{Case 3.1.2.} Assume that $\mathfrak{e}'''$ is an free leg. 
    
    We apply the same argument as case 2 and case 3.1.1 in this case. Cut $\mathfrak{e}'$ into $\mathfrak{e}'^{(1)}\in \mathcal{C}_{\mathfrak{n}_*}\cup \mathcal{C}_2$ and $\mathfrak{e}'^{(2)}\in \mathcal{C}_1$, and then cut $\mathfrak{e}''$ into $\mathfrak{e}''^{(1)}\in \mathcal{C}_{\mathfrak{n}_*}$ and $\mathfrak{e}''^{(2)}\in \mathcal{C}_2$. Let $\mathfrak{e}'^{(1)}$ be a free leg (or a fixed leg) and $\mathfrak{e}'^{(2)}$ be a fixed leg (or a free leg) if $\mathcal{C}_1$ contains a free leg (or a fixed leg). Let $\mathfrak{e}''^{(1)}$ be a free leg (or a fixed leg) and $\mathfrak{e}''^{(2)}$ be a fixed leg (or a free leg) if $\mathcal{C}_2$ contains a free leg (or a fixed leg). Cutting in this way, all couples contain both fixed and free legs.
    
    As in case 3.1.1, we have
    \begin{equation}
    \begin{split}
     \sup_{\{c_{\mathfrak{l}}\}_{\mathfrak{l}}}\#Eq(\mathcal{C},\{c_{\mathfrak{l}}\}_{\mathfrak{l}})
     \le&
     \sup_{\{c_{\mathfrak{l}_1}\}_{\mathfrak{l}_1\in \text{leg}(\mathcal{C}_{\mathfrak{n}_*}\cup \mathcal{C}_2)} } \# Eq(\mathcal{C}_{\mathfrak{n}_*}\cup \mathcal{C}_2,\{c_{\mathfrak{l}_1}\}) \sup_{\{c_{\mathfrak{l}_2}\}_{\mathfrak{l}_2\in \text{leg}(\mathcal{C}_1)} }\# Eq(\mathcal{C}_{1}, \{c_{\mathfrak{l}_2}\})
     \\
     \lesssim& L^{O(\theta)} Q^{\chi(\mathcal{C}_1)}\sup_{\{c_{\mathfrak{l}_1}\}_{\mathfrak{l}_1\in \text{leg}(\mathcal{C}_{\mathfrak{n}_*}\cup \mathcal{C}_2)} } \# Eq(\mathcal{C}_{\mathfrak{n}_*}\cup \mathcal{C}_2,\{c_{\mathfrak{l}_1}\})
     \\
     \lesssim& L^{O(\theta)} Q^{\chi(\mathcal{C}_1)} \sup_{\{c_{\mathfrak{l}_1}\}_{\mathfrak{l}_1\in \text{leg}(\mathcal{C}_{\mathfrak{n}_*})} } \# Eq(\mathcal{C}_{\mathfrak{n}_*},\{c_{\mathfrak{l}_1}\}) \sup_{\{c_{\mathfrak{l}_2}\}_{\mathfrak{l}_2\in \text{leg}(\mathcal{C}_2)} }\# Eq(\mathcal{C}_{2}, \{c_{\mathfrak{l}_2}\})
     \\
     \lesssim& L^{O(\theta)} Q^{\chi(\mathcal{C}_1)+\chi(\mathcal{C}_2)+\chi(\mathcal{C}_{\mathfrak{n}_*})}=L^{O(\theta)} Q^{\chi(\mathcal{C})}.
    \end{split}
    \end{equation}
    
    \textbf{Case 3.2.} Two of $\mathfrak{e}'$, $\mathfrak{e}''$, $\mathfrak{e}'''$ (w.l.o.g. assume that they are $\mathfrak{e}'$, $\mathfrak{e}''$) connects $\mathfrak{n}_*$ and one of the two components (w.l.o.g. assume that it's $\mathcal{C}_1$). The last one (w.l.o.g. assume that it's $\mathfrak{e}'''$) connects $\mathfrak{n}_*$ and the last components (w.l.o.g. assume that it's $\mathcal{C}_2$). In this case $\mathcal{C}_2$ must contain a leg by \eqref{eq.nnenlnle}.
    
    One example of this case is shown in Figure \ref{fig.examplecase3.2}
    \begin{figure}[H]
     \centering
     \scalebox{0.4}{
     \begin{tikzpicture}[level distance=80pt, sibling distance=100pt]
     \node[] at (0,0) (1) {};
     \node[fillcirc] at (0, -3) (2) {};
     \node[scale = 1.3, draw, circle, minimum size=1cm, scale=2] at (-3,-7) (3) {$\mathcal{C}_1$};
     \node[scale = 1.3, draw, circle, minimum size=1cm, scale=2] at (3,-7) (4) {$\mathcal{C}_2$};
     \draw[-{Stealth[length=5mm, width=3mm]}] (1) edge (2);
     \draw[{Stealth[length=5mm, width=3mm]}-, bend left = 30] (2) edge (3);
     \draw[-{Stealth[length=5mm, width=3mm]}, bend right = 30] (2) edge (3);
     \draw[-{Stealth[length=5mm, width=3mm]}] (2) edge (4);
     \node[scale =3, rotate = 44] at (-0.4,-5.2) {$\times$};
     \node[scale =3, rotate = 44] at (-1.8,-3.8) {$\times$};
     \node[scale =3, rotate = 44] at (1.18,-4.5) {$\times$};
     \end{tikzpicture}
     }
     \caption{Example of case 3.2.}
     \label{fig.examplecase3.2}
    \end{figure}
     
    \textbf{Case 3.2.1.} Assume that $\mathcal{C}_1$ contains a leg. 
    
    We apply the same argument as case 2 or case 3.1 in this case. Cut $\mathfrak{e}'$, $\mathfrak{e}''$ into $\mathfrak{e}'^{(1)}, \mathfrak{e}''^{(1)}\in \mathcal{C}_{\mathfrak{n}_*}\cup \mathcal{C}_2$ and $\mathfrak{e}'^{(2)}, \mathfrak{e}''^{(2)}\in \mathcal{C}_1$, and then cut $\mathfrak{e}'''$ into $\mathfrak{e}'''^{(1)}\in \mathcal{C}_{\mathfrak{n}_*}$ and $\mathfrak{e}'''^{(2)}\in \mathcal{C}_2$. Let $\mathfrak{e}'^{(1)}$, $\mathfrak{e}''^{(1)}$ be free legs (or fixed legs) and $\mathfrak{e}'^{(2)}$, $\mathfrak{e}''^{(2)}$ be fixed legs (or a free legs) if $\mathcal{C}_1$ contains a free leg (or a fixed leg). Let $\mathfrak{e}'''^{(1)}$ be a free leg (or a fixed leg) and $\mathfrak{e}'''^{(2)}$ be a fixed leg (or a free leg) if $\mathcal{C}_2$ contains a free leg (or a fixed leg). Cutting in this way, all couples contain both fixed and free legs.
    
    As in case 3.1, we have
    \begin{equation}
    \begin{split}
     \sup_{\{c_{\mathfrak{l}}\}_{\mathfrak{l}}}\#Eq(\mathcal{C},\{c_{\mathfrak{l}}\}_{\mathfrak{l}})
     \le&
     \sup_{\{c_{\mathfrak{l}_1}\}_{\mathfrak{l}_1\in \text{leg}(\mathcal{C}_{\mathfrak{n}_*}\cup \mathcal{C}_2)} } \# Eq(\mathcal{C}_{\mathfrak{n}_*}\cup \mathcal{C}_2,\{c_{\mathfrak{l}_1}\}) \sup_{\{c_{\mathfrak{l}_2}\}_{\mathfrak{l}_2\in \text{leg}(\mathcal{C}_1)} }\# Eq(\mathcal{C}_{1}, \{c_{\mathfrak{l}_2}\})
     \\
     \lesssim& L^{O(\theta)} Q^{\chi(\mathcal{C}_1)}\sup_{\{c_{\mathfrak{l}_1}\}_{\mathfrak{l}_1\in \text{leg}(\mathcal{C}_{\mathfrak{n}_*}\cup \mathcal{C}_2)} } \# Eq(\mathcal{C}_{\mathfrak{n}_*}\cup \mathcal{C}_2,\{c_{\mathfrak{l}_1}\})
     \\
     \lesssim& L^{O(\theta)} Q^{\chi(\mathcal{C}_1)} \sup_{\{c_{\mathfrak{l}_1}\}_{\mathfrak{l}_1\in \text{leg}(\mathcal{C}_{\mathfrak{n}_*})} } \# Eq(\mathcal{C}_{\mathfrak{n}_*},\{c_{\mathfrak{l}_1}\}) \sup_{\{c_{\mathfrak{l}_2}\}_{\mathfrak{l}_2\in \text{leg}(\mathcal{C}_2)} }\# Eq(\mathcal{C}_{2}, \{c_{\mathfrak{l}_2}\})
     \\
     \lesssim& L^{O(\theta)} Q^{\chi(\mathcal{C}_1)+\chi(\mathcal{C}_2)+\chi(\mathcal{C}_{\mathfrak{n}_*})}=L^{O(\theta)} Q^{\chi(\mathcal{C})}.
    \end{split}
    \end{equation}
    
    \textbf{Case 3.2.2.} Assume that $\mathcal{C}_1$ contains no legs. Then $\mathcal{C}_2$ must contain free legs since $\mathcal{C}$ contains free legs. 
    
    We apply the same argument as case 2 and case 3.1. Cut $\mathfrak{e}'''$ into $\mathfrak{e}'''^{(1)}\in \mathcal{C}_{\mathfrak{n}_*}\cup \mathcal{C}_1$ and $\mathfrak{e}'''^{(2)}\in \mathcal{C}_2$. Let $\mathfrak{e}'''^{(1)}$ be a free leg and $\mathfrak{e}'''^{(2)}$ be a fixed leg. Cutting in this way, all couples contain both fixed and free legs.
    
    As in case 3.1, we have
    \begin{equation}\label{eq.case3.2.2expand}
    \begin{split}
     \sup_{\{c_{\mathfrak{l}}\}_{\mathfrak{l}}}\#Eq(\mathcal{C},\{c_{\mathfrak{l}}\}_{\mathfrak{l}})
     \le&
     \sup_{\{c_{\mathfrak{l}_1}\}_{\mathfrak{l}_1\in \text{leg}(\mathcal{C}_{\mathfrak{n}_*}\cup \mathcal{C}_1)} } \# Eq(\mathcal{C}_{\mathfrak{n}_*}\cup \mathcal{C}_1,\{c_{\mathfrak{l}_1}\}) \sup_{\{c_{\mathfrak{l}_2}\}_{\mathfrak{l}_2\in \text{leg}(\mathcal{C}_2)} }\# Eq(\mathcal{C}_{2}, \{c_{\mathfrak{l}_2}\})
     \\
     \lesssim& L^{O(\theta)} Q^{\chi(\mathcal{C}_2)}\sup_{\{c_{\mathfrak{l}_1}\}_{\mathfrak{l}_1\in \text{leg}(\mathcal{C}_{\mathfrak{n}_*}\cup \mathcal{C}_1)} } \# Eq(\mathcal{C}_{\mathfrak{n}_*}\cup \mathcal{C}_1,\{c_{\mathfrak{l}_1}\}).
    \end{split}
    \end{equation}
    
    $\mathcal{C}_{\mathfrak{n}_*}\cup \mathcal{C}_1$ has only one free leg $\mathfrak{e}'''^{(1)}$ and a fixed leg $\mathfrak{l}$ and both of these two edges are connected to $\mathfrak{n}_*$. Therefore, $\mathcal{C}_{\mathfrak{n}_*}\cup \mathcal{C}_1$ satisfies exactly the condition of case 1.2. Applying the conclusion there gives
    \begin{equation}\label{eq.case3.2.2expand'}
     \sup_{\{c_{\mathfrak{l}_1}\}_{\mathfrak{l}_1\in \text{leg}(\mathcal{C}_{\mathfrak{n}_*}\cup \mathcal{C}_1)} } \# Eq(\mathcal{C}_{\mathfrak{n}_*}\cup \mathcal{C}_1,\{c_{\mathfrak{l}_1}\})\lesssim L^{O(\theta)} Q^{\chi(\mathcal{C}_{\mathfrak{n}_*}\cup \mathcal{C}_1)}.
    \end{equation}
    
    Combine \eqref{eq.case3.2.2expand} and \eqref{eq.case3.2.2expand'}. We get 
    
    \begin{equation}
    \begin{split}
     \sup_{\{c_{\mathfrak{l}}\}_{\mathfrak{l}}}\#Eq(\mathcal{C},\{c_{\mathfrak{l}}\}_{\mathfrak{l}})
     \lesssim& L^{O(\theta)} Q^{\chi(\mathcal{C}_2)}\sup_{\{c_{\mathfrak{l}_1}\}_{\mathfrak{l}_1\in \text{leg}(\mathcal{C}_{\mathfrak{n}_*}\cup \mathcal{C}_1)} } \# Eq(\mathcal{C}_{\mathfrak{n}_*}\cup \mathcal{C}_1,\{c_{\mathfrak{l}_1}\})
     \\
     \lesssim& L^{O(\theta)} Q^{\chi(\mathcal{C}_2)+\chi(\mathcal{C}_{\mathfrak{n}_*}\cup \mathcal{C}_1)}=L^{O(\theta)} Q^{\chi(\mathcal{C})}.
    \end{split}
    \end{equation}
    
    Therefore, we complete the proof of Proposition \ref{prop.countingind} and thus the proof of Proposition \ref{prop.countingconn} and \ref{prop.counting}.
    
    \end{proof}








% \subsection{An upper bound of tree terms}\label{sec.treetermsupperbound}
% In this section, we prove the following upper bound of the variance of $\mathcal{J}_{T,k}$.

% \begin{prop}\label{prop.treetermsvariance}
% Assume that $L^{-d}\le \alpha\le L^{-2}$. For any $\theta>0$ and sufficiently large $p$, we have
% \begin{equation}
%     \sup_k\langle k\rangle^p\, \mathbb{E}|(\mathcal{J}_T)_k|^2\le L^{O(\theta)} \rho^{l(T)}.
% \end{equation}
% Here $\rho=\alpha^3t^2$.
% \end{prop}
% \begin{proof}
% By \eqref{eq.termTp} and \eqref{eq.termexp}, we know that $\mathcal{J}_{T,k}$ is a linear combination of $Term(T,p)$.
% \begin{equation}
% \begin{split}
%     \mathbb{E}|\mathcal{J}_{T,k}|^2=\left(\frac{\lambda^2}{L^{2d}}\right)^{2l(T)}
%     \sum_{p\in \mathcal{P}_F(\{k_1,\cdots, k_{2l(T)+1}, k'_1,\cdots, k'_{2l(T)+1}\})} Term(T, p).
% \end{split}
% \end{equation}

% Since $\alpha=\frac{\lambda^2}{L^{d}}$, so $\frac{\lambda^2}{L^{2d}}=\alpha L^{-d}$. Since the number of elements in $\mathcal{P}_F$ can be bounded by a constant, by Lemma \ref{lem.Tpvariance} proved below, we get
% \begin{equation}\label{eq.proptreetermsvariance1}
% \begin{split}
%     \mathbb{E}|\mathcal{J}_{T,k}|^2\lesssim (\alpha L^{-d})^{2l(T)}
%     L^{O(\theta)} Q^{n} L^{\frac{1}{2} dn_d} t^{2l(T)} \langle k\rangle^{-q}.
% \end{split}
% \end{equation}

% By definition, $n$ and $n_d$ are total number of $\bullet$ nodes and $\circ$ nodes respectively of the two paired trees. The two paired trees of the same type $T$ contains $2l(T)$ branching nodes in total (each one contains $l(T)$). We thus get $2l(T)=n+n_d$. Replacing $2l(T)$ by $n+n_d$ in \eqref{eq.proptreetermsvariance1}, we get
% \begin{equation}
% \begin{split}
%     \mathbb{E}|\mathcal{J}_{T,k}|^2\lesssim& (\alpha L^{-d})^{n+n_d}
%     L^{O(\theta)} Q^{n} L^{\frac{1}{2} dn_d} t^{n+n_d} \langle k\rangle^{-q}
%     \\
%     =& L^{O(\theta)} (\alpha L^{-d} Qt)^{n} (\alpha L^{-d}L^{\frac{1}{2} d}t)^{n_d}\langle k\rangle^{-q}
%     \\
%     =& L^{O(\theta)} (\alpha^{3/2} t)^{n} (t\alpha L^{-d/2})^{n_d}\langle k\rangle^{-q}
%     \\
%     \lesssim& L^{O(\theta)} (\alpha^{3/2} t)^{n+n_d} \langle k\rangle^{-q}= L^{O(\theta)} (\alpha^{3} t^2)^{l(T)}. \langle k\rangle^{-q}
% \end{split}
% \end{equation}
% Here in the last step, we use the fact that $t\alpha L^{-d/2}\le \alpha^{3/2} t$ if $\alpha\ge L^{-d} $.

% Therefore, we complete the proof of this proposition.
% \end{proof}


% \begin{lem}\label{lem.Tpvariance} Assume that $\alpha\le L^{-2}$ and the initial data satisfies $\sqrt{n_{\textrm{in}}(k)}\lesssim \langle k\rangle^{-q}$. Let $\mathcal{C}$ be the couple constructed from tree $T$ and pairing $p$. Then for any $\theta>0$ and sufficiently large $q$ depending on $T$, we have
% \begin{equation}
%     \sup_k\langle k\rangle^q\, |(Term(T,p))_k|\le L^{O(\theta)} Q^{n} L^{\frac{1}{2} dn_d} t^{2l(T)}.
% \end{equation}
% Here $n$, $n_d$ are defined in Proposition \ref{prop.counting}
% \end{lem}
% \begin{proof} By \eqref{eq.termTp}, we get

% \begin{equation}
% \begin{split}
%     Term(T, p)=&\sum_{k_1,\, k_2,\, \cdots,\, k_{2l(T)+1}}\sum_{k'_1,\, k'_2,\, \cdots,\, k'_{2l(T)+1}} H^T_{k_1\cdots k_{2l(T)+1}} H^{T}_{k'_1\cdots k'_{2l(T)+1}}
%     \\
%     & \delta_{p}(k_1,\cdots, k_{2l(T)+1}, k'_1,\cdots, k'_{2l(T)+1})\sqrt{n_{\textrm{in}}(k_1)}\cdots\sqrt{n_{\textrm{in}}(k'_1)}\cdots
% \end{split}
% \end{equation}

% By \eqref{eq.boundcoef}, we get
% \begin{equation}
%     H^T_{k_1\cdots k_{2l+1}}\lesssim \sum_{\{q_{\mathfrak{n}}\}_{\mathfrak{n}\in T'}\in Q}\prod_{\mathfrak{n}\in T'}\frac{t\alpha}{|q_{\mathfrak{n}}|+\alpha}\ \delta_{\cap_{\mathfrak{n}\in T'} S_{\mathfrak{n}}}.
% \end{equation}

% By \eqref{eq.q_n}, $q_{\mathfrak{n}}$ is a linear combination of $\Omega_{\mathfrak{n}}$, so there exists constants $c_{\mathfrak{n},\mathfrak{n}'}$ such that $q_{\mathfrak{n}}=\sum_{\mathfrak{n}'}c_{\mathfrak{n},\mathfrak{n}'}\Omega_{\mathfrak{n}'}$. Let $c$ be the matrix $[c_{\mathfrak{n},\mathfrak{n}'}]$ and $\mathscr{M}$ be the set of all possible such matrices, then the number of elements in $\mathscr{M}$ can be bounded by a constant. Let $c(\Omega)$ be the vector $\{\sum_{\mathfrak{n}'}c_{\mathfrak{n},\mathfrak{n}'}\Omega_{\mathfrak{n}'}\}_{\mathfrak{n}}$

% With this notation, we know that $q_{\mathfrak{n}}=c(\Omega)_{\mathfrak{n}}$ and the right hand side of above inequality becomes $\sum_{c\in \mathscr{M} }\prod_{\mathfrak{n}\in T'} \frac{t\alpha}{|c(\Omega)_{\mathfrak{n}}|+\alpha}$. Therefore, we have
% \begin{equation}
% \begin{split}
%     Term(T, p)\lesssim& \sum_{k_1,\, \cdots,\, k_{2l(T)+1},\, k'_1,\, \cdots,\, k'_{2l(T)+1}} \sum_{c\in \mathscr{M} }\prod_{\mathfrak{n}\in T_{\text{in}}}\frac{t\alpha}{|c(\Omega)_{\mathfrak{n}}|+\alpha}\ \delta_{\cap_{\mathfrak{n}\in T_{\text{in}}} S_{\mathfrak{n}}} \sum_{c'\in \mathscr{M}}\prod_{\mathfrak{n}\in T_{\text{in}}}\frac{t\alpha}{|c'(\Omega)_{\mathfrak{n}}|+\alpha}\ \delta_{\cap_{\mathfrak{n}\in T_{\text{in}}} S'_{\mathfrak{n}}} 
%     \\
%     &\delta_{p}(k_1,\cdots, k_{2l(T)+1}, k'_1,\cdots, k'_{2l(T)+1})\langle k_1\rangle^{-q}\cdots\langle k_1'\rangle^{-q}\cdots 
% \end{split}
% \end{equation}
% Here we have used the assumption that $\sqrt{n_{\textrm{in}}(k)}\le \langle k\rangle^{-q}$.

% Given tree $T$ and pairing $p$, we can construct a couple $\mathcal{C}$. Assigning a number $\sigma_{\mathfrak{n}}\in \mathbb{Z}_{\alpha^{-1}}$ for each node $\mathfrak{n}$ and a number $k\in \mathbb{Z}_{L}$ for the legs, we can define the assoicated equation $\widetilde{Eq}(\mathcal{C}, \{\sigma_{\mathfrak{n}}\}_{\mathfrak{n}},k)$
% \begin{equation}
%     \widetilde{Eq}(\mathcal{C}, \{\sigma_{\mathfrak{n}}\}_{\mathfrak{n}},k)=\{k_{\mathfrak{e}}\in \mathbb{Z}^d,\ \forall \mathfrak{e}\in \mathcal{C}:MC_{\mathfrak{n}},\  EC_{\mathfrak{n}},\ \forall \mathfrak{n}\in \mathcal{C}.\ k_{\mathfrak{l}}=k_{\mathfrak{l}}'=k.\}
% \end{equation}
% $\widetilde{Eq}(\mathcal{C}, \{\sigma_{\mathfrak{n}}\}_{\mathfrak{n}},k)$ is the same as $Eq(\mathcal{C})$ defined in \eqref{eq.diophantineeqpairedsigma'}, except that $k_{\mathfrak{e}}$ may not satisfy $|k_{\mathfrak{e}}| \lesssim L^+$ . With this definition, we know that 
% \begin{equation}
%     \sum_{k_1,\, \cdots,\, k_{2l(T)+1},\, k'_1,\, \cdots,\, k'_{2l(T)+1}}=\sum_{\sigma_{\mathfrak{n}}\in \mathbb{Z}_{\alpha^{-1}}}\sum_{\widetilde{Eq}(\mathcal{C}, \{\sigma_{\mathfrak{n}}\}_{\mathfrak{n}},k)},
% \end{equation}
% which implies that
% \begin{equation}
%     Term(T, p)\lesssim \sum_{\sigma_{\mathfrak{n}}\in \mathbb{Z}_{\alpha^{-1}}}\sum_{\widetilde{Eq}(\mathcal{C}, \{\sigma_{\mathfrak{n}}\}_{\mathfrak{n}},k)} \sum_{c\in \mathscr{M} }\prod_{\mathfrak{n}\in T_{\text{in}}}\frac{t\alpha}{|c(\Omega)_{\mathfrak{n}}|+\alpha} \sum_{c'\in \mathscr{M}}\prod_{\mathfrak{n}\in T_{\text{in}}}\frac{t\alpha}{|c'(\Omega)_{\mathfrak{n}}|+\alpha} \left(\langle k_1\rangle^{-q}\cdots\langle k_1'\rangle^{-q}\cdots\right).
% \end{equation}
% Since $\widetilde{Eq}(\mathcal{C}, \{\sigma_{\mathfrak{n}}\}_{\mathfrak{n}},k)$ contains equations in $\delta_{\cap_{\mathfrak{n}\in T_{\text{in}}} S_{\mathfrak{n}}}$, $\delta_{\cap_{\mathfrak{n}\in T_{\text{in}}} S'_{\mathfrak{n}}}$ and $\delta_{p}$, so we may remove the three indicator functions from above sum.

% In $\widetilde{Eq}(\mathcal{C}, \{\sigma_{\mathfrak{n}}\}_{\mathfrak{n}},k)$, $|\Omega_{\mathfrak{n}}-\sigma_{\mathfrak{n}}|\le C\alpha$ and then we have $|c(\Omega)_{\mathfrak{n}}-c(\{\sigma_{\mathfrak{n}}\})_{\mathfrak{n}}|\lesssim C\alpha$. We have the freedom of choosing $C$ and we take it sufficiently small so that $|c(\Omega)_{\mathfrak{n}}-c(\{\sigma_{\mathfrak{n}}\})_{\mathfrak{n}}|\le \frac{1}{2}\alpha$. This implies that $|c(\Omega)_{\mathfrak{n}}|+\alpha\gtrsim |c(\{\sigma_{\mathfrak{n}}\})_{\mathfrak{n}}|+\alpha$, so we have
% \begin{equation}\label{eq.lemboundtermTp}
% \begin{split}
%     &Term(T, p)
%     \\
%     \lesssim& \sum_{\sigma_{\mathfrak{n}}\in \mathbb{Z}_{\alpha^{-1}}}\sum_{\widetilde{Eq}(\mathcal{C}, \{\sigma_{\mathfrak{n}}\}_{\mathfrak{n}},k)} \sum_{c,c'\in \mathscr{M} }\prod_{\mathfrak{n}\in T_{\text{in}}}\frac{t\alpha}{|c(\{\sigma_{\mathfrak{n}}\})_{\mathfrak{n}}|+\alpha} \prod_{\mathfrak{n}\in T_{\text{in}}}\frac{t\alpha}{|c'(\{\sigma_{\mathfrak{n}}\})_{\mathfrak{n}}|+\alpha} \left(\langle k_1\rangle^{-q}\cdots\langle k_1'\rangle^{-q}\cdots\right)
%     \\
%     =&\sum_{c,c'\in \mathscr{M} } \sum_{\sigma_{\mathfrak{n}}\in \mathbb{Z}_{\alpha^{-1}}} \prod_{\mathfrak{n}\in T_{\text{in}}}\frac{t\alpha}{|c(\{\sigma_{\mathfrak{n}}\})_{\mathfrak{n}}|+\alpha} \prod_{\mathfrak{n}\in T_{\text{in}}}\frac{t\alpha}{|c'(\{\sigma_{\mathfrak{n}}\})_{\mathfrak{n}}|+\alpha} \underbrace{\sum_{\widetilde{Eq}(\mathcal{C}, \{\sigma_{\mathfrak{n}}\}_{\mathfrak{n}},k)}\left(\langle k_1\rangle^{-q}\cdots\langle k_1'\rangle^{-q}\cdots\right)}_{\widetilde{\text{sum}}(\mathcal{C}, \{\sigma_{\mathfrak{n}}\}_{\mathfrak{n}},k)}
% \end{split}
% \end{equation}

% Let us prove an upper bound for $\widetilde{\text{sum}}(\mathcal{C}, \{\sigma_{\mathfrak{n}}\}_{\mathfrak{n}},k)$. By definition of $\widetilde{Eq}(\mathcal{C}, \{\sigma_{\mathfrak{n}}\}_{\mathfrak{n}},k)$ and $Eq(\mathcal{C}, \{\sigma_{\mathfrak{n}}\}_{\mathfrak{n}},k)$, we know that $\{k_{\mathfrak{e}}\}\in \widetilde{Eq}(\mathcal{C}, \{\sigma_{\mathfrak{n}}\}_{\mathfrak{n}},k)\backslash Eq(\mathcal{C}, \{\sigma_{\mathfrak{n}}\}_{\mathfrak{n}},k)$ if and only if for some $\mathfrak{e}_{*}$, $|k_{\mathfrak{e}_*}|\gtrsim L^+$. Since $k_{\mathfrak{e}}$ are linear combinations of $k_1$, $\cdots$, $k_{2l(T)+1}$, $|k_{\mathfrak{e}_*}|\gtrsim L^+$ implies that for some $j_{*}$, $|k_{j_*}|\gtrsim L^+$. Then we have
% \begin{equation}
%     \widetilde{Eq}(\mathcal{C}, \{\sigma_{\mathfrak{n}}\}_{\mathfrak{n}},k)\subseteq Eq(\mathcal{C}, \{\sigma_{\mathfrak{n}}\}_{\mathfrak{n}},k)\cup \{\{k_{\mathfrak{e}}\}: |k_{j_*}|\gtrsim L^+\, \text{for some }j_*\}.
% \end{equation}

% Take $q$ sufficiently large so that $(L^+)^{-q}\le L^{-10d\, l(T)-10d}$. We get
% \begin{equation}\label{eq.lemboundtermTpsum}
% \begin{split}
%     &\widetilde{\text{sum}}(\mathcal{C}, \{\sigma_{\mathfrak{n}}\}_{\mathfrak{n}},k)=\sum_{\widetilde{Eq}(\mathcal{C}, \{\sigma_{\mathfrak{n}}\}_{\mathfrak{n}},k)}\left(\langle k_1\rangle^{-q}\cdots\langle k_1'\rangle^{-q}\cdots\right)
%     \\
%     \le & \sum_{Eq(\mathcal{C}, \{\sigma_{\mathfrak{n}}\}_{\mathfrak{n}},k)}\left(\langle k_1\rangle^{-q}\cdots\langle k_1'\rangle^{-q}\cdots\right)+\sum_{|k_{j_*}|\gtrsim L^+\, \text{for some }j_*}
%     \left(\langle k_1\rangle^{-q}\cdots\langle k_1'\rangle^{-q}\cdots\right)
%     \\
%     \le &\#Eq(\mathcal{C}, \{\sigma_{\mathfrak{n}}\}_{\mathfrak{n}},k)+O(L^{-8d\, l(T)-8d})
%     \\
%     \le &O(L^{O(\theta)} Q^{n} L^{\frac{1}{2} dn_d})+O(L^{-8d\, l(T)-8d}).
% \end{split}
% \end{equation}
% Here in the last inequality we applied \eqref{eq.countingbd1} and $n$, $n_d$ were defined in Proposition \ref{prop.counting}.

% Combining \eqref{eq.lemboundtermTp} and \eqref{eq.lemboundtermTpsum}, we get
% \begin{equation}
% \begin{split}
%     &Term(T, p)
%     \\
%     \lesssim&\sum_{c,c'\in \mathscr{M} } \sum_{\sigma_{\mathfrak{n}}\in \mathbb{Z}_{\alpha^{-1}}} \prod_{\mathfrak{n}\in T_{\text{in}}}\frac{t\alpha}{|c(\{\sigma_{\mathfrak{n}}\})_{\mathfrak{n}}|+\alpha} \prod_{\mathfrak{n}\in T_{\text{in}}}\frac{t\alpha}{|c'(\{\sigma_{\mathfrak{n}}\})_{\mathfrak{n}}|+\alpha} (\#Eq(\mathcal{C}, \{\sigma_{\mathfrak{n}}\}_{\mathfrak{n}},k)+O(L^{-8d\, l(T)-8d}))
%     \\
%     \lesssim & L^{O(\theta)} Q^{n} L^{\frac{1}{2} dn_d}\sum_{c,c'\in \mathscr{M} } \sum_{\substack{\sigma_{\mathfrak{n}}\in \mathbb{Z}_{\alpha^{-1}}\\ |\sigma_{\mathfrak{n}}|\lesssim 1}} \prod_{\mathfrak{n}\in T_{\text{in}}}\frac{t\alpha}{|c(\{\sigma_{\mathfrak{n}}\})_{\mathfrak{n}}|+\alpha} \prod_{\mathfrak{n}\in T_{\text{in}}}\frac{t\alpha}{|c'(\{\sigma_{\mathfrak{n}}\})_{\mathfrak{n}}|+\alpha}  + O(L^{-6d\, l(T)-6d})
% \end{split}
% \end{equation}
% Here $|\sigma_{\mathfrak{n}}|\lesssim L^+$ in the sum in the third line comes from the fact that $\#Eq(\mathcal{C}, \{\sigma_{\mathfrak{n}}\}_{\mathfrak{n}},k)=0$ if some $\sigma_{\mathfrak{n}}\gtrsim L^+$. This fact is true because in $Eq(\mathcal{C}, \{\sigma_{\mathfrak{n}}\}_{\mathfrak{n}},k)$, all $|k_{\mathfrak{e}}|\lesssim L^+$

% We claim that 
% \begin{equation}\label{eq.lemTpvarianceclaim}
%      \sum_{\substack{\sigma_{\mathfrak{n}}\in \mathbb{Z}_{\alpha^{-1}}\\ |\sigma_{\mathfrak{n}}|\lesssim 1}} \prod_{\mathfrak{n}\in T_{\text{in}}}\frac{t\alpha}{|c(\{\sigma_{\mathfrak{n}}\})_{\mathfrak{n}}|+\alpha} \prod_{\mathfrak{n}\in T_{\text{in}}}\frac{t\alpha}{|c'(\{\sigma_{\mathfrak{n}}\})_{\mathfrak{n}}|+\alpha}\lesssim t^{2l(T)}
% \end{equation}

% Given this claim, we know that 
% \begin{equation}
%     Term(T, p)\lesssim L^{O(\theta)} Q^{n} L^{\frac{1}{2} dn_d} t^{2l(T)},
% \end{equation}
% which proves the lemma.

% \textbf{Correct error!!}

% Now prove the claim. In a tree $T$, there are $l(T)$ branching nodes, so there are $l(T)$ nodes in $T_{\text{in}}$. Label these nodes by $h=1,\cdots,l(T)$ and denote $\sigma_{\mathfrak{n}}$ by $\sigma_{h}$ if $\mathfrak{n}$ is labelled by $h$. Since $\sigma_{h}\in \mathbb{Z}_{\alpha^{-1}}$, there exists $m_{h}\in \mathbb{Z}$ such that $\sigma_{h}=\alpha m_{h}$. \eqref{eq.lemTpvarianceclaim} is thus equivalent to 
% \begin{equation}\label{eq.lemTpvarianceclaim1}
%     \sum_{\substack{m_{h}\in \mathbb{Z}\\ |m_{h}|\lesssim \alpha^{-1}}} \prod_{h=1}^{l(T)}\frac{t}{|c(\{m_{h}\})_{h}|+1} \prod_{h=1}^{l(T)}\frac{t}{|c'(\{m_{h}\})_{h}|+1}\lesssim t^{2l(T)}
% \end{equation}

% Before proving \eqref{eq.lemTpvarianceclaim1}, let's first look at one of its special case. If $c,c'=Id$, then $c(\{m_{h}\})_{h}=c'(\{m_{h}\})_{h}=m_h$. The right hand side of \eqref{eq.lemTpvarianceclaim1} becomes
% \begin{equation}
% \begin{split}
%     t^{2l(T)}\sum_{\substack{m_{h}\in \mathbb{Z}\\ |m_{h}|\lesssim \alpha^{-1}}} \prod_{h=1}^{l(T)}\frac{1}{(|m_{h}|+1)^2} = t^{2l(T)}\prod_{h=1}^{l(T)}\left(\sum_{\substack{m_{h}\in \mathbb{Z}\\ |m_{h}|\lesssim \alpha^{-1}}} \frac{1}{(|m_{h}|+1)^2}\right)
%     \lesssim t^{2l(T)}.
% \end{split}
% \end{equation}
% Here the inequality is because $\sum_{\substack{j\in \mathbb{Z}\\ |j|\lesssim \alpha^{-1}}} \frac{1}{(|j|+1)^2}\le 2$.

% Now we prove \eqref{eq.lemTpvarianceclaim1}. Use Cauchy inequality we get
% \begin{equation}
% \begin{split}
%     &\sum_{\substack{m_{h}\in \mathbb{Z}\\ |m_{h}|\lesssim \alpha^{-1}}} \prod_{h=1}^{l(T)}\frac{t}{|c(\{m_{h}\})_{h}|+1} \prod_{h=1}^{l(T)}\frac{t}{|c'(\{m_{h}\})_{h}|+1}
%     \\
%     \lesssim & \left(\sum_{\substack{m_{h}\in \mathbb{Z}\\ |m_{h}|\lesssim \alpha^{-1}}} \prod_{h=1}^{l(T)}\frac{t^2}{(|c(\{m_{h}\})_{h}|+1)^2} \right)^{\frac{1}{2}} 
%     \left(\sum_{\substack{m_{h}\in \mathbb{Z}\\ |m_{h}|\lesssim \alpha^{-1}}} \prod_{h=1}^{l(T)}\frac{t^2}{(|c'(\{m_{h}\})_{h}|+1)^2} \right)^{\frac{1}{2}}.
% \end{split}
% \end{equation}

% Therefore, we just need to show that 
% \begin{equation}
%     \sum_{m_{h}\in \mathbb{Z}} \prod_{h=1}^{l(T)}\frac{1}{(|c(\{m_{h}\})_{h}|+1)^2}\lesssim 1 
% \end{equation}

% By Euler-Maclaurin formula and change of variable formula, we get
% \begin{equation}
% \begin{split}
%     \sum_{m_{h}\in \mathbb{Z}} \prod_{h=1}^{l(T)}\frac{1}{(|c(\{m_{h}\})_{h}|+1)^2}\le& \int  \prod_{h=1}^{l(T)}\frac{1}{(|c(\{m_{h}\})_{h}|+1)^2}\prod_{h=1}^{l(T)} dm_{h}
%     \\
%     =& \int  \prod_{h=1}^{l(T)}\frac{1}{(|m_{h}|+1)^2}|\text{det}\ c|\prod_{h=1}^{l(T)}  dm_{h}
%     \\
%     =&\prod_{h=1}^{l(T)}\int \frac{1}{(|m_{h}|+1)^2}  dm_{h}
%     \\
%     \lesssim & 1
% \end{split}
% \end{equation}

% Now we complete the proof of the claim and thus the proof of the lemma.
% \end{proof}

% \textbf{{Large deviation of polynomials of Gaussians}
% }


\subsection{A heuristic explanation of the derivation of the wave kinetic equation}\label{sec.heuristic}


\subsection{An upper bound of coefficients in expansion series}\label{sec.uppcoef} In this section, we derive an upper bound for coefficients $H^T_{k_1\cdots k_{2l+1}}$.

Notice that in \eqref{eq.coefterm}, $H^T_{k_1\cdots k_{l+1}}$ are integral of some oscillatory functions. An upper bound can be derived by the standard integration by parts arguments.

Associate each $\mathfrak{n}\in T_{\text{in}}$ with two variables $a_{\mathfrak{n}}$, $b_{\mathfrak{n}}$. Then we define
\begin{equation}\label{eq.defF_T}
F_{T}(t,\{a_{\mathfrak{n}}\}_{\mathfrak{n}\in T_{\text{in}}},\{b_{\mathfrak{n}}\}_{\mathfrak{n}\in T_{\text{in}}})=\int_{\cup_{\mathfrak{n}\in T_{\text{in}}} A_{\mathfrak{n}}} e^{\sum_{\mathfrak{n}\in T_{\text{in}}} it_{\mathfrak{n}} a_{\mathfrak{n}} - \nu(t_{\widehat{\mathfrak{n}}}-t_{\mathfrak{n}})b_{\mathfrak{n}}} \prod_{\mathfrak{n}\in T_{\text{in}}} dt_{\mathfrak{n}} 
\end{equation}

\begin{lem}\label{lem.boundcoef'}
If $c'(t_{\mathfrak{n}})\lesssim 1$, then we have the following upper bound for $F_{T}(t,\{a_{\mathfrak{n}}\}_{\mathfrak{n}},\{b_{\mathfrak{n}}\}_{\mathfrak{n}})$,
\begin{equation}\label{eq.boundcoef'}
    \sup_{\{b_{\mathfrak{n}}\}_{\mathfrak{n}}\lesssim 1} |F_{T}(t,\{a_{\mathfrak{n}}\}_{\mathfrak{n}},\{b_{\mathfrak{n}}\}_{\mathfrak{n}})|\lesssim \sum_{\{d_{\mathfrak{n}}\}_{\mathfrak{n}\in T_{\text{in}}}\in\{0,1\}^{l(T)}}\prod_{\mathfrak{n}\in T_{\text{in}}}\frac{t\alpha}{|q_{\mathfrak{n}}|+\alpha}.
\end{equation}
Fix a sequence $\{d_{\mathfrak{n}}\}_{\mathfrak{n}\in T_{\text{in}}}$ whose elements $d_{\mathfrak{n}}$ takes boolean values $\{0,1\}$. We define the sequence $\{q_{\mathfrak{n}}\}_{\mathfrak{n}\in T_{\text{in}}}$ by the following recursive formula
\begin{equation}\label{eq.q_n'}
    q_{\mathfrak{n}}=
    \begin{cases}
    a_{\mathfrak{r}}, \qquad\qquad \textit{ if $\mathfrak{n}=$ the root $\mathfrak{r}$.}
    \\
    a_{\mathfrak{n}}+d_{\mathfrak{n}}q_{\mathfrak{n}'},\ \ \textit{ if }\mathfrak{n}\neq\mathfrak{r}\textit{ and }\mathfrak{n}'\textit{ is the parent of }\mathfrak{n}.
    \end{cases}
\end{equation}

% \begin{equation}
% Q=\{\{q_{\mathfrak{n}}\}_{\mathfrak{n}\in T_{\text{in}}}:q_{\mathfrak{n}}= q_{\mathfrak{n}}(),\ \forall\mathfrak{n}\in T_{\text{in}}\}    
% \end{equation}
% and the set $Q_{\mathfrak{n}}$ is defined inductively by 
% \begin{equation}\label{eq.treeterm}
%     Q_{\mathfrak{n}}=
%     \begin{cases}
%     \{a_{\mathfrak{r}}\}, \qquad\qquad\qquad\qquad \textit{ if $\mathfrak{n}=$ the root $\mathfrak{r}$.}
%     \\
%     \mathcal{T}_1(\mathcal{J}_{T_{\mathfrak{n}_1}}, \mathcal{J}_{T_{\mathfrak{n}_2}}, \mathcal{J}_{T_{\mathfrak{n}_3}}), \textit{ if $\mathfrak{n}\neq\mathfrak{r}$.}
%     \end{cases}
% \end{equation}
\end{lem}
\begin{proof} The lemma is proved by induction. 

For a tree $T$ contains only one node $\mathfrak{r}$, $F_{T}=1$ and \eqref{eq.boundcoef'} is obviously true.

Assume that \eqref{eq.boundcoef'} is true for trees with $\le n-1$ nodes. We prove the $n$ nodes case. 

For general $T$, let $T_1$, $T_2$, $T_3$ be the three subtrees and $\mathfrak{n}_1$, $\mathfrak{n}_2$, $\mathfrak{n}_3$ be the three children of the root $\mathfrak{r}$, then by the definition of $F_T$\eqref{eq.defF_T}, we get
\begin{equation}
\begin{split}
    &F_{T}(t)=\int_{\cup_{\mathfrak{n}\in T_{\text{in}}} A_{\mathfrak{n}}} e^{-i\sum_{\mathfrak{n}\in T_{\text{in}}} (t_{\mathfrak{n}}a_{\mathfrak{n}}+\alpha c(t_{\mathfrak{n}}) b_{\mathfrak{n}})} \prod_{\mathfrak{n}\in T_{\text{in}}} dt_{\mathfrak{n}}    
    \\
    =&\int_{\cup_{\mathfrak{n}\in T_{\text{in}}} A_{\mathfrak{n}}}e^{-i(t_{\mathfrak{r}}a_{\mathfrak{r}}+\alpha c(t_{\mathfrak{r}})b_{\mathfrak{r}})} e^{-i\sum_{\mathfrak{n}\in T_{1,\text{in}}\cup T_{2,\text{in}}\cup T_{3,\text{in}}} (t_{\mathfrak{n}}a_{\mathfrak{n}}+\alpha c(t_{\mathfrak{n}}) b_{\mathfrak{n}})}  \left(dt_{\mathfrak{r}}\prod_{j=1}^3\prod_{\mathfrak{n}\in T_{j,\text{in}}}dt_{\mathfrak{n}}  \right)
    \\
    =&\int_{\cup_{\mathfrak{n}\in T_{\text{in}}} A_{\mathfrak{n}}}e^{-i(t_{\mathfrak{r}}a_{\mathfrak{r}}+\alpha\, \text{sgn}(a_{\mathfrak{r}}))} e^{-i\alpha(c(t_{\mathfrak{n}}) b_{\mathfrak{n}}- \text{sgn}(a_{\mathfrak{r}}))} e^{-i\sum_{\mathfrak{n}\in \mathfrak{n}\in T_{1,\text{in}}\cup T_{2,\text{in}}\cup T_{3,\text{in}}} (t_{\mathfrak{n}}a_{\mathfrak{n}}+\alpha c(t_{\mathfrak{n}}) b_{\mathfrak{n}})}  \left(dt_{\mathfrak{r}}\prod_{j=1}^3\prod_{\mathfrak{n}\in T_{j,\text{in}}}dt_{\mathfrak{n}}  \right)
\end{split}
\end{equation}

We do integration by parts in above integrals using Stokes formula. Notice that for $t_{\mathfrak{r}}$, there are four inequality constrains, $t_{\mathfrak{r}}\le t$ and $t_{\mathfrak{r}}\ge t_{\mathfrak{n}_1},t_{\mathfrak{n}_2}, t_{\mathfrak{n}_3}$. 


% \begin{equation}\label{eq.lemboundcoefexpand}
% \begin{split}
%     F_{T}(t)=&\frac{i}{a_{\mathfrak{r}}+\alpha \text{sgn}(a_{\mathfrak{r}}) }\int_{\cup_{\mathfrak{n}\in T_{\text{in}}} A_{\mathfrak{n}}} \frac{d}{dt_{\mathfrak{r}}}e^{-i(t_{\mathfrak{r}}a_{\mathfrak{r}}+\alpha\, \text{sgn}(a_{\mathfrak{r}}))} 
%     \\
%     &\qquad\qquad\qquad\qquad\qquad e^{-i\alpha(c(t_{\mathfrak{n}}) b_{\mathfrak{n}}- \text{sgn}(a_{\mathfrak{r}}))} e^{-i\sum_{\mathfrak{n}\in T_{\text{in}}} (t_{\mathfrak{n}}a_{\mathfrak{n}}+\alpha c(t_{\mathfrak{n}}) b_{\mathfrak{n}})}  \left(dt_{\mathfrak{r}}\prod_{j=1}^3\prod_{\mathfrak{n}\in T_{j,\text{in}}}dt_{\mathfrak{n}}  \right)
%     \\
%     =&\frac{i}{a_{\mathfrak{r}}+\alpha \text{sgn}(a_{\mathfrak{r}}) }\left(\int_{\cup_{\mathfrak{n}\in T_{\text{in}}} A_{\mathfrak{n}},\ t_{\mathfrak{r}}=t}-\int_{\cup_{\mathfrak{n}\in T_{\text{in}}} A_{\mathfrak{n}},\ t_{\mathfrak{r}}=t_{\mathfrak{n}_1}}-\int_{\cup_{\mathfrak{n}\in T_{\text{in}}} A_{\mathfrak{n}},\ t_{\mathfrak{r}}=t_{\mathfrak{n}_2}}-\int_{\cup_{\mathfrak{n}\in T_{\text{in}}} A_{\mathfrak{n}},\ t_{\mathfrak{r}}=t_{\mathfrak{n}_3}}\right) 
%     \\
%     &\qquad\qquad\qquad e^{-i(t_{\mathfrak{r}}a_{\mathfrak{r}}+\alpha\, \text{sgn}(a_{\mathfrak{r}}))}e^{-i\alpha(c(t_{\mathfrak{n}}) b_{\mathfrak{n}}- \text{sgn}(a_{\mathfrak{r}}))} e^{-i\sum_{\mathfrak{n}\in T_{\text{in}}} (t_{\mathfrak{n}}a_{\mathfrak{n}}+\alpha c(t_{\mathfrak{n}}) b_{\mathfrak{n}})}  \left(dt_{\mathfrak{r}}\prod_{j=1}^3\prod_{\mathfrak{n}\in T_{j,\text{in}}}dt_{\mathfrak{n}}  \right)
%     \\
%     -&\frac{i}{a_{\mathfrak{r}}+\alpha \text{sgn}(a_{\mathfrak{r}}) }\int_{\cup_{\mathfrak{n}\in T_{\text{in}}} A_{\mathfrak{n}}} e^{-i(t_{\mathfrak{r}}a_{\mathfrak{r}}+\alpha\, \text{sgn}(a_{\mathfrak{r}}))} 
%     \\
%     &\qquad\qquad\qquad\qquad\qquad \frac{d}{dt_{\mathfrak{r}}}e^{-i\alpha(c(t_{\mathfrak{n}}) b_{\mathfrak{n}}- \text{sgn}(a_{\mathfrak{r}}))} e^{-i\sum_{\mathfrak{n}\in T_{\text{in}}} (t_{\mathfrak{n}}a_{\mathfrak{n}}+\alpha c(t_{\mathfrak{n}}) b_{\mathfrak{n}})}  \left(dt_{\mathfrak{r}}\prod_{j=1}^3\prod_{\mathfrak{n}\in T_{j,\text{in}}}dt_{\mathfrak{n}}  \right)
%     \\
%     =& \frac{i}{a_{\mathfrak{r}}+\alpha \text{sgn}(a_{\mathfrak{r}}) }(F_{I}-F_{T^{(1)}}-F_{T^{(2)}}-F_{T^{(3)}}-F_{II})
% \end{split}
% \end{equation}

\begingroup
\allowdisplaybreaks
\begin{align}\label{eq.lemboundcoefexpand}
    &F_{T}(t)=\frac{i}{a_{\mathfrak{r}}+\alpha \text{sgn}(a_{\mathfrak{r}}) }\int_{\cup_{\mathfrak{n}\in T_{\text{in}}} A_{\mathfrak{n}}} \frac{d}{dt_{\mathfrak{r}}}e^{-i(t_{\mathfrak{r}}a_{\mathfrak{r}}+\alpha\, \text{sgn}(a_{\mathfrak{r}}))} \notag
    \\
    &\qquad\qquad\qquad e^{-i\alpha(c(t_{\mathfrak{n}}) b_{\mathfrak{n}}- \text{sgn}(a_{\mathfrak{r}}))} e^{-i\sum_{\mathfrak{n}\in T_{1,\text{in}}\cup T_{2,\text{in}}\cup T_{3,\text{in}}} (t_{\mathfrak{n}}a_{\mathfrak{n}}+\alpha c(t_{\mathfrak{n}}) b_{\mathfrak{n}})}  \left(dt_{\mathfrak{r}}\prod_{j=1}^3\prod_{\mathfrak{n}\in T_{j,\text{in}}}dt_{\mathfrak{n}}  \right)\notag
    \\
    =&\frac{i}{a_{\mathfrak{r}}+\alpha \text{sgn}(a_{\mathfrak{r}}) }\left(\int_{\cup_{\mathfrak{n}\in T_{\text{in}}} A_{\mathfrak{n}},\ t_{\mathfrak{r}}=t}-\int_{\cup_{\mathfrak{n}\in T_{\text{in}}} A_{\mathfrak{n}},\ t_{\mathfrak{r}}=t_{\mathfrak{n}_1}}-\int_{\cup_{\mathfrak{n}\in T_{\text{in}}} A_{\mathfrak{n}},\ t_{\mathfrak{r}}=t_{\mathfrak{n}_2}}-\int_{\cup_{\mathfrak{n}\in T_{\text{in}}} A_{\mathfrak{n}},\ t_{\mathfrak{r}}=t_{\mathfrak{n}_3}}\right) \notag
    \\
    & e^{-i(t_{\mathfrak{r}}a_{\mathfrak{r}}+\alpha\, \text{sgn}(a_{\mathfrak{r}}))}e^{-i\alpha(c(t_{\mathfrak{n}}) b_{\mathfrak{n}}- \text{sgn}(a_{\mathfrak{r}}))} e^{-i\sum_{\mathfrak{n}\in T_{1,\text{in}}\cup T_{2,\text{in}}\cup T_{3,\text{in}}} (t_{\mathfrak{n}}a_{\mathfrak{n}}+\alpha c(t_{\mathfrak{n}}) b_{\mathfrak{n}})}  \left(dt_{\mathfrak{r}}\prod_{j=1}^3\prod_{\mathfrak{n}\in T_{j,\text{in}}}dt_{\mathfrak{n}}  \right) 
    \\
    -&\frac{i}{a_{\mathfrak{r}}+\alpha \text{sgn}(a_{\mathfrak{r}}) }\int_{\cup_{\mathfrak{n}\in T_{\text{in}}} A_{\mathfrak{n}}} e^{-i(t_{\mathfrak{r}}a_{\mathfrak{r}}+\alpha\, \text{sgn}(a_{\mathfrak{r}}))} \notag
    \\
    &\qquad\qquad\qquad \frac{d}{dt_{\mathfrak{r}}}e^{-i\alpha(c(t_{\mathfrak{n}}) b_{\mathfrak{n}}- \text{sgn}(a_{\mathfrak{r}}))} e^{-i\sum_{\mathfrak{n}\in T_{1,\text{in}}\cup T_{2,\text{in}}\cup T_{3,\text{in}}} (t_{\mathfrak{n}}a_{\mathfrak{n}}+\alpha c(t_{\mathfrak{n}}) b_{\mathfrak{n}})}  \left(dt_{\mathfrak{r}}\prod_{j=1}^3\prod_{\mathfrak{n}\in T_{j,\text{in}}}dt_{\mathfrak{n}}  \right) \notag
    \\
    =& \frac{i}{a_{\mathfrak{r}}+\alpha \text{sgn}(a_{\mathfrak{r}}) }(F_{I}-F_{T^{(1)}}-F_{T^{(2)}}-F_{T^{(3)}}-F_{II})\notag
\end{align}
\endgroup








% \begin{equation}\label{eq.lemboundcoefexpand}
%     \begin{split}
%         F_{T}(t)=&\frac{i}{a_{\mathfrak{r}}+\alpha \text{sgn}(a_{\mathfrak{r}}) }\int_{\cup_{\mathfrak{n}\in T_{\text{in}}} A_{\mathfrak{n}}} \frac{d}{dt_{\mathfrak{r}}}e^{-i(t_{\mathfrak{r}}a_{\mathfrak{r}}+\alpha\, \text{sgn}(a_{\mathfrak{r}}))} 
%         \\
%         &\qquad\qquad\qquad\qquad\qquad e^{-i\alpha(c(t_{\mathfrak{n}}) b_{\mathfrak{n}}- \text{sgn}(a_{\mathfrak{r}}))} e^{-i\sum_{\mathfrak{n}\in T_{1,\text{in}}\cup T_{2,\text{in}}\cup T_{3,\text{in}}} (t_{\mathfrak{n}}a_{\mathfrak{n}}+\alpha c(t_{\mathfrak{n}}) b_{\mathfrak{n}})}  \left(dt_{\mathfrak{r}}\prod_{j=1}^3\prod_{\mathfrak{n}\in T_{j,\text{in}}}dt_{\mathfrak{n}}  \right)
%     \end{split}
%     \end{equation}
%     \begin{flalign*}
%     \hspace{1.3cm}
%     =&\frac{i}{a_{\mathfrak{r}}+\alpha \text{sgn}(a_{\mathfrak{r}}) }\left(\int_{\cup_{\mathfrak{n}\in T_{\text{in}}} A_{\mathfrak{n}},\ t_{\mathfrak{r}}=t}-\int_{\cup_{\mathfrak{n}\in T_{\text{in}}} A_{\mathfrak{n}},\ t_{\mathfrak{r}}=t_{\mathfrak{n}_1}}-\int_{\cup_{\mathfrak{n}\in T_{\text{in}}} A_{\mathfrak{n}},\ t_{\mathfrak{r}}=t_{\mathfrak{n}_2}}-\int_{\cup_{\mathfrak{n}\in T_{\text{in}}} A_{\mathfrak{n}},\ t_{\mathfrak{r}}=t_{\mathfrak{n}_3}}\right) &&
%     \\
%     &\qquad\qquad\qquad e^{-i(t_{\mathfrak{r}}a_{\mathfrak{r}}+\alpha\, \text{sgn}(a_{\mathfrak{r}}))}e^{-i\alpha(c(t_{\mathfrak{n}}) b_{\mathfrak{n}}- \text{sgn}(a_{\mathfrak{r}}))} e^{-i\sum_{\mathfrak{n}\in T_{1,\text{in}}\cup T_{2,\text{in}}\cup T_{3,\text{in}}} (t_{\mathfrak{n}}a_{\mathfrak{n}}+\alpha c(t_{\mathfrak{n}}) b_{\mathfrak{n}})}  \left(dt_{\mathfrak{r}}\prod_{j=1}^3\prod_{\mathfrak{n}\in T_{j,\text{in}}}dt_{\mathfrak{n}}  \right) &&
%     \end{flalign*}
%     \begin{flalign*}
%     \hspace{1.3cm}
%     -&\frac{i}{a_{\mathfrak{r}}+\alpha \text{sgn}(a_{\mathfrak{r}}) }\int_{\cup_{\mathfrak{n}\in T_{\text{in}}} A_{\mathfrak{n}}} e^{-i(t_{\mathfrak{r}}a_{\mathfrak{r}}+\alpha\, \text{sgn}(a_{\mathfrak{r}}))} &&
%         \\
%         &\qquad\qquad\qquad\qquad\qquad \frac{d}{dt_{\mathfrak{r}}}e^{-i\alpha(c(t_{\mathfrak{n}}) b_{\mathfrak{n}}- \text{sgn}(a_{\mathfrak{r}}))} e^{-i\sum_{\mathfrak{n}\in T_{1,\text{in}}\cup T_{2,\text{in}}\cup T_{3,\text{in}}} (t_{\mathfrak{n}}a_{\mathfrak{n}}+\alpha c(t_{\mathfrak{n}}) b_{\mathfrak{n}})}  \left(dt_{\mathfrak{r}}\prod_{j=1}^3\prod_{\mathfrak{n}\in T_{j,\text{in}}}dt_{\mathfrak{n}}  \right) &&
%     \end{flalign*}
%     \begin{flalign*}
%     \hspace{1.3cm}
%     = \frac{i}{a_{\mathfrak{r}}+\alpha \text{sgn}(a_{\mathfrak{r}}) }(F_{I}-F_{T^{(1)}}-F_{T^{(2)}}-F_{T^{(3)}}-F_{II}) &&
%     \end{flalign*}


Here $T^{(j)}$, $j=1,2,3$ are trees that is obtained by deleting the root $\mathfrak{r}$, adding edges that connecting $\mathfrak{n}_j$ with other two nodes and defining $\mathfrak{n}_j$ to be the new root. For $T^{(j)}$, we can define the term $F_{T^{(j)}}$ by \eqref{eq.defF_T}. It can be shown that $F_{T^{(j)}}$ defined in this way is the same as the $\int_{\cup_{\mathfrak{n}\in T_{\text{in}}} A_{\mathfrak{n}},\ t_{\mathfrak{r}}=t_{\mathfrak{n}_j}}$ term in the second equality of \eqref{eq.lemboundcoefexpand}, so the last equality of \eqref{eq.lemboundcoefexpand} is true. $F_{I}$ is the $\int_{\cup_{\mathfrak{n}\in T_{\text{in}}} A_{\mathfrak{n}},\ t_{\mathfrak{r}}=t}$ term and $F_{II}$ is the last term containing $\frac{d}{dt_{\mathfrak{r}}}$.

We can apply the induction assumption to $F_{T^{(j)}}$ and show that $\frac{i}{a_{\mathfrak{r}}+\alpha \text{sgn}(a_{\mathfrak{r}})} F_{T^{(j)}}$ can be bounded by the right hand side of \eqref{eq.boundcoef'}.

A direct calculation gives that 
\begin{equation}
    F_{I}(t)=e^{-i(ta_{\mathfrak{r}}+\alpha c(t)b_{\mathfrak{r}})} F_{T_1}(t)F_{T_2}(t)F_{T_3}(t).
\end{equation}
Then the induction assumption implies that $\frac{i}{a_{\mathfrak{r}}+\alpha \text{sgn}(a_{\mathfrak{r}})} F_{I}$ can be bounded by the right hand side of \eqref{eq.boundcoef'}.

Another direct calculation gives that 
\begin{equation}
    F_{II}(t)=\int^t_0 e^{-i(t_{\mathfrak{r}}a_{\mathfrak{r}}+\alpha c(t_{\mathfrak{r}})b_{\mathfrak{r}})} \frac{d}{dt_{\mathfrak{r}}}e^{-i\alpha(c(t_{\mathfrak{n}}) b_{\mathfrak{n}}- \text{sgn}(a_{\mathfrak{r}}))} F_{T_1}(t_{\mathfrak{r}})F_{T_2}(t_{\mathfrak{r}})F_{T_3}(t_{\mathfrak{r}}) dt_{\mathfrak{r}}.
\end{equation}
Applying the induction assumption
\begin{equation}
\begin{split}
    \left| \frac{i}{a_{\mathfrak{r}}+\alpha \text{sgn}(a_{\mathfrak{r}})} F_{II}(t)\right|\le& \frac{\alpha t}{|a_{\mathfrak{r}}|+\alpha}\prod_{j=1}^3\left(\sum_{\{d_{\mathfrak{n}}\}_{\mathfrak{n}\in T_{j,\text{in}}}\in\{0,1\}^{l(T)}}\prod_{\mathfrak{n}\in T_{j,\text{in}}}\frac{t\alpha}{|q_{\mathfrak{n}}|+\alpha}\right)
    \\
    \le& \sum_{\{d_{\mathfrak{n}}\}_{\mathfrak{n}\in T_{\text{in}}}\in\{0,1\}^{l(T)}}\prod_{\mathfrak{n}\in T_{\text{in}}}\frac{t\alpha}{|q_{\mathfrak{n}}|+\alpha}.
\end{split}
\end{equation}.

Combining the bounds of $F_{I}$, $F_{T^{(1)}}$, $F_{T^{(2)}}$, $F_{T^{(3)}}$, $F_{II}$, we conclude that $F_T$ can be bounded by the right hand side of \eqref{eq.boundcoef'} and thus complete the proof of Lemma \ref{lem.boundcoef'}.
\end{proof}
 
% For general $T$, let $T_1$, $T_2$, $T_3$ be the three subtrees of the root, then we have the following recursive formula for $F_{T}$
% \begin{equation}\label{eq.lemboundcoefrecur}
%     F_{T}(t)=\int^t_0 e^{-i(t_{\mathfrak{r}}a_{\mathfrak{r}}+\alpha c(t_{\mathfrak{r}})b_{\mathfrak{r}})} F_{T_1}(t_{\mathfrak{r}})F_{T_2}(t_{\mathfrak{r}})F_{T_3}(t_{\mathfrak{r}}) dt_{\mathfrak{r}}.
% \end{equation}





A straight forward application of above lemma gives following upper bound of the coefficients $H^T_{k_1\cdots k_{2l+1}}$. 

\begin{lem}\label{lem.boundcoef}
Let $m(t)=\mathbb{E}M(t)$, then by mass conservation $m(t_{\mathfrak{n}})\lesssim 1$ for all $t_{\mathfrak{n}}$. We have the following upper bound for $H^T_{k_1\cdots k_{2l+1}}$,
\begin{equation}\label{eq.boundcoef}
    H^T_{k_1\cdots k_{2l+1}}\lesssim \sum_{\{d_{\mathfrak{n}}\}_{\mathfrak{n}\in T_{\text{in}}}\in\{0,1\}^{l(T)}}\prod_{\mathfrak{n}\in T_{\text{in}}}\frac{t\alpha}{|q_{\mathfrak{n}}|+\alpha}\ \delta_{\cap_{\mathfrak{n}\in T_{\text{in}}} S_{\mathfrak{n}}}.
\end{equation}
Fix a sequence $\{d_{\mathfrak{n}}\}_{\mathfrak{n}\in T_{\text{in}}}$ whose elements $d_{\mathfrak{n}}$ takes boolean values $\{0,1\}$. We define the sequence $\{q_{\mathfrak{n}}\}_{\mathfrak{n}\in T_{\text{in}}}$ by the following recursive formula
\begin{equation}\label{eq.q_n}
    q_{\mathfrak{n}}=
    \begin{cases}
    \Omega_{\mathfrak{r}}, \qquad\qquad \textit{ if }\mathfrak{n}=\textit{ the root }\mathfrak{r}.
    \\
    \Omega_{\mathfrak{n}}+d_{\mathfrak{n}}q_{\mathfrak{n}'},\ \ \textit{ if }\mathfrak{n}\neq\mathfrak{r}\textit{ and }\mathfrak{n}'\textit{ is the parent of }\mathfrak{n}.
    \end{cases}
\end{equation}


\end{lem}
\begin{proof}
This is a direct corollary of Lemma \ref{eq.boundcoef'} if we take $a_{\mathfrak{n}}=\Omega_{\mathfrak{n}}$, $b_{\mathfrak{n}}=\widetilde{\Omega}_{\mathfrak{n}}/(2\alpha \int^t_{0}m(s) ds)$ and $c(t_{\mathfrak{n}})=2\int^t_{0}m(s) ds$. 
\end{proof}

















